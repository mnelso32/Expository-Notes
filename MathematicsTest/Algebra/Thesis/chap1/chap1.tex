%% LyX 2.2.3 created this file.  For more info, see http://www.lyx.org/.
%% Do not edit unless you really know what you are doing.
\documentclass[12pt,english]{article}
\usepackage[osf]{mathpazo}
\renewcommand{\sfdefault}{lmss}
\renewcommand{\ttdefault}{lmtt}
\usepackage[T1]{fontenc}
\usepackage[latin9]{inputenc}
\usepackage[paperwidth=30cm,paperheight=35cm]{geometry}
\geometry{verbose,tmargin=2cm,bmargin=2cm}
\setlength{\parindent}{0bp}
\usepackage{amsmath}
\usepackage{amssymb}

\makeatletter
%%%%%%%%%%%%%%%%%%%%%%%%%%%%%% User specified LaTeX commands.
\usepackage{tikz}
\usetikzlibrary{matrix,arrows,decorations.pathmorphing}
\usetikzlibrary{shapes.geometric}
\usepackage{tikz-cd}
\usepackage{amsthm}
\usepackage{xparse,etoolbox}

\theoremstyle{plain}
\newtheorem{theorem}{Theorem}[section]
\newtheorem{lemma}[theorem]{Lemma}
\newtheorem{prop}{Proposition}[section]
\newtheorem*{cor}{Corollary}
\theoremstyle{definition}
\newtheorem{defn}{Definition}[section]
\newtheorem{ex}{Exercise} 
\newtheorem{example}{Example}[section]
\theoremstyle{remark}
\newtheorem*{rem}{Remark}
\newtheorem*{note}{Note}
\newtheorem{case}{Case}
\usepackage{graphicx}
\usepackage{amssymb}
\usepackage{tikz-cd}
\usetikzlibrary{calc,arrows,decorations.pathreplacing}
\tikzset{mydot/.style={circle,fill,inner sep=1.5pt},
commutative diagrams/.cd,
  arrow style=tikz,
  diagrams={>=latex},
}

\usepackage{babel}
\usepackage{hyperref}
\hypersetup{
    colorlinks,
    citecolor=blue,
    filecolor=blue,
    linkcolor=blue,
    urlcolor=blue
}
\usepackage{pgfplots}
\usetikzlibrary{decorations.markings}
\pgfplotsset{compat=1.9}


\newcommand{\blocktheorem}[1]{%
  \csletcs{old#1}{#1}% Store \begin
  \csletcs{endold#1}{end#1}% Store \end
  \RenewDocumentEnvironment{#1}{o}
    {\par\addvspace{1.5ex}
     \noindent\begin{minipage}{\textwidth}
     \IfNoValueTF{##1}
       {\csuse{old#1}}
       {\csuse{old#1}[##1]}}
    {\csuse{endold#1}
     \end{minipage}
     \par\addvspace{1.5ex}}
}

\raggedbottom

\blocktheorem{theorem}% Make theo into a block
\blocktheorem{defn}% Make defi into a block
\blocktheorem{lemma}% Make lem into a block
\blocktheorem{rem}% Make rem into a block
\blocktheorem{cor}% Make col into a block
\blocktheorem{prop}% Make prop into a block


\usepackage[bottom]{footmisc}

\makeatother

\usepackage{babel}
\begin{document}
~~~In this thesis, we study some homological constructions over
a ring $R$ of characteristic $2$. The reason we specialize to case
where the ring has characteristic $2$ is merely for simplicity. Indeed,
nearly all of our results can be generalized by replacing $R$ with
any ring with arbitary characteristic.

~~~In the second chapter, we introduce preliminary material. Section
2.1 of deals with the theory of Gr�bner bases. The main references
we used for this section are \cite{CLO15} and \cite{GP08}. Section
2.2 deals with graded rings and modules. In this section we used various
references, namely \cite{GP08}, \cite{BH98}, and \cite{E99}. Section
2.3 deals with homological algebra. Here, our main reference is \cite{E99},
but we also used \cite{KCExterior}, \cite{BH98}, and \cite{stack}
as well. Section 2.4 deals with simplicial complexes and simplicial
homology. We use \cite{MP99} as a reference here. 

~~~In the third chapter, we study some homological constructions
over a field $K$ of characteristic $2$. In section 3.1, we construct
some chain complexes over $K$. We first show how the polynomial ring
$K[x_{1},\dots,x_{n}]$ can be equipped with the differential $d:=\sum_{\lambda=1}^{n}\partial_{x_{\lambda}}$
so that it becomes a chain complex over $K$. We then use the theory
of Gr�bner bases to show how $K[x_{1},\dots,x_{n}]/I$ can be equipped
with a differential so that it becomes a chain over $K$. In section
3.2, we study differential graded $K$-algebras. In particular, we
classify which of these chain complexes are differential graded $K$-algebras.
In section 3.3, we study the homology of these chain complexes, and
in section 3.4, we give a topological interpretation of these homologies.
Namely, we show how the homology of a chain complex we construct in
3.1 corresponds to the simplicial homology of a simplicial complex. 

~~~In the fourth and final chapter, we study some homological constructions
over a ring $R$ of characteristic $2$. In contrast to the third
chapter, which has more of a topological flavor, this chapter has
more of an algebraic flavor. In this chapter, we show that every finitely-generated
commutative differential graded $R$-algebra is isomorphic to an $R$-algebra
of the form $R[x_{1},\dots,x_{n}]/I$ equipped a differential defined
as linear combination of partial derivatives. We also show how Koszul
complexes and blowup algebras can be interpretted as differential
graded $R$-algebras in this way. We end this chapter with some basic
homology calculations. 
\end{document}
