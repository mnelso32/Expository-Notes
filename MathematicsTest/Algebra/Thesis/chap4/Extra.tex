%% LyX 2.3.3 created this file.  For more info, see http://www.lyx.org/.
%% Do not edit unless you really know what you are doing.
\documentclass[12pt,english]{article}
\usepackage[osf]{mathpazo}
\renewcommand{\sfdefault}{lmss}
\renewcommand{\ttdefault}{lmtt}
\usepackage[T1]{fontenc}
\usepackage[latin9]{inputenc}
\usepackage[paperwidth=30cm,paperheight=35cm]{geometry}
\geometry{verbose,tmargin=2cm,bmargin=2cm}
\setlength{\parindent}{0bp}
\usepackage{amsmath}
\usepackage{amssymb}

\makeatletter
%%%%%%%%%%%%%%%%%%%%%%%%%%%%%% User specified LaTeX commands.
\usepackage{tikz}
\usetikzlibrary{matrix,arrows,decorations.pathmorphing}
\usetikzlibrary{shapes.geometric}
\usepackage{tikz-cd}
\usepackage{amsthm}
\usepackage{xparse,etoolbox}

\theoremstyle{plain}
\newtheorem{theorem}{Theorem}[section]
\newtheorem{lemma}[theorem]{Lemma}
\newtheorem{prop}{Proposition}[section]
\newtheorem*{cor}{Corollary}
\theoremstyle{definition}
\newtheorem{defn}{Definition}[section]
\newtheorem{ex}{Exercise} 
\newtheorem{example}{Example}[section]
\theoremstyle{remark}
\newtheorem*{rem}{Remark}
\newtheorem*{note}{Note}
\newtheorem{case}{Case}
\usepackage{graphicx}
\usepackage{amssymb}
\usepackage{tikz-cd}
\usetikzlibrary{calc,arrows,decorations.pathreplacing}
\tikzset{mydot/.style={circle,fill,inner sep=1.5pt},
commutative diagrams/.cd,
  arrow style=tikz,
  diagrams={>=latex},
}

\usepackage{babel}
\usepackage{hyperref}
\hypersetup{
    colorlinks,
    citecolor=blue,
    filecolor=blue,
    linkcolor=blue,
    urlcolor=blue
}
\usepackage{pgfplots}
\usetikzlibrary{decorations.markings}
\pgfplotsset{compat=1.9}


\newcommand{\blocktheorem}[1]{%
  \csletcs{old#1}{#1}% Store \begin
  \csletcs{endold#1}{end#1}% Store \end
  \RenewDocumentEnvironment{#1}{o}
    {\par\addvspace{1.5ex}
     \noindent\begin{minipage}{\textwidth}
     \IfNoValueTF{##1}
       {\csuse{old#1}}
       {\csuse{old#1}[##1]}}
    {\csuse{endold#1}
     \end{minipage}
     \par\addvspace{1.5ex}}
}

\raggedbottom

\blocktheorem{theorem}% Make theo into a block
\blocktheorem{defn}% Make defi into a block
\blocktheorem{lemma}% Make lem into a block
\blocktheorem{rem}% Make rem into a block
\blocktheorem{cor}% Make col into a block
\blocktheorem{prop}% Make prop into a block


\usepackage[bottom]{footmisc}

\makeatother

\usepackage{babel}
\begin{document}

\subsection*{Classifying $d$-Stable Ideals}

~~~Let $(R[x_{1},\dots,x_{n}]/I,r_{1},\dots,r_{n})$ be a differential
graded $R$-algebra. Suppose that there are \textbf{$t_{1},\dots,t_{m}\in R$
}such that $\langle r_{1},\dots,r_{n}\rangle=\langle t_{1},\dots,t_{m}\rangle$
and $(R[y_{1},\dots,y_{m}]/I,t_{1},\dots,t_{m})$ is also a differential
graded $R$-algebra. Then for all $1\leq\lambda\leq n$ and $1\leq\mu\leq n$,
there are $a_{\lambda\mu}$ and $b_{\lambda\mu}$ in $R$ such that
\[
r_{\lambda}=\sum_{\mu=1}^{m}a_{\lambda\mu}t_{\mu}\text{ and }t_{\mu}=\sum_{\lambda=1}^{n}b_{\lambda\mu}r_{\lambda}.
\]
~~~Let $\varphi:R[x_{1},\dots,x_{n}]\to R[y_{1},\dots,y_{m}]$
be the unique graded $R$-algebra homomorphism such that $\varphi(x_{\lambda})=\sum_{\mu=1}^{m}a_{\lambda\mu}y_{\mu}$
for all $\lambda=1,\dots,n$. Then $\varphi$ induces a graded $R$-algebra
homomorphism $\overline{\varphi}:R[x_{1},\dots,x_{n}]/I\to R[y_{1},\dots,y_{m}]/\langle\varphi(I)\rangle$
which in turn induces a homomorphism of differential graded $R$-algebras
$\overline{\varphi}:(R[x_{1},\dots,x_{n}]/I,r_{1},\dots,r_{n})\to(R[y_{1},\dots,y_{m}]/\langle\varphi(I)\rangle,t_{1},\dots,t_{m})$.
Indeed, let us denote the differentials as
\[
d_{r}:=\sum_{\lambda=1}^{n}r_{\lambda}\partial_{x_{\lambda}}\text{ and }d_{t}:=\sum_{\mu=1}^{m}t_{\mu}\partial_{y_{\mu}}.
\]

We first show that $\varphi d_{r}=d_{t}\varphi$. It is enough to
show that $\varphi d_{r}(x_{\lambda})=d_{t}\varphi(x_{\lambda})$
for all $\lambda=1,\dots,n$. We have 
\begin{align*}
d_{t}\varphi(x_{\lambda}) & =d_{t}\left(\sum_{\mu=1}^{m}a_{\lambda\mu}y_{\mu}\right)\\
 & =\sum_{\mu=1}^{m}a_{\lambda\mu}t_{\mu}\\
 & =r_{\lambda}\\
 & =d_{r}(x_{\lambda})\\
 & =\varphi(d_{r}(x_{\lambda})).
\end{align*}
Now we show that $(R[y_{1},\dots,y_{m}]/\langle\varphi(I)\rangle,t_{1},\dots,t_{m})$
is a differential graded $R$-algebra. We do this by showing that
$\langle\varphi(I)\rangle$ is $d_{t}$-stable. Let $\sum_{\kappa=1}^{r}g_{\kappa}\varphi(f_{\kappa})\in\varphi(I)$.
Then
\begin{align*}
d_{t}\left(\sum_{\kappa=1}^{r}g_{\kappa}\varphi(f_{\kappa})\right) & =\sum_{\kappa=1}^{r}d_{t}(g_{\kappa})\varphi(f_{\kappa})+\sum_{\kappa=1}^{r}g_{\kappa}d_{t}(\varphi(f_{\kappa}))\\
 & =\sum_{\kappa=1}^{r}d_{t}(g_{\kappa})\varphi(f_{\kappa})+\sum_{\kappa=1}^{r}g_{\kappa}\varphi(d_{r}(f_{\kappa}))\in\langle\varphi(I)\rangle.
\end{align*}

Similarly, let $\psi:R[y_{1},\dots,y_{m}]\to R[x_{1},\dots,x_{n}]$
be the unique graded $R$-algebra homomorphism such that $\psi(y_{\mu})=\sum_{\lambda=1}^{n}b_{\lambda\mu}x_{\lambda}$
for all $\mu=1,\dots,m$. Then $\varphi$ induces a graded $R$-algebra
homomorphism $\overline{\psi}:R[y_{1},\dots,y_{m}]/\langle\varphi(I)\rangle\to R[x_{1},\dots,x_{n}]/\langle\psi(\varphi(I))\rangle$
which in turn induces a homomorphism of differential graded $R$-algebras
$\overline{\psi}(R[y_{1},\dots,y_{m}]/\langle\varphi(I)\rangle,t_{1},\dots,t_{m})\to(R[x_{1},\dots,x_{n}]/\langle\psi(\varphi(I))\rangle,r_{1},\dots,r_{n})$. 

\subsubsection{Evalutation Map}

~~~Let $(S/I,r_{1},\dots,r_{n})$ be a differential graded $R$-algebra
such that $I$ is contained in $\langle x_{1},\dots,x_{n}\rangle$.
Let $Q=\langle r_{1},\dots,r_{n}\rangle$ and $\text{Ev}_{r}:S\to R$
be the unique $R$-algebra homomorphism such that $\text{Ev}_{r}(x_{\lambda})=r_{\lambda}$
for all $\lambda=1,\dots,n$. We are interested in the ideal $\text{Ev}_{r}(I)$
in $R$. Clearly we have $\text{Ev}_{r}(I)\subset Q$. Suppose $a\in Q\backslash\text{Ev}_{r}(I)$.
Then $a=\sum_{\lambda=1}^{n}a_{\lambda}r_{\lambda}$ for some $a_{\lambda}\in R$.
This implies $x:=\sum_{\lambda=1}^{n}a_{\lambda}x_{\lambda}\notin I$.
Now $J=I+\langle x,a\rangle$ is an ideal strictly larger than $I$
such that $J$ is $d$-stable $\text{Ev}_{r}(J)$ is strictly larger
than $\text{Ev}_{r}(I)$. This implies that we can find an ideal $I$
such that $\text{Ev}_{r}(I)=Q$.

\begin{defn}\label{defn} We say $I$ is $(Q,d)$-maximal if $I$
is $d$-stable and $\text{Ev}_{r}(I)=Q$. \end{defn}

~~~~Assume that $I$ is $(Q,d)$-maximal. By the paragraph above,
a change of basis of the ideal $Q$ induces an $R$-algebra map $\phi:(S/I,r_{1},\dots,r_{n})\to(S/J,r_{1},\dots,r_{n}),$
where $\text{Ev}_{r}(I)=\text{Ev}_{r}(J)$. Thus $J$ is $(Q,d)$-maximal
too. 

\section{The Case of a Local Ring}

~~~Assume that $R$ is a local ring with maximal ideal $\mathfrak{m}=\langle r_{1},\dots,r_{n}\rangle$
and residue field $K:=R/\mathfrak{m}$. Let $(S/I,r_{1},\dots,r_{n})$
be a differential graded $R$-algebra such that $I$ is contained
in $\langle x_{1},\dots,x_{n}\rangle$ and let $\text{Ev}_{r}:S\to R$
be the unique $R$-algebra homomorphism such that $\text{Ev}_{r}(x_{\lambda})=r_{\lambda}$
for all $\lambda=1,\dots,n$. Then $\text{Ev}_{r}$ induces an $R$-algebra
homomorphism $\text{Ev}_{r}:S/I\to K$, since $\text{Ev}_{r}(I)\subset\mathfrak{m}$. 

\subsection{Long Exact Sequence}

~~~It is straightforward to check that 

\begin{equation}\label{sesdga}\begin{tikzcd}[row sep=5] 0 \arrow[r] & (S_w (-j) /(I:g) \text{,}\overline{d} ) \arrow[r, "\cdot g"] & (S/I \text{,} \overline{d}) \arrow[r] & (S/\langle I \text{,} g \rangle \text{,} \overline{d}) \arrow[r] & 0 \\ & \overline{f} \arrow[r,mapsto,shorten >=0.5cm,shorten <=0.5cm] & \overline{fg} \end{tikzcd}\end{equation}

is short exact sequence of chain complexes. The short exact sequence
(\ref{sesdga}) gives rise to a long exact sequence in homology:

\begin{center}\begin{tikzcd}[row sep=40]  && \cdots \arrow[r] \arrow[d, phantom, ""{coordinate, name=Z'}] & H_{i+1} (S_w  / \langle I \text{,} g \rangle  ) \arrow[dll, " \lambda  ", swap, rounded corners, to path={ -- ([xshift=2ex]\tikztostart.east) |- (Z') [near end]\tikztonodes -| ([xshift=-2ex]\tikztotarget.west) -- (\tikztotarget)}] 



\\  & H_{i-j} (S_w /( I:g )) \arrow[r, "\cdot g"] & H_{i} (S_w / I) \arrow[r] \arrow[d, phantom, ""{coordinate, name=Z}] & H_{i} (S_w / \langle I \text{,} g \rangle  ) \arrow[dll, " \lambda ", swap, rounded corners, to path={ -- ([xshift=2ex]\tikztostart.east) |- (Z) [near end]\tikztonodes -| ([xshift=-2ex]\tikztotarget.west) -- (\tikztotarget)}] 

\\ & H_{i-j-1} (S_w /( I:g ) ) \arrow[r, "\cdot g "] & H_{i-1} (S_w / I ) \arrow[r] & \cdots 

\end{tikzcd}\end{center}

~~~Let us work out the details of the connecting map: Let $\overline{f}$
be a homogeneous element in $S_{w}/\langle I,g\rangle$ which represents
a class in $H_{i}(S_{w}/\langle I,g\rangle)$. In particular, $f\in S$
and $d(f)\in\langle I,g\rangle$. We lift $\overline{f}\in S_{w}/\langle I,g\rangle$
to $S_{w}/I$ and then apply $d$ to get $\overline{d(f)}\in S_{w}/I$.
Since $d(f)\in\langle I,g\rangle$, we can write $d(f)=p+gq$ where
$p\in I$. Thus, $\overline{d(f)}=\overline{gq}$, and this pulls
back to $\overline{q}$ in $S_{w}/(I:g)$. 

~~~Let $(A,d_{A})$ be a differential graded $R$-algebra. If we
start with a chain complex over $R$, then we can construct a differential
graded $A$-module. Indeed, suppose that $(B,d_{B})$ is a chain complex
over $R$. Then $A\otimes_{R}B$ is an $A$-module and a graded $R$-module
whose homogeneous component in degree $k$ is
\[
(A\otimes_{R}B)_{k}:=\bigoplus_{i+j=k}A_{i}\otimes_{R}B_{j}.
\]

We define a differential $d$ on $A\otimes_{R}B$ by first definining
it on the elementary tensors as
\[
d(a\otimes b):=d_{A}(a)\otimes b+(-1)^{\text{deg}(a)}a\otimes d_{B}(b),
\]


\subsubsection{Tensor product of differential graded $R$-algebras}

~~~Let $(R[x_{1},\dots,x_{n}]/I,d_{r})$ and $(R[y_{1},\dots,y_{m}]/J,d_{t})$
be two differential graded $R$-algebras, where 
\[
d_{r}:=\sum_{\lambda=1}^{n}r_{\lambda}\partial_{x_{\lambda}}\text{ and }d_{t}:=\sum_{\mu=1}^{m}t_{\mu}\partial_{y_{\mu}}.
\]
for $r_{\lambda},t_{\mu}\in R$ for all $\lambda=1,\dots,n$ and $\mu=1,\dots,m$.
Then their tensor product over $R$ is 
\[
(R[x_{1},\dots,x_{n}]/I,d_{r})\otimes_{R}(R[y_{1},\dots,y_{m}]/J,d_{t})\cong(R[x_{1},\dots,x_{n},y_{1},\dots,y_{m}]/(I+J),d_{r}+d_{t}).
\]

\begin{example}\label{example} The Koszul complex $\mathcal{K}(r_{1},\dots,r_{n})$
can be realized as a tensor product:
\[
\mathcal{K}(r_{1},\dots,r_{n})\cong\mathcal{K}(r_{1})\otimes\cdots\otimes\mathcal{K}(r_{n}).
\]

\end{example} 

~~~Let $M$ be an $R$-module, and let $(S/I,r_{1},\dots,r_{n})$
be a differential graded $R$-algebra. Recall that $(M\otimes_{R}S/I,d)$
is an $(S/I)$-module. 
\end{document}
