%% LyX 2.2.3 created this file.  For more info, see http://www.lyx.org/.
%% Do not edit unless you really know what you are doing.
\documentclass[12pt,english]{article}
\usepackage[osf]{mathpazo}
\renewcommand{\sfdefault}{lmss}
\renewcommand{\ttdefault}{lmtt}
\usepackage[T1]{fontenc}
\usepackage[latin9]{inputenc}
\usepackage[paperwidth=30cm,paperheight=35cm]{geometry}
\geometry{verbose,tmargin=3cm,bmargin=3cm}
\setlength{\parindent}{0bp}
\usepackage{amsmath}
\usepackage{amssymb}

\makeatletter
%%%%%%%%%%%%%%%%%%%%%%%%%%%%%% User specified LaTeX commands.
\usepackage{tikz}
\usetikzlibrary{matrix,arrows,decorations.pathmorphing}
\usetikzlibrary{shapes.geometric}
\usepackage{tikz-cd}
\usepackage{amsthm}
\theoremstyle{plain}
\newtheorem{theorem}{Theorem}[section]
\newtheorem{lemma}[theorem]{Lemma}
\newtheorem{prop}{Proposition}[section]
\newtheorem*{cor}{Corollary}
\theoremstyle{definition}
\newtheorem{defn}{Definition}[section]
\newtheorem{ex}{Exercise} 
\newtheorem{example}{Example}[section]
\theoremstyle{remark}
\newtheorem*{rem}{Remark}
\newtheorem*{note}{Note}
\newtheorem{case}{Case}
\usepackage{graphicx}
\usepackage{amssymb}
\usepackage{tikz-cd}
\usetikzlibrary{calc,arrows,decorations.pathreplacing}
\tikzset{mydot/.style={circle,fill,inner sep=1.5pt},
commutative diagrams/.cd,
  arrow style=tikz,
  diagrams={>=latex},
}

\usepackage{babel}
\usepackage{hyperref}
\hypersetup{
    colorlinks,
    citecolor=black,
    filecolor=black,
    linkcolor=black,
    urlcolor=black
}
\usepackage{pgfplots}
\usetikzlibrary{decorations.markings}
\pgfplotsset{compat=1.9}


\newcommand{\dotcup}{\ensuremath{\mathaccent\cdot\cup}}

\makeatother

\usepackage{babel}
\begin{document}

\title{Stirling Numbers}

\maketitle
~~~Our goal in this exposition is to show 
\begin{equation}
\sum_{j=0}^{n}s(n,j)S(j,k)=\sum_{j=0}^{n}S(n,j)s(j,k)=\delta_{n,k}\label{eq:1}
\end{equation}

for all $n,k\ge0$. Another way to express (\ref{eq:1}) is by using
matrices: Let $S_{1},S_{2},$ and $I$ be the infinite matrices 
\[
S_{1}=\begin{pmatrix}s(1,1) & 0 & 0 & \cdots\\
s(2,1) & s(2,2) & 0 & \cdots\\
s(3,1) & s(3,2) & s(3,3) & \cdots\\
\vdots & \vdots & \vdots & \ddots
\end{pmatrix},\quad S_{2}=\begin{pmatrix}S(1,1) & 0 & 0 & \cdots\\
S(2,1) & S(2,2) & 0 & \cdots\\
S(3,1) & S(3,2) & S(3,3) & \cdots\\
\vdots & \vdots & \vdots & \ddots
\end{pmatrix},\quad I=\begin{pmatrix}1 & 0 & 0 & \cdots\\
0 & 1 & 0 & \cdots\\
0 & 0 & 1 & \cdots\\
\vdots & \vdots & \vdots & \ddots
\end{pmatrix}.
\]

Then 
\[
S_{1}S_{2}=S_{2}S_{1}=I.
\]

Our method to proving this is to realize these matrices as an appropriate
change of coordinates. 

\subsubsection*{Stirling Numbers of the Second Kind}

\begin{theorem} Let $f$ be an infinitely differentiable function.
Then for all $n\in\mathbb{N}$,
\[
\frac{d^{n}}{dx^{n}}f(e^{x}-1)=\sum_{k=1}^{n}S(n,k)e^{kx}f^{(k)}(e^{x}-1).
\]

\end{theorem}

\begin{proof} We prove this by induction on $n$. The base case $n=1$
is easy to check:
\[
\frac{d}{dx}f(e^{x}-1)=S(1,1)e^{x}f'(e^{x}-1).
\]
Now assume it is true for $n$. Then 
\begin{align*}
\frac{d^{n+1}}{dx^{n+1}}f(e^{x}-1) & =\sum_{k=1}^{n}S(n,k)\frac{d}{dx}\left(e^{kx}f^{(k)}(e^{x}-1)\right)\\
 & =\sum_{k=1}^{n}S(n,k)\left(\frac{d}{dx}\left(e^{kx}\right)f^{(k)}(e^{x}-1)+e^{kx}\frac{d}{dx}\left(f^{(k)}(e^{x}-1)\right)\right)\\
 & =\sum_{k=1}^{n}S(n,k)\left(ke^{kx}f^{(k)}(e^{x}-1)+e^{(k+1)x}f^{(k+1)}(e^{x}-1)\right)\\
 & =\sum_{k=1}^{n+1}S(n+1,k)e^{kx}f^{(k)}(e^{x}-1).
\end{align*}

\end{proof}

\begin{rem} Note that we used the recurrence $S(n+1,k)=kS(n,k)+S(n,k-1)$
in the fourth step. \end{rem}

\begin{cor} Let
\[
f(x)=\sum_{n=1}^{\infty}\frac{a_{n}}{n!}x^{n}
\]

be the exponential generating function of the sequence $(a_{n})$
and let $(b_{n})$ be the sequence obtained from the matrix equation
\[
\begin{pmatrix}S(1,1) & 0 & 0 & \cdots\\
S(2,1) & S(2,2) & 0 & \cdots\\
S(3,1) & S(3,2) & S(3,3) & \cdots\\
\vdots & \vdots & \vdots & \ddots
\end{pmatrix}\begin{pmatrix}a_{1}\\
a_{2}\\
a_{3}\\
\vdots
\end{pmatrix}=\begin{pmatrix}b_{1}\\
b_{2}\\
b_{3}\\
\vdots
\end{pmatrix},
\]

i.e. $b_{n}=\sum_{k=1}^{n}S(n,k)a_{n}$. Then $f(e^{x}-1)$ is the
exponential generating function of $(b_{n})$:
\[
f(e^{x}-1)=\sum_{n=1}^{\infty}\frac{b_{n}}{n!}x^{n}.
\]

\end{cor}

\begin{proof} The coefficient $b_{n}$ is obtained by 
\begin{align*}
b_{n} & =\frac{d^{n}}{dx^{n}}f(e^{x}-1)_{\mid0}\\
 & =\sum_{k=1}^{n}S(n,k)e^{kx}f^{(k)}(e^{x}-1)_{\mid0}\\
 & =\sum_{k=1}^{n}S(n,k)f^{(k)}(0)\\
 & =\sum_{k=1}^{n}S(n,k)a_{n}
\end{align*}

\end{proof}

~~~We have shown how $S_{2}$ corresponds to the change of coordinates
$x\mapsto e^{x}-1$. This seems to indicate that $S_{1}$ should correspond
to the change of coordinates $x\mapsto\log(x+1)$, since $e^{\log(x+1)}-1=x$.
This is indeed the case, as we show in the next section.

\subsubsection*{Stirling Numbers of the First Kind}

\begin{theorem} Let $f$ be an infinitely differentiable function.
Then for all $n\in\mathbb{N}$,
\[
\frac{d^{n}}{dx^{n}}f(\log(x+1))=\frac{1}{(x+1)^{n}}\sum_{k=1}^{n}s(n,k)f^{(k)}(\log(x+1)).
\]

\end{theorem}

\begin{proof} We prove this by induction on $n$. The base case $n=1$
is easy to check:
\[
\frac{d}{dx}f(\log(x+1))=\frac{1}{x+1}s(1,1)f'(\log(x+1)).
\]
Now assume it is true for $n$. Then 
\begin{align*}
\frac{d^{n+1}}{dx^{n+1}}f(\log(x+1)) & =\frac{d}{dx}\left(\frac{1}{(x+1)^{n}}\sum_{k=1}^{n}s(n,k)f^{(k)}(\log(x+1))\right)\\
 & =\frac{d}{dx}\left(\frac{1}{(x+1)^{n}}\right)\sum_{k=1}^{n}s(n,k)f^{(k)}(\log(x+1))+\frac{1}{(x+1)^{n}}\sum_{k=1}^{n}s(n,k)\frac{d}{dx}\left(f^{(k)}(\log(x+1))\right)\\
 & =\frac{-n}{(x+1)^{n+1}}\sum_{k=1}^{n}s(n,k)f^{(k)}(\log(x+1))+\frac{1}{(x+1)^{n}}\sum_{k=1}^{n}s(n,k)\left(f^{(k+1)}(\log(x+1))\right)\\
 & =\frac{1}{(x+1)^{n+1}}\sum_{k=1}^{n}s(n+1,k)f^{(k)}(\log(x+1)).
\end{align*}

\end{proof}

\begin{rem} Note that we used the recurrence $s(n+1,k)=s(n,k-1)-ns(n,k)$
in the fourth step. \end{rem}

\begin{cor} Let
\[
f(x)=\sum_{n=1}^{\infty}\frac{a_{n}}{n!}x^{n}
\]
be the exponential generating function of the sequence $(a_{n})$
and let $(b_{n})$ be the sequence obtained from the matrix equation
and 
\[
\begin{pmatrix}s(1,1) & 0 & 0 & \cdots\\
s(2,1) & s(2,2) & 0 & \cdots\\
s(3,1) & s(3,2) & s(3,3) & \cdots\\
\vdots & \vdots & \vdots & \ddots
\end{pmatrix}\begin{pmatrix}a_{1}\\
a_{2}\\
a_{3}\\
\vdots
\end{pmatrix}=\begin{pmatrix}b_{1}\\
b_{2}\\
b_{3}\\
\vdots
\end{pmatrix},
\]

i.e. $\sum_{k=1}^{n}s(n,k)a_{n}$. Then $f(\log(x+1))$ is the exponential
generating function of $(b_{n})$:
\[
f(\log(x+1))=\sum_{n=1}^{\infty}\frac{b_{n}}{n!}x^{n}.
\]

\end{cor}

\begin{proof} The coefficient $b_{n}$ is obtained by 
\begin{align*}
b_{n} & =\frac{d^{n}}{dx^{n}}f(\log(x+1)){}_{\mid0}\\
 & =\frac{1}{(x+1)^{n}}\sum_{k=1}^{n}s(n,k)f^{(k)}(\log(x+1)){}_{\mid0}\\
 & =\sum_{k=1}^{n}s(n,k)f^{(k)}(0)\\
 & =\sum_{k=1}^{n}s(n,k)a_{n}
\end{align*}

\end{proof}

\subsubsection*{Change of Coordinates}

~~~What does the change of coordinates $x\mapsto e^{x}-1$ look
like? It takes the real line and squashes it in one direction while
stretching it in the other, like this:

\begin{center}
\begin{tikzpicture}
\begin{axis}[axis lines = none,xmin=-20,xmax=20,ymin=-20,ymax=20,scale=2 ]

\addplot [domain=-30:15,samples=100,] {10};
\addplot [domain=-1:15,samples=100,] {-10}; 




\node[circle, fill=black, inner sep=1.5pt, label=below left: $$] (a) at (axis cs:-0.5,-10) {$$};
\node[circle, fill=black, inner sep=1.5pt, label=below:$ 0 $] (b) at (axis cs:0,-10) {$$};
\node[circle, fill=black, inner sep=1.5pt, label=below:$ e - 1 $] (c) at (axis cs:2.8,-10) {$$};
\node[circle, fill=black, inner sep=1.5pt, label=below right:$$] (d) at (axis cs:7.8,-10) {$$};

\node[circle, fill=black, inner sep=1.5pt, label=above:$$] (a') at (axis cs:-8,10) {$$};
\node[circle, fill=black, inner sep=1.5pt, label=above:$ 0 $] (b') at (axis cs:-1,10) {$$};
\node[circle, fill=black, inner sep=1.5pt, label=above:$ 1 $] (c') at (axis cs:0,10) {$$};
\node[circle, fill=black, inner sep=1.5pt, label=above:$$] (d') at (axis cs:1,10) {$$};

\draw[-Latex,opacity=0.25] (b)--(b');
\draw[-Latex,opacity=0.25] (b')--(b);
\draw[-Latex,opacity=0.25] (a)--(a');
\draw[-Latex,opacity=0.25] (a')--(a);
\draw[-Latex,opacity=0.25] (c)--(c');
\draw[-Latex,opacity=0.25] (c')--(c);
\draw[-Latex,opacity=0.25] (d)--(d');
\draw[-Latex,opacity=0.25] (d')--(d);


\end{axis}\end{tikzpicture}\end{center}

Now let $f(x)$ be a function on the real line. What should the corresponding
squashed-stretched function be on the squashed-stretched real line
be? It should be $f(\log(x+1))$. The idea is that we want to use
the inverse change of coordinates $x\mapsto\log(x+1)$ from the squashed-stretched
real line to the real line, and pull back $f(x)$. So for example,
the squashed-stretched version of the function $\sin(x)$ is $\sin(\log(x+1)):$ 

\begin{center}
\begin{tikzpicture}
\begin{axis}[axis lines = none,xmin=-20,xmax=20,ymin=-20,ymax=20,scale=2 ]

\addplot [domain=-20:15,samples=60,color=blue] {sin(deg(x + 1))+ 10 };
\addplot [domain=-0.999999:15,samples=1000,color=red] {sin(deg(ln(x+1))) - 10};
\addplot [domain=-30:15,samples=100,] {10};
\addplot [domain=-1:15,samples=100,] {-10}; 
\addlegendentry{$ \sin (x) $} 
\addlegendentry{$ \sin ( \log (x+1) ) $} 




\node[circle, fill=black, inner sep=1.5pt, label=below left: $$] (a) at (axis cs:-0.5,-10) {$$};
\node[circle, fill=black, inner sep=1.5pt, label=below:$ 0 $] (b) at (axis cs:0,-10) {$$};
\node[circle, fill=black, inner sep=1.5pt, label=below:$ e - 1 $] (c) at (axis cs:2.8,-10) {$$};
\node[circle, fill=black, inner sep=1.5pt, label=below right:$$] (d) at (axis cs:7.8,-10) {$$};

\node[circle, fill=black, inner sep=1.5pt, label=above:$$] (a') at (axis cs:-8,10) {$$};
\node[circle, fill=black, inner sep=1.5pt, label=above:$ 0 $] (b') at (axis cs:-1,10) {$$};
\node[circle, fill=black, inner sep=1.5pt, label=above:$ 1 $] (c') at (axis cs:0,10) {$$};
\node[circle, fill=black, inner sep=1.5pt, label=above:$$] (d') at (axis cs:1,10) {$$};

\draw[-Latex,opacity=0.25] (b)--(b');
\draw[-Latex,opacity=0.25] (b')--(b);
\draw[-Latex,opacity=0.25] (a)--(a');
\draw[-Latex,opacity=0.25] (a')--(a);
\draw[-Latex,opacity=0.25] (c)--(c');
\draw[-Latex,opacity=0.25] (c')--(c);
\draw[-Latex,opacity=0.25] (d)--(d');
\draw[-Latex,opacity=0.25] (d')--(d);


\end{axis}\end{tikzpicture}\end{center}

The squashed-stretched version of the funciton $e^{x}$ is $e^{\log(x+1)}=x+1$: 

\begin{center}
\begin{tikzpicture}
\begin{axis}[axis lines = none,xmin=-20,xmax=20,ymin=-20,ymax=20,scale=2 ]

\addplot [domain=-20:6,samples=60,color=blue] {2^(x-1)+ 10 };
\addplot [domain=-1:12,samples=60,color=red] { x - 9 };
\addplot [domain=-30:15,samples=100,] {10};
\addplot [domain=-1:15,samples=100,] {-10}; 
\addlegendentry{$ e^x $} 
\addlegendentry{$ x + 1 $} 




\node[circle, fill=black, inner sep=1.5pt, label=below left: $$] (a) at (axis cs:-0.5,-10) {$$};
\node[circle, fill=black, inner sep=1.5pt, label=below:$ 0 $] (b) at (axis cs:0,-10) {$$};
\node[circle, fill=black, inner sep=1.5pt, label=below:$ e - 1 $] (c) at (axis cs:2.8,-10) {$$};
\node[circle, fill=black, inner sep=1.5pt, label=below right:$$] (d) at (axis cs:7.8,-10) {$$};

\node[circle, fill=black, inner sep=1.5pt, label=above:$$] (a') at (axis cs:-8,10) {$$};
\node[circle, fill=black, inner sep=1.5pt, label=above:$ 0 $] (b') at (axis cs:-1,10) {$$};
\node[circle, fill=black, inner sep=1.5pt, label=above:$ 1 $] (c') at (axis cs:0,10) {$$};
\node[circle, fill=black, inner sep=1.5pt, label=above:$$] (d') at (axis cs:1,10) {$$};

\draw[-Latex,opacity=0.25] (b)--(b');
\draw[-Latex,opacity=0.25] (b')--(b);
\draw[-Latex,opacity=0.25] (a)--(a');
\draw[-Latex,opacity=0.25] (a')--(a);
\draw[-Latex,opacity=0.25] (c)--(c');
\draw[-Latex,opacity=0.25] (c')--(c);
\draw[-Latex,opacity=0.25] (d)--(d');
\draw[-Latex,opacity=0.25] (d')--(d);


\end{axis}\end{tikzpicture}\end{center}

Let $f(x)$ be a function on the real line with $f(0)=0$. If we know
the taylor expansion of a function $f(x)$ centered at $0$, then
using the stirling matrix $S_{1}$, we can calculate the taylor expansion
of the squashed-stretched version of $f(x)$ centered at $0$. That's
very interesting because the squashing and stretching of the real
line seems to have nothing to do with combinatorics and linear algebra
at first. We will end with an example of how this works in the case
$f(x)=\sin(x)$: The taylor expansion of $\sin(x)$ is 
\[
\sin(x)=x-\frac{1}{3!}x^{3}+\frac{1}{5!}x^{5}-\cdots.
\]

Now calculate $S_{1}(a_{n})$ where $(a_{n})$ corresponds to the
coefficients in $\sin(x)$: 

\[
\begin{pmatrix}1 & 0 & 0 & 0 & 0 & 0 & \cdots\\
-1 & 1 & 0 & 0 & 0 & 0 & \cdots\\
2 & -3 & 1 & 0 & 0 & 0 & \cdots\\
-6 & 11 & -6 & 1 & 0 & 0 & \cdots\\
24 & -50 & 35 & -10 & 1 & 0 & \cdots\\
\vdots & \vdots & \vdots & \vdots & \vdots & \vdots & \ddots
\end{pmatrix}\begin{pmatrix}1\\
0\\
-1\\
0\\
1\\
\vdots
\end{pmatrix}=\begin{pmatrix}1\\
-1\\
1\\
0\\
-10\\
\vdots
\end{pmatrix}.
\]

So the taylor expansion of $\sin(\log(x+1))$ centered at $0$ is
\[
\sin(\log(x+1))=x-\frac{1}{2!}x^{2}+\frac{1}{3!}x^{3}-\frac{10}{5!}x^{5}+\cdots.
\]

\end{document}
