%% LyX 2.3.3 created this file.  For more info, see http://www.lyx.org/.
%% Do not edit unless you really know what you are doing.
\documentclass[12pt,english]{article}
\usepackage[osf]{mathpazo}
\renewcommand{\sfdefault}{lmss}
\renewcommand{\ttdefault}{lmtt}
\usepackage[T1]{fontenc}
\usepackage[latin9]{inputenc}
\usepackage[paperwidth=30cm,paperheight=35cm]{geometry}
\geometry{verbose,tmargin=2cm,bmargin=2cm}
\setlength{\parindent}{0bp}
\usepackage{amsmath}
\usepackage{amssymb}

\makeatletter
\@ifundefined{date}{}{\date{}}
%%%%%%%%%%%%%%%%%%%%%%%%%%%%%% User specified LaTeX commands.
\usepackage{tikz}
\usetikzlibrary{matrix,arrows,decorations.pathmorphing}
\usetikzlibrary{shapes.geometric}
\usepackage{tikz-cd}
\usepackage{amsthm}
\usepackage{xparse,etoolbox}

\theoremstyle{plain}
\newtheorem{theorem}{Theorem}[section]
\newtheorem{lemma}[theorem]{Lemma}
\newtheorem{prop}{Proposition}[section]
\newtheorem*{cor}{Corollary}
\theoremstyle{definition}
\newtheorem{defn}{Definition}[section]
\newtheorem{ex}{Exercise} 
\newtheorem{example}{Example}[section]
\theoremstyle{remark}
\newtheorem*{rem}{Remark}
\newtheorem*{note}{Note}
\newtheorem{case}{Case}
\usepackage{graphicx}
\usepackage{amssymb}
\usepackage{tikz-cd}
\usetikzlibrary{calc,arrows,decorations.pathreplacing}
\tikzset{mydot/.style={circle,fill,inner sep=1.5pt},
commutative diagrams/.cd,
  arrow style=tikz,
  diagrams={>=latex},
}

\usepackage{babel}
\usepackage{hyperref}
\hypersetup{
    colorlinks,
    citecolor=blue,
    filecolor=blue,
    linkcolor=blue,
    urlcolor=blue
}
\usepackage{pgfplots}
\usetikzlibrary{decorations.markings}
\pgfplotsset{compat=1.9}


\newcommand{\blocktheorem}[1]{%
  \csletcs{old#1}{#1}% Store \begin
  \csletcs{endold#1}{end#1}% Store \end
  \RenewDocumentEnvironment{#1}{o}
    {\par\addvspace{1.5ex}
     \noindent\begin{minipage}{\textwidth}
     \IfNoValueTF{##1}
       {\csuse{old#1}}
       {\csuse{old#1}[##1]}}
    {\csuse{endold#1}
     \end{minipage}
     \par\addvspace{1.5ex}}
}

\raggedbottom

\blocktheorem{theorem}% Make theo into a block
\blocktheorem{defn}% Make defi into a block
\blocktheorem{lemma}% Make lem into a block
\blocktheorem{rem}% Make rem into a block
\blocktheorem{cor}% Make col into a block
\blocktheorem{prop}% Make prop into a block


\usepackage[bottom]{footmisc}

\makeatother

\usepackage{babel}
\begin{document}
\title{Linear Analysis Homework 3}
\author{Michael Nelson}
\maketitle

\subsection*{Problem 1}

\begin{prop}\label{propproblem1} Let $\mathcal{K}$ be a closed subspace
of a Hilbert space $\mathcal{H}$ and let $x\in\mathcal{H}$. Then
\begin{enumerate}
\item $\text{P}_{\mathcal{K}}x=x$ if and only if $x\in\mathcal{K}$.
\item $\|\text{P}_{\mathcal{K}}x\|=\|x\|$ if and only if $x\in\mathcal{K}$.
\item $\langle\text{P}_{\mathcal{K}}x,x\rangle=\|\text{P}_{\mathcal{K}}x\|^{2}$.
\end{enumerate}
\end{prop}

\begin{proof}\label{proof} \hfill
\begin{enumerate}
\item If $\text{P}_{\mathcal{K}}x=x$, then it is clear that $x\in\mathcal{K}$
since $\text{P}_{\mathcal{K}}x\in\mathcal{K}$. For the reverse direction,
suppose $x\in\mathcal{K}$. Then
\begin{align*}
0 & =\|x-x\|\\
 & \geq d(x,\mathcal{K})\\
 & =\|x-\text{P}_{\mathcal{K}}x\|\\
 & \geq0
\end{align*}
implies $\|x-x\|=d(x,\mathcal{K})=\|x-\text{P}_{\mathcal{K}}x\|$,
and so by uniqueness of $\text{P}_{\mathcal{K}}x$, we must have $x=\text{P}_{\mathcal{K}}x$.
\item If $x\in\mathcal{K}$, then it is clear that $\|\text{P}_{\mathcal{K}}x\|=\|x\|$
since $x=\text{P}_{\mathcal{K}}x$ by 1. For the reverse direction,
suppose $\|\text{P}_{\mathcal{K}}x\|=\|x\|$. Since $\langle x-\text{P}_{\mathcal{K}}x,\text{P}_{\mathcal{K}}x\rangle=0$,
the Pythagorean Theorem\footnote{Theorem~(\ref{pythagoreantheorem}) in the Appendix. }
implies
\begin{align*}
\|x\|^{2} & =\|x-\text{P}_{\mathcal{K}}x+\text{P}_{\mathcal{K}}x\|^{2}\\
 & =\|x-\text{P}_{\mathcal{K}}x\|^{2}+\|\text{P}_{\mathcal{K}}x\|^{2}\\
 & =\|x-\text{P}_{\mathcal{K}}x\|^{2}+\|x\|^{2}.
\end{align*}
Thus $\|x-\text{P}_{\mathcal{K}}x\|^{2}=0$, which implies $x=\text{P}_{\mathcal{K}}x$
since the metric is positive definite.
\item We have
\begin{align*}
0 & =\langle x-\text{P}_{\mathcal{K}}x,\text{P}_{\mathcal{K}}x\rangle\\
 & =\langle x,\text{P}_{\mathcal{K}}x\rangle-\langle\text{P}_{\mathcal{K}}x,\text{P}_{\mathcal{K}}x\rangle\\
 & =\langle x,\text{P}_{\mathcal{K}}x\rangle-\|\text{P}_{\mathcal{K}}x\|^{2},
\end{align*}
which implies $\langle x,\text{P}_{\mathcal{K}}x\rangle=\|\text{P}_{\mathcal{K}}x\|^{2}$.
Since $\|\text{P}_{\mathcal{K}}x\|^{2}$ is a real number, this implies
$\langle\text{P}_{\mathcal{K}}x,x\rangle=\|\text{P}_{\mathcal{K}}x\|^{2}$. 
\end{enumerate}
\end{proof}

\subsection*{Problem 2}

\begin{prop}\label{prop} Let $\mathcal{K}_{1}$ and $\mathcal{K}_{2}$
be closed subspaces of a Hilbert space $\mathcal{H}$. Then $\mathcal{K}_{1}\subseteq\mathcal{K}_{2}$
if and only if $\langle\text{P}_{\mathcal{K}_{1}}x,x\rangle\leq\langle\text{P}_{\mathcal{K}_{2}}x,x\rangle$
for all $x\in\mathcal{H}$. \end{prop}

\begin{proof}\label{proof} By Proposition~(\ref{propproblem1}),
we can replace the condition $\langle\text{P}_{\mathcal{K}_{1}}x,x\rangle\leq\langle\text{P}_{\mathcal{K}_{2}}x,x\rangle$
for all $x\in\mathcal{H}$ with $\|\text{P}_{\mathcal{K}_{1}}x\|^{2}\leq\|\text{P}_{\mathcal{K}_{2}}x\|^{2}$
for all $x\in\mathcal{H}$. Suppose $\mathcal{K}_{1}\subseteq\mathcal{K}_{2}$.
Then
\begin{align*}
\|x-\text{P}_{\mathcal{K}_{2}}x\| & =d(x,\mathcal{K}_{2})\\
 & =\inf\{\|x-y\|\mid y\in\mathcal{K}_{2}\}\\
 & \leq\inf\{\|x-y\|\mid y\in\mathcal{K}_{1}\}\\
 & =d(x,\mathcal{K}_{1})\\
 & =\|x-\text{P}_{\mathcal{K}_{1}}x\|.
\end{align*}
Therefore by the Pythagorean Theorem, we have
\begin{align*}
\|\text{P}_{\mathcal{K}_{1}}x\|^{2} & =\|x\|^{2}-\|x-\text{P}_{\mathcal{K}_{1}}x\|^{2}\\
 & \leq\|x\|^{2}-\|x-\text{P}_{\mathcal{K}_{2}}x\|^{2}\\
 & =\|\text{P}_{\mathcal{K}_{2}}x\|^{2}.
\end{align*}

~~~Conversely, suppose $\|\text{P}_{\mathcal{K}_{1}}x\|^{2}\leq\|\text{P}_{\mathcal{K}_{2}}x\|^{2}$
for all $x\in\mathcal{H}$. Equivalently, by the Pythagorean Theorem,
we have
\begin{align*}
\|x-\text{P}_{\mathcal{K}_{1}}x\|^{2} & =\|x\|^{2}-\|\text{P}_{\mathcal{K}_{1}}x\|^{2}\\
 & \leq\|x\|^{2}-\|\text{P}_{\mathcal{K}_{2}}x\|^{2}\\
 & =\|x-\text{P}_{\mathcal{K}_{2}}x\|^{2}
\end{align*}
for all $x\in\mathcal{K}_{1}$. Now let $x\in\mathcal{K}_{1}$. Then
$x=\text{P}_{\mathcal{K}_{1}}x$ by Proposition~(\ref{propproblem1}).
Thus
\begin{align*}
0 & =\|x-x\|^{2}\\
 & =\|x-\text{P}_{\mathcal{K}_{1}}x\|^{2}\\
 & \geq\|x-\text{P}_{\mathcal{K}_{2}}x\|^{2},
\end{align*}
which implies $x=\text{P}_{\mathcal{K}_{2}}x$ since the metric is
positive definite. Applying Proposition~(\ref{propproblem1}) again,
we see that $x\in\mathcal{K}_{2}$, and hence $\mathcal{K}_{1}\subseteq\mathcal{K}_{2}$.
\end{proof}

\subsection*{Problem 3}

\begin{prop}\label{prop} Let $\mathcal{K}$ be a closed subspace
of a Hilbert space $\mathcal{H}$. Then $\|\text{P}_{\mathcal{K}^{\perp}}x\|=d(x,\mathcal{K})$
for all $x\in\mathcal{H}$. \end{prop}

\begin{proof} From a theorem\footnote{Theorem~(\ref{theoremorthprop}) in the Appendix.}
we proved in class, we know that $x$ can be uniquely decomposed as
\begin{equation}
x=\text{P}_{\mathcal{K}}x+(x-\text{P}_{\mathcal{K}}x),\label{eq:decomp}
\end{equation}
for unique$\text{P}_{\mathcal{K}}x\in\mathcal{K}$ and unique $x-\text{P}_{\mathcal{K}}x\in\mathcal{K}^{\perp}$.
Since $\mathcal{K}^{\perp}$ is another closed subspace\footnote{This was also shown in class and is given in Theorem~(\ref{theoremorthprop})
in the Appendix} of $\mathcal{H}$, we can uniquely decompose $x$ as
\begin{equation}
x=\text{P}_{\mathcal{K}^{\perp}}x+(x-\text{P}_{\mathcal{K}^{\perp}}x)\label{eq:decomp2}
\end{equation}
for unique $\text{P}_{\mathcal{K}^{\perp}}x\in\mathcal{K}^{\perp}$
and unique $x-\text{P}_{\mathcal{K}^{\perp}}x\in(\mathcal{K}^{\perp})^{\perp}=\mathcal{K}$.
It follows from uniqueness of (\ref{eq:decomp}) and (\ref{eq:decomp2})
that
\[
\text{P}_{\mathcal{K}^{\perp}}x=x-\text{P}_{\mathcal{K}}x\quad\text{and}\quad\text{P}_{\mathcal{K}}x=x-\text{P}_{\mathcal{K}^{\perp}}x
\]
In particular, we have
\begin{align*}
d(x,\mathcal{K}) & =\|x-\text{P}_{\mathcal{K}}x\|\\
 & =\|\text{P}_{\mathcal{K}^{\perp}}x\|.
\end{align*}

\end{proof}

\subsection*{Problem 4}

\begin{prop}\label{propspan} Let $\mathcal{V}$ be an inner-product
space and let $E\subseteq V$. Define
\[
\text{Span}(E):=\left\{ \sum_{i=1}^{n}\lambda_{i}v_{i}\mid n\in\mathbb{N}\text{, }\lambda_{i}\in\mathbb{C}\text{, and }v_{i}\in E\text{ for }1\leq i\leq n\right\} 
\]
Then
\begin{enumerate}
\item $\text{Span}(E)$ is a subspace of $\mathcal{V}$.
\item $\text{Span}(E)$ is the smallest subspace containing $E$.
\end{enumerate}
\end{prop}

\begin{proof}\label{proof}\hfill
\begin{enumerate}
\item Let $\lambda\in\mathbb{C}$ and let $v,w\in\text{Span}(E)$ where
\[
v=\sum_{i=1}^{m}\lambda_{i}v_{i}\text{ and }w=\sum_{j=1}^{n}\mu_{j}w_{j}
\]
where $\lambda_{i},\mu_{j}\in\mathbb{C}$ and $v_{i},w_{j}\in E$
for all $1\leq i\leq m$ and $1\leq j\leq n$. Then
\begin{align*}
\lambda v+w & =\sum_{i=1}^{m}\lambda\lambda_{i}v_{i}+\sum_{j=1}^{n}\lambda\mu_{j}w_{j}\\
 & =\sum_{i=1}^{m}\kappa_{k}u_{k}\in\text{Span}(E),
\end{align*}
where $\kappa_{k}=\lambda\lambda_{k}$ and $u_{k}=v_{k}$ for $1\leq k\leq m$
and $\kappa_{k}=\lambda\mu_{k-m}$ and $u_{k}=w_{k-m}$ for $m<k\leq m+n$.
Therefore $\text{Span}(E)$ is a subspace of $\mathcal{V}$.
\item Let $\mathcal{U}$ be any subspace of $\mathcal{V}$ which contains
$E$. Suppose that $v\in\text{Span}(E)$, where
\begin{equation}
v=\sum_{i=1}^{n}\lambda_{i}v_{i}\label{eq:finitelinear}
\end{equation}
where $\lambda_{i}\in\mathbb{C}$ and $v_{i}\in E$ for all $1\leq i\leq n$.
As $\mathcal{U}$ is a subspace of $\mathcal{V}$, it must be closed
under taking finite linear combinations of elements in $\mathcal{U}$.
Since for each $1\leq i\leq n$, we have $\lambda_{i}\in\mathbb{C}$
and $v_{i}\in E\subseteq\mathcal{U}$, it is clear that from (\ref{eq:finitelinear})
that $v\in\mathcal{U}$. Thus $\text{Span}(E)\subseteq\mathcal{U}$. 
\end{enumerate}
\end{proof}

\subsection*{Problem 5}

\begin{prop}\label{propspanbar} Let $\mathcal{V}$ be an inner-product
space and let $E\subseteq V$. Define the closed span of $E$, denoted
$\overline{\text{Span}}(E)$, as the closure of $\text{Span}(E)$.
Then
\begin{enumerate}
\item $\overline{\text{Span}}(E)$ is a closed subspace of $\mathcal{V}$.
\item $\overline{\text{Span}}(E)$ is the smallest closed subspace containing
$E$.
\end{enumerate}
\end{prop}

\begin{proof}\label{proof}\hfill
\begin{enumerate}
\item By Proposition~(\ref{propspan}), we know that $\text{Span}(E)$ is
a subspace. By a theorem\footnote{Theorem~(\ref{theoremclosureofsubspaceissubspace}) in the Appendix.}
which we proved in class, the closure of a subspace is a closed subspace.
Therefore $\overline{\text{Span}}(E)$ is a closed subspace of $\mathcal{V}$.
\item Let $\mathcal{U}$ be any closed subspace of $\mathcal{V}$ which
contains $E$. By Proposition~(\ref{propspan}), we know that $\text{Span}(E)\subseteq\mathcal{U}$.
Therefore
\begin{align*}
\overline{\text{Span}}(E) & =\overline{\text{Span}(E)}\\
 & \subseteq\overline{\mathcal{U}}\\
 & =\mathcal{U},
\end{align*}
where $\mathcal{U}=\overline{\mathcal{U}}$ since $\mathcal{U}$ is
closed (this was proved in the second homework).
\end{enumerate}
\end{proof}

\subsection*{Problem 6}

\begin{prop}\label{prop} Let $\mathcal{H}$ be a Hilbert space and
let $E\subseteq\mathcal{H}$. Then
\[
(E^{\perp})^{\perp}=\overline{\text{Span}}(E).
\]
\end{prop}

\begin{proof}\label{proof} First note that $E\subseteq(E^{\perp})^{\perp}$.
Indeed, if $x\in E$, then $\langle x,y\rangle=0$ for all $y\in E^{\perp}$,
hence $x\in(E^{\perp})^{\perp}$. Also, from a theorem\footnote{Theorem~(\ref{theoremorthprop}) in the Appendix}
we proved in class, we know that $(E^{\perp})^{\perp}$ is a closed
subspace. Thus $(E^{\perp})^{\perp}$ is a closed subspace which contains
$E$, which implies $(E^{\perp})^{\perp}\supseteq\overline{\text{Span}}(E)$
by Proposition~(\ref{propspanbar}). 

~~~Converesely, since taking orthononal complements is inclusion-reversing\footnote{Theorem~(\ref{theoremorthprop}) in the Appendix},
$E\subseteq\overline{\text{Span}}(E)$ implies $E^{\perp}\supseteq\overline{\text{Span}}(E)^{\perp}$
which implies $(E^{\perp})^{\perp}\subseteq(\overline{\text{Span}}(E)^{\perp})^{\perp}=\overline{\text{Span}}(E)$,
where the last equality follows from a theorem\footnote{Theorem~(\ref{theoremorthprop}) in the Appendix}
in class. \end{proof}

\subsection*{Appendix}

\begin{theorem}\label{pythagoreantheorem} (Pythagorean Theorem) Let
$x$ and $y$ be nonzero vectors in $\mathcal{V}$ such that $\langle x,y\rangle=0$
(we call such vectors \textbf{orthogonal }to one another). Then
\[
\|x+y\|^{2}=\|x\|^{2}+\|y\|^{2}.
\]
\end{theorem}

\begin{proof} We have
\begin{align*}
\|x+y\|^{2} & =\langle x+y,x+y\rangle\\
 & =\langle x,x\rangle+\langle x,y\rangle+\langle y,x\rangle+\langle y,y\rangle\\
 & =\langle x,x\rangle+\langle y,y\rangle\\
 & =\|x\|^{2}+\|y\|^{2}
\end{align*}

\end{proof}

\begin{theorem}\label{theoremclosureofsubspaceissubspace} Let $\mathcal{U}$
be a subspace of $\mathcal{V}$. Then $\overline{\mathcal{U}}$ is
a subspace of $\mathcal{V}$. \end{theorem}

\begin{proof} Let $x,y\in\overline{\mathcal{U}}$ and $\lambda\in\mathbb{C}$.
Let $(x_{n})$ and $(y_{n})$ be two sequences of elements in $\mathcal{U}$
such that $x_{n}\to x$ and $y_{n}\to y$. Then $(\lambda x_{n}+y_{n})$
is a sequence of elements in $\mathcal{U}$ such that $\lambda x_{n}+y_{n}\to\lambda x+y$.
Therefore $\lambda x+y\in\overline{\mathcal{U}}$, which implies $\overline{\mathcal{U}}$
is a subspace of $\mathcal{V}$. \end{proof}

\begin{theorem}\label{theoremorthprop} Let $\mathcal{H}$ be a Hilbert
space and let $\mathcal{K}\subseteq\mathcal{L}\subseteq\mathcal{H}$.
Then
\begin{enumerate}
\item we have $\mathcal{K}^{\perp}\supseteq\mathcal{L}^{\perp}$.
\item $\mathcal{K}^{\perp}$ is a closed subspace of $\mathcal{H}$.
\item If $\mathcal{K}$ is a closed subspace of $\mathcal{H}$, then every
$x\in\mathcal{H}$ can be decomposed in a \emph{unique }way as a sum
of a vector in $\mathcal{K}$ and a vector in $\mathcal{K}^{\perp}$.
In other words, we have $\mathcal{H}=\mathcal{K}\oplus\mathcal{K}^{\perp}$.
\item If $\mathcal{K}$ is a closed subspace of $\mathcal{H}$, then $(\mathcal{K}^{\perp})^{\perp}=\mathcal{K}$.
\end{enumerate}
\end{theorem}

\begin{proof}\label{proof} \hfill
\begin{enumerate}
\item We have
\begin{align*}
x\in\mathcal{L}^{\perp} & \implies\langle x,y\rangle=0\text{ for all }y\in\mathcal{L}\\
 & \implies\langle x,y\rangle=0\text{ for all }y\in\mathcal{K}\\
 & \implies x\in\mathcal{K}^{\perp}.
\end{align*}
Thus $\mathcal{K}^{\perp}\supseteq\mathcal{L}^{\perp}$.
\item First we show that $\mathcal{K}^{\perp}$ is a subspace of $\mathcal{V}$.
Let $x,z\in\mathcal{K}^{\perp}$ and $\lambda\in\mathbb{C}$. Then
\begin{align*}
\langle x+\lambda z,y\rangle & =\langle x,y\rangle+\lambda\langle z,y\rangle\\
 & =0
\end{align*}
for all $y\in\mathcal{K}$. This implies $\mathcal{K}^{\perp}$ is
a subspace of $\mathcal{V}$. Now we will show that $\mathcal{K}^{\perp}$
is closed. Let $(x_{n})$ be a sequence of points in $\mathcal{K}^{\perp}$
such that $x_{n}\to x$ for some $x\in\mathcal{H}$. Then since $\langle x_{n},y\rangle=0$
for all $n\in\mathbb{N}$ and $y\in\mathcal{K}$, we have
\begin{align*}
\langle x,y\rangle & =\lim_{n\to\infty}\langle x_{n},y\rangle\\
 & =\lim_{n\to\infty}0\\
 & =0.
\end{align*}
 for all $y\in\mathcal{K}$. Therefore $x\in\mathcal{K}^{\perp}$,
which implies $\mathcal{K}^{\perp}$ is closed.
\item Let $x\in\mathcal{H}$. Then $x=\text{P}_{\mathcal{K}}x+x-\text{P}_{\mathcal{K}}x$
where $\text{P}_{\mathcal{K}}x\in\mathcal{K}$ and $x-\text{P}_{\mathcal{K}}x\in\mathcal{K}^{\perp}$.
This establishes existence. For uniqueness, first note that $\mathcal{K}\cap\mathcal{K}^{\perp}=\{0\}$.
Indeed, if $y\in\mathcal{K}\cap\mathcal{K}^{\perp}$, then we must
have $\langle y,y\rangle=0$, which implies $y=0$. Now suppose that
$x=y+z$ is another decomposition of $x$ where $y\in\mathcal{K}$
and $z\in\mathcal{K}^{\perp}$. Then we have
\[
(\text{P}_{\mathcal{K}}x)+(x-\text{P}_{\mathcal{K}}x)=x=y+z
\]
implies $\text{P}_{\mathcal{K}}x-y=(x-\text{P}_{\mathcal{K}}x)-z$
which implies $\text{P}_{\mathcal{K}}x-y\in\mathcal{K}\cap\mathcal{K}^{\perp}=\{0\}$
and $(x-\text{P}_{\mathcal{K}}x)-z\in\mathcal{K}\cap\mathcal{K}^{\perp}=\{0\}$.
\item Let $x\in\mathcal{K}$. Then $\langle x,y\rangle=0$ for all $y\in\mathcal{K}^{\perp}$.
Thus $x\in(\mathcal{K}^{\perp})^{\perp}$, and so $\mathcal{K}\subseteq(\mathcal{K}^{\perp})^{\perp}$.
Conversely, let $x\in(\mathcal{K}^{\perp})^{\perp}$. Then $\langle x,y\rangle=0$
for all $y\in\mathcal{K}^{\perp}$. In particular, we have
\begin{align*}
\|x-\text{P}_{\mathcal{K}}x\|^{2} & =\langle x-\text{P}_{\mathcal{K}}x,x-\text{P}_{\mathcal{K}}x\rangle\\
 & =\langle x,x-\text{P}_{\mathcal{K}}x\rangle-\langle\text{P}_{\mathcal{K}}x,x-\text{P}_{\mathcal{K}}x\rangle\\
 & =0-0\\
 & =0,
\end{align*}
which implies $x=\text{P}_{\mathcal{K}}x$. This implies $x\in\mathcal{K}$,
and hence $(\mathcal{K}^{\perp})^{\perp}\subseteq\mathcal{K}$. 
\end{enumerate}
\end{proof}
\end{document}
