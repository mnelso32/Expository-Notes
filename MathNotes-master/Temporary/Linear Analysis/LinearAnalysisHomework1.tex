%% LyX 2.3.3 created this file.  For more info, see http://www.lyx.org/.
%% Do not edit unless you really know what you are doing.
\documentclass[12pt,english]{article}
\usepackage[osf]{mathpazo}
\renewcommand{\sfdefault}{lmss}
\renewcommand{\ttdefault}{lmtt}
\usepackage[T1]{fontenc}
\usepackage[latin9]{inputenc}
\usepackage[paperwidth=30cm,paperheight=35cm]{geometry}
\geometry{verbose,tmargin=2cm,bmargin=2cm}
\setlength{\parindent}{0bp}
\usepackage{mathrsfs}
\usepackage{amsmath}
\usepackage{amssymb}

\makeatletter
\@ifundefined{date}{}{\date{}}
%%%%%%%%%%%%%%%%%%%%%%%%%%%%%% User specified LaTeX commands.
\usepackage{tikz}
\usetikzlibrary{matrix,arrows,decorations.pathmorphing}
\usetikzlibrary{shapes.geometric}
\usepackage{tikz-cd}
\usepackage{amsthm}
\usepackage{xparse,etoolbox}

\theoremstyle{plain}
\newtheorem{theorem}{Theorem}[section]
\newtheorem{lemma}[theorem]{Lemma}
\newtheorem{prop}{Proposition}[section]
\newtheorem*{cor}{Corollary}
\theoremstyle{definition}
\newtheorem{defn}{Definition}[section]
\newtheorem{ex}{Exercise} 
\newtheorem{example}{Example}[section]
\theoremstyle{remark}
\newtheorem*{rem}{Remark}
\newtheorem*{note}{Note}
\newtheorem{case}{Case}
\usepackage{graphicx}
\usepackage{amssymb}
\usepackage{tikz-cd}
\usetikzlibrary{calc,arrows,decorations.pathreplacing}
\tikzset{mydot/.style={circle,fill,inner sep=1.5pt},
commutative diagrams/.cd,
  arrow style=tikz,
  diagrams={>=latex},
}

\usepackage{babel}
\usepackage{hyperref}
\hypersetup{
    colorlinks,
    citecolor=blue,
    filecolor=blue,
    linkcolor=blue,
    urlcolor=blue
}
\usepackage{pgfplots}
\usetikzlibrary{decorations.markings}
\pgfplotsset{compat=1.9}


\newcommand{\blocktheorem}[1]{%
  \csletcs{old#1}{#1}% Store \begin
  \csletcs{endold#1}{end#1}% Store \end
  \RenewDocumentEnvironment{#1}{o}
    {\par\addvspace{1.5ex}
     \noindent\begin{minipage}{\textwidth}
     \IfNoValueTF{##1}
       {\csuse{old#1}}
       {\csuse{old#1}[##1]}}
    {\csuse{endold#1}
     \end{minipage}
     \par\addvspace{1.5ex}}
}

\raggedbottom

\blocktheorem{theorem}% Make theo into a block
\blocktheorem{defn}% Make defi into a block
\blocktheorem{lemma}% Make lem into a block
\blocktheorem{rem}% Make rem into a block
\blocktheorem{cor}% Make col into a block
\blocktheorem{prop}% Make prop into a block


\usepackage[bottom]{footmisc}

\makeatother

\usepackage{babel}
\begin{document}
\title{Linear Analysis Homework 1}
\author{Michael Nelson}
\maketitle

\subsection*{Problem 1}

\begin{prop}\label{prop} (Polarization Identity) For $x,y\in\mathscr{V}$
we have
\[
4\langle x,y\rangle=\|x+y\|^{2}+i\|x+iy\|^{2}-\|x-y\|^{2}-i\|x-iy\|^{2}
\]
\end{prop}

\begin{proof} We calculate
\begin{align*}
\|x+y\|^{2} & =\langle x+y,x+y\rangle\\
 & =\langle x,x\rangle+\langle x,y\rangle+\langle y,x\rangle+\langle y,y\rangle,
\end{align*}
and
\begin{align*}
i\|x+iy\|^{2} & =i\langle x+iy,x+iy\rangle\\
 & =i\langle x,x\rangle+i\langle x,iy\rangle+i\langle iy,x\rangle+i\langle iy,iy\rangle\\
 & =i\langle x,x\rangle+\langle x,y\rangle-\langle y,x\rangle+i\langle y,y\rangle,
\end{align*}
and
\begin{align*}
-\|x-y\|^{2} & =-\langle x-y,x-y\rangle\\
 & =-\langle x,x\rangle-\langle x,-y\rangle-\langle-y,x\rangle-\langle-y,-y\rangle\\
 & =-\langle x,x\rangle+\langle x,y\rangle+\langle y,x\rangle-\langle y,y\rangle,
\end{align*}
and
\begin{align*}
-i\|x-iy\|^{2} & =-i\langle x-iy,x-iy\rangle\\
 & =-i\langle x,x\rangle-i\langle x,-iy\rangle-i\langle-iy,x\rangle-i\langle-iy,-iy\rangle\\
 & =-i\langle x,x\rangle+\langle x,y\rangle-\langle y,x\rangle-i\langle y,y\rangle.
\end{align*}
Adding these together gives us our desired result. \end{proof}

\subsection*{Problem 2}

\begin{prop}\label{propparallelogram} (Parallelogram Identity) For
$x,y\in\mathscr{V}$ we have
\[
\|x-y\|^{2}+\|x+y\|^{2}=2\|x\|^{2}+2\|y\|^{2}
\]
\end{prop}

\begin{proof} We calculate
\begin{align*}
\|x+y\|^{2} & =\langle x+y,x+y\rangle\\
 & =\langle x,x\rangle+\langle x,y\rangle+\langle y,x\rangle+\langle y,y\rangle\\
 & =\|x\|^{2}+2\text{Re}(\langle x,y\rangle)+\|y\|^{2}.
\end{align*}
and
\begin{align*}
\|x-y\|^{2} & =\langle x-y,x-y\rangle\\
 & =\langle x,x\rangle+\langle x,-y\rangle+\langle-y,x\rangle+\langle-y,-y\rangle\\
 & =\|x\|^{2}-2\text{Re}(\langle x,y\rangle)+\|y\|^{2}.
\end{align*}
Adding these together gives us our desired result. \end{proof}

The geometric interpretation of Proposition~(\ref{propparallelogram})
in the case where $\mathscr{V}=\mathbb{R}^{3}$ can be seen below:

\hfill

\hfill

\hfill

\hfill

\hfill

\hfill

\hfill

\hfill

\hfill

\hfill

\hfill

\hfill

\hfill

\hfill

\hfill

\hfill

\subsection*{Problem 3}

\begin{prop}\label{prop} (Pythagorean Theorem) Let $x$ and $y$
be nonzero vectors in $\mathscr{V}$ such that $\langle x,y\rangle=0$
(we call such vectors \textbf{orthogonal }to one another). Then
\[
\|x+y\|^{2}=\|x\|^{2}+\|y\|^{2}.
\]
\end{prop}

\begin{proof} We have
\begin{align*}
\|x+y\|^{2} & =\langle x+y,x+y\rangle\\
 & =\langle x,x\rangle+\langle x,y\rangle+\langle y,x\rangle+\langle y,y\rangle\\
 & =\langle x,x\rangle+\langle y,y\rangle\\
 & =\|x\|^{2}+\|y\|^{2}.
\end{align*}
\end{proof}

\subsection*{Problem 4}

\begin{prop}\label{prop} Let $(x_{n})$ and $(y_{n})$ be two sequences
in $\mathscr{V}$. Then the following statements hold:
\begin{enumerate}
\item If $x_{n}\to x$ and $y_{n}\to y$, then $x_{n}+y_{n}\to x+y$.
\item If $x_{n}\to x$ and $y_{n}\to y$, then $\langle x_{n},y_{n}\rangle\to\langle x,y\rangle$.
In particular, $\|x_{n}\|\to\|x\|$. 
\end{enumerate}
\end{prop}

\begin{proof} \hfill
\begin{enumerate}
\item Let $\varepsilon>0$. Choose $N\in\mathbb{N}$ such that $n\geq N$
implies $\|x_{n}-x\|<\varepsilon/2$ and $\|y_{n}-y\|<\varepsilon/2$.
Then $n\geq N$ implies
\begin{align*}
\|(x_{n}+y_{n})-(x+y)\| & \leq\|x_{n}-x\|+\|y_{n}-y\|\\
 & <\varepsilon/2+\varepsilon/2\\
 & =\varepsilon.
\end{align*}
\item Since $y_{n}\to y$, there exists $M\geq0$ such that $\|y_{n}\|\leq M$
for all $n\in\mathbb{N}$. Choose such an $M$ and let $\varepsilon>0$.
Choose $N\in\mathbb{N}$ such that $n\geq N$ implies $\|x_{n}-x\|<\varepsilon/2M$
and $\|y_{n}-y\|<\varepsilon/2\|x\|$. Then $n\geq N$ implies
\begin{align*}
|\langle x_{n},y_{n}\rangle-\langle x,y\rangle| & =|\langle x_{n},y_{n}\rangle-\langle x,y_{n}\rangle+\langle x,y_{n}\rangle-\langle x,y\rangle|\\
 & \leq|\langle x_{n},y_{n}\rangle-\langle x,y_{n}\rangle|+|\langle x,y_{n}\rangle-\langle x,y\rangle|\\
 & =|\langle x_{n}-x,y_{n}\rangle|+|\langle x,y_{n}-y\rangle|\\
 & \leq\|x_{n}-x\|\|y_{n}\|+\|x\|\|y_{n}-y\|\\
 & \leq\|x_{n}-x\|M+\|x\|\|y_{n}-y\|\\
 & <\varepsilon/2+\varepsilon/2\\
 & =\varepsilon.
\end{align*}
To see that $\|x_{n}\|\to\|x\|$, we just set $y_{n}=x_{n}$. Then
\begin{align*}
\|x_{n}\| & =\sqrt{\langle x_{n},x_{n}\rangle}\\
 & \to\sqrt{\langle x,x\rangle}\\
 & =\|x\|,
\end{align*}
where we were allowed to take limits inside the square root function
since the square root function is continuous on $\mathbb{R}_{\geq0}$. 
\end{enumerate}
\end{proof}

\subsection*{Problem 5}

\begin{prop}\label{prop} Let $\langle\cdot,\cdot\rangle\colon\text{M}_{m\times n}(\mathbb{R})\times\text{M}_{m\times n}(\mathbb{R})\to\mathbb{R}$
be given by
\[
\langle A,B\rangle=\text{Tr}(B^{\top}A),
\]
for all $A,B\in\text{M}_{n}(\mathbb{C})$. Then the pair $(\text{M}_{n}(\mathbb{C}),\langle\cdot,\cdot\rangle)$
forms an inner-product space. \end{prop}

\begin{proof}\label{proof} Linearity in the first argument follows
from distributivity of matrix multiplication and from linearity of
the trace function: Let $A,B,C\in\text{M}_{m\times n}(\mathbb{R})$.
Then
\begin{align*}
\langle A+B,C\rangle & =\text{Tr}(C^{\top}(A+B))\\
 & =\text{Tr}(C^{\top}A+C^{\top}B)\\
 & =\text{Tr}(C^{\top}A)+\text{Tr}(C^{\top}B)\\
 & =\langle A,C\rangle+\langle B,C\rangle.
\end{align*}

Symmetry of $\langle\cdot,\cdot\rangle$ follows from the fact that
$\text{Tr}(A)=\text{Tr}(A^{\top})$ for all $A\in\text{M}_{m\times n}(\mathbb{R})$:
Let $A,B\in\text{M}_{m\times n}(\mathbb{R})$. Then
\begin{align*}
\langle A,B\rangle & =\text{Tr}(B^{\top}A)\\
 & =\text{Tr}((B^{\top}A)^{\top})\\
 & =\text{Tr}(A^{\top}B)\\
 & =\langle B,A\rangle.
\end{align*}

Finally, to see positive-definiteness of $\langle\cdot,\cdot\rangle$,
let
\[
A=\begin{pmatrix}a_{11} & \cdots & a_{1n}\\
\vdots & \ddots & \vdots\\
a_{m1} & \cdots & a_{mn}
\end{pmatrix}\in\text{M}_{m\times n}(\mathbb{R}).
\]
Then
\begin{align*}
\langle A,A\rangle & =\text{Tr}(A^{\top}A)\\
 & =\text{Tr}\begin{pmatrix}a_{11} & \cdots & a_{m1}\\
\vdots & \ddots & \vdots\\
a_{1n} & \cdots & a_{mn}
\end{pmatrix}\begin{pmatrix}a_{11} & \cdots & a_{1n}\\
\vdots & \ddots & \vdots\\
a_{m1} & \cdots & a_{mn}
\end{pmatrix}\\
 & =\sum_{j=1}^{n}\sum_{i=1}^{m}a_{ij}^{2}.
\end{align*}
is a sum of its entries squared. This implies positive-definiteness.
\end{proof}

\subsection*{Problem 6a}

\begin{prop}\label{prop} Let $\langle\cdot,\cdot\rangle\colon\mathbb{C}^{n}\times\mathbb{C}^{n}\to\mathbb{C}$
be given by
\[
\langle x,y\rangle=\langle(x_{1},\dots,x_{n}),(y_{1},\dots,y_{n})\rangle=\sum_{i=1}^{n}x_{i}\overline{y}_{i}.
\]
for all $x,y\in\mathbb{C}^{n}$. Then the pair $(\mathbb{C}^{n},\langle\cdot,\cdot\rangle)$
forms an inner-product space. \end{prop}

\begin{proof}\label{proof} For linearity in the first argument follows
from linearity, let $x,y,z\in\mathbb{C}^{n}$. Then
\begin{align*}
\langle x+y,z\rangle & =\sum_{i=1}^{n}(x_{i}+y_{i})\overline{z}_{i}\\
 & =\sum_{i=1}^{n}x_{i}\overline{z}_{i}+\sum_{i=1}^{n}y_{i}\overline{z}_{i}\\
 & =\langle x,z\rangle+\langle y,z\rangle.
\end{align*}

For conjugate symmetry of $\langle\cdot,\cdot\rangle$, let $x,y\in\mathbb{C}^{n}$.
Then
\begin{align*}
\langle x,y\rangle & =\sum_{i=1}^{n}x_{i}\overline{y}_{i}\\
 & =\sum_{i=1}^{n}\overline{\overline{x_{i}\overline{y}_{i}}}\\
 & =\sum_{i=1}^{n}\overline{y_{i}\overline{x}_{i}}\\
 & =\overline{\langle y,x\rangle}.
\end{align*}

For positive-definiteness of $\langle\cdot,\cdot\rangle$, let $x\in\mathbb{C}^{n}$.
Then
\begin{align*}
\langle x,x\rangle & =\sum_{i=1}^{n}x_{i}\overline{x}_{i}\\
 & =\sum_{i=1}^{n}|x_{i}|^{2}.
\end{align*}
is a sum of its components absolute squared. This implies positive-definiteness.
\end{proof}

\subsection*{Problem 6b}

This follows from an easy application of Cauchy-Schwarz, but here's
another method (which turns out to be equivalent to Cauchy-Schwarz).
We need the following two lemmas:

\begin{lemma}\label{lemmainequality} Let $a$ and $b$ be nonnegative
real numbers. Then we have
\begin{equation}
2ab\leq a^{2}+b^{2}.\label{eq:inequality-1-1}
\end{equation}
\end{lemma}

\begin{proof} We have 
\begin{align*}
0 & \leq(a-b)^{2}\\
 & =a^{2}-2ab+b^{2}.
\end{align*}
Therefore the inequality (\ref{eq:inequality-1-1}) follows from adding
$2ab$. \end{proof}

\begin{lemma}\label{lemmainequality2} Let $a_{1},\dots,a_{n}$ and
$b_{1},\dots,b_{n}$ be nonnegative real numbers. Then
\[
\left(\sum_{i=1}^{n}a_{i}b_{i}\right)^{2}\leq\left(\sum_{i=1}^{n}a_{i}^{2}\right)\left(\sum_{i=1}^{n}b_{i}^{2}\right).
\]
\end{lemma}

\begin{proof}\label{proof} We have
\begin{align*}
\left(\sum_{i=1}^{n}a_{i}b_{i}\right)^{2} & =\sum_{i=1}^{n}a_{i}^{2}b_{i}^{2}+\sum_{1\leq i<j\leq n}2a_{i}b_{j}a_{j}b_{i}\\
 & \leq\sum_{i=1}^{n}a_{i}^{2}b_{i}^{2}+\sum_{1\leq i<j\leq n}(a_{i}^{2}b_{j}^{2}+a_{j}^{2}b_{i}^{2})\\
 & =\left(\sum_{i=1}^{n}a_{i}^{2}\right)\left(\sum_{i=1}^{n}b_{i}^{2}\right)
\end{align*}
where the inequality in the second line follows from Lemma~(\ref{lemmainequality})
applied to $a_{i}b_{j}$ and $a_{j}b_{i}$. \end{proof}

\begin{cor}\label{cor} Let $x,y\in\mathbb{C}^{n}$. Then
\[
\sum_{i=1}^{n}|x_{i}||y_{i}|\leq\sqrt{\sum_{i=1}^{n}|x_{i}|^{2}}\sqrt{\sum_{i=1}^{n}|y_{i}|^{2}}.
\]
\end{cor}

\begin{proof}\label{proof} This follows from by taking squares on
both sides and applying Lemma~(\ref{lemmainequality2}) since the
$|x_{i}|$ and $|y_{i}|$ are nonnegative real numbers. \end{proof}

\subsection*{Problem 7a}

\begin{prop}\label{prop} Let $\ell^{2}(\mathbb{N})$ be the set of
all sequence $(x_{n})$ in $\mathbb{C}$ such that 
\[
\sum_{n=1}^{\infty}|x_{n}|^{2}<\infty
\]
and let $\langle\cdot,\cdot\rangle\colon\ell^{2}(\mathbb{N})\times\ell^{2}(\mathbb{N})\to\mathbb{C}$
be given by
\[
\langle(x_{n}),(y_{n})\rangle=\sum_{n=1}^{\infty}x_{n}\overline{y}_{n}.
\]
for all $(x_{n}),(y_{n})\in\ell^{2}(\mathbb{N})$. Then the pair $(\ell^{2}(\mathbb{N}),\langle\cdot,\cdot\rangle)$
forms an inner-product space. \end{prop}

\begin{proof}\label{proof} We first need to show that $\ell^{2}(\mathbb{N})$
is indeed a vector space. In fact, we will show that $\ell^{2}(\mathbb{N})$
is a subspace of $\mathbb{C}^{N}$, the set of all sequences in $\mathbb{C}$.
Let $(x_{n}),(y_{n})\in\ell^{2}(\mathbb{N})$ and $\lambda\in\mathbb{C}$.
Then Lemma~(\ref{lemmainequality}) implies
\begin{align*}
\sum_{n=1}^{\infty}|\lambda x_{n}+y_{n}|^{2} & \leq\sum_{n=1}^{\infty}|\lambda x_{n}|^{2}+\sum_{n=1}^{\infty}|y_{n}|^{2}+\sum_{n=1}^{\infty}2|\lambda x_{n}||y_{n}|\\
 & \leq\lambda^{2}\sum_{n=1}^{\infty}|x_{n}|^{2}+\sum_{n=1}^{\infty}|y_{n}|^{2}+\lambda^{2}\sum_{n=1}^{\infty}|x_{n}|^{2}+|y_{n}|^{2}\\
 & <\infty.
\end{align*}
Therefore $(\lambda x_{n}+y_{n})\in\ell^{2}(\mathbb{N})$, which implies
$\ell^{2}(\mathbb{N})$ is a subspace of $\mathbb{C}^{\mathbb{N}}$. 

~~~Next, let us show that the inner product converges, and hence
is defined everywhere. Let $(x_{n}),(y_{n})\in\ell^{2}(\mathbb{N})$.
Then it follows from Lemma~(\ref{lemmainequality}) that
\begin{align*}
\sum_{n=1}^{\infty}|x_{n}\overline{y}_{n}| & =\sum_{n=1}^{\infty}|x_{n}||y_{n}|\\
 & \leq\sum_{n=1}^{\infty}\frac{|x_{n}|^{2}+|y_{n}|^{2}}{2}\\
 & =\frac{1}{2}\sum_{n=1}^{\infty}|x_{n}|^{2}+\frac{1}{2}\sum_{n=1}^{\infty}|y_{n}|^{2}\\
 & <\infty.
\end{align*}
Therefore $\sum_{n=1}^{\infty}x_{n}\overline{y}_{n}$ is absolutely
convergent, which implies it is convergent. (We can't use Cauchy-Schwarz
here since we haven't yet shown that $\langle\cdot,\cdot\rangle$
is in fact an inner-product).

~~~Finally, let us shows that $\langle\cdot,\cdot\rangle$ is an
inner-product. Linearity in the first argument follows from distrubitivity
of multiplication and linearity of taking infinite sums. For conjugate
symmetry, let $(x_{n}),(y_{n})\in\ell^{2}(\mathbb{N})$. Then
\begin{align*}
\langle(x_{n}),(y_{n})\rangle & =\sum_{n=1}^{\infty}x_{n}\overline{y}_{n}\\
 & =\lim_{N\to\infty}\sum_{n=1}^{N}x_{n}\overline{y}_{n}\\
 & =\overline{\overline{\lim_{N\to\infty}\sum_{n=1}^{N}x_{n}\overline{y}_{n}}}\\
 & =\overline{\lim_{N\to\infty}\sum_{n=1}^{N}\overline{x_{n}\overline{y}_{n}}}\\
 & =\overline{\lim_{N\to\infty}\sum_{n=1}^{N}y_{n}\overline{x}_{n}}\\
 & =\overline{\sum_{n=1}^{\infty}y_{n}\overline{x}_{n}}\\
 & =\overline{\langle(y_{n}),(x_{n})\rangle},
\end{align*}
where we were allowed to bring the conjugate inside the limit since
the conjugate function is continuous on $\mathbb{C}$. For positive-definiteness,
let $(x_{n})\in\ell^{2}(\mathbb{N})$. Then 
\begin{align*}
\langle(x_{n}),(x_{n})\rangle & =\sum_{n=1}^{\infty}x_{n}\overline{x}_{n}\\
 & =\sum_{n=1}^{\infty}|x_{n}|^{2}\\
 & \geq0.
\end{align*}
If $\sum_{n=1}^{\infty}|x_{n}|^{2}=0$, then clearly we must have
$x_{n}=0$ for all $n$. \end{proof}

\subsection*{Problem 7b}

\begin{prop}\label{prop} Let $(x_{n})\in\ell^{2}(\mathbb{N})$ such
that $\sum_{n=1}^{\infty}|x_{n}|^{2}=1$. Then 
\begin{equation}
\sum_{n=1}^{\infty}\frac{|x_{n}|}{2^{n}}\leq\frac{1}{\sqrt{3}}.\label{eq:cauchyschwarzinequality}
\end{equation}
where the inequality (\ref{eq:cauchyschwarzinequality}) becomes an
equality if and only if $|x_{n}|=\sqrt{3}\cdot2^{-n}$ for all $n$.
\end{prop}

\begin{proof} By Cauchy-Schwarz, we have 
\begin{align*}
\sum_{n=1}^{\infty}\frac{|x_{n}|}{2^{n}} & =|\langle(|x_{n}|),(2^{-n})\rangle|\\
 & \leq\|(|x_{n}|)\|\|(2^{-n})\|\\
 & =\sqrt{\sum_{n=1}^{\infty}|x_{n}|^{2}}\sqrt{\sum_{n=1}^{\infty}2^{-2n}}\\
 & 1\cdot\sqrt{\sum_{n=0}^{\infty}\left(\frac{1}{4}\right)^{n}-1}\\
 & =\sqrt{\frac{1}{1-1/4}-1}\\
 & =\sqrt{\frac{4}{3}-1}\\
 & =\frac{1}{\sqrt{3}}.
\end{align*}
where the inequality becomes an equality if and only if $(|x_{n}|)$
and $(2^{-n})$ are linearly dependent. This means that there is a
$\lambda\in\mathbb{C}$ such that $|x_{n}|=\lambda2^{-n}$ for all
$n$. To find this $\lambda$, write
\begin{align*}
1 & =\sum_{n=1}^{\infty}|x_{n}|^{2}\\
 & =\sum_{n=1}^{\infty}|\lambda2^{-n}|^{2}\\
 & =|\lambda|^{2}\sum_{n=1}^{\infty}\left(\frac{1}{4}\right)^{n}\\
 & =\frac{|\lambda|^{2}}{3}.
\end{align*}
Thus, any $\lambda\in\mathbb{C}$ such that $|\lambda|=\sqrt{3}$
works. (Actually, we must have $\lambda=\sqrt{3}$ since $\lambda=|x_{n}|2^{n}$
is positive). \end{proof}

\subsection*{Problem 8}

\begin{prop}\label{prop} Let $f\in C[0,1]$ such that $\int_{0}^{1}|f(x)|^{2}dx=1$.
Then 
\[
\int_{0}^{1}|f(x)|\sin(\pi x)dx\leq\frac{1}{\sqrt{2}},
\]
where the inequality becomes an equality if and only if $|f(x)|=\sqrt{2}\sin(\pi x)$.
\end{prop}

\begin{proof} First note that
\begin{align*}
\int_{0}^{1}\sin^{2}(\pi x)dx & =\int_{0}^{1}\cos^{2}(\pi x)dx\\
 & =\int_{0}^{1}(1-\sin^{2}(\pi x))dx
\end{align*}
implies $\int_{0}^{1}\sin^{2}(\pi x)dx=1/2$, where in the first equality
above we used integration by parts with $u=\sin(\pi x)$ and $dv=\sin(\pi x)dx$.
Therefore, by Cauchy-Schwarz, we have
\begin{align*}
\int_{0}^{1}|f(x)|\sin(\pi x)dx & \leq\sqrt{\int_{0}^{1}|f(x)|^{2}dx}\cdot\sqrt{\int_{0}^{1}\sin^{2}(\pi x)dx}\\
 & =1\cdot\frac{1}{\sqrt{2}}\\
 & =\frac{1}{\sqrt{2}},
\end{align*}
where the inequality becomes an equality if and only if $|f(x)|$
and $\sin(\pi x)$ and linearly dependent. This means that there is
a $\lambda\in\mathbb{C}$ such that $|f(x)|=\lambda\sin(\pi x)$ for
all $x$. To find this $\lambda$, write
\begin{align*}
1 & =\int_{0}^{1}|f(x)|^{2}dx\\
 & =\int_{0}^{1}|\lambda\sin(\pi x)|^{2}dx\\
 & =|\lambda|^{2}\int_{0}^{1}\sin^{2}(\pi x)dx\\
 & =\frac{|\lambda|^{2}}{2}.
\end{align*}
Thus, any $\lambda\in\mathbb{C}$ such that $|\lambda|=\sqrt{2}$
works. (Actually, we must have $\lambda=\sqrt{2}$ since $\lambda=|f(x)|/\sin(\pi x)$
is positive). \end{proof}

\begin{rem}\label{rem} If we tried to apply Lemma~(\ref{lemmainequality})
at each $x\in[0,1]$, we'd only get the weaker result: 
\begin{align*}
\int_{0}^{1}|f(x)|\sin(\pi x)dx & \leq\frac{1}{2}\left(\int_{0}^{1}|f(x)|^{2}dx+\int_{0}^{1}\sin^{2}(\pi x)dx\right)\\
 & =\frac{1}{2}+\frac{1}{4}\\
 & =\frac{3}{4}.
\end{align*}
\end{rem}
\end{document}
