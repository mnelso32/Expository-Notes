%% LyX 2.3.3 created this file.  For more info, see http://www.lyx.org/.
%% Do not edit unless you really know what you are doing.
\documentclass[12pt,english]{article}
\usepackage[osf]{mathpazo}
\renewcommand{\sfdefault}{lmss}
\renewcommand{\ttdefault}{lmtt}
\usepackage[T1]{fontenc}
\usepackage[latin9]{inputenc}
\usepackage[paperwidth=30cm,paperheight=35cm]{geometry}
\geometry{verbose,tmargin=2cm,bmargin=2cm}
\setlength{\parindent}{0bp}
\usepackage{amsmath}
\usepackage{amssymb}

\makeatletter
\@ifundefined{date}{}{\date{}}
%%%%%%%%%%%%%%%%%%%%%%%%%%%%%% User specified LaTeX commands.
\usepackage{tikz}
\usetikzlibrary{matrix,arrows,decorations.pathmorphing}
\usetikzlibrary{shapes.geometric}
\usepackage{tikz-cd}
\usepackage{amsthm}
\usepackage{xparse,etoolbox}

\theoremstyle{plain}
\newtheorem{theorem}{Theorem}[section]
\newtheorem{lemma}[theorem]{Lemma}
\newtheorem{prop}{Proposition}[section]
\newtheorem*{cor}{Corollary}
\theoremstyle{definition}
\newtheorem{defn}{Definition}[section]
\newtheorem{ex}{Exercise} 
\newtheorem{example}{Example}[section]
\theoremstyle{remark}
\newtheorem*{rem}{Remark}
\newtheorem*{note}{Note}
\newtheorem{case}{Case}
\usepackage{graphicx}
\usepackage{amssymb}
\usepackage{tikz-cd}
\usetikzlibrary{calc,arrows,decorations.pathreplacing}
\tikzset{mydot/.style={circle,fill,inner sep=1.5pt},
commutative diagrams/.cd,
  arrow style=tikz,
  diagrams={>=latex},
}

\usepackage{babel}
\usepackage{hyperref}
\hypersetup{
    colorlinks,
    citecolor=blue,
    filecolor=blue,
    linkcolor=blue,
    urlcolor=blue
}
\usepackage{pgfplots}
\usetikzlibrary{decorations.markings}
\pgfplotsset{compat=1.9}


\newcommand{\blocktheorem}[1]{%
  \csletcs{old#1}{#1}% Store \begin
  \csletcs{endold#1}{end#1}% Store \end
  \RenewDocumentEnvironment{#1}{o}
    {\par\addvspace{1.5ex}
     \noindent\begin{minipage}{\textwidth}
     \IfNoValueTF{##1}
       {\csuse{old#1}}
       {\csuse{old#1}[##1]}}
    {\csuse{endold#1}
     \end{minipage}
     \par\addvspace{1.5ex}}
}

\raggedbottom

\blocktheorem{theorem}% Make theo into a block
\blocktheorem{defn}% Make defi into a block
\blocktheorem{lemma}% Make lem into a block
\blocktheorem{rem}% Make rem into a block
\blocktheorem{cor}% Make col into a block
\blocktheorem{prop}% Make prop into a block


\usepackage[bottom]{footmisc}

\makeatother

\usepackage{babel}
\begin{document}

\subsection*{Problem 6}

\begin{prop}\label{prop} Let $\mathcal{H}$ be a separable Hilbert
space and let $T\colon\mathcal{H}\to\mathcal{H}$ be a compact self-adjoint
operator. Then there exists a sequence $T_{m}$ of operators with
finite dimensional range such that $\|T-T_{m}\|\to0$ and $m\to\infty$.
\end{prop}

\begin{proof} Choose an orthonormal basis $(e_{n})$ consisting of
eigenvectors of $T$ and let $(\lambda_{n})$ be the corresponding
sequence of eigenvalues. By reindexing if necessary, we may assume
that $|\lambda_{n}|\geq|\lambda_{n+1}|$ for all $n\in\mathbb{N}$.
For each $m\in\mathbb{N}$, we define $T_{m}\colon\mathcal{H}\to\mathcal{H}$
by
\[
T_{m}x=\sum_{n=1}^{m}\lambda_{n}\langle x,e_{n}\rangle e_{n}
\]
for all $x\in\mathcal{H}$. Observe that $\text{im}(T_{m})=\text{span}(\{e_{1},\dots,e_{m}\})$
is finite dimensional. We claim that $\|T-T_{m}\|\to0$ and $m\to\infty$.
Indeed, let $\varepsilon>0$ and let $\Lambda$ denote the set of
all eigenvalues of $T$. If $\Lambda$ is finite, then the claim is
clear by the spectral theorem for compact self-adjoint operators,
so assume $\Lambda$ is infinite. Then $0$ must be an accumulation
point of $\Lambda$. In particular, $|\lambda_{n}|\to0$ as $n\to\infty$.
Choose $N\in\mathbb{N}$ such that $n\geq N$ implies $|\lambda_{n}|<\varepsilon$.
Then for all $x\in B_{1}[0]$, we have
\begin{align*}
\|Tx-T_{m}x\|^{2} & =\left\Vert \sum_{n=1}^{\infty}\lambda_{n}\langle x,e_{n}\rangle e_{n}-\sum_{n=1}^{m}\lambda_{n}\langle x,e_{n}\rangle e_{n}\right\Vert ^{2}\\
 & =\left\Vert \sum_{n=m+1}^{\infty}\lambda_{n}\langle x,e_{n}\rangle e_{n}\right\Vert ^{2}\\
 & =\sum_{n=m+1}^{\infty}|\lambda_{n}\langle x,e_{n}\rangle|^{2}\\
 & \leq|\lambda_{N}|^{2}\sum_{n=m+1}^{\infty}|\langle x,e_{n}\rangle|^{2}\\
 & \leq|\lambda_{N}|^{2}\|x\|^{2}\\
 & <\varepsilon^{2}.
\end{align*}
This implies $\|T-T_{m}\|\to0$ and $m\to\infty$. \end{proof}
\end{document}
