%% LyX 2.3.3 created this file.  For more info, see http://www.lyx.org/.
%% Do not edit unless you really know what you are doing.
\documentclass[12pt,english]{article}
\usepackage[osf]{mathpazo}
\renewcommand{\sfdefault}{lmss}
\renewcommand{\ttdefault}{lmtt}
\usepackage[T1]{fontenc}
\usepackage[latin9]{inputenc}
\usepackage[paperwidth=30cm,paperheight=35cm]{geometry}
\geometry{verbose,tmargin=2cm,bmargin=2cm}
\setlength{\parindent}{0bp}
\usepackage{amsmath}
\usepackage{amssymb}

\makeatletter
\@ifundefined{date}{}{\date{}}
%%%%%%%%%%%%%%%%%%%%%%%%%%%%%% User specified LaTeX commands.
\usepackage{tikz}
\usetikzlibrary{matrix,arrows,decorations.pathmorphing}
\usetikzlibrary{shapes.geometric}
\usepackage{tikz-cd}
\usepackage{amsthm}
\usepackage{xparse,etoolbox}

\theoremstyle{plain}
\newtheorem{theorem}{Theorem}[section]
\newtheorem{lemma}[theorem]{Lemma}
\newtheorem{prop}{Proposition}[section]
\newtheorem*{cor}{Corollary}
\theoremstyle{definition}
\newtheorem{defn}{Definition}[section]
\newtheorem{ex}{Exercise} 
\newtheorem{example}{Example}[section]
\theoremstyle{remark}
\newtheorem*{rem}{Remark}
\newtheorem*{note}{Note}
\newtheorem{case}{Case}
\usepackage{graphicx}
\usepackage{amssymb}
\usepackage{tikz-cd}
\usetikzlibrary{calc,arrows,decorations.pathreplacing}
\tikzset{mydot/.style={circle,fill,inner sep=1.5pt},
commutative diagrams/.cd,
  arrow style=tikz,
  diagrams={>=latex},
}

\usepackage{babel}
\usepackage{hyperref}
\hypersetup{
    colorlinks,
    citecolor=blue,
    filecolor=blue,
    linkcolor=blue,
    urlcolor=blue
}
\usepackage{pgfplots}
\usetikzlibrary{decorations.markings}
\pgfplotsset{compat=1.9}


\newcommand{\blocktheorem}[1]{%
  \csletcs{old#1}{#1}% Store \begin
  \csletcs{endold#1}{end#1}% Store \end
  \RenewDocumentEnvironment{#1}{o}
    {\par\addvspace{1.5ex}
     \noindent\begin{minipage}{\textwidth}
     \IfNoValueTF{##1}
       {\csuse{old#1}}
       {\csuse{old#1}[##1]}}
    {\csuse{endold#1}
     \end{minipage}
     \par\addvspace{1.5ex}}
}

\raggedbottom

\blocktheorem{theorem}% Make theo into a block
\blocktheorem{defn}% Make defi into a block
\blocktheorem{lemma}% Make lem into a block
\blocktheorem{rem}% Make rem into a block
\blocktheorem{cor}% Make col into a block
\blocktheorem{prop}% Make prop into a block


\usepackage[bottom]{footmisc}

\makeatother

\usepackage{babel}
\begin{document}
\title{Linear Analysis Homework 4}
\author{Michael Nelson}
\maketitle

\subsection*{Problem 1}

\begin{prop}\label{prop} Let $\mathcal{H}$ be a hilbert space and
let $T\colon\mathcal{H\to\mathcal{H}}$ be a bounded operator. Then
\begin{enumerate}
\item $\|T\|=\sup\{\|Tx\|\mid\|x\|=1\}$;
\item $\|T\|=\sup\left\{ \frac{\|Tx\|}{\|x\|}\mid x\in\mathcal{H}\backslash\{0\}\right\} .$
\end{enumerate}
\end{prop}

\begin{proof}\label{proof} \hfill

\hfill

1. First note that
\begin{align*}
\sup\{\|Tx\|\mid\|x\|=1\} & \leq\sup\{\|Tx\|\mid\|x\|\leq1\}\\
 & =\|T\|.
\end{align*}
We prove the reverse inequality by contradiction. Assume that $\|T\|>\sup\{\|Tx\|\mid\|x\|=1\}$.
Choose $\varepsilon>0$ such that
\begin{equation}
\|T\|-\varepsilon>\sup\{\|Tx\|\mid\|x\|=1\}\label{eq:contra}
\end{equation}
Next, choose $x\in\mathcal{H}$ such that $\|x\|\leq1$ and $\|Tx\|\geq\|T\|-\varepsilon$.
Then since $\|x\|\leq1$ and $\left\Vert \frac{x}{\|x\|}\right\Vert =1$,
we have
\begin{align*}
\|T\| & \geq\left\Vert T\left(\frac{x}{\|x\|}\right)\right\Vert \\
 & =\frac{\|Tx\|}{\|x\|}\\
 & \geq\|Tx\|\\
 & >\|T\|-\varepsilon,
\end{align*}
and this contradicts (\ref{eq:contra}).

\hfill

2. We have
\begin{align*}
\sup\left\{ \frac{\|Tx\|}{\|x\|}\mid x\in\mathcal{H}\backslash\{0\}\right\}  & =\sup\left\{ \left\Vert T\left(\frac{x}{\|x\|}\right)\right\Vert \mid x\in\mathcal{H}\backslash\{0\}\right\} \\
 & =\sup\left\{ \left\Vert Ty\right\Vert \mid\|y\|=1\right\} \\
 & =\|T\|,
\end{align*}
where the last equality follows from 1. \end{proof}

\subsection*{Problem 2}

\begin{prop}\label{prop} Let $k\in C[a,b]$. Then the operator $T\colon C[a,b]\to C[a,b]$
defined by
\[
Tf=kf
\]
for all $f\in C[a,b]$ is bounded. It's norm will be explicitly computed
in the proof below. \end{prop}

\begin{proof}\label{proof} We first show it is linear. Let $f,g\in C[a,b]$
and let $\lambda,\mu\in\mathbb{C}$. Then we have
\begin{align*}
T(\lambda f+\mu g) & =k(\lambda f+\mu g)\\
 & =\lambda kf+\mu kg\\
 & =\lambda T(f)+\mu T(g).
\end{align*}
Thus, $T$ is linear.

~~~Next we show it is bounded. If $k=0$, then $\|T\|=0$, so assume
$k\neq0$. Since $k$ is continuous on the compact interval $[a,b]$,
there exists $c\in[a,b]$ such that $|k(x)|\leq|k(c)|$ for all $x\in[a,b]$.
Choose such a $c\in[a,b]$ and let $f\in C[a,b]$ such that $\|f\|\leq1$.
Then
\begin{align*}
\|Tf\| & =\|kf\|\\
 & =\sqrt{\int_{a}^{b}|k(x)|^{2}|f(x)|^{2}dx}\\
 & \leq|k(c)|\sqrt{\int_{a}^{b}|f(x)|^{2}dx}\\
 & \leq|k(c)|.
\end{align*}
implies $\|T\|\leq|k(c)|$, and hence $T$ is bounded. 

~~~To find the norm of $T$, let $\varepsilon>0$ such that $\varepsilon<|k(c)|$.
Without loss of generality, assume that $c<b$ (if $c=b$, then we
swap the role of $b$ with $a$ in the argument which follows). Choose
$c'\in(c,b)$ such that $|k(x)|\geq|k(c)|-\varepsilon$ for all $x\in(c,c')$
(such a $c'$ must exist since $k$ is continuous) and choose $f$
to be a nonzero continuous function in $C[a,b]$ which vanishes outside
the interval $(c,c')$. Then
\[
|k(x)||f(x)|\geq(|k(c)|-\varepsilon)|f(x)|
\]
for all $x\in(a,b)$. In particular, this implies
\begin{align*}
\|Tf\| & =\|kf\|\\
 & =\sqrt{\int_{a}^{b}|k(x)f(x)|^{2}dx}\\
 & \geq\sqrt{\int_{a}^{b}(|k(c)|-\varepsilon)|f(x)|^{2}dx}\\
 & =(|k(c)|-\varepsilon)\sqrt{\int_{a}^{b}|f(x)|^{2}dx}\\
 & =(|k(c)|-\varepsilon)\|f\|.
\end{align*}
Therefore $\|T(f/\|f\|)\|\geq|k(c)|-\varepsilon$, and this implies
\begin{equation}
\|T\|\geq|k(c)|-\varepsilon\label{ini}
\end{equation}
Since (\ref{ini}) holds for all $\varepsilon>0$, we must have $\|T\|\geq|k(c)|$.
Thus $\|T\|=|k(c)|$. \end{proof}

\subsection*{Problem 3}

\begin{prop}\label{prop} Let $\{x_{n}\mid n\in\mathbb{N}\}$ be a
linearly independent set of vectors in a Hilbert space $\mathcal{H}$.
Consider the so called Gram-Schmidt process: set $e_{1}=\frac{1}{\|x_{1}\|}x_{1}$.
Proceed inductively. If $e_{1},e_{2},\dots,e_{n-1}$ are computed,
compute $e_{n}$ in two steps by
\[
f_{n}:=x_{n}-\sum_{k=1}^{n-1}\langle x_{n},e_{k}\rangle e_{k},\text{ and then set }e_{n}:=\frac{1}{\|f_{n}\|}f_{n}.
\]

Then
\begin{enumerate}
\item for every $N\in\mathbb{N}$ we have $\text{span}\{x_{1},x_{2},\dots,x_{N}\}=\text{span}\{e_{1},e_{2},\dots,e_{N}\}$;
\item the set $\{e_{n}\mid n\in\mathbb{N}\}$ is an orthonormal set in $\mathcal{H}$;
\item if $\overline{\text{span}}\{x_{n}\mid n\in\mathbb{N}\}=\mathcal{H}$,
then $\{e_{n}\mid n\in\mathbb{N}\}$ is an orthonormal basis for $\mathcal{H}$.
\end{enumerate}
\end{prop}

\begin{proof}\label{proof} \hfill

1. Let $N\in\mathbb{N}$. Then for each $1\leq n\leq N$, we have
\[
x_{n}=\sum_{k=1}^{n-1}\langle x_{n},e_{k}\rangle e_{k}+\|f_{n}\|e_{n}.
\]
This implies $\text{span}\{x_{1},x_{2},\dots,x_{N}\}\subseteq\text{span}\{e_{1},e_{2},\dots,e_{N}\}$.
We show the reverse inclusion by induction on $n$ such that $1\leq n\leq N$.
The base case $n=1$ being $\text{span}\{x_{1}\}\supseteq\text{span}\{e_{1}\}$,
which holds since $e_{1}=\frac{1}{\|x_{1}\|}x_{1}$. Now suppose for
some $n$ such that $1\leq n<N$ we have
\begin{equation}
\text{span}\{x_{1},x_{2},\dots,x_{k}\}\supseteq\text{span}\{e_{1},e_{2},\dots,e_{k}\}\label{eq:inductionstep}
\end{equation}
for all $1\leq k\leq n$. Then
\[
e_{n+1}=\frac{1}{\|f_{n}\|}x_{n}-\sum_{k=1}^{n}\frac{1}{\|f_{n}\|}\langle x_{n},e_{k}\rangle e_{k}\in\text{span}\{x_{1},x_{2},\dots,x_{n}\}.
\]
where we used the induction step (\ref{eq:inductionstep}) on the
$e_{k}$'s ($1\leq k\leq n$). Therefore
\[
\text{span}\{x_{1},x_{2},\dots,x_{k}\}\supseteq\text{span}\{e_{1},e_{2},\dots,e_{k}\}
\]
for all $1\leq k\leq n+1$, and this proves our claim. 

\hfill

2. By construction, we have $\langle e_{n},e_{n}\rangle=1$ for all
$n\in\mathbb{N}$. Thus, it remains to show that $\langle e_{m},e_{n}\rangle=0$
whenever $m\neq n$. We prove by induction on $n\geq2$ that $\langle e_{n},e_{m}\rangle=0$
for all $m<n$. Proving this also give us $\langle e_{m},e_{n}\rangle=0$
for all $m<n$, since
\begin{align*}
\langle e_{m},e_{n}\rangle & =\overline{\langle e_{n},e_{m}\rangle}\\
 & =\overline{0}\\
 & =0.
\end{align*}
The base case is
\begin{align*}
\langle e_{2},e_{1}\rangle & =\frac{1}{\|x_{1}\|\|f_{2}\|}\left\langle \left(x_{2}-\frac{\langle x_{2},x_{1}\rangle}{\langle x_{1},x_{1}\rangle}x_{1}\right),x_{1}\right\rangle \\
 & =\frac{1}{\|x_{1}\|\|f_{2}\|}\left(\langle x_{2},x_{1}\rangle-\langle x_{2},x_{1}\rangle\right)\\
 & =0
\end{align*}
Now suppose that $n>2$ and that $\langle e_{n},e_{m}\rangle=0$ for
all $m<n$. Then
\begin{align*}
\langle e_{n+1},e_{m}\rangle & =\frac{1}{\|f_{n+1}\|}\langle x_{n+1}-\sum_{k=1}^{n}\langle x_{n+1},e_{k}\rangle e_{k},e_{m}\rangle\\
 & =\frac{1}{\|f_{n+1}\|}\left(\langle x_{n+1},e_{m}\rangle-\sum_{k=1}^{n}\langle x_{n+1},e_{k}\rangle\langle e_{k},e_{m}\rangle\right)\\
 & =\frac{1}{\|f_{n+1}\|}\left(\langle x_{n+1},e_{m}\rangle-\langle x_{n+1},e_{m}\rangle\langle e_{m},e_{m}\rangle\right)\\
 & =\frac{1}{\|f_{n+1}\|}\left(\langle x_{n+1},e_{m}\rangle-\langle x_{n+1},e_{m}\rangle\right)\\
 & =0,
\end{align*}
for all $m<n+1$, where we used the induction hypothesis to get from
the second line to the third line. This proves the induction step,
which finishes the proof of part 2 of the proposition.

\hfill

3. By 2, we know that $\{e_{n}\mid n\in\mathbb{N}\}$ is an orthonormal
set. Thus, it suffices to show that $\{e_{n}\mid n\in\mathbb{N}\}$
is complete. To do this, we use the criterion that the set $\{e_{n}\mid n\in\mathbb{N}\}$
is complete if and only if the only $x\in\mathcal{H}$ such that $\langle x,e_{n}\rangle=0$
for all $n\in\mathbb{N}$ is $x=0$. 

~~~Let $x\in\mathcal{H}$ and suppose $\langle x,e_{n}\rangle=0$
for all $n\in\mathbb{N}$. Then
\begin{align*}
\langle x,x_{n}\rangle & =\left\langle x,\sum_{k=1}^{n-1}\langle x_{n},e_{k}\rangle e_{k}+\|f_{n}\|e_{n}\right\rangle \\
 & =\sum_{k=1}^{n-1}\langle x_{n},e_{k}\rangle\langle x,e_{k}\rangle+\|f_{n}\|\langle x,e_{n}\rangle\\
 & =0
\end{align*}
for all $n\in\mathbb{N}$. Since $\{x_{n}\mid n\in\mathbb{N}\}$ is
complete, this implies $x=0$. Therefore $\{e_{n}\mid n\in\mathbb{N}\}$
is complete. \end{proof}

\subsection*{Problem 4}

\begin{example}\label{example} The first three Legendre polynomials
are
\[
P_{1}(x)=1,\quad P_{2}(x)=x,\quad,P_{3}(x)=\frac{1}{2}(3x^{2}-1).
\]
We apply Gram-Schmidt process to the polynomials $1,x,x^{2}$ in the
space $C[-1,1]$ to get scalar multiples of the Legendre polynomials
above. First we set $f_{1}(x)=1$ and then calculate
\begin{align*}
\|f_{1}(x)\| & =\sqrt{\int_{-1}^{1}dx}\\
 & =\sqrt{2}.
\end{align*}
Thus we set $e_{1}(x)=1/\sqrt{2}$. Next we calculate
\begin{align*}
f_{1}(x) & =x-\left\langle x,\frac{1}{\sqrt{2}}\right\rangle \frac{1}{\sqrt{2}}\\
 & =x-\frac{1}{2}\int_{-1}^{1}xdx\\
 & =x.
\end{align*}
Next we calculate
\begin{align*}
\|f_{1}(x)\| & =\sqrt{\int_{-1}^{1}x^{2}dx}\\
 & =\sqrt{\frac{2}{3}}.
\end{align*}
Thus we set $e_{2}(x)=\sqrt{3/2}x$. Next we calculate
\begin{align*}
f_{2}(x) & =x^{2}-\left\langle x^{2},\sqrt{\frac{3}{2}}x\right\rangle \sqrt{\frac{3}{2}}x-\left\langle x^{2},\sqrt{\frac{1}{2}}\right\rangle \sqrt{\frac{1}{2}}\\
 & =x^{2}-\frac{3}{2}x\int_{-1}^{1}x^{3}dx-\frac{1}{2}\int_{-1}^{1}x^{2}dx\\
 & =x^{2}-\frac{1}{3}.
\end{align*}
Then we finally calculate
\begin{align*}
\|f_{2}(x)\| & =\sqrt{\int_{-1}^{1}\left(x^{2}-\frac{1}{3}\right)^{2}dx}\\
 & =\sqrt{\int_{-1}^{1}\left(x^{4}-\frac{2}{3}x^{2}+\frac{1}{9}\right)dx}\\
 & =\sqrt{\int_{-1}^{1}x^{4}dx-\frac{2}{3}\int_{-1}^{1}x^{2}dx+\frac{1}{9}\int_{-1}^{1}dx}\\
 & =\sqrt{\frac{2}{5}-\frac{4}{9}+\frac{2}{9}}\\
 & =\sqrt{\frac{8}{45}}.
\end{align*}
Thus we set $e_{3}(x)=\sqrt{45/8}(x^{2}-1/3)$. Now observe that
\begin{align*}
P_{1}(x) & =\sqrt{2}e_{1}(x)\\
P_{2}(x) & =\sqrt{\frac{2}{3}}e_{2}(x)\\
P_{3}(x) & =\sqrt{\frac{2}{5}}e_{3}(x)
\end{align*}

\end{example}

\subsection*{Problem 5}

For this problem, we needed to establish some basic results which
we proved in the Appendix. 

\begin{prop}\label{prop} The expression
\begin{equation}
\int_{-1}^{1}|x^{3}-a-bx-cx^{2}|^{2}dx.\label{eq:min}
\end{equation}
is minimized in $a,b,c\in\mathbb{C}$ if and only if $a=0$, $b=3/5$,
and $c=0$. \end{prop}

\begin{proof} Let 
\[
\mathcal{H}=\{p(x)\in\mathbb{C}[x]\mid\deg(p(x))\leq3\}\quad\text{and}\quad\mathcal{K}=\{p(x)\in\mathbb{C}[x]\mid\deg(p(x))\leq2\}.
\]
Then $\mathcal{H}$ and $\mathcal{K}$ are subspaces of $C[-1,1]$,
Proposition~(\ref{propsubspace}) implies they are inner-product spaces
with the inner-product inherited from $C[-1,1]$. Since $\mathcal{H}$
is finite dimensional, Proposition~(\ref{propfinistrivial}) implies
$\mathcal{H}$ is a separable Hilbert space. Since $\mathcal{K}$
is a finite dimensional subspace of $\mathcal{H}$, Proposition~(\ref{propfinsubisclosed})
implies $\mathcal{K}$ is closed in $\mathcal{H}$. Let $\{e_{1},e_{2},e_{3}\}$
be the orthonormal basis computed in problem 4. A proposition proved
in class implies
\begin{align*}
\text{P}_{\mathcal{K}}(x^{3}) & =\langle x^{3},e_{1}\rangle e_{1}+\langle x^{3},e_{2}\rangle e_{2}+\langle x^{3},e_{3}\rangle e_{3}\\
 & =\frac{1}{2}\int_{-1}^{1}x^{3}dx+\frac{3}{2}x\int_{-1}^{1}x^{4}dx+\frac{45}{8}\left(x^{2}-\frac{1}{3}\right)\int_{-1}^{1}x^{3}\left(x^{2}-\frac{1}{3}\right)dx\\
 & =\frac{3}{5}x.
\end{align*}
where we used the fact that $x^{3}(x^{2}-1/3)$ is an odd function
to get $\int_{-1}^{1}x^{3}(x^{2}-1/3)dx=0$. Therefore
\begin{align*}
\int_{-1}^{1}\left|x^{3}-\frac{3}{5}x\right|^{2}dx & =\|x^{3}-\text{P}_{\mathcal{K}}(x^{3})\|^{2}\\
 & =\inf\left\{ \|x^{3}-(a+bx+cx^{2})\|^{2}\mid a+bx+cx^{2}\in\mathcal{K}\right\} \\
 & =\inf\left\{ \int_{-1}^{1}|x^{3}-a-bx-cx^{2}|^{2}dx\mid a,b,c\in\mathbb{C}\right\} .
\end{align*}
By uniqueness of $\text{P}_{\mathcal{K}}x^{3}$, (\ref{eq:min}) is
minimized in $a,b,c\in\mathbb{C}$ if and only if $a=0$, $b=3/5$,
and $c=0$. \end{proof}

\subsection*{Problem 6}

\begin{prop}\label{prop} $\ell^{2}(\mathbb{N})$ is a Hilbert space.
\end{prop}

\begin{proof} Let $(a^{n})_{n\in\mathbb{N}}$ be a Cauchy sequence
in $\ell^{2}(\mathbb{N})$. 

\hfill 

\textbf{Step 1: }We show that for each $k\in\mathbb{N}$, the sequence
of $k$th coordinates $(a_{k}^{n})_{n\in\mathbb{N}}$ is a Cauchy
sequence of complex numbers, and hence must converge (as $\mathbb{C}$
is complete). Let $k\in\mathbb{N}$ and let $\varepsilon>0$. Choose
$N\in\mathbb{N}$ such that $m,n\geq N$ implies $\|a^{n}-a^{m}\|<\varepsilon^{2}$.
Then $n,m\geq N$ implies
\begin{align*}
|a_{k}^{n}-a_{k}^{m}|^{2} & \leq\sum_{i=1}^{\infty}|a_{i}^{n}-a_{i}^{m}|^{2}\\
 & =\|a^{n}-a^{m}\|\\
 & <\varepsilon^{2},
\end{align*}
which implies $|a_{k}^{n}-a_{k}^{m}|<\varepsilon$. Therefore $(a_{k}^{n})_{n\in\mathbb{N}}$
is a Cauchy sequence of complex numbers. In particular, the sequence
$(a_{k}^{n})_{n\in\mathbb{N}}$ converges to some element, say $a_{k}^{n}\to a_{k}$. 

\hfill

\textbf{Step 2: }We show that the sequence $(a_{k})_{k\in\mathbb{N}}$
defined in step 1 is square summable. Since $(a^{n})$ is a Cauchy
sequence of elements in $\ell^{2}(\mathbb{N})$, there exists an $M>0$
such that $\|a^{n}\|<M$ for all $n\in\mathbb{N}$ (see Lemma~((\ref{lemmacauchybound})
for a proof of this). Choose such an $M>0$ and let $\varepsilon>0$.
Choose $N\in\mathbb{N}$ such that
\[
|a_{k}|^{2}<|a_{k}^{N}|^{2}+\varepsilon/K
\]
for all $1\leq k\leq K$. Then
\begin{align*}
\sum_{k=1}^{K}|a_{k}|^{2} & <\sum_{k=1}^{K}|a_{k}^{N}|^{2}+\varepsilon\\
 & \leq\|a^{N}\|+\varepsilon\\
 & \leq M+\varepsilon.
\end{align*}
Taking the limit $K\to\infty$, we see that
\begin{align*}
\|a\| & =\sum_{k=1}^{\infty}|a_{k}|^{2}\\
 & \leq M+\varepsilon\\
 & \leq0.
\end{align*}
In particular, $a$ is square summable. 

\hfill

\textbf{Step 3: }Let $a$ be the sequence $(a_{k})_{k\in\mathbb{N}}$
defined in step 1. We show that $a^{n}\to a$ in the $\ell^{2}$ norm.
Let $\varepsilon>0$ and let $K\in\mathbb{N}$. Choose $N\in\mathbb{N}$
such that $m,n\geq N$ implies $\|a^{n}-a^{m}\|^{2}<\varepsilon/2$.
Then
\begin{align*}
\sum_{k=1}^{K}|a_{k}^{n}-a_{k}^{m}|^{2} & \leq\sum_{k=1}^{\infty}|a_{k}^{n}-a_{k}^{m}|^{2}\\
 & =\|a^{n}-a^{m}\|^{2}\\
 & <\varepsilon/2
\end{align*}
for all $n,m\geq N$. Since $a_{k}^{m}\to a_{k}$ as $m\to\infty$
implies
\[
\sum_{k=1}^{K}|a_{k}^{n}-a_{k}^{m}|^{2}\to\sum_{k=1}^{K}|a_{k}^{n}-a_{k}|^{2}
\]
as $m\to\infty$, we see that after taking the limit $m\to\infty$
, we have
\begin{equation}
\sum_{k=1}^{K}|a_{k}^{n}-a_{k}|^{2}\leq\varepsilon/2.\label{lim}
\end{equation}
for all $n\geq N$. Taking the limit $K\to\infty$ in (\ref{lim})
gives us
\[
\|a^{n}-a\|^{2}<\varepsilon
\]
for all $n\geq N$. It follows that $a^{n}\to a$. \end{proof}

\subsection*{Problem 7}

\begin{prop}\label{prop} $C[a,b]$ is not a Hilbert space. \end{prop}

\begin{proof} For each $n\in\mathbb{N}$, define $f_{n}\in C[a,b]$
by
\[
f_{n}(x)=\begin{cases}
0 & x\in[a,c-\frac{1}{n}]\\
nx+1-nc & x\in[c-\frac{1}{n},c]\\
1 & x\in[c,b],
\end{cases}
\]
where $c=\frac{a+b}{2}$. We will show that the sequence $(f_{n})$
is a Cauchy sequence which is not convergent. 

\hfill

\textbf{Step 1:} We first show that the sequence $(f_{n})$ is a Cauchy
sequence. Let $\varepsilon>0$ and let $m,n\in\mathbb{N}$ such that
$n\geq m$. Then
\begin{align*}
\|f_{n}-f_{m}\|^{2} & =\int_{c-\frac{1}{m}}^{c-\frac{1}{n}}\left|mx+1-mc\right|^{2}dx+\int_{c-\frac{1}{n}}^{c}\left|nx+1-nc-\left(mx+1-mc\right)\right|^{2}dx\\
 & =\int_{c-\frac{1}{m}}^{c-\frac{1}{n}}\left|m(x-c)+1\right|^{2}dx+(n-m)^{2}\int_{c-\frac{1}{n}}^{c}\left|x-c\right|^{2}dx\\
 & \leq\left(\frac{1}{m}-\frac{1}{n}\right)\left|1-\frac{m}{n}\right|^{2}+\frac{(n-m)^{2}}{n^{3}}\\
 & \leq\frac{1}{m}-\frac{1}{n}+\frac{(n-m)^{2}}{n^{3}}.
\end{align*}

Choose $N\in\mathbb{N}$ such that $n\geq m\geq N$ implies
\[
\frac{1}{m}-\frac{1}{n}+\frac{(n-m)^{2}}{n^{3}}<\varepsilon^{2}.
\]
Then $n\geq m\geq N$ implies $\|f_{n}-f_{m}\|<\varepsilon$. Therefore
$(f_{n})$ is a Cauchy sequence.

\hfill

\textbf{Step 2: }We show that the sequence $(f_{n})$ is not convergent.
Assume for a contradiction that $f_{n}\to f$ where $f\in C[a,b]$.
Then
\begin{align*}
\|f_{n}-f\|^{2} & =\int_{a}^{c-\frac{1}{n}}|f(x)|^{2}dx+\int_{c-\frac{1}{n}}^{c}|f(x)-f_{n}(x)|^{2}dx+\int_{c}^{b}|f(x)-1|^{2}dx\\
 & \leq(c-a-\frac{1}{n})\sup_{x\in[a,c-\frac{1}{n}]}|f(x)|^{2}+\int_{c-\frac{1}{n}}^{c}|f(x)-f_{n}(x)|^{2}dx+(b-c)\sup_{x\in[c,c-\frac{1}{n}]}|f(x)-1|^{2}dx.
\end{align*}
Since $\|f_{n}-f\|\to0$ as $n\to\infty$, we see that (after taking
the limit $n\to\infty$) we must have
\[
f(x)=\begin{cases}
0 & \text{if }x\in[a,c]\\
1 & \text{if }x\in[c,b]
\end{cases}
\]
but this is not a continuous function. Thus we obtain a contradiction.
\end{proof}

\section*{Appendix}

\begin{prop}\label{propsubspace} Let $(V,\langle\cdot,\cdot\rangle)$
be an inner-product space and let $W$ be a subspace of $V$. Then
$(W,\langle\cdot,\cdot\rangle|_{W\times W}\rangle$ is an inner-product
space, where $\langle\cdot,\cdot\rangle|_{W\times W}\colon W\times W\to\mathbb{C}$
is the restriction of $\langle\cdot,\cdot\rangle$ to $W\times W$.
\end{prop}

\begin{proof} All of the required properties for $\langle\cdot,\cdot\rangle|_{W\times W}$
to be an inner-product are \emph{inherited }by $\langle\cdot,\cdot\rangle$
since $W$ is a subset of $V$. For instance, let $x,y,z\in V$ and
let $\lambda\in\mathbb{C}$. Then
\begin{align*}
\langle x+\lambda y,z\rangle|_{W\times W} & =\langle x+\lambda y,z\rangle\\
 & =\langle x,z\rangle+\lambda\langle y,z\rangle\\
 & =\langle x,z\rangle|_{W\times W}+\lambda\langle y,z\rangle|_{W\times W}
\end{align*}
gives us linearity in the first argument. The other properties follow
similarly. \end{proof}

\begin{rem}\label{rem} As long as context is clear, then we denote
$\langle\cdot,\cdot\rangle|_{W\times W}$ simply by $\langle\cdot,\cdot\rangle$.
\end{rem}

\begin{prop}\label{propfinistrivial} Let $(V,\langle\cdot,\cdot\rangle)$
be an $n$-dimensional inner-product space. Then $(V,\langle\cdot,\cdot\rangle)$
is unitarily equivalent to $(\mathbb{C}^{n},\langle\cdot,\cdot\rangle_{\text{e}}\rangle$,
where $\langle\cdot,\cdot\rangle_{\text{e}}$ is the standard Euclidean
inner-product on $\mathbb{C}^{n}$. In particular, $(V,\langle\cdot,\cdot\rangle)$
is a separable Hilbert space. \end{prop}

\begin{proof} Let $\{v_{1},\dots,v_{n}\}$ be a basis for $V$. By
applying the Gram-Schmidt process to $\{v_{1},\dots,v_{n}\}$, we
can get an orthonormal basis, say $\{u_{1},\dots,u_{n}\}$, of $V$.
Let $\varphi\colon V\to\mathbb{C}^{n}$ be the unique linear isomorphism
such that
\[
\varphi(u_{i})=e_{i}
\]
where $e_{i}$ is the standard $i$th coordinate vector in $\mathbb{C}^{n}$
for all $1\leq i\leq n$. Then $\varphi$ is a unitary equivalence.
Indeed, it is an isomorphism since it restricts to a bijection on
basis sets. Moreover we have
\begin{align*}
\langle u_{i},u_{j}\rangle & =\langle\varphi(u_{i}),\varphi(u_{j})\rangle_{\text{e}}=\langle e_{i},e_{j}\rangle_{\text{e}}
\end{align*}
for all $1\leq i,j\leq n$. This implies
\[
\langle x,y\rangle=\langle\varphi(x),\varphi(y)\rangle_{\text{e}}
\]
for all $x,y\in V$. \end{proof}

\begin{prop}\label{propfinsubisclosed} Let $\mathcal{V}$ be an inner-product
space over $\mathbb{C}$ and let $\mathcal{W}$ be a finite dimensional
subspace of $\mathcal{V}$. Then $\mathcal{W}$ is a closed. \end{prop}

\begin{proof} Let $\{w_{1},\dots,w_{k}\}$ be an orthonormal basis
for $\mathcal{W}$ and let $(x_{n})$ be a sequence of vectors in
$\mathcal{W}$ such that $x_{n}\to x$ where $x\in\mathcal{V}$. For
each $n\in\mathbb{N}$, express $x_{n}$ in terms of the basis $\{w_{1},\dots,w_{k}\}$
say as
\[
x_{n}=\lambda_{1n}w_{1}+\cdots+\lambda_{kn}w_{k},
\]
where $\lambda_{1n},\dots,\lambda_{kn}\in\mathbb{C}$. Since $x_{n}\to x$
as $n\to\infty$, the sequence $(x_{n})$ is a Cauchy sequence. This
implies the sequence $(\lambda_{jn})_{n\in\mathbb{N}}$ of complex
numbers is a Cauchy sequence, for each $1\leq j\leq k$. Indeed, letting
$\varepsilon>0$, choose $N\in\mathbb{N}$ such that $n,m\geq N$
implies $\|x_{n}-x_{m}\|<\varepsilon$. Then $n,m\geq N$ implies
\begin{align*}
|\lambda_{jn}-\lambda_{jm}| & \leq|\lambda_{1n}-\lambda_{1m}|+\cdots+|\lambda_{kn}-\lambda_{km}|\\
 & =\|(\lambda_{1n}-\lambda_{1m})w_{1}+\cdots+(\lambda_{kn}-\lambda_{km})w_{k}\|\\
 & =\|x_{n}-x_{m}\|\\
 & <\varepsilon
\end{align*}
for each $1\leq j\leq k$. Now since $\mathbb{C}$ is complete, we
must have $\lambda_{jn}\to\lambda_{j}$ as $n\to\infty$ for some
$\lambda_{j}\in\mathbb{C}$ for all $1\leq j\leq n$. In particular,
we have
\begin{align*}
x & =\lim_{n\to\infty}x_{n}\\
 & =\lim_{n\to\infty}\left(\lambda_{1n}w_{1}+\cdots+\lambda_{kn}w_{k}\right)\\
 & =\lim_{n\to\infty}(\lambda_{1n}w_{1})+\cdots+\lim_{n\to\infty}(\lambda_{kn}w_{k})\\
 & =\lambda_{1}w_{1}+\cdots+\lambda_{k}w_{k},
\end{align*}
and this implies $x\in\mathcal{W}$, which implies $\mathcal{W}$
is closed. \end{proof}

\begin{lemma}\label{lemmacauchybound} Let $(x_{n})$ be a Cauchy
sequence in $\mathcal{V}$. Then $(x_{n})$ is bounded. \end{lemma}

\begin{proof} Let $\varepsilon>0$. Choose $N\in\mathbb{N}$ such
that $m,n\geq N$ implies $\|x_{n}-x_{m}\|<\varepsilon$. Thus, fixing
$m\in\mathbb{N}$, we see that $n\geq N$ implies 
\[
\|x_{n}\|<\|x_{m}\|+\varepsilon.
\]
Now we let 
\[
M=\max\{\|x_{1}\|,\|x_{2}\|,\dots,\|x_{m}\|+\varepsilon\}.
\]
Then $M$ is a bound for $(x_{n})$. \end{proof}

\begin{prop}\label{propcauchy} Let $(x_{n})$ and $(y_{n})$ be Cauchy
sequences of vectors in $\mathcal{V}$. Then $(\langle x_{n},y_{n}\rangle)$
is a Cauchy sequence of complex numbers. \end{prop}

\begin{proof} Let $\varepsilon>0$. Choose $M_{x}$ and $M_{y}$
such that $\|x_{n}\|<M_{x}$ and $\|y_{n}\|<M_{y}$ for all $n\in\mathbb{N}$.
We can do this by Lemma~(\ref{lemmacauchybound}). Next, choose $N\in\mathbb{N}$
such that $n,m\geq N$ implies $\|x_{n}-x_{m}\|<\frac{\varepsilon}{2M_{y}}$
and $\|y_{n}-y_{m}\|<\frac{\varepsilon}{2M_{x}}$. Then $n,m\geq M$
implies 
\begin{align*}
|\langle x_{n},y_{n}\rangle-\langle x_{m},y_{m}\rangle| & =|\langle x_{n},y_{n}\rangle-\langle x_{m},y_{n}\rangle+\langle x_{m},y_{n}\rangle-\langle x_{m},y_{m}\rangle|\\
 & =|\langle x_{n}-x_{m},y_{n}\rangle+\langle x_{m},y_{n}-y_{m}\rangle|\\
 & \leq|\langle x_{n}-x_{m},y_{n}\rangle|+|\langle x_{m},y_{n}-y_{m}\rangle|\\
 & \leq\|x_{n}-x_{m}\|\|y_{n}\|+\|x_{m}\|\|y_{n}-y_{m}\|\\
 & \leq\|x_{n}-x_{m}\|M_{y}+M_{x}\|y_{n}-y_{m}\|\\
 & <\varepsilon.
\end{align*}
This implies $(\langle x_{n},y_{n}\rangle)$ is a Cauchy sequence
of complex numbers in $\mathbb{C}$. The latter statement in the proposition
follows from the fact that $\mathbb{C}$ is complete. \end{proof}

\subsection*{Homework 2, Problem 5 }

\begin{prop}\label{prop} Let $\mathcal{V}$ be an inner-product space,
let $\mathcal{A}$ be a subspace of $\mathcal{V}$, let $x\in\mathcal{V}$,
and let $\lambda\in\mathbb{C}$. Then
\[
d(\lambda x,\mathcal{A})=|\lambda|d(x,\mathcal{A}).
\]
\end{prop}

\begin{proof} Choose a sequence $(y_{n})$ of elements in $\mathcal{A}$
such that
\[
\|x-y_{n}\|<d(x,\mathcal{A})+\frac{1}{|\lambda n|}
\]
for all $n\in\mathbb{N}$. Then since $\mathcal{A}$ is a subspace,
we have
\begin{align*}
d(\lambda x,\mathcal{A}) & \leq\|\lambda x-\lambda y_{n}\|\\
 & =|\lambda|\|x-y_{n}\|\\
 & <|\lambda|\left(d(x,\mathcal{A})+\frac{1}{|\lambda n|}\right)\\
 & =|\lambda|d(x,\mathcal{A})+\frac{1}{n}
\end{align*}
for all $n\in\mathbb{N}$. In particular, this implies $d(\lambda x,\mathcal{A})\leq|\lambda|d(x,\mathcal{A})$. 

~~~Conversely, choose a sequence $(z_{n})$ of elements in $\mathcal{A}$
such that
\[
\|\lambda x-z_{n}\|<d(\lambda x,\mathcal{A})+\frac{1}{n}
\]

Then since $\mathcal{A}$ is a subspace, we have
\begin{align*}
|\lambda|d(x,\mathcal{A}) & \leq|\lambda|\|x-z_{n}/|\lambda|\|\\
 & =\|\lambda x-z_{n}\|\\
 & <d(\lambda x,\mathcal{A})+\frac{1}{n}
\end{align*}
for all $n\in\mathbb{N}$. In particular, this implies $|\lambda|d(x,\mathcal{A})\leq d(\lambda x,\mathcal{A})$.
\end{proof}

\begin{prop}\label{prop} Let $\mathcal{V}$ be an inner-product space,
let $\mathcal{A}$ be a subspace of $\mathcal{V}$, and let $x,y\in\mathcal{V}$.
Then
\[
d(x+y,\mathcal{A})\leq d(x,\mathcal{A})+d(y,\mathcal{A}).
\]
\end{prop}

\begin{proof} Choose a sequences $(w_{n})$ and $(z_{n})$ of elements
in $\mathcal{A}$ such that
\[
\|x-w_{n}\|<d(x,\mathcal{A})+\frac{1}{2n}\quad\text{and}\quad\|y-z_{n}\|<d(y,\mathcal{A})+\frac{1}{2n}
\]
for all $n\in\mathbb{N}$. Then since $\mathcal{A}$ is a subspace,
we have
\begin{align*}
d(x+y,\mathcal{A}) & \leq\|(x+y)-(w_{n}+z_{n})\|\\
 & \leq\|x-w_{n}\|+\|y-z_{n}\|\\
 & <d(x,\mathcal{A})+\frac{1}{2n}+d(y,\mathcal{A})+\frac{1}{2n}\\
 & =d(x,\mathcal{A})+d(y,\mathcal{A})+\frac{1}{n}
\end{align*}
for all $n\in\mathbb{N}$. In particular, this implies $d(x+y,\mathcal{A})\leq d(x,\mathcal{A})+d(y,\mathcal{A})$.
\end{proof}
\end{document}
