%% LyX 2.2.3 created this file.  For more info, see http://www.lyx.org/.
%% Do not edit unless you really know what you are doing.
\documentclass[12pt,english]{article}
\usepackage[osf]{mathpazo}
\renewcommand{\sfdefault}{lmss}
\renewcommand{\ttdefault}{lmtt}
\usepackage[T1]{fontenc}
\usepackage[latin9]{inputenc}
\usepackage[paperwidth=30cm,paperheight=35cm]{geometry}
\setlength{\parindent}{0bp}
\usepackage{amsmath}
\usepackage{amssymb}

\makeatletter
%%%%%%%%%%%%%%%%%%%%%%%%%%%%%% User specified LaTeX commands.
\usepackage{tikz}
\usetikzlibrary{matrix,arrows,decorations.pathmorphing}
\usetikzlibrary{shapes.geometric}
\usepackage{tikz-cd}
\usepackage{amsthm}
\theoremstyle{plain}
\newtheorem{theorem}{Theorem}[section]
\newtheorem{lemma}[theorem]{Lemma}
\newtheorem{prop}{Proposition}[section]
\newtheorem*{cor}{Corollary}
\theoremstyle{definition}
\newtheorem{defn}{Definition}[section]
\newtheorem{ex}{Exercise} 
\newtheorem{example}{Example}[section]
\theoremstyle{remark}
\newtheorem*{rem}{Remark}
\newtheorem*{note}{Note}
\newtheorem{case}{Case}
\usepackage{graphicx}
\usepackage{amssymb}
\usepackage{tikz-cd}
\usetikzlibrary{calc,arrows,decorations.pathreplacing}
\tikzset{mydot/.style={circle,fill,inner sep=1.5pt},
commutative diagrams/.cd,
  arrow style=tikz,
  diagrams={>=latex},
}

\usepackage{babel}
\usepackage{hyperref}
\hypersetup{
    colorlinks,
    citecolor=black,
    filecolor=black,
    linkcolor=black,
    urlcolor=black
}
\usepackage{pgfplots}
\usetikzlibrary{decorations.markings}
\pgfplotsset{compat=1.9}

\makeatother

\usepackage{babel}
\begin{document}

\title{Real Analysis}
\maketitle

\section{Preliminaries}

\subsection{The extended real numbers}

\begin{defn} The \textbf{extended real number system }is $[-\infty,\infty]:=\mathbb{R}\cup\{-\infty,\infty\}$.
Let $X$ be a set. A function $f:X\to[-\infty,\infty]$ is called
an \textbf{extended real-valued function}. \end{defn}

~~~For any $x\in\mathbb{R}$, we have 
\begin{enumerate}
\item $x\pm\infty=\infty\pm x:=\pm\infty$. 
\item $x\cdot\pm\infty=\pm\infty\cdot x:=\begin{cases}
\pm\infty & \mbox{if }x>0\\
\mp\infty & \mbox{if }x<0
\end{cases}$ .
\item $\pm\infty\pm\infty=\pm\infty$.
\item $\infty\cdot\pm\infty=\pm\infty$ and $-\infty\cdot\pm\infty=\mp\infty$ 
\item $0\cdot\infty=\infty\cdot0=0\cdot-\infty=-\infty\cdot0=0$ (this will
show up when we integrate a function which takes values $\infty$
over a set of measure zero).
\item $\infty-\infty$ remains undefined. 
\end{enumerate}
~~~For any $S\subseteq\mathbb{R}$ recall $\mbox{inf }S=\mbox{greatest lower limit }S$
and $\mbox{sup }S=\mbox{least upper limit of }S$. We define $\mbox{inf }\emptyset=\infty$
and $\mbox{sup }\emptyset=-\infty$. Then for all $S\subseteq\mathbb{R}$,
$\mbox{inf }S$ and $\mbox{sup }S$ belong to $[-\infty,\infty]$
. 

\subsection{Algebra of Sets}

\begin{defn} Let $X$ be a set. A nonempty collection $\mathcal{A}$
of subsets of $X$ is called an \textbf{algebra }(or \textbf{Boolean
algebra})\textbf{ }if
\begin{enumerate}
\item it is closed under finite unions: $A\cup B\in\mathcal{A}$ for all
$A$ and $B$ in $\mathcal{A}$. 
\item it is closed under complements: $A^{c}\in\mathcal{A}$ for all $A$
in $\mathcal{A}$. 
\end{enumerate}
\end{defn}

\begin{rem} \hfill
\begin{enumerate}
\item ~~~If $\mathcal{A}$ is an algebra, then it is closed under finite
intersections: $A\cap B\in\mathcal{A}$ for all $A$ and $B$ in $\mathcal{A}$.
This follows because 
\begin{align*}
A\cap B & =\left(\left(A\cap B\right)^{c}\right)^{c}\\
 & =\left(A^{c}\cup B^{c}\right)^{c}\qquad\mbox{(De Morgan).}
\end{align*}
\item $X$ and $\emptyset$ must belong to $\mathcal{A}$ too. Since $\mathcal{A}$
is nonempty, there exists $A\in\mathcal{A}$. Then $A\cup A^{c}=X$
and $X^{c}=\emptyset$. 
\end{enumerate}
\end{rem}

\begin{prop} For any nonempty collection $\mathcal{C}$ of subsets
of $X$, there exists a smallest algebra $\mathcal{A}$ which contains
$\mathcal{C}$. \end{prop}

\begin{proof} Let $\mathcal{A}$ be the set of all elements of the
form $A\cup B$ or $A^{c}$ where $A$ and $B$ belong to $\mathcal{C}$.
Then $\mathcal{A}$ is a nonempty collection which contains $\mathcal{C}$
and is closed under finite unions and complements. If $\mathcal{B}$
is any algebra which contains $\mathcal{C}$, then it must contain
all elements of the form $A\cup B$ or $A^{c}$ where $A$ and $B$
belong to $\mathcal{C}$. So $\mathcal{B}$ must contain $\mathcal{A}$.
Therefore, $\mathcal{A}$ is the smallest algebra containing $\mathcal{C}$.
\end{proof}

\begin{defn} The smallest algebra containing $\mathcal{C}$ is called
the \textbf{algebra generated by }$\mathcal{C}$. \end{defn}

\begin{prop}\label{disjointify} Let $\mathcal{A}$ be an algebra
of subsets of $X$ and $\{A_{i}\}$ a sequence of sets in $\mathcal{A}$.
Then there exists $\{B_{i}\}$ of sets in $\mathcal{A}$ such that
$B_{n}\subseteq A_{n}$, $B_{i}\cap B_{j}=\emptyset$ for all $i\neq j$,
and $\bigcup_{i}B_{i}=\bigcup_{i}A_{i}$. \end{prop}

\begin{proof} Define $B_{1}=A_{1}$ and $B_{n}=A_{n}\setminus(A_{1}\cup\cdots\cup A_{n-1})=A_{n}\cap A_{1}^{c}\cap\cdots\cap A_{n-1}^{c}$
for all $n>1$. Clearly we have $B_{n}\subseteq A_{n}$ for all $n$.
Given $B_{i}$ and $B_{j}$ with $i\neq j$, we want to show $B_{i}\cap B_{j}=\emptyset$.
Without loss of generality, assume $i<j$. Then $B_{j}$ consists
of all elements which belong to $A_{j}$ but do not belong to $A_{k}$
for $1\leq k<j$. In particular, $A_{i}\cap B_{j}=\emptyset$. Since
$B_{i}\subseteq A_{i}$, we also have $B_{i}\cap B_{j}=\emptyset$.
It remains to show $\bigcup_{i}B_{i}=\bigcup_{i}A_{i}$. Since $B_{i}\subseteq A_{i}$
for all $i$, we have $\bigcup_{i}B_{i}\subseteq\bigcup_{i}A_{i}$.
To show the reverse inclusion, suppose $x\in\bigcup_{i}A_{i}$. Choose
$m$ to be the smallest natural number such that $x\in A_{m}$. Then
$x$ does not belong to $A_{1},\dots,A_{m-1}$. In other words, $x\in B_{m}$.
Thus $\bigcup_{i}B_{i}\supseteq\bigcup_{i}A_{i}$. \end{proof}

\begin{defn} An algebra $\mathcal{A}$ of subsets of $X$ is called
a $\sigma$-\textbf{algebra }if it is closed under countable unions,
i.e. if $\{A_{i}\}_{i=1}^{\infty}$ is countable collection of members
in $\mathcal{A}$, then $\bigcup_{i}A_{i}$ is a member of $\mathcal{A}$.
\end{defn}

\begin{rem} A $\sigma$-algebra is closed under countable intersections.
In fact, let $\{A_{i}\}\subseteq\mathcal{A}$ be countable. Then $\bigcap_{i}A_{i}=\left(\left(\bigcap_{i}A_{i}\right)^{c}\right)^{c}=\left(\bigcup_{i}A_{i}^{c}\right)^{c}$.
\end{rem}

\begin{prop} For any nonempty collection $\mathcal{C}$ of subsets
of $X$, then there exists a smallest $\sigma$-algebra $\mathcal{A}$
which contains $\mathcal{C}$. \end{prop}

\begin{proof} Let $\mathcal{A}$ be the set of all elements of the
form $\bigcup\limits _{i=1}^{\infty}A_{i}$ or $A^{c}$ where $A_{i}$
and $A$ belong to $\mathcal{C}$. \end{proof}

\begin{defn} $\mathcal{A}$ is called the $\sigma$-algebra \textbf{generated
by }$\sigma$. \end{defn}

\begin{defn} The collection $\mathcal{B}$ of \textbf{borel sets
}of $\mathbb{R}$ is the $\sigma$-algebra generated by the open sets.
\end{defn}

\begin{rem} $\mathcal{B}$ is the $\sigma$-algebra generated by
the closed sets: Let $A$ be a closed set. Then $A^{c}$ is open,
so $A^{c}\in\mathcal{B}$. Since $\mathcal{B}$ is an algebra, $A\in\mathcal{B}$.
Now let $\mathcal{B}'$ be any $\sigma$-algebra containing all closed
sets. Let $A$ be an open set, so that $A^{c}$ is a closed set. Then
$A^{c}\in\mathcal{B}'$. Since $\mathcal{B}'$ is a $\sigma$-algebra,
$A\in\mathcal{B}'$. Since $\mathcal{B}$ is the smallest $\sigma$-algebra
which contains all open sets, we must have $\mathcal{B}\subseteq\mathcal{B}'$.
\end{rem}

\section{Lebesgue Measure}

\begin{defn} Let $\mathcal{A}$ be a $\sigma$-algebra of subsets
of $X$. A function $\mu:\mathcal{A}\to[0,\infty]$ is called a \textbf{measure
}if for each sequence $\{A_{n}\}$ of disjoint sets in $\mathcal{A}$,
\[
\mu\left(\bigcup_{n=1}^{\infty}A_{n}\right)=\sum_{n=1}^{\infty}\mu(A_{n}).
\]

\end{defn}

\begin{theorem} Let $\mu$ be a measure on a $\sigma$-algebra $\mathcal{A}$. 
\begin{enumerate}
\item If there is a set $A\in\mathcal{A}$ such that $\mu(A)<\infty$, then
$\mu(\emptyset)=0$.
\item If $A_{1},\dots,A_{n}\in\mathcal{A}$ are pairwise disjoint, then
$\mu(A_{1}\cup\cdots\cup A_{n})=\mu(A_{1})+\cdots+\mu(A_{n})$.
\item If $A,B\in\mathcal{A}$ and $A\subseteq B$, then $\mu(A)\leq\mu(B)$.
\item Let $\{A_{n}\}$ be any sequence in $\mathcal{A}$. Then $\mu\left(\bigcup_{n=1}^{\infty}A_{n}\right)\leq\sum_{n=1}^{\infty}\mu(A_{n})$. 
\end{enumerate}
\end{theorem}

\begin{proof}\hfill
\begin{enumerate}
\item Let $A_{1}=A$ and $A_{n}=\emptyset$ for all $n\geq2$. Then 
\begin{align*}
\mu(A) & =\mu\left(\bigcup_{n=1}^{\infty}A_{n}\right)\\
 & =\sum_{n=1}^{\infty}\mu(A_{n})\\
 & =\mu(A)+\sum_{n=2}^{\infty}\mu(\emptyset).
\end{align*}
 Canceling $\mu(A)$ from both sides, we obtain $\sum_{n=2}^{\infty}\mu(\emptyset)=0$,
hence $\mu(\emptyset)$. 
\item Let $A_{i}=\emptyset$ for all $i\geq n+1$. Then 
\begin{align*}
\mu\left(\bigcup_{i=1}^{n}A_{i}\right) & =\mu\left(\bigcup_{i=1}^{\infty}A_{i}\right)\\
 & =\sum_{i=1}^{n}\mu(A_{i})+\sum_{i=n+1}^{\infty}\mu(\emptyset)\\
 & =\sum_{i=1}^{n}\mu(A_{i}).
\end{align*}
\item By finite additivity, we have 
\begin{align*}
\mu(B) & =\mu(A\cup(B\setminus A))\\
 & =\mu(A)+\mu(B\setminus A)\\
 & \geq\mu(A).
\end{align*}
\item By Prop~(\ref{disjointify}), there exists a disjoint sequence $\{B_{n}\}$
in $\mathcal{A}$ such that $B_{n}\subseteq A_{n}$ for all $n\geq1$
and $\bigcup B_{n}=\bigcup A_{n}$. Therefore 
\begin{align*}
\mu\left(\bigcup_{n=1}^{\infty}A_{n}\right) & =\mu\left(\bigcup_{n=1}^{\infty}B_{n}\right)\\
 & =\sum_{n=1}^{\infty}\mu(B_{n})\\
 & \leq\sum_{n=1}^{\infty}\mu(A_{n}).
\end{align*}
\end{enumerate}
\end{proof}

\begin{defn} Let $\mathcal{A}$ be a $\sigma$-algebra of $\mathbb{R}$.
A measure $\mu:\mathcal{A}\to[0,\infty]$ is called \textbf{translation
invariant} if for any set $A\in\mathcal{A}$ and any $x\in\mathbb{R}$,
$A+x\in\mathcal{A}$ and $\mu(A+x)=\mu(A)$, where $A+x=\{a+x\mid a\in A\}$.\end{defn}

\begin{example} Let $\mathcal{A}=\mathbf{P}(\mathbb{R})$. Define
\[
\mu(A)=\begin{cases}
\infty & \mbox{if }A\mbox{ is infinite}\\
n & \mbox{if }A\mbox{ is finite with }n\mbox{ elements.}
\end{cases}
\]

Then $\mu$ is a translation invariant measure, called the \textbf{counting
measure}. \end{example}

\subsection{Outer Measure}

\begin{defn} For any set $A\subseteq\mathbb{R}$, define the outer
measure of $A$ to be 
\[
\mu^{\star}(A)=\mbox{inf}\left\{ \sum_{n}\ell(I_{n})\mid\{I_{n}\}\mbox{ is a countable collection of bounded open intervals, }A\subseteq\bigcup I_{n}\right\} 
\]
 \end{defn}

\begin{prop} Let $I\subseteq\mathbb{R}$ be an interval. Then $\mu^{\star}(I)=\ell(I)$.
\end{prop}

\begin{proof} Case $1$: $I=[a,b]$ is compact. For any $\varepsilon>0$,
since $\{(a-\varepsilon,b+\varepsilon)\}$ covers $[a,b]$, we have
$\mu^{\star}(I)\le\ell(a-\varepsilon,b+\varepsilon)$. Since this
holds for all $\varepsilon<0$, we have $\mu^{\star}(I)\le\ell(a,b)$
(If $\mu^{\star}(I)>\ell(a,b)=b-a$, then there would exist some $\varepsilon>0$
such that $\mu^{\star}(I)>b-a+2\varepsilon$. But we know that $\mu^{\star}(I)\le\ell(a-\varepsilon,b+\varepsilon)=b-a+2\varepsilon$,
which is a contradiction) To show the reverse direction $(\geq)$,
let $\{I_{n}\}$ be any cover of $[a,b]$ consisting of countably
many open intervals. Since $I$ is compact, there exists a finite
subcover $\{\tilde{I}_{n}\}$. There exists an interval, say $\tilde{I}_{1}=(a_{1},b_{1})$
such that $a_{1}<a<b_{1}$. If $b_{1}>b$, we stop the process, otherwise
there exists an interval, say $\tilde{I}_{2}=(a_{2},b_{2})$ such
that $a_{2}<b_{1}<b_{2}$. This process must stop since there are
only finitely many intervals and we obtain $\{\tilde{I}_{j}\}$. Thus
\begin{align*}
\sum\ell(I_{n}) & \geq\sum\ell(\tilde{I}_{n})\\
 & \geq\sum_{j=1}^{k}\ell(\tilde{I}_{j})\\
 & =(b_{k}-a_{k})+(b_{k-1}-a_{k-1})+\cdots+(b_{1}-a_{1})\\
 & =(b_{k}-a)+(b_{k-1}-a_{k})+\cdots+(b_{1}-a_{2})\\
 & >b-a.
\end{align*}

Thus $\mu^{\star}(I)\geq b-a$ since $\mu^{\star}(I)$ is the greatest
lower bound. 

~~~Case $2:$ $I$ is any bounded interval ($(a,b),[a,b),(a,b],\dots)$.
Then for $\varepsilon>0$, there exists a closed bounded interval
$J\subseteq I$ such that $\ell(I)-\varepsilon\le\ell(J)$. For instance,
if $I=(a,b]$, then $J=[a+\varepsilon,b]$ has this property. So
\begin{align*}
\ell(I)-\varepsilon & \leq\ell(J)\\
 & =\mu^{\star}(J)\\
 & \le\mu^{\star}(I)\\
 & \le\mu^{\star}(\bar{I})\\
 & =\ell(\bar{I})\\
 & =\ell(I).
\end{align*}

Since $\varepsilon>0$ is arbitrary, $\mu^{\star}(I)=\ell(I)$. 

~~~Case $3:$ $I$ is an unbounded interval ($[a,\infty),(-\infty,b],(-\infty,\infty),\dots$).
For any number $N$, there exists a compact interval $J\subseteq I$
such that $\mu^{\star}(I)\geq\mu^{\star}(J)=\ell(J)\geq N$. Hence
$\mu^{\star}(I)=\infty=\ell(I)$.\end{proof}

\begin{prop} $\mu^{\star}$ is countably subadditive, i.e., for any
countable collection $\{A_{n}\}$ of subsets of $\mathbb{R}$, 
\begin{equation}
\mu^{\star}\left(\bigcup_{n=1}^{\infty}A_{n}\right)\leq\sum_{n=1}^{\infty}\mu^{\star}(A_{n}).\label{eq:subadditivity}
\end{equation}

\end{prop}

\begin{proof} Suppose $\mu^{\star}(A_{n})=\infty$ for some $n$.
Then both sides of (\ref{eq:subadditivity}) are $\infty$. So we
assume $\mu^{\star}(A_{n})<\infty$ for all $n$. Let $\varepsilon>0$.
By the definition of $\mu^{\star}$, for each $n$, there exists a
countable cover $\{I_{n,m}\}_{m}$ consisting of bounded open intervals
such that 
\[
\sum_{m=1}^{\infty}\ell(I_{n,m})\leq\mu^{\star}(A_{n})+\frac{\varepsilon}{2^{n}}.
\]

Now $\{I_{n,m}\}_{n,m}$, as $n$ and $m$ vary, is a countable cover
of $\bigcup A_{n}$ consisting of open bounded intervals. So 
\begin{align*}
\mu^{\star}\left(\bigcup A_{n}\right) & \leq\sum_{n}\sum_{m}\ell(I_{n,m})\\
 & \leq\sum_{n}\mu^{\star}(A_{n})+\frac{\varepsilon}{2^{n}}\\
 & =\sum_{n}\mu^{\star}(A_{n})+\varepsilon.
\end{align*}

Since $\varepsilon>0$ is arbitrary, we have (\ref{eq:subadditivity}).
\end{proof}

\begin{cor} \hfill
\begin{enumerate}
\item If $A$ is countable, then $\mu^{\star}(A)=0$.
\item The set $[0,1]$ is uncountable.
\end{enumerate}
\end{cor}

\begin{proof}\hfill
\begin{enumerate}
\item Let $A=\{a_{i}\}$, then
\begin{align*}
\mu^{\star}(A) & \leq\sum\mu^{\star}(a_{i})\\
 & =\sum0\\
 & =0
\end{align*}
So $\mu^{\star}(A)=0$. 
\item If $[0,1]$ were countable, then we must have $\mu^{\star}[0,1]=0$.
But $\mu^{\star}[0,1]=\ell[0,1]=1$. 
\end{enumerate}
\end{proof}

\subsection{Measurable Sets and Lebesgue Mesaure}

$\mu^{\star}$ is defined on $\mathbf{P}(\mathbb{R})$, is translation
invariant, and $\mu^{\star}(I)=\ell(I)$. However, it is not countably
additive, nor even finitely additive. So $\mu^{\star}$ is \emph{not
}a measure on $\mathbf{P}(\mathbb{R})$.

\begin{defn} (Caratheodory, 1914). A set $E\subseteq\mathbb{R}$
is (Lebesgue) \textbf{measurable }if for each set $A\subseteq\mathbb{R}$,
we have $\mu^{\star}(A)=\mu^{\star}(A\cap E)+\mu^{\star}(A\cap E^{c})$.
\end{defn}

\begin{rem} \hfill
\begin{enumerate}
\item When proving measurability, we need only prove $\mu^{\star}(A)\geq\mu^{\star}(A\cap E)+\mu^{\star}(A\cap E^{c})$
since $\mu^{\star}$ is subadditive.
\item If there exists a non-measurable set, then $\mu^{\star}$ is not finitely
additive.
\end{enumerate}
\end{rem}

\begin{prop} If $\mu^{\star}(E)=0$, then $E$ is measurable. In
particular, any countable set is measurable. \end{prop}

\begin{proof} Let $A\subseteq\mathbb{R}$. Then 
\begin{align*}
\mu^{\star}(A\cap E)+\mu^{\star}(A\cap E^{c}) & \leq\mu^{\star}(E)+\mu^{\star}(A\cap E^{c})\\
 & =\mu^{\star}(A\cap E^{c})\\
 & \leq\mu^{\star}(A).
\end{align*}
Therefore $E$ is measurable. \end{proof}

\begin{prop}\label{measurablealgebra} The measurable sets form an
algebra, i.e.
\begin{enumerate}
\item $E_{1}$ and $E_{2}$ is measurable implies $E_{1}\cup E_{2}$ is
measurable.
\item $E$ is measurable implies $E^{c}$ is measurable.
\end{enumerate}
\end{prop}

\begin{proof} \hfill
\begin{enumerate}
\item Let $A\subseteq\mathbb{R}$. Since $E_{1}$ and $E_{2}$ are measurable
\begin{align*}
\mu^{\star}(A) & =\mu^{\star}(A\cap E_{1})+\mu^{\star}(A\cap E_{1}^{c})\\
 & =\mu^{\star}(A\cap E_{1})+\mu^{\star}(A\cap E_{1}^{c}\cap E_{2})+\mu^{\star}(A\cap E_{1}^{c}\cap E_{2}^{c})\\
 & \geq\mu^{\star}((A\cap E_{1})\cup(A\cap E_{1}^{c}\cap E_{2}))+\mu^{\star}(A\cap E_{1}^{c}\cap E_{2}^{c})\\
 & =\mu^{\star}(A\cap(E_{1}\cup E_{2}))+\mu^{\star}(A\cap(E_{1}\cup E_{2})^{c}).
\end{align*}
 Therefore $E_{1}\cup E_{2}$ is measurable.
\item Let $A\subseteq\mathbb{R}$. Since $E$ is measurable,
\begin{align*}
\mu^{\star}(A) & =\mu^{\star}(A\cap E)+\mu^{\star}(A\cap E^{c})\\
 & =\mu^{\star}(A\cap E^{c})+\mu^{\star}(A\cap E)\\
 & =\mu^{\star}(A\cap E^{c})+\mu^{\star}(A\cap(E^{c})^{c}).
\end{align*}
\end{enumerate}
\end{proof}

\begin{cor} The family $\mathcal{M}$ of measurable sets is an algebra.
\end{cor}

\begin{proof} Follows from Prop~(\ref{measurablealgebra}) directly.
\end{proof}

\begin{lemma} Let $A$ be any set and let $E_{1},\dots,E_{n}$ be
a finite sequence of disjoint measurable sets. Then 
\[
\mu^{\star}\left(A\cap\left(\bigcup_{i=1}^{n}E_{i}\right)\right)=\sum_{i=1}^{n}\mu^{\star}(A\cap E_{i}).
\]
\end{lemma}

\begin{rem} Letting $A=\mathbb{R}$ gives finite additivity of $\mu^{\star}$
on $\mathcal{M}$. \end{rem}

\begin{proof} Use induction. The equality holds for $n=1$. Assume
that it holds for some $n\geq1$. By the measurability of $E_{n+1}$,
\begin{align*}
\mu^{\star}\left(A\cap\left(\bigcup_{i=1}^{n+1}E_{i}\right)\right) & =\mu^{\star}\left(A\cap\left(\bigcup_{i=1}^{n+1}E_{i}\right)\cap E_{n+1}\right)+\mu^{\star}\left(A\cap\left(\bigcup_{i=1}^{n+1}E_{i}\right)\cap E_{n+1}^{c}\right)\\
 & =\mu^{\star}\left(A\cap E_{n+1}\right)+\mu^{\star}\left(A\cap\left(\bigcup_{i=1}^{n}E_{i}\right)\right)\\
 & =\mu^{\star}\left(A\cap E_{n+1}\right)+\sum_{i=1}^{n}\mu^{\star}(A\cap E_{i})\\
 & =\sum_{i=1}^{n+1}\mu^{\star}(A\cap E_{i}).
\end{align*}

\end{proof}

\begin{prop} The union of a countable collection of measurable sets
is measurable. \end{prop}

\begin{proof} Let $E$ be a countable union of sets in $\mathcal{M}$.
By Prop~(\ref{measurablealgebra}), $\mathcal{M}$ is an algebra.
Therefore we can disjointify $E$ into sets $\{E_{i}\}_{i=1}^{\infty}$.
Now for all $A\subseteq\mathbb{R}$, 
\[
\mu^{\star}(A)\leq\mu^{\star}(A\cap E)+\mu^{\star}(A\cap E^{c})
\]
 since $\mu^{\star}$ is countably subadditive. Define
\[
F_{n}:=\bigcup_{i=1}^{n}E_{i}\subseteq E.
\]

Then $F_{n}\in\mathcal{M}$ and $F_{n}^{c}\subset E^{c}$. Hence for
all $n\geq1$,
\begin{align*}
\mu^{\star}(A) & =\mu^{\star}(A\cap F_{n})+\mu^{\star}(A\cap F_{n}^{c})\\
 & \geq\mu^{\star}\left(A\cap\left(\bigcup_{i=1}^{n}E_{i}\right)\right)+\mu^{\star}(A\cap E^{c})\\
 & =\sum_{i=1}^{n}\mu^{\star}(A\cap E_{i})+\mu^{\star}(A\cap E^{c}).
\end{align*}

Since this holds for all $n\geq1$, we can take the limit as $n\to\infty$.
Therefore 
\begin{align*}
\mu^{\star}(A) & \geq\sum_{i=1}^{\infty}\mu^{\star}(A\cap E_{i})+\mu^{\star}(A\cap E^{c})\\
 & \geq\mu^{\star}\left(\bigcup_{i=1}^{\infty}\left(A\cap E_{i}\right)\right)+\mu^{\star}(A\cap E^{c})\\
 & =\mu^{\star}\left(A\cap\left(\bigcup_{i=1}^{\infty}E_{i}\right)\right)+\mu^{\star}(A\cap E^{c})\\
 & =\mu^{\star}\left(A\cap E\right)+\mu^{\star}(A\cap E^{c}).
\end{align*}

\end{proof}

\begin{prop} For any $a\in\mathbb{R}$, the interval $(a,\infty)$
is measurable. \end{prop}

\begin{proof} Let $A\subseteq\mathbb{R}$ and define $A_{1}:=A\cap(a,\infty)$,
$A_{2}:=A\cap(a,\infty)^{c}$. It suffices to show $\mu^{\star}(A)\geq\mu^{\star}(A_{1})+\mu^{\star}(A_{2})$.
If $\mu^{\star}(A)=\infty$, the inequality holds trivially. So we
assume $\mu^{\star}(A)<\infty$. 
\begin{enumerate}
\item Case $1$: Suppose $a\notin A$. Let $\varepsilon>0$ be arbitrary.
By the definition of $\mu^{\star}$, there exists a countable cover
$\{I_{n}\}$ of $A$ consisting of bounded open intervals such that
$\sum\ell(I_{n})\leq\mu^{\star}(A)+\varepsilon$. For each $n$, define
$I_{n}^{1}:=I_{n}\cap(a,\infty)$ and $I_{n}^{2}:=I_{n}\cap(-\infty,a)$.
Then 
\[
A_{1}\subseteq\bigcup I_{n}^{1}\qquad A_{2}\subseteq\bigcup I_{n}^{2}
\]
 Hence, 
\begin{align*}
\mu^{\star}(A_{1})+\mu^{\star}(A_{2}) & \leq\sum\ell(I_{n}^{1})+\sum\ell(I_{n}^{2})\\
 & =\sum\ell(I_{n})\\
 & \leq\mu^{\star}(A)+\varepsilon
\end{align*}
since $\varepsilon>0$ is arbitrary, Case $1$ holds. 
\item Case $2:$ Suppose $a\in A$. Then
\begin{align*}
\mu^{\star}(A) & =\mu^{\star}(A\cap\{a\})+\mu^{\star}(A\cap\{a\}^{c})\\
 & =\mu^{\star}(\{a\})+\mu^{\star}(A\setminus\{a\})\\
 & =\mu^{\star}(A\setminus\{a\}),
\end{align*}
 and this reduces to Case $1$.
\end{enumerate}
\end{proof}

\begin{cor} Every Borel set is measurable. In particular, open sets
and closed sets are measurable. \end{cor}

\begin{proof} Since the collection $\mathcal{M}$ of measurable sets
is a $\sigma$-algebra, and for all $a\in\mathbb{R}$, $(-\infty,a]=(a,\infty)^{c}$
is measurable, also for all $b\in\mathbb{R}$, 
\[
(-\infty,b)=\bigcup_{n=1}^{\infty}(-\infty,b-\frac{1}{n}]
\]

Hence $(-\infty,b)$ is measurable, and thus each open interval $(a,b)=(a,\infty)\cap(b,\infty)$
is measurable. Each open set in $\mathbb{R}$ is a countable union
of open intervals and must be measurable. Since the collection $\mathcal{B}$
of Borel sets is the $\sigma$-algebra generated by the open sets,
we have $\mathcal{B}\subseteq\mathcal{M}$.\end{proof}

\begin{theorem} The collection $\mathcal{M}$ of measurable sets
is a $\sigma$-algebra that contains the sigma algebra $\mathcal{B}$
of Borel sets. In particular, it contains intervals, open sets, closed
sets, $G_{\delta}$ sets, and $F_{\sigma}$ sets. \end{theorem}

\textbf{Recall: }Let $A\subseteq\mathbb{R}$ and $t\in\mathbb{R}$,
then $A+t=\{a+t\mid a\in A\}=t+A$. 

\begin{ex} 
\begin{enumerate}
\item Let $\{A_{\alpha}\}_{\alpha\in I}$ be a family of subsets of $\mathbb{R}$
and $t\in\mathbb{R}$, then $\left(\bigcap_{\alpha\in I}A_{\alpha}\right)+t=\bigcap_{\alpha\in I}\left(A_{\alpha}+t\right)$
and $\left(\bigcup_{\alpha\in I}A_{\alpha}\right)+t=\bigcup_{\alpha\in I}\left(A_{\alpha}+t\right)$. 
\item For any $A\subseteq\mathbb{R}$ and $t\in\mathbb{R}$, $(A+t)^{c}=A^{c}+t$. 
\end{enumerate}
\end{ex}

\begin{proof}
\begin{enumerate}
\item h
\item Suppose $x\in A^{c}+t$. Then $x=b+t$ where $b\notin A$. If $x\in A+t$,
then $b+t=a+t$ for some $a\in A$, but this implies $b=a\in A$,
which is a contradiction. So $x\in(A+t)^{c}$. Conversely, suppose
$x\in(A+t)^{c}$. Since $x\in\mathbb{R}$, $x=b+t$ for some $b\in\mathbb{R}$.
Since $x\in(A+t)^{c}$, $b\notin A$. Therefore $x\in A^{c}+t$. 
\end{enumerate}
\end{proof}

\begin{rem} Translation commutes with intersection, union, and complement.
\end{rem}

\begin{prop} The translate of a measurable set is measurable. \end{prop}

\begin{proof} Let $E\in\mathcal{M}$ and $x\in\mathbb{R}$. To show
$E+x\in\mathcal{M}$, let $A\subseteq\mathbb{R}$ be arbitrary. Then
\begin{align*}
\mu^{\star}\left(A\cap\left(E+x\right)\right)+\mu^{\star}\left(A\cap\left(E+x\right)^{c}\right) & =\mu^{\star}(A\cap\left(E+x\right)-x)+\mu^{\star}\left(A\cap\left(E+x\right)^{c}-x\right)\\
 & =\mu^{\star}\left(\left(A-x\right)\cap E\right)+\mu^{\star}\left(\left(A-x\right)\cap E^{c}\right)\\
 & =\mu^{\star}(A-x)\\
 & =\mu^{\star}(A).
\end{align*}

\end{proof}

\subsection{Inner and Outer Approximation}

\begin{prop}Let $E\in\mathcal{M}$ with $\mu^{\star}(E)<\infty$.
Then for any $A\subseteq\mathbb{R}$ (not necessarily measurable)
with $E\subseteq A$
\begin{equation}
\mu^{\star}(A\setminus E)=\mu^{\star}(A)-\mu^{\star}(E).\label{eq:EminusA}
\end{equation}

\end{prop}

\begin{rem} Note that (\ref{eq:EminusA}) is undedfined if $\mu^{\star}(E)=\infty$.\end{rem}

\begin{proof} By the measurability of $E$, 
\begin{align*}
\mu^{\star}(A) & =\mu^{\star}(A\cap E)+\mu^{\star}(A\setminus E)\\
 & =\mu^{\star}(E)+\mu^{\star}(A\setminus E)
\end{align*}

which agrees with the desired property.\end{proof}

\begin{theorem}\label{theorem11} Let $E\subseteq\mathbb{R}$. Then
the following are equivalent
\begin{enumerate}
\item $E$ is measurable.
\item Given an $\varepsilon>0$, there exists an open set $O\supseteq E$
such that $\mu^{\star}(O\setminus E)<\varepsilon$.
\item There exists a $G_{\delta}$ set $G\supseteq E$ such that $\mu^{\star}(G\setminus E)=0$.
\item Given an $\varepsilon>0$, there exists a closed set $F\subseteq E$
such that $\mu^{\star}(E\setminus F)<\varepsilon$.
\item There exists an $F_{\delta}$ set $F\subseteq E$ such that $\mu^{\star}(E\setminus F)=0$.
\end{enumerate}
\end{theorem}

\begin{proof} We'll only prove $(1)\implies(2)\implies(3)\implies(1)$
and leave the rest as an exercise. 

\hfill

$(1)\implies(2):$ Case $1$: $\mu^{\star}(E)<\infty$. Then there
exists a countable cover $\{I_{n}\}$ of $E$ consisting of open bounded
intervals such that $\sum\ell(I_{n})<\mu^{\star}(E)+\varepsilon$.
Let $O$ be defined as $\bigcup I_{n}$. Then 
\begin{align*}
\mu^{\star}(O\setminus E) & =\mu^{\star}(O)-\mu^{\star}(E)\\
 & \leq\sum\mu^{\star}(I_{n})-\mu^{\star}(E)\\
 & =\sum\ell(I_{n})-\mu^{\star}(E)\\
 & <\mu^{\star}(E)+\varepsilon-\mu^{\star}(E)\\
 & =\varepsilon.
\end{align*}

Case $2:$ $\mu^{\star}(E)=\infty$. Let $E_{n}=E\cap[n,n+1)$ for
all $n\in\mathbb{Z}$. Then $E=\bigcup_{n\in\mathbb{Z}}E_{n}$, with
the union being disjoint and $\mu^{\star}(E_{n})\leq1<\infty$. Note
that $E_{n}\in\mathcal{M}$. Let $\varepsilon>0$. By Case $1$, there
exists an open set $O_{n}$ with $E_{n}\subseteq O_{n}$ such that
$\mu^{\star}(O_{n}\setminus E_{n})<\varepsilon/3\cdot2^{|n|}$ for
all $n\in\mathbb{Z}$. Let $O=\bigcup_{n\in\mathbb{Z}}O_{n}$. Then
$E\subseteq O$ and 
\begin{align*}
\mu^{\star}(O\setminus E) & =\mu^{\star}\left(\left(\bigcup_{n\in\mathbb{Z}}O_{n}\right)\setminus E\right)\\
 & =\mu^{\star}\left(\bigcup_{n\in\mathbb{Z}}(O_{n}\setminus E)\right)\\
 & \leq\sum_{n\in\mathbb{Z}}\mu^{\star}(O_{n}\setminus E)\\
 & \leq\sum_{n=\infty}^{-1}\mu^{\star}(O_{n}\setminus E)+\mu^{\star}(O_{0}\setminus E)+\sum_{n=1}^{\infty}\mu^{\star}(O_{n}\setminus E)\\
 & =\varepsilon.
\end{align*}

\hfill

$(2)\implies(3):$ By $(2)$, for each $n\in\mathbb{N}$, there exists
an open set $O_{n}\subseteq\mathbb{R}$ with $E\subseteq O_{n}$ such
that $\mu^{\star}(O_{n}\setminus E)<1/n$. Let 
\[
G=\bigcap_{n=1}^{\infty}O_{n}.
\]

Then $G$ is a $G_{\delta}$ set and $E\subseteq G$. Moreover, for
all $n\in\mathbb{N}$, 
\[
G\setminus E\subseteq O_{n}\setminus E
\]

and thus by monotonicity, 
\[
\mu^{\star}(G\setminus E)\leq\mu^{\star}(O_{n}\setminus E)<\frac{1}{n}.
\]

This holds for all $n$ and thus $\mu^{\star}(G\setminus E)=0$.

\hfill

$(3)\implies(1):$ Since $\mu^{\star}(G\setminus E)=0$, we know $G\setminus E$
is measurable. Since $G$ is a $G_{\delta}$-set, $G$ is measurable.
Since $E=G\cap(G\setminus E)^{c}$, $E$ is measurable. \end{proof}

\begin{rem}Theorem~(\ref{theorem11}) leads to inner and outer regularity
of Lebesgue measure. \end{rem}

\begin{theorem} (One of Littlewood's Principles) Let $E\in\mathcal{M}$
with $\mu^{\star}(E)<\infty$. For any $\varepsilon>0$, there exists
a finite disjoint collection of open bounded intervals $\{I_{k}\}_{k=1}^{n}$
whose union $O=\bigcup_{n=1}^{\infty}I_{k}$ satisfies
\[
\mu^{\star}(E\setminus O)+\mu^{\star}(O\setminus E)<\varepsilon.
\]
 \end{theorem}

\begin{rem} \hfill
\begin{enumerate}
\item $(E\setminus O)\cup(O\setminus E)$ is called the \textbf{symmetric
difference }of $E$ and $O$. 
\item This principle says that each measurable set of finite outer measure
is nearly a finite union of intervals. 
\end{enumerate}
\end{rem}

\subsection{Lebesgue measure, continuity, and Borel Cartelli Lemma}

\begin{defn} The restriction of $\mu^{\star}$ to the $\sigma$-algebra
$\mathcal{M}$ of measurable sets, denoted by $\mu$, is called ($1$-dimensional)
\textbf{Lebesgue mesaure}. Thus, for all $E\in\mathcal{M}$, $\mu(E)=\mu^{\star}(E)$.
\end{defn}

~~~The following proposition shows that $\mu$ is indeed a measure
on $\mathcal{M}$. 

\begin{prop} Let $\{E_{n}\}$ be a sequence of measurable sets. Then
\[
\mu\left(\bigcup E_{n}\right)\leq\sum\mu(E_{n}).
\]

Furthermore, if the $E_{n}$ are pairwise disjoint, then 
\[
\mu\left(\bigcup E_{n}\right)=\sum\mu(E_{n}),
\]

i.e. $\mu$ is a measure on $\mathcal{M}$.\end{prop}

\begin{proof} The countable subadditivity of $\mu$ follows from
that of $\mu^{\star}$. Assume now that $E_{n}$ are pairwise disjoint.
For any $k\in\mathbb{N}$, we have
\begin{align*}
\mu\left(\bigcup_{n=1}^{\infty}E_{n}\right) & \geq\mu\left(\bigcup_{n=1}^{k}E_{n}\right)\\
 & =\sum_{n=1}^{k}\mu(E_{n}).
\end{align*}

Letting $k\to\infty$, we obtain 
\[
\mu\left(\bigcup_{n=1}^{\infty}E_{n}\right)\geq\sum_{n=1}^{\infty}\mu(E_{n}).
\]

\end{proof}

\begin{theorem} Lebesgue measure $\mu$ defined on the $\sigma$-algebra
$\mathcal{M}$ of (Lebesgue) measurable sets, assigns length to any
interval, is translation invariant, and is countably additive. \end{theorem}

\begin{defn} A countable collection of sets $\{E_{n}\}$ is said
to be \textbf{ascending }(or \textbf{increasing}) if $E_{n}\subseteq E_{n+1}$
for all $n$ and is \textbf{descending }(or \textbf{decreasing}) if
$E_{n}\supseteq E_{n+1}$ for all $n$. \end{defn}

\begin{theorem} (Continuity property of Lebesgue measure) Let $\{E_{n}\}_{n\in\mathbb{N}}$
be a countable sequence of measurable sets. 
\begin{enumerate}
\item If $\{E_{n}\}_{n\in\mathbb{N}}$ is ascending, then $\mu\left(\bigcup_{n\in\mathbb{N}}E_{n}\right)=\lim_{n\to\infty}\mu(E_{n})$.
\item If $\{E_{n}\}_{n\in\mathbb{N}}$ is descending and $\mu(E_{1})<\infty$,
then $\mu\left(\bigcap_{n\in\mathbb{N}}E_{n}\right)=\lim_{n\to\infty}\mu(E_{n})$.
\end{enumerate}
\end{theorem}

\begin{proof} We only prove $(2)$. The proof of $(1)$ is in the
text. Let $E:=\bigcap_{n\in\mathbb{N}}E_{n}$ and for each $i\in\mathbb{N}$,
define $F_{i}:=E_{i}\setminus E_{i+1}$. Then $E_{1}\setminus E=\bigcup_{n\in N}F_{i}$.
Also, the $F_{i}$ are disjoint and hence 
\begin{align*}
\mu(E_{1})-\mu(E) & =\mu\left(E_{1}\setminus E\right)\\
 & =\mu\left(\bigcup_{n\in N}F_{i}\right)\\
 & =\sum_{n\in\mathbb{N}}\mu(F_{i})\\
 & =\lim_{n\to\infty}\sum_{i=1}^{n}\mu(F_{i})\\
 & =\lim_{n\to\infty}\sum_{i=1}^{n}\left(\mu(E_{i})-\mu(E_{i+1})\right)\\
 & =\lim_{n\to\infty}\left(\mu(E_{1})-\mu(E_{n+1})\right)\\
 & =\mu(E_{1})-\lim_{n\to\infty}\mu(E_{n+1}).
\end{align*}

Since $\mu(E_{1})<\infty$, we can cancel $\mu(E_{1})$ on both sides
to obtain our desired result. \end{proof}

\begin{rem} The result does not hold without assuming $\mu(E_{1})<\infty$.
Counterexample: let $E_{n}=(n,\infty)$ for all $n\geq1$. \end{rem}

\begin{defn} Let $E\in\mathcal{M}$. We say that a property holds
\textbf{almost everywhere }on $E$ or it holds for \textbf{almost
all }$x\in E$ if there exists a subset $E_{0}\subseteq E$ for which
$\mu(E_{0})=0$ and the property holds for all $x\in E\setminus E_{0}$.
\end{defn}

\begin{example} Define $f:[0,1]\to\{0,1\}$ by 
\[
f(x)=\begin{cases}
1 & \mbox{if \ensuremath{x} is irrational.}\\
0 & \mbox{if \ensuremath{x} is rational.}
\end{cases}
\]

Then $f(x)=1$ for almost all $x\in[0,1]$. \end{example}

\section{Cantor Set}

\begin{prop} The Cantor Set $C$ is compact, uncountable, and of
measure zero. \end{prop}

\begin{proof} Proof by contradiction. Suppose $C$ were countable.
Let $C=\{x_{k}\}_{k=1}^{\infty}$ be an enumeration of $C$. We will
construct a decreasing sequence of Cantor intervals that avoids all
the $x_{k}$'s as follows. One of the Cantor intervals of $C_{1}$
fails to contain $x_{1}$; denote it by $F_{1}$. One of the two Cantor
intervals of $C_{2}$ contained in $F_{1}$ fails to contain $x_{1}$,
denote it by $F_{2}$. Continue to obtain a decreasing sequence $\{F_{k}\}_{k=1}^{\infty}$
of nonempty compact sets. Let 
\[
F:=\bigcap_{k=1}^{\infty}F_{k}.
\]

Then $\emptyset\neq F\subseteq\bigcap_{k=1}^{\infty}C_{k}=C$. Choose
$x\in F$. Then for all $k\in\mathbb{N}$, $x\neq x_{k}$ as $x\in F_{k}$
but $x_{k}\notin F_{k}$. This contradicts that $C=\{x_{k}\}_{k=1}^{\infty}$.
Thus, $C$ is uncountable.\end{proof}

\begin{prop} There is a subset of the Cantor set that is measurable
but is not a Borel set. \end{prop}

\begin{rem} We cannot construct it, but only show it exists. \end{rem}

\begin{proof} We only give an outline of the proof. The Cantor-Lebesgue
function $\psi(x)=\varphi(x)+x$, where $\varphi(x)$ is the Cantor
function, maps the Cantor set to a set of positive measure. Some subset
of the Cantor set is thus mapped to a nonmeasurable set. Such a subset
of $C$ cannot be a Borel set since a continuous strictly increasing
function maps Borel sets to Borel sets. \end{proof}

~~~All inclusions below are proper 
\[
\{\emptyset,\mathbb{R}\}\subset\mathcal{B}\subset\mathcal{M}\subset\mathbf{P}(\mathbb{R}).
\]

Cantor Function
\[
\frac{1}{4}=\frac{1}{1+3}=1-3+3^{2}-3^{3}+\cdots
\]

\begin{rem} 
\begin{enumerate}
\item We saw that $\mu(C)=0$; yet $C$ is uncountable. In $1919$, Hausdorff
extended Caratheodory's construction to obtain a family of measures
$H^{s}$, where $0\leq s\leq n$.
\[
\mathcal{H}_{\delta}^{s}(E):=\mbox{inf}\left\{ \sum_{i=1}^{\infty}\left|U_{i}\right|^{s}\mid E\subseteq\bigcup_{i=1}^{\infty}U_{i},\,\mbox{diam}(U_{i})\leq\delta\right\} 
\]
 where $\mbox{diam}(F)=\mbox{sup}\{|x-y|\mid x,y\in F\}$. Then we
define 
\[
\mathcal{H}^{s}(E):=\lim_{\delta\to0^{+}}\mathcal{H}_{\delta}^{s}(E)
\]
 to be the \textbf{Hausdorff measure }of $E$. For $s=\frac{\log2}{\mbox{log}3}$,
we have $H^{s}(C)=1$. 
\end{enumerate}
\end{rem}

\begin{defn} Let $\mathcal{F}$ be a nonempty family of nonempty
sets. A \textbf{choice function }$f$ on $\mathcal{F}$ is a function
from $\mathcal{F}\to\bigcup\limits _{F\in\mathcal{F}}F$ with the
property that each set $F\in\mathcal{F}$, $f(F)$ is a member of
$F$. \textbf{(Zermelo's) Axiom of Choice}: Let $\mathcal{F}$ be
a nonempty collection of nonempty sets, then there is a choice function
on $\mathcal{F}$. \end{defn}

~~~We show $[0,1)$ contains a nonmeasurable set. Define an equivalence
relation $\sim$ on $[0,1)$ so that $x\sim y$ if $x-y\in\mathbb{Q}$.
By the Axiom of Choice, there exists a $P\subseteq[0,1)$ which contains
exactly one element from each equivalence class. Let $\{r_{i}\}_{i=0}^{\infty}$
be an enumberation of $[0,1)\cap\mathbb{Q}$ with $r_{0}=0$. Define
$P_{i}=P+r_{i}$ modulo $1$. Then $\{P_{i}\}$ is a pairwise disjoint
sequence of sets with $\bigcup P_{i}=[0,1)$. Assuming that the $P_{i}$
are all measurable, we have 
\begin{align*}
1 & =\mu[0,1)\\
 & =\sum_{i=1}^{\infty}\mu(P_{i})\\
 & =\sum_{i=1}^{\infty}\mu(P)\\
 & \in\{0,\infty\},
\end{align*}

which is a contradiction. 

\section{Measurable Functions}

\subsection{Lebesgue Measurable Functions}

~~~Recall that $\mathcal{M}$ denotes the collection of Lebesgue
measurable subsets of $\mathbb{R}$. 

\begin{prop}\label{prop1} Let $f:E\subseteq\mathbb{R}\to[-\infty,\infty]$
where $E\in\mathcal{M}$. Then the following are equivalent:
\begin{enumerate}
\item For all $c\in R$, $f^{-1}(c,\infty]=\{x\in E\mid f(x)>c\}\in\mathcal{M}$. 
\item For all $c\in R$, $f^{-1}[c,\infty]=\{x\in E\mid f(x)\geq c\}\in\mathcal{M}$. 
\item For all $c\in R$, $f^{-1}[-\infty,c)=\{x\in E\mid f(x)<c\}\in\mathcal{M}$. 
\item For all $c\in R$, $f^{-1}[-\infty,c]=\{x\in E\mid f(x)\leq c\}\in\mathcal{M}$. 
\end{enumerate}
Furthermore, these conditions imply that for all $c\in[-\infty,\infty]$,
$f^{-1}(c)=\{x\in E\mid f(x)=c\}\in\mathcal{M}$. 

\end{prop}

\begin{proof}\hfill
\begin{itemize}
\item $(1)\implies(2):$ $f^{-1}[c,\infty]=\bigcap_{n=1}^{\infty}f^{-1}(c-\frac{1}{n},\infty]\in\mathcal{M}$. 
\item $(2)\implies(3):$ $f^{-1}[-\infty,c)=E\setminus f^{-1}[c,\infty]\in\mathcal{M}$. 
\item $(3)\implies(4):$ $f^{-1}[-\infty,c]=\bigcap_{n=1}^{\infty}f^{-1}[-\infty,c+\frac{1}{n})\in\mathcal{M}$. 
\item $(4)\implies(1):$ $f^{-1}(c,\infty]=E\setminus f^{-1}[-\infty,c]\in\mathcal{M}$. 
\end{itemize}
~~~Now we show that these conditions imply $f^{-1}(c)=\{x\in E\mid f(x)=c\}\in\mathcal{M}$
for all $c\in[-\infty,\infty]$.
\begin{itemize}
\item Case $1$: If $c\in\mathbb{R}$, then $f^{-1}(c)=f^{-1}[c,\infty]\setminus f^{-1}(c,\infty]\in\mathcal{M}$. 
\item Case $2:$ If $c=\infty$, then $f^{-1}(\infty)=\bigcap_{n=0}^{\infty}f^{-1}(n,\infty]\in\mathcal{M}$.
\item Case $3:$ If $c=-\infty$, then $f^{-1}(-\infty)=\bigcap_{n=0}^{\infty}f^{-1}[-\infty,-n)\in\mathcal{M}$.
\end{itemize}
\end{proof}

\begin{defn} We say $f:E\subseteq\mathbb{R}\to[-\infty,\infty]$
is \textbf{Lebesgue measurable }if its domain $E$ is measurable and
it satisfies any of the first conditions in Prop~(\ref{prop1})\end{defn}

\begin{prop}\label{prop2} Let $E\subseteq\mathcal{M}$ and $f:E\to\mathbb{R}$.
Then $f$ is measurable if and only if for each open set $O\subseteq\mathbb{R}$,
the preimage $f^{-1}(O)=\{x\in E\mid f(x)\in O\}$ is measurable.
\end{prop}

\begin{proof} $(\impliedby)$ Let $c\in\mathbb{R}$. Since $(c,\infty)$
is open, by assumption $f^{-1}(c,\infty)$ is measurable. Hence by
definition, $f$ is measurable $(\infty\notin\mathbb{R})$. $(\implies)$
Assume $f$ is measurable. Each open interval $I_{k}$ is an open
interval of the form $I_{k}=A_{k}\cap B_{k}$ where $A_{k}=(a_{k},\infty)$
and $B_{k}=(-\infty,b_{k})$. Then $f^{-1}(A_{k})$ and $f^{-1}(B_{k})$
are measurable, hence 
\begin{align*}
f^{-1}(I_{k}) & =f^{-1}(A_{k}\cap B_{k})\\
 & =f^{-1}(A_{k})\cap f^{-1}(A_{k})\in\mathcal{M}
\end{align*}

Let $O$ be an open set. Then 
\[
O=\bigcup_{k=1}^{\infty}I_{k}
\]

So 
\begin{align*}
f^{-1}(O) & =f^{-1}\left(\bigcup_{k=1}^{\infty}I_{k}\right)\\
 & =\bigcup_{k=1}^{\infty}f^{-1}(I_{k})\in\mathcal{M}
\end{align*}

\end{proof}

\begin{rem} \hfill
\begin{enumerate}
\item Topology of $[-\infty,\infty]$ are countable (equivalently arbitrary)
unions of sets of the form $[-\infty,b)$, $(a,b)$, and $(a,\infty]$. 
\item Proposition~(\ref{prop2}) holds for $f:E\to[-\infty,\infty]$. In
the proof of $(\implies)$, replace $(c,\infty)$ by $(c,\infty]$.
For $(\impliedby)$, allow $I_{k}$ to be of the form $[-\infty,b)$
or $(a,\infty]$.
\item The converse of the last statement of Prop~(\ref{prop1}) does not
hold, i.e. $(5)$ does not imply $(1)$ (or $(2)-(4)$) . Counterexample:
Let $N$ be a nonmeasurable set in $[0,1]$, define 
\[
f(x)=\begin{cases}
x & \mbox{if }x\in N\\
-x & \mbox{if }x\in[0,1]\setminus N\\
0 & \mbox{otherwise. }
\end{cases}
\]
\end{enumerate}
\end{rem}

\begin{prop} Let $E\in\mathcal{M}$ and $f:E\subseteq\mathbb{R}\to\mathbb{R}$
be continuous. Then $f$ is measurable. \end{prop}

\begin{proof} Let $O\subseteq\mathbb{R}$ be open. Then $f^{-1}(O)$
is open in $E$ and thus there exists an open set $U$ in $\mathbb{R}$
such that $f^{-1}(O)=E\cap U\in\mathcal{M}$. By Proposition~(\ref{prop2}),
$f$ is measurable. \end{proof}

\begin{prop} Let $I=[a,b]\subseteq\mathbb{R}$ be an interval and
$f:I\to\mathbb{R}$ be monotone (increasing or decreasing). Then $f$
is measurable.\end{prop}

\begin{proof} Assume $f$ is monotone increasing. The case $f$ is
monotone decreasing is similar. Let $c\in\mathbb{R}$ and set $d=\mbox{inf}(\mbox{Im}f\cap(c,\infty))$.
Then there exists an $x\in I$ such that $f(x)=d$ and $f^{-1}(c,\infty)=f^{-1}(d,\infty)=[x,b]\in\mathcal{M}$.
\end{proof}

\begin{defn} We say that two functions $f$ and $g$ are \textbf{equal
almost everywhere}. Denoted $f=g$ a.e. if $f$ and $g$ have the
same domain and $\mu(x\mid f(x)\neq g(x))=0$. \end{defn}

\begin{example} Let 
\[
f(x)=\begin{cases}
1 & \mbox{if }x\in\mathbb{Q}\\
0 & \mbox{if }x\in[0,1]-\mathbb{Q}
\end{cases}
\]

So $f=0$ a.e. \end{example}

\begin{prop} Let $E\in\mathcal{M}$ and $f:E\subseteq\mathbb{R}\to[-\infty,\infty]$. 
\begin{enumerate}
\item If $f$ is measurable and $f=g$ a.e. on $E$, then $g$ is measurable.
\item Let $D\subseteq E$ be measurable. Then $f$ is measurable if and
only if the restrictions of $f$ to $D$ and $E-D$ are measurable. 
\end{enumerate}
\end{prop}

\begin{proof} We prove $(1)$ only and leave $(2)$ as an exercise.
Let $A=\{x\in E\mid f(x)\neq g(x)\}$. Then for all $c\in\mathbb{R}$, 

\begin{align*}
\{x\in E\mid g(x)>c\} & =\left(\{x\in E\mid f(x)\geq c\}\cup\{x\in A\mid g(x)>c\}\right)-\{x\in A\mid g(x)\leq c\}\\
 & =(A_{1}\cup A_{2})-A_{3}
\end{align*}

$A_{1}\in\mathcal{M}$ because $f$ is measurable and $\mu(A_{2})=\mu(A_{3})=\mu(A)=0$
by monotoncity, so they are measurable. \end{proof}

\begin{theorem} Let $E\in\mathcal{M}$, $f,g:E\to\mathbb{R}$ be
measurable, and $\alpha\in\mathbb{R}$. Then
\begin{enumerate}
\item $f+\alpha$ is measurable.
\item $\alpha f$ is measurable. 
\item $f+g$ is measurable
\item $fg$ is measurable. 
\end{enumerate}
\end{theorem}

\begin{rem} If $f,g$ are extended real-valued function, $f+g$ and
$f-g$ could equal $\infty-\infty$, which is undefined. \end{rem}

\begin{proof}\hfill
\begin{enumerate}
\item Since $\{x\in E\mid f(x)+\alpha<c\}=f^{-1}(-\infty,c-\alpha)\in\mathcal{M}$,
$f+\alpha$ is measurable. 
\item Since
\[
\{x\in E\mid\alpha f(x)<c\}=\begin{cases}
f^{-1}(-\infty,c/\alpha)\in\mathcal{M} & \mbox{if }\alpha>0\\
f^{-1}(c/\alpha,\infty)\in\mathcal{M} & \mbox{if }\alpha<0\\
\{x\in E\mid0<c\}\in\mathcal{M} & \mbox{if }\alpha=0
\end{cases}
\]
 $\alpha f$ is measurable. 
\item Note that $(f+g)(x)<c$ if and only if there exists an $r\in\mathbb{Q}$
such that $f(x)<r$ and $r<c-g(x)$. Hence 
\[
(f+g)^{-1}(-\infty,c)=\bigcup_{r\in\mathbb{Q}}f^{-1}(-\infty,r)\cap g^{-1}(-\infty,c-r)\in\mathcal{M}.
\]
\item We first prove $f^{2}$ is measurable:
\[
(f^{2})^{-1}(c,\infty)=\begin{cases}
f^{-1}(\sqrt{c},\infty)\cup f^{-1}(-\infty,-\sqrt{c})\in\mathcal{M} & c\geq0\\
E\in\mathcal{M} & c<0
\end{cases}
\]
 Next, note that 
\[
fg=\frac{1}{4}\left((f+g)^{2}-(f-g)^{2}\right)\in\mathcal{M}.
\]
\end{enumerate}
\end{proof}

\begin{defn} Let $f:(X,\mathcal{A},\mu)\to(Y,\mathcal{B},\mathcal{\nu})$
be a function between measurable spaces. We say $f$ is \textbf{measurable
}if for all $B\in\mathcal{B}$, $f^{-1}(B)\in\mathcal{A}$. \end{defn}

~~~Functions we study now are $f:(\mathbb{R},\mathcal{M})\to(\mathbb{R},\mathcal{B})$
or $([-\infty,\infty],\mathcal{B})$. We do not use $\mathcal{M}$
in the image space in order to include all continuous functions. In
fact, $f$ is measurable if and only if $f^{-1}(B)\in\mathcal{M}$
for all $B\in\mathcal{B}$. It is possible that the preimage of a
Lebesgue measurable set under a continuous function is not Lebesgue
measurable. So using $\mathcal{M}$ would exclude some continuous
functions. Since 
\[
\mathcal{B}\subset\mathcal{M}\subset\mathcal{P}(\mathbb{R}),
\]

with strict inclusions, the composition of a measurable function with
a continuous function is \emph{not }necessarily measurable. A counterexample
is given on page $57$ in the book. Nevertheless, the composition
of a continuous function and a measurable function is measurable. 

\begin{prop} Let $E\in\mathcal{M}$, $g:E\to\mathbb{R}$ be measurable,
and $f:\mathbb{R}\to\mathbb{R}$ be continuous. Then the composition
$f\circ g$ is measurable. In particular, if $f$ is measurable and
$p\geq0$, then $|f|^{p}$ is measurable. \end{prop}

\begin{proof} Let $O\subseteq\mathbb{R}$ be open. Then 
\[
(f\circ g)^{-1}(O)=g^{-1}(f^{-1}(O))\in\mathcal{M}
\]

since the preimage of an open set by a measurable function is measurable.
\end{proof}

\begin{proof} \hfill
\begin{enumerate}
\item Let $h:=\mbox{min}\{f_{1},\dots,f_{n}\}$, then for all $c\in\mathbb{R}$,
the set $\{x\in E\mid h(x)>c\}=\bigcap_{k=1}^{n}\{x\in E\mid f_{k}(x)>c\}$.
The proof for $\mbox{max}\{f_{1},\dots,f_{n}\}$ is similar. 
\item Follows from $(1)$.
\end{enumerate}
\end{proof}

\subsection{Differential Pairwise Limits and Simple Approximation}

\begin{defn} Let $f,f_{n}:E\subseteq\mathbb{R}\to[-\infty,\infty]$
(or $\mathbb{R})$ be a sequence of functions for $n\geq1$ and let
$A\subseteq E$. 
\begin{enumerate}
\item We say that $\{f_{n}\}$ \textbf{converges pointwise }on $A$ if 
\[
\mbox{lim}_{n\to\infty}f_{n}(x)=f(x)\qquad\forall x\in A.
\]
\item We say that $\{f_{n}\}$ \textbf{converges pointwise a.e. }on $A$
if there exists $B\subseteq A$ with $\mu(B)=0$ such that $\{f_{n}\}$
converges pointwise on $A-B$.
\item We say that $\{f_{n}\}$ \textbf{converges uniformly }on $A$ if for
all $\varepsilon>0$, there exists $N\in\mathbb{N}$ such that for
all $n\geq N$, 
\[
|f_{n}(x)-f(x)|<\varepsilon\qquad\forall x\in A.
\]
\end{enumerate}
\end{defn}

\begin{prop} Let $E\in\mathcal{M}$ and $f_{n}:E\to[-\infty,\infty]$
for $n\geq1$ be a sequence of measurable functions. Then $\mbox{sup}(f_{n})$,
$\mbox{inf}(f_{n})$, $\mbox{limsup}(f_{n})$, and $\mbox{liminf}(f_{n})$
are measurable functions.\end{prop}

\begin{proof} Let $g(x):=\mbox{sup}(f_{n}(x))$. Then $g^{-1}[-\infty,c]=\bigcap_{n=1}^{\infty}f_{n}^{-1}[-\infty,c]\in\mathcal{M}$.
The proof of $\mbox{inf}(f_{n})$ is similar. To prove that $\mbox{liminf}(f_{n})$
is measurable, note that 
\begin{align*}
\mbox{liminf}(f_{n})(x) & =\lim_{n\to\infty}(\mbox{inf}\{f_{k}(x)\mid k\geq n\})\\
 & =\mbox{sup}(\mbox{inf}\{f_{k}(x)\mid k\geq n\}).
\end{align*}

The proof of $\mbox{limsup}(f_{n})$ is similar. \end{proof}

\begin{prop} Let $E\in\mathcal{M}$ and $f_{n}:E\to[-\infty,\infty]$
(or $\mathbb{R})$ for $n\geq1$ be a sequence of measurable functions.
If $\{f_{n}\}$ converges pointwise a.e. on $E$ to $f$, then $f$
is measurable. \end{prop}

\begin{proof} Let $E_{0}\subseteq E$ with $\mu(E_{0})=0$ such that
$f_{n}$ converges to $f$ on $E\setminus E_{0}$. Since $f_{\mid E_{0}}$
is measurable (the preimage of every measurable set under $f_{\mid E_{0}}$
is a subset of $E_{0}$, a set with measure zero, is measurable.),
$f$ is measurable if and only if $f_{\mid E\setminus E_{0}}$ is
measurable. Replacing $E$ with $E\setminus E_{0}$ if necessary,
we may assume that $f_{n}$ converges pointwise to $f$ on $E$. Let
$\varepsilon>0$. Then for all $c\in\mathbb{R}$, we have 
\begin{equation}
f^{-1}[-\infty,c)=\bigcup_{n=1}^{\infty}\bigcap_{k=n}^{\infty}f_{k}^{-1}[-\infty,c-1/n).\label{eq:setequal}
\end{equation}

\end{proof}

\begin{defn} \hfill
\begin{enumerate}
\item Let $A\subseteq\mathbb{R}$. The \textbf{characteristic function $\chi_{_{A}}$
of $A$ }is the function defined as 
\[
\chi_{_{A}}(x)=\begin{cases}
1 & \mbox{if }x\in A\\
0 & \mbox{if }x\notin A.
\end{cases}
\]
\item Let $E\in\mathcal{M}$. The function $\varphi:E\to\mathbb{R}$ is
called \textbf{simple }if it is measurable and assume only a finite
number of values, i.e. there exists distinct values $c_{1},\dots,c_{n}$
such that
\[
\varphi=\sum_{i=1}^{n}c_{i}\chi_{_{A_{i}}}
\]
 where $A_{i}:=\{x\in E\mid\varphi(x)=c_{i}\}\in\mathcal{M}$. 
\end{enumerate}
\end{defn}

\begin{rem} \hfill
\begin{enumerate}
\item $\chi_{_{A}}$ is measurable if and only if $A$ is measurable. $(\implies)$
$A=\{x\in\mathbb{R}\mid\chi_{_{A}}(x)=1\}\in\mathcal{M}$. $(\impliedby)$
For all $c\in\mathbb{R}$, 
\[
\{x\in\mathbb{R}\mid\chi_{_{A}}(x)\geq c\}=\begin{cases}
\emptyset & \mbox{if }c>1\\
A & \mbox{if }1\geq c>0\\
\mathbb{R} & \mbox{if }0\geq c.
\end{cases}
\]
\item The sum, product, and differences of two simple functions are simple
(homework). 
\item Let $\varphi:[a,b]\to\mathbb{R}$. Then $\varphi$ is called a \textbf{step
function }if there exists a partition 
\[
a=x_{0}<x_{1}<\cdots<x_{n}=b
\]
such that for all $0\leq i\le n-1$, $\varphi$ is constant on $[x_{i},x_{i+1})$
and $[x_{n-1},x_{n}]$. Step functions are simple functions, however
the converse is not necessarily true. A counterexample is given by
$f:[0,1]\to\mathbb{R}$, where 
\[
f(x)=\begin{cases}
1 & \mbox{if }x\mbox{ is irrational}\\
-1 & \mbox{if }x\mbox{ is rational}
\end{cases}
\]
 then $f(x)=\chi_{_{[0,1]-\mathbb{Q}}}-\chi_{_{[0,1]\cap\mathbb{Q}}}$. 
\end{enumerate}
\end{rem}

\begin{lemma} (Simple Approximation Lemma) Let $E\in\mathcal{M}$
and $f:E\subseteq\mathbb{R}\to\mathbb{R}$ be measurable and bounded,
i.e. there exists $M\geq0$ for which $|f(x)|\leq M$ for all $x\in E$.
Then for all $\varepsilon>0$, there exists simple functions $\varphi_{\varepsilon}$
and $\psi_{\varepsilon}$ defined on $E$ such that 
\[
\varphi_{\varepsilon}\leq f\leq\psi_{\varepsilon}\quad\mbox{and}\quad0\le\psi_{\varepsilon}-\varphi_{\varepsilon}<\varepsilon
\]
on $E$. \end{lemma}

\begin{proof} Let $c,d\in\mathbb{R}$ such that $f(E)\in[c,d]$ and
let 
\[
c=x_{0}<x_{1}<\cdots<x_{n}=d
\]

be a partition of $[c,d]$ such that 
\[
0<x_{k}-x_{k-1}<\varepsilon
\]

for all $0\le k\leq n$. Define $I_{k}=[x_{k-1},x_{k})$ and $E_{k}=f^{-1}(I_{k})$
for all $1\leq k\leq n$. Since $f$ is measurable, $E_{k}$ is measurable
for all $1\leq k\leq n$. Define simple functions 
\[
\varphi_{\varepsilon}=\sum_{k=1}^{n}x_{k-1}\chi_{_{E_{k}}}\quad\mbox{and}\quad\psi_{\varepsilon}=\sum_{k=1}^{n}x_{k}\chi_{_{E_{k}}}
\]

Let $x\in E$. Then there exists a unique $k$, $1\leq k\leq n$,
such that $x_{k-1}\leq f(x)\leq x_{k}$ and thus $\varphi_{\varepsilon}(x)=x_{k-1}\leq f(x)\leq x_{k}=\psi_{\varepsilon}(x)$.
Moreover, $\psi_{\varepsilon}(x)-\varphi_{\varepsilon}(x)=x_{k}-x_{k-1}<\varepsilon$. 

\end{proof}

\begin{lemma} (Simple Approximation Theorem) Let $E\in\mathcal{M}$
and $f:E\subseteq\mathbb{R}\to[-\infty,\infty]$. Then $f$ is measurable
if and only if there exists a sequence $\{\varphi_{n}\}$ of simple
functions on $E$ which converges pointwise on $E$ to $f$ and has
the property that $|\varphi_{n}|\leq|f|$ on $E$ for all $n$. If
$f$ is nonnegative, we may choose $\{\varphi_{n}\}$ to be nonnegative.\end{lemma}

\begin{proof} $(\impliedby)$ Since simple functions are measurable
and the limit of measurable functions is measurable, $f$ is measurable.
$(\implies)$ Omitted. In text use the Simple Approximation Lemma.\end{proof}

\subsubsection*{Littlewood's three principles}
\begin{enumerate}
\item Every measurable set is nearly a finite union of intervals.
\item Every measurable function is nearly continuous. (Lusin's Theorem)
\item Every pointwise convergent sequence of measurable functions is nearly
uniformly convergent. (Egoroff's Theorem)
\end{enumerate}
\begin{lemma}\label{lemma10} Let $E\in\mathcal{M}$ such that $\mu(E)<\infty$
and let $\{f_{n}\}$ be a sequence of measurable functions defined
on $E$. Suppose $\{f_{n}\}$ converges pointwise to a real-valued
function $f$ on $E$. Then given $\eta>0$ and $\delta>0$, there
exists a measurable set $A\subseteq E$ and $N\in\mathbb{N}$ such
that 
\[
\mu(E\setminus A)<\delta\quad\mbox{ and }\quad|f_{n}(x)-f(x)|<\eta
\]
 for all $x\in A$ and for all $n\geq N$.\end{lemma}

\begin{proof} For each $n\in N$, denote $h_{n}:=|f-f_{n}|$. Then
$h_{n}$ is measurable. It is well-defined since $f$ is real-valued.
Define for all $N\in\mathbb{N}$, 
\[
E_{N}:=\{x\in E\mid|f_{n}(x)-f(x)|<\eta,\,\forall n\geq N\}=\bigcap_{n=N}^{\infty}h_{n}^{-1}[0,\eta)\in\mathcal{M}
\]

Then $\{E_{N}\}_{N=1}^{\infty}$ is an ascending sequence of measurable
sets. Moreover,
\[
E=\bigcup_{N=1}^{\infty}E_{N},
\]

since $\{f_{n}\}$ converges pointwise to $f$ on $E$. By the continuity
property of Lebesgue measure, 
\[
\lim_{N\to\infty}\mu(E_{N})=\mu(E)<\infty.
\]

Thus we may choose a sufficiently large $N_{0}\in\mathbb{N}$ such
that $\mu(E_{N_{0}})>\mu(E)-\delta$. Define $A:=E_{N_{0}}$. Then
\[
\mu(E\setminus A)=\mu(E)-\mu(E_{N_{0}})<\infty
\]

and by the definition of $E_{N_{0}}$, 
\[
|f_{n}(x)-f(x)|<\eta
\]

for all $n\geq N_{0}$.\end{proof}

\begin{rem} Lemma~(\ref{lemma10}) holds if $\{f_{n}\}$ converges
a.e. on $E$. The proof is similar. \end{rem}

\begin{theorem} (Egoroff's Theorem) Let $E\in\mathcal{M}$ with $\mu(E)<\infty$
and let $f_{n}:E\to[-\infty,\infty]$ be a sequence of measurable
functions that converges a.e. to a real-valued function $f$ on $E$.
Then given $\varepsilon>0$, there exists a closed set $F\subseteq E$
such that $f_{n}\to f$ uniformly on $F$ and $\mu(E\setminus F)<\varepsilon$.\end{theorem}

\begin{proof} Given $n\in\mathbb{N}$ and $\varepsilon>0$, by Lemma~(\ref{lemma10})
and by the remark that follows, there exists measurable sets $A_{n}\subseteq E$
and $N_{n}\in\mathbb{N}$ such that
\[
\mu(E\setminus A_{n})<\frac{\varepsilon}{2^{n+1}},\qquad\mbox{and}\qquad|f_{k}(x)-f(x)|<\frac{1}{n}
\]

for all $x\in A_{n}$ and for all $k\geq N_{n}$. Define 
\[
A:=\bigcap_{n=1}^{\infty}A_{n}.
\]

Then 
\begin{align*}
\mu(E\setminus A) & =\mu\left(E\setminus\left(\bigcap_{n=1}^{\infty}A_{n}\right)\right)\\
 & =\mu\left(\bigcup_{n=1}^{\infty}(E\setminus A_{n})\right)\\
 & \leq\sum_{n=1}^{\infty}\mu\left(E\setminus A_{n}\right)\\
 & \leq\sum_{n=1}^{\infty}\frac{\varepsilon}{2^{n+1}}\\
 & =\frac{\varepsilon}{2}.
\end{align*}

~~~Let $\delta>0$. Then there exists $n_{0}\in\mathbb{N}$ such
that $1/n_{0}<\delta$. Note that $x\in A$ implies $x\in A_{n_{0}}$.
Thus, for all $n\geq N_{n_{0}}$, we have 
\[
|f_{n}(x)-f(x)|<\frac{1}{n_{0}}<\delta
\]

for all $x\in A_{n_{0}}$. Therefore $f_{n}\to f$ uniformly on $A$.
Finally, by Theorem 2.11, there exists a closed set $F\subseteq A$
such that $\mu(A\setminus F)<\varepsilon/2$. Thus, $\mu(E\setminus F)$
\begin{align*}
\mu(E\setminus F) & =\mu(E\setminus A)+\mu(A\setminus F)\\
 & <\frac{\varepsilon}{2}+\frac{\varepsilon}{2}\\
 & =\varepsilon.
\end{align*}

Therefore $f_{n}\to f$ uniformly on $F$. \end{proof}

\begin{prop} Let $E\in\mathcal{M}$ and $f$ be a simple function
defined on $E$. Then for all $\varepsilon>0$, there exists a continuous
function $g$ on $\mathbb{R}$ and a closed set $F\subseteq E$ such
that $f=g$ on $F$ and $\mu(E\setminus F)<\varepsilon$. \end{prop}

\begin{proof} Omitted. In text. \end{proof}

\begin{theorem}\label{theorem11.5} Let $E\in\mathcal{M}$ and $f:E\to\mathbb{R}$
be measurable. Then for all $\varepsilon>0$, there exists continuous
function $g$ on $\mathbb{R}$ and a closed set $F\subseteq E$ such
that $f=g$ on $F$ and $\mu(E\setminus F)<\varepsilon$. \end{theorem}

\begin{proof} Omitted. In text. \end{proof}

\section{The Lebesgue Integral}

\subsection{The Riemann Integral}

~~~This section is a reading assignment. Pay attention to defininition
$1$, i.e. the definition of the Riemann Integral. Recall
\begin{enumerate}
\item The Dirichlet function 
\[
f(x)=\begin{cases}
1 & \mbox{if }x\in\mathbb{Q}\\
0 & \mbox{if }x\in[0,1]-\mathbb{Q}
\end{cases}
\]
 is \emph{not }Riemann integrable
\item There exists an increasing sequence of functions $\{f_{n}\}$ that
are Riemann integrable on $[0,1]$ and satisfies 
\begin{enumerate}
\item $|f_{n}|\leq1$ on $[0,1]$ for all $n$, and
\item $f_{n}\to f$ pointwise on $[0,1]$ 
\end{enumerate}
\end{enumerate}

\subsection{The Lebesgue Integral of a Bounded Measurable Funcion over a Set
of Finite Measure}

~~~Let $E\in\mathcal{M}$ and $\varphi:E\to\mathbb{R}$ be a simple
function, and
\begin{equation}
\varphi:=\sum_{i=1}^{n}a_{i}\chi_{_{E_{i}}}\label{eq:canon}
\end{equation}

be the canonical representation, i.e. $a_{1},\dots,a_{n}$ are distinct
and $E_{i}=\{x\in E\mid\varphi(x)=a_{i}\}\in\mathcal{M}$. 

\begin{defn} Let $E\in\mathcal{M}$ with $\mu(E)<\infty$ and $\varphi:E\to\mathbb{R}$
be a simple function with canonical representation (\ref{eq:canon}).
Define the \textbf{Lebesgue integral }of $\varphi$ over $E$ 
\[
\int_{E}\varphi=\int_{E}\varphi(x)dx=\sum_{i=1}^{n}a_{i}\mu(E_{i}).
\]
\end{defn}

\begin{example} Let $f$ be the Dirichlet function. Then 
\begin{align*}
\int_{[0,1]}f & =1\cdot\mu\left([0,1]\cap\mathbb{Q}\right)+0\cdot\mu\left([0,1]-\mathbb{Q}\right)\\
 & =1\cdot0+0\cdot1\\
 & =0.
\end{align*}
\end{example}

\begin{lemma}\label{lemma1} Let $E\in\mathcal{M}$ with $\mu(E)<\infty$,
$E_{i}$ be disjoint measurable subsets of $E$ for $1\leq i\leq n$,
and
\[
\varphi=\sum_{i=1}^{n}a_{i}\chi_{_{E_{i}}}
\]
 where $a_{i}$ are not necessarily distinct. Then 
\[
\int_{E}\varphi=\sum_{i=1}^{n}a_{i}\mu(E_{i}).
\]
\end{lemma}

\begin{proof} Let 
\[
A_{a}:=\{x\in E\mid\varphi(x)=a\}=\bigcup_{a_{i}=a}E_{i}.
\]

Then by finite additivity of $\mu$, 
\[
\mu(A_{a})=\sum_{a_{i}=a}\mu(E_{i}),
\]

hence 
\begin{align*}
\int_{E}\varphi & =\sum_{a\in\mathbb{R}}a\mu(A_{a})\\
 & =\sum_{a\in\mathbb{R}}a\sum_{a_{i}=a}\mu(E_{i})\\
 & =\sum_{i=1}^{n}a_{i}\mu(E_{i}).
\end{align*}

\end{proof}

~~~For the Riemann Integral, we partition the domain of $f$. For
the Lebesgue Integral, we partition the range of $f$. 

\begin{prop}\label{prop2} Let $\varphi$ and $\psi$ be simple functions
defined on $E\in\mathcal{M}$ with $\mu(E)<\infty$. Then 
\begin{enumerate}
\item (Linearity) $\int_{E}(a\varphi+b\psi)=a\int_{E}\varphi+b\int_{E}\psi$
\item (Monotonicity) If $\varphi\leq\psi$ on $E$, then $\int_{E}\varphi\leq\int_{E}\psi$
\end{enumerate}
\end{prop}

\begin{proof} \hfill
\begin{enumerate}
\item Let 
\[
\varphi=\sum_{i=1}^{m}a_{i}\chi_{_{A_{i}}}\qquad\mbox{and}\qquad\psi=\sum_{j=1}^{n}b_{j}\chi_{_{B_{j}}}
\]
be canonical representations of $\varphi$ and $\psi$ respectively.
Then 
\[
\mathcal{F}=\{A_{i}\cap B_{j}\mid A_{i}\cap B_{j}\neq\emptyset,\,1\leq i\le m,\,1\leq j\leq n\}=\{E_{1},\dots,E_{N}\}
\]
 is a disjoint collection whose union is $E$, and we can write 
\[
\varphi=\sum_{k=1}^{N}c_{k}\chi_{_{E_{k}}}\qquad\mbox{and}\qquad\psi=\sum_{k=1}^{N}d_{k}\chi_{_{E_{k}}}
\]
Hence 
\[
a\varphi+b\psi=\sum_{k=1}^{N}(ac_{k}+bd_{k})\chi_{_{E_{k}}}
\]
and by Lemma~(\ref{lemma1}), 
\begin{align*}
\int_{E}(a\varphi+b\psi) & =\sum_{k=1}^{N}(ac_{k}+bd_{k})\mu(E_{k})\\
 & =a\int_{E}\varphi+b\int_{E}\psi.
\end{align*}
\item $\varphi\leq\psi$ implies $\eta:=\psi-\varphi\geq0$. By linearity,
\begin{align*}
\int_{E}\psi-\int_{E}\varphi & =\int_{E}(\psi-\varphi)\\
 & =\int_{E}\eta\\
 & \ge0.
\end{align*}
\end{enumerate}
\end{proof}

\begin{rem}\hfill
\begin{enumerate}
\item If $\varphi=\sum_{i=1}^{n}a_{i}\chi_{_{E_{i}}}$, where $E_{i}$ are
measurable with $\mu(E_{i})<\infty$ but the $E_{i}$ are \emph{not
necessarily disjoint}, then we still have Proposition~(\ref{prop2}).
\item We still have monotonicity if $\varphi\leq\psi$ almost everywhere. 
\end{enumerate}
\end{rem}

\begin{prop}\label{prop3} Let $f$ be defined and bounded on $E\in\mathcal{M}$
with $\mu(E)<\infty$. Then $f$ is measurable if and only if 
\[
\mbox{sup}\left\{ \int_{E}\varphi\mid\varphi\mbox{ is simple, }\varphi\leq f\right\} =\mbox{inf}\left\{ \int_{E}\psi\mid\psi\mbox{ is simple, }\psi\geq f\right\} .
\]
\end{prop}

\begin{proof} $(\implies)$ By the Simple Approximation Lemma, for
all $n\in\mathbb{N}$, there exists simple functions $\varphi_{n}$
and $\psi_{n}$ such that
\[
\varphi_{n}\leq f\leq\psi_{n}\mbox{ on }E\mbox{ and }0\leq\psi_{n}-\varphi_{n}\leq\frac{1}{n}.
\]
This implies
\begin{align*}
0 & \leq\int_{E}\psi_{n}-\int_{E}\varphi_{n}\\
 & =\int_{E}(\psi_{n}-\varphi_{n})\\
 & \leq\int_{E}\frac{1}{n}\\
 & =\frac{\mu(E)}{n}.
\end{align*}
 Letting $n\to\infty$, we have $\lim_{n\to\infty}\left(\int_{E}\psi_{n}-\int_{E}\varphi_{n}\right)=0$.
Thus, 
\[
0\leq\mbox{inf}\left\{ \int_{E}\psi\mid\psi\mbox{ is simple, }\psi\geq f\right\} -\mbox{sup}\left\{ \int_{E}\varphi\mid\varphi\mbox{ is simple, }\varphi\leq f\right\} \leq\int_{E}\psi_{n}-\int_{E}\varphi_{n}
\]

and letting $n\to\infty$ proves the stated equality. 

~~~$(\impliedby)$ Omitted.

\end{proof}

\begin{defn}\label{defn3} Let $f$ be a bounded measurable function
defined on a set $E\in\mathcal{M}$ with $\mu(E)<\infty$. The (\textbf{Lebesgue})
\textbf{integral }of $f$ over $E$ is defined as 
\[
\int_{E}f:=\mbox{sup}\left\{ \int_{E}\varphi\mid\varphi\mbox{ is simple, }\varphi\leq f\right\} =\mbox{inf}\left\{ \int_{E}\psi\mid\psi\mbox{ is simple, }\psi\geq f\right\} .
\]
\end{defn}

\begin{rem}\hfill 
\begin{enumerate}
\item If $E=[a,b]$, we denote $\int_{E}f=\int_{a}^{b}f$.
\item If $E_{1}\subseteq E$ and $E_{1}\in\mathcal{M}$, then $\int_{E_{1}}f=\int_{E}f\cdot\chi_{_{E_{i}}}.$ 
\end{enumerate}
\end{rem}

\begin{prop}\label{prop4} Let $f$ be a bounded function defined
on $[a,b]$. If $f$ is Riemann integrable, then $f$ is measurable
and 
\[
R\int_{a}^{b}f(x)dx=\int_{a}^{b}f(x)dx
\]

where the lefthand side is the Riemann integral and the righthand
side is the Lebesgue integral. \end{prop}

\begin{proof} Since step functions are simple functions, 
\begin{align*}
LR\int_{a}^{b}f(x)dx & \leq\mbox{sup}\left\{ \int_{a}^{b}\varphi\mid\varphi\mbox{ step function, }\varphi\leq f\right\} \\
 & \leq\mbox{sup}\left\{ \int_{a}^{b}\varphi\mid\varphi\mbox{ is simple, }\varphi\leq f\right\} \\
 & \mbox{\ensuremath{\leq}inf}\left\{ \int_{a}^{b}\psi\mid\psi\mbox{ is simple, }\psi\geq f\right\} \\
 & \mbox{\ensuremath{\leq}inf}\left\{ \int_{a}^{b}\psi\mid\psi\mbox{ step function, }\psi\geq f\right\} \\
 & \leq UR\int_{a}^{b}f(x)dx
\end{align*}

where $LR$ and $UR$ means lower and upper riemann integral respectively.
If $f$ is Riemann integrable, then all inequalities become equalities
and the result follows. \end{proof}

\subsection{$4.2$ continued}

\begin{prop}\label{prop7} Let $\{f_{n}\}$ be a sequence of bounded
measurable functions on $E\in\mathcal{M}$ with $\mu(E)<\infty$.
If $f_{n}\to f$ uniformly on $E$, then 
\[
\lim_{n\to\infty}\int_{E}f_{n}=\int_{E}f.
\]
\end{prop}

\begin{proof} Let $\varepsilon>0$. Then by uniform convergence,
there exists $N\in\mathbb{N}$ such that for all $n\geq N$, 
\[
|f_{n}-f|<\frac{\varepsilon}{\mu(E)+1}
\]

on $E$. Hence, 
\begin{align*}
\left|\int_{E}f_{n}-\int_{E}f\right| & =\left|\int_{E}(f_{n}-f)\right|\\
 & \leq\int_{E}\left|f_{n}-f\right|\\
 & \leq\int_{E}\frac{\varepsilon}{\mu(E)+1}\\
 & <\varepsilon.
\end{align*}
\end{proof}

\begin{rem} By Prop~(\ref{prop7}), uniform convergence cannot be
replaced by pointwise convergence. Here's a counterexample: Let $f_{n}:[0,1]\to\mathbb{R}$
be given by 
\[
f_{n}(x)=\begin{cases}
0 & \mbox{if }x\geq1/n\\
n & \mbox{if }x<1/n
\end{cases}
\]

Then $f_{n}\to0$ pointwise, however 
\begin{align*}
\lim_{n\to\infty}\int_{[0,1]}f_{n} & =\lim_{n\to\infty}1\\
 & =1\\
 & \neq0\\
 & =\int_{[0,1]}0.
\end{align*}
\end{rem}

\begin{theorem}\label{theorem8} (Bounded Convergence Theorem) Let
$\{f_{n}\}$ be a sequence of measurable functions defined on a measurable
set $E$ with $\mu(E)<\infty$. Suppose
\begin{enumerate}
\item There exists $M\geq0$ such that $|f_{n}(x)|\leq M$ for all $n\in\mathbb{N}$
and $x\in E$.
\item $f_{n}(x)\to f(x)$ pointwise for all $x\in E$.
\end{enumerate}
Then 
\[
\lim_{n\to\infty}\int_{E}f_{n}=\int_{E}f.
\]

\end{theorem}

\begin{proof} Let $\varepsilon>0$ be given. By Lemma $3.10$ (Lemma
to Erogoff's Theorem), there exists a measurable set $A\subseteq E$
and a number $N\in\mathbb{N}$ such that $\mu(E\setminus A)<\varepsilon$
and $\left|f_{n}(x)-f(x)\right|<\varepsilon$ for all $n\geq N$ and
$x\in A$. Thus, for all $n\geq N$
\begin{align*}
\left|\int_{E}f_{n}-\int_{E}f\right| & \leq\int_{E}\left|f_{n}-f\right|\\
 & =\int_{A}\left|f_{n}-f\right|+\int_{E\setminus A}\left|f_{n}-f\right|\\
 & \leq\int_{A}\varepsilon+\int_{E\setminus A}2M\\
 & =\varepsilon\mu(A)+2M\mu(E-A)\\
 & <\varepsilon(\mu(A)+2M)
\end{align*}

where $\mu(A)+2M$ is a constant, and we are done. \end{proof}

\subsection{Integrals of nonmeasurable functions. }

~~~We make the following notation (not standard): Denote $\mathcal{BF}$
to be the set of bounded measurable functions that vanish outside
a set of finite measure
\[
\mathcal{BF}:=\left\{ h:\mbox{dom}(h)\to\mathbb{R},\mbox{ }h\mbox{ is bounded measurable and }\mu(x\in\mbox{dom}(h)\mid h(x)\neq0)<\infty\right\} .
\]

Let $h\in\mathcal{BF}$ and $E_{0}=\{x\in\mbox{\mbox{dom}}(h)\mid h(x)\neq0\}$.
Define 
\[
\int_{\mbox{dom}(h)}h:=\int_{E}h.
\]

\begin{defn} Let $f$ be a nonnegative measurable function (not necessarily
bounded) defined on $E\in\mathcal{M}$ (not necessarily have finite
measure). Define 
\[
\int_{E}f:=\mbox{sup}\left\{ \int_{E}h\mid h\in\mathcal{BF},\,0\leq h\leq f\mbox{ on }E\right\} \in[0,\infty].
\]
 \end{defn}

\begin{rem} Note that 
\[
\int_{E}f\neq\inf_{g\geq f}\int_{E}g
\]
 because $f$ may be unbounded and hence $\int_{E}g$ need not be
defined. \end{rem}

\begin{prop}\label{prop9} (Chebychev's inequality) Let $f$ be a
nonnegative measurable function on $E\in\mathcal{M}$. Then 
\[
\mu\left\{ x\in E\mid f(x)\geq\lambda\right\} \leq\frac{1}{\lambda}\int_{E}f
\]
for all $\lambda>0$. \end{prop}

\begin{proof} Let $E_{0}:=\{x\in E\mid f(x)\geq\lambda\}$. Define
$E_{0}^{(n)}:=E_{0}\cap[-n,n]$ for all $n\geq1$. Then $\{E_{0}^{(n)}\}$
is ascending with 
\[
\bigcup_{n=1}^{\infty}E_{0}^{(n)}=E.
\]
Also, $\lambda\chi_{_{E_{0}^{(n)}}}\in\mathcal{BF}$ and $\lambda\chi_{_{E_{0}^{(n)}}}\leq f$
on $E$ for all $n\geq1$. Therefore 
\begin{align*}
\int_{E}f & \geq\int_{E}\lambda\chi_{_{E_{0}^{(n)}}}\\
 & =\lambda\int_{E}\chi_{_{E_{0}^{(n)}}}\\
 & =\lambda\mu\left(E_{0}^{(n)}\right),
\end{align*}
and letting $n\to\infty$, we obtain 
\[
\int_{E}f\geq\lambda\mu(E_{0}).
\]
\end{proof}

\begin{prop}\label{prop10} Let $E\in\mathcal{M}$ and $f:E\to[0,\infty]$
be measurable. Then 
\[
\int_{E}f=0\quad\iff\quad f=0\mbox{ almost everywhere on }E
\]
\end{prop}

\begin{proof} $(\impliedby)$ Let $h\in\mathcal{BF}$ such that $0\leq h\leq f$
on $E$. Then $h=0$ almost everywhere on $E$, so 
\[
\int_{E}f=\sup_{\substack{0\leq h\leq f\\
h\in\mathcal{BF}
}
}\int_{E}h=\sup_{\substack{0\leq h\leq f\\
h\in\mathcal{BF}
}
}0=0.
\]

$(\implies)$ Let $E_{+}:=\{x\in E\mid f(x)>0\}$ and $E_{+}^{(n)}:=\{x\in E\mid f(x)>1/n\}$
for all $n\in\mathbb{N}$. Then 
\[
E_{+}=\bigcup_{n=1}^{\infty}E_{+}^{(n)}.
\]

So by Chebychev's inequality, 
\[
\mu\left(E_{+}^{(n)}\right)\leq n\int_{E}f=0.
\]

Therefore 
\[
\mu(E_{+})\leq\sum_{n=1}^{\infty}\mu\left(E_{+}^{(n)}\right)=0.
\]

\end{proof}

\begin{theorem}\label{theorem11} Let $f$ and $g$ be nonnegative
measurable functions defined on $E\in\mathcal{M}$. Then 
\begin{enumerate}
\item (positive homogenity): $\int_{E}cf=c\int_{E}f$ for all constants
$c\ge0$. 
\item (additivity): $\int_{E}(f+g)=\int_{E}f+\int_{E}g$
\item (monotonicity): If $f\leq g$ almost everywhere, then $\int_{E}f\leq\int_{E}g$. 
\end{enumerate}
\end{theorem}

\begin{proof} Omitted. \end{proof}

\begin{theorem} (Additivity over domain of integration). Let $E\in\mathcal{M}$
and $f:E\to[0,\infty]$ be measurable. If $A$ and $B$ are disjoint
measurable subsets of $E$, then 
\[
\int_{A\cup B}f=\int_{A}f+\int_{B}f.
\]

In particular, if $E_{0}\subseteq E$ and $\mu(E_{0})=0$, then 
\[
\int_{E}f=\int_{E-E_{0}}f.
\]
\end{theorem}

\begin{proof} Omitted. \end{proof}

\begin{theorem} (Fatou's Lemma) Let $\{f_{n}\}$ be a sequence of
nonnegative measurable functions such that $f_{n}(x)\to f(x)$ pointwise
almost everywhere on a measurable set $E$. Then 
\[
\int_{E}f\leq\mbox{liminf}\left(\int_{E}f_{n}\right).
\]
\end{theorem}

\begin{proof} We first assume that $f_{n}(x)\to f(x)$ pointwise
everywhere on $E$. Let $h\in\mathcal{BF}$ such that $h\leq f$ and
$h$ vanishes outside $E_{h}\subseteq E$. Define $h_{n}(x):=\min\{h(x),f_{n}(x)\}$
for all $n\in\mathbb{N}$. Then $h_{n}(x)\to h(x)$ on $E_{h}$ since
$h$ is bounded, so are all $h_{n}$. Hence by the Bounded Convergence
Theorem, 
\begin{align*}
\int_{E}h & =\int_{E_{h}}h\\
 & =\lim_{n\to\infty}\int_{E_{h}}h_{n}\\
 & \leq\underline{\lim}_{n\to\infty}\int_{E}f_{n}.
\end{align*}

Thus, 
\[
\int_{E}f=\sup\int_{E}h\leq\underline{\lim}_{n\to\infty}\int_{E}f_{n}.
\]

Finally, if $f_{n}\to f$ a.e. let $F\subseteq E$ such that $\mu(F)=0$
and $f_{n}\to f$ everywhere on $E-F$. Then
\begin{align*}
\int_{E}f & =\int_{E-F}f+\int_{F}f\\
 & =\int_{E-F}f\\
 & \leq\underline{\lim}_{n\to\infty}\int_{E-F}f_{n}\\
 & \leq\underline{\lim}_{n\to\infty}\int_{E}f_{n}.
\end{align*}

This completes the proof.\end{proof}

\begin{rem} The example following Prop $8$ shows that strict inequality
may hold. \end{rem}

\begin{theorem}\label{theorem14} (Monotone Convergence Theorem) Let
$E\in\mathcal{M}$, $\{f_{n}\}$ be an increasing sequence of nonnegative
measurable functions on $E$, and let $f=\lim_{n\to\infty}f_{n}(x)$
a.e. on $E$. Then 
\[
\int_{E}f=\lim_{n\to\infty}\int_{E}f_{n}.
\]
\end{theorem}

\begin{proof} By Fataou's Lemma, 
\begin{equation}
\int_{E}f\leq\underline{\lim}_{n\to\infty}\int_{E}f_{n}.\label{eq:1}
\end{equation}

On the other hand, by Theorem~(\ref{theorem11}), we have 
\begin{equation}
\int_{E}f_{n}\leq\int_{E}f\qquad\implies\qquad\lim\int_{E}f_{n}\leq\int_{e}f\label{eq:2}
\end{equation}

Combining (\ref{eq:1}) and (\ref{eq:2}) proves the theorem. \end{proof}

\begin{cor}\label{cor15} (Interchanging $\int$ and $\sum$) Let
$\{u_{n}\}$ be a given sequence of nonnegative measurable functions
on $E\in\mathcal{M}$ and let $f:=\sum_{n=1}^{\infty}u_{n}$. Then
\[
\int_{E}f=\sum_{n=1}^{\infty}\int_{E}u_{n}
\]
 \end{cor}

\begin{proof} Let $f_{n}:=\sum_{k=1}^{n}u_{k}$. So $\{f_{n}\}$
is increasing. Hence by the Monotone Convergence Theorem,
\begin{align*}
\int_{E}f & =\lim_{n\to\infty}\int_{E}f_{n}\\
 & =\lim_{n\to\infty}\int_{E}\sum_{k=1}^{n}u_{k}\\
 & \leq\lim_{n\to\infty}\sum_{k=1}^{n}\int_{E}u_{k}\\
 & \leq\sum_{k=1}^{\infty}\int_{E}u_{k}.
\end{align*}
 \end{proof}

\begin{defn}\label{defn5} A nonnegative measure for $f$ is called
\textbf{integrable }over $E\in\mathcal{M}$ if 
\[
\int_{E}f<\infty.
\]
\end{defn}

\begin{prop}\label{prop16} Let $E\in\mathcal{M}$ and $f,g:E\to[0,\infty]$
be measurable functions. If $g\leq f$ on $E$ and $f$ is integrable
over $E$, then $g$ is integrable over $E$ and 
\[
\int_{E}(f-g)=\int_{E}f-\int_{E}g.
\]
\end{prop}

\begin{proof} 
\begin{align*}
\infty & >\int_{E}f\\
 & =\int_{E}((f-g)+g)\\
 & =\int_{E}(f-g)+\int_{E}g
\end{align*}
implies $\int_{E}g<\infty$ and $\int_{E}(f-g)-\int_{E}f-\int_{E}g$
implies $g$ is integrable and equality in the proposition holds.
\end{proof}

\begin{prop} Let $E\in\mathcal{M}$ and $f:E\to[0,\infty]$ be measurable.
If $f$ is integrable over $E$, then $f$ is finite a.e. on $E$.
\end{prop}

\begin{proof} Let $n\in\mathbb{N}$. Then by Chebychev's inequality,
we have 
\begin{align*}
\mu\left\{ x\in E\mid f(x)=\infty\right\}  & \leq\mu\left\{ x\in E\mid f(x)\geq n\right\} \\
 & \leq\frac{1}{n}\int_{E}f.
\end{align*}

Since $\int_{E}f$ is finite, letting $n\to\infty$ gives us $\mu\left\{ x\in E\mid f(x)=\infty\right\} =0$.
\end{proof}

\begin{theorem}\label{theorem18} (Beppo Levi's Theorem) Let $E\in\mathcal{M}$
and $f_{n}:E\to[0,\infty]$ be an increasing sequence of nonnegative
measurable functions. Suppose $\{\int_{E}f_{n}\}$ is bounded. Then
$\{f_{n}\}$ converges pointwise to a measurable function on $E$
that is finite a.e. on $E$ and 
\[
\lim_{n\to\infty}\int_{E}f_{n}=\int_{E}f<\infty.
\]
\end{theorem}

\begin{proof} Omitted. Let $f(x):=\lim_{n\to\infty}f_{n}(x)$ and
use Monotone Convergence Theorem. \end{proof}

\subsection{The General Lebesgue Integral}

~~~Recall that for $E\subseteq\mathbb{R}$ and $f:E\to[-\infty,\infty]$,
$f^{+}(x):=\max\{f(x),0\}$ is the positive part of $f$, and $f^{-}(x):=\max\{-f(x),0\}$
is the negative part of $f$. Then 
\[
f=f^{+}-f^{-},\qquad|f|=f^{+}+f^{-}
\]
\[
(-f)^{+}=f^{-}\qquad(-f)^{-}=f^{+}.
\]

\begin{prop}\label{prop19} Let $E\in\mathcal{M}$ and $f:E\to[-\infty,\infty]$
be measurable. Then $f^{+}$ and $f^{-}$ are integrable over $E$
if and only if $|f|$ is integrable over $E$. \end{prop}

\begin{proof} $(\implies)$ Assume $f^{+}$ and $f^{-}$ are integrable
over $E$. Then 
\begin{align*}
\int_{E}|f| & =\int_{E}(f^{+}+f^{-})\\
 & =\int_{E}f^{+}+\int_{E}f^{-}\\
 & <\infty
\end{align*}

implies $|f|$ is integrable over $E$. 

~~~$(\impliedby)$ Assume $|f|$ is integrable over $E$. Since
$0\leq f^{+}\leq|f|$ and $0\leq f^{-}\leq|f|$, we have $\int_{E}f^{+}\leq\int_{E}|f|<\infty$
and $\int_{E}f^{-}\leq\int_{E}|f|<\infty$ by monotonicity.\end{proof}

\begin{defn} Let $E\in\mathcal{M}$ and $f:E\to[-\infty,\infty]$
be measurable. Then $f$ is said to be \textbf{integrable }over $E$
if $|f|$ is integrable over $E$ (equivalently, $f^{+}$ and $f^{-}$
are integrable over $E$). In this case, 
\[
\int_{E}f:=\int_{E}f^{+}-\int_{E}f^{-}\in\mathbb{R}.
\]

\end{defn}

\begin{prop}\label{prop20} Let $E\in\mathcal{M}$ and $f:E\to[-\infty,\infty]$
be integrable over $E$. Then $f$ is finite a.e. on $E$ and 
\[
\int_{E}f=\int_{E-E_{0}}f
\]

if $E_{0}\subseteq E$ and $\mu(E_{0})=0$.\end{prop}

\begin{proof} Since $f$ is integrable, $|f|$ is integrable, thererefore
$|f|$ is finite a.e. by Proposition~(\ref{prop17}) which implies
$f$ is finite a.e. Also, 
\begin{align*}
\int_{E}f & =\int_{E}f^{+}-\int_{E}f^{-}\\
 & =\int_{E-E_{0}}f^{+}-\int_{E-E_{0}}f^{-}\\
 & =\int_{E-E_{0}}f.
\end{align*}
\end{proof}

\begin{prop}\label{prop21} (Integral Comparison Test). Let $E\in\mathcal{M}$
and $f:E\to[-\infty,\infty]$ be measurable. Suppose there exists
$g:E\to[0,\infty]$ such that $g$ is integrable over $E$ and $|f|\leq g$
on $E$. Then $f$ is integrable over $E$ and 
\begin{equation}
\left|\int_{E}f\right|\leq\int_{E}|f|.\label{eq:absconv}
\end{equation}
\end{prop}

\begin{rem}\label{rem1} (\ref{absconv}) holds for all integrable
$f$. (Replace $g$ by $|f|$). \end{rem}

\begin{proof} Since $\int_{E}|f|\leq\int_{E}g<\infty$ by monotonicity,
$f$ is integrable over $E$. Also 
\begin{align*}
\left|\int_{E}f\right| & =\left|\int_{E}f^{+}-\int_{E}f^{-}\right|\\
 & \leq\left|\int_{E}f^{+}\right|+\left|\int_{E}f^{-}\right|\\
 & \leq\int_{E}f^{+}+\int_{E}f^{-}\\
 & =\int_{E}|f|.
\end{align*}
\end{proof}

\begin{theorem}\label{theorem22} Let $E\in\mathcal{M}$ and $f,g:E\to[-\infty,\infty]$
be integrable over $E$. Then 
\begin{enumerate}
\item For all $c\in\mathbb{R}$, $cf$ is integrable over $E$ and $\int_{E}cf=c\int_{E}f$.
\item $f+g$ is integrable over $E$ and $\int_{E}(f+g)=\int_{E}f+\int_{E}g$. 
\item $f\leq g$ a.e. on $E$ implies $\int_{E}f\leq\int_{E}g$.
\end{enumerate}
\end{theorem}

\begin{proof} Omitted.\end{proof}

\begin{cor}\label{cor23} (Additivity over domain of integration).
Let $E\in\mathcal{M}$ and $f:E\to[-\infty,\infty]$ be integrable
over $E$. If $A,B\subseteq E$ are disjoint and measurable, then
\[
\int_{A\cup B}f=\int_{A}f+\int_{B}f.
\]
\end{cor}

\begin{proof} Omitted. \end{proof}

\begin{theorem}\label{theorem24} (Lebesgue Bounded Convergence Theorem).
Let $E\in\mathcal{M}$. Suppose 
\begin{enumerate}
\item $\{f_{n}\}$ is a sequence of measurable functions such that $f_{n}\to f$
pointwise a.e. on $E$; 
\item There exists an integrable function $g$ on $E$ such that $|f_{n}|\leq g$
pointwise everywhere on a.e. all $E$ for all $n\in\mathbb{N}$. 
\end{enumerate}
Then $f$ is integrable over $E$ and 
\[
\int_{E}f=\lim_{n\to\infty}\int_{E}f_{n}
\]

\end{theorem}

\begin{proof} Each $f_{n}$ is integrable since $|f_{n}|\leq g$
and $|f|\leq g$ implies integrability of $f$. Also, $|f_{n}|\leq g$
if and only if $-g\leq f_{n}\leq g$ if and only if $g-f_{n}\geq0$
and $g+f_{n}\geq0$. Applying Fatou's Lemma to $g-f_{n}$, we have
\begin{align*}
\int_{E}(g-f) & \leq\mbox{liminf}_{n\to\infty}\int_{E}(g-f_{n})
\end{align*}
and
\begin{align*}
\int_{E}g-\int_{E}f & \leq\mbox{liminf}_{n\to\infty}(\int_{E}g-\int_{E}f_{n})\\
 & \leq\int_{E}g-\mbox{liminf}_{n\to\infty}\int_{E}f_{n}
\end{align*}

implies 
\begin{equation}
\int_{E}f\geq\mbox{limsup}_{n\to\infty}\int_{E}f_{n}.\label{eq:limsup}
\end{equation}

Repeating the process for $g+f_{n}\geq0$, we get 
\begin{align}
\int_{E}(g+f) & \leq\mbox{liminf}_{n\to\infty}\int_{E}(g+f_{n})\nonumber \\
\int_{E}g+\int_{E}f & \leq\mbox{liminf}_{n\to\infty}\int_{E}g+\int_{E}f_{n}\nonumber \\
 & =\int_{E}g+\mbox{liminf}_{n\to\infty}\int_{E}f_{n}\nonumber \\
\int_{E}f & \leq\mbox{liminf}_{n\to\infty}\int_{E}f_{n}\label{eq:liminf}
\end{align}

Combining (\ref{eq:limsup}) and (\ref{eq:liminf}) proves the theorem.

\end{proof}

\begin{theorem}\label{theorem25} (General Lebesgue Dominated Convergence
Theorem). Let $E\in\mathcal{M}$. Suppose 
\begin{enumerate}
\item $\{g_{n}\}$ is a sequence of nonnegative measurable functions that
converge a.e. on $E$ to an integrable function $g$;
\item $\{f_{n}\}$ is a sequence of measurable functions such that $|f_{n}|\leq g_{n}$
and $f_{n}\to f$ a.e. on $E$. 
\item $\lim_{n\to\infty}\int_{E}g_{n}=\int_{E}g<\infty$.
\end{enumerate}
Then 
\[
\int_{E}f=\lim_{n\to\infty}\int_{E}f_{n}.
\]

\end{theorem}

\begin{proof} Exercise. Note that setting $g_{n}=g$ for all $n$
implies Lebesgue Dominated Convergence Theorem. \end{proof}

\subsection{Countable Additivity over Domain of Integration}

\begin{theorem}\label{theorem26} Let $E\in\mathcal{M}$, $f:E\to[-\infty,\infty]$
be integrable over $E$, and $\{E_{n}\}$ be a disjoint countable
collection of measurable subsets of $E$ with $\bigcup_{n=1}^{\infty}E_{n}=E$.
Then 
\[
\int_{E}f=\sum_{n=1}^{\infty}\int_{E_{n}}f.
\]
\end{theorem}

\begin{rem}\label{rem2} This generalizes Theorem~(\ref{cor23}).
\end{rem}

\begin{proof} Let $n\in\mathbb{N}$, $U_{n}:=\bigcup_{k=1}^{n}E_{k}$,
and $f_{n}:=f\cdot\chi_{_{U}}$. Then $f_{n}$ is a measurable function
on $E$, $|f_{n}|\leq|f|$ on $E$, and $f_{n}\to f$ pointwise on
$E$. Thus, by Lebesgue Dominated Convergence Theorem, 
\begin{align*}
\int_{E}f & =\lim_{n\to\infty}\int_{E}f_{n}\\
 & =\lim_{n\to\infty}\int_{E}f\cdot\chi_{_{U_{n}}}\\
 & =\lim_{n\to\infty}\int_{U_{n}}f\\
 & =\lim_{n\to\infty}\sum_{k=1}^{n}\int_{E_{k}}f\\
 & =\sum_{n=1}^{\infty}\int_{E_{k}}f.
\end{align*}
 \end{proof}

\begin{defn}\label{defn1} We say $\mathcal{I}$ is a \textbf{Vitali
cover }of $E\subseteq\mathbb{R}$ if for all $\varepsilon>0$ and
$x\in E$, there exists $I\in\mathcal{I}$ such that $x\in I$ and
$\ell(I)<\varepsilon$. \end{defn}

\begin{example}\label{example1} $\mathcal{I}_{1}:=\left\{ [x-1/n,x+1/n]\mid x\in[0,1],\,n\in\mathbb{N}\right\} $
is a Vitali cover of $[0,1]$, however $\tilde{\mathcal{I}}=\left\{ [x-1,x+1]\mid x\in[0,1]\right\} $
is not a Vitali cover of $[0,1]$. \end{example}

\begin{lemma}\label{lemma2.5} (Vitali Cover Lemma) Let $E\subseteq\mathbb{R}$
such that $\mu^{\star}(E)<\infty$ and let $\mathcal{I}$ be a Vitali
cover of $E$ consisting of intervals. Then given $\varepsilon>0$,
there exists finite disjoint subcollection $\{I_{1},\dots,I_{N}\}\in\mathcal{I}$
such that 
\[
\mu^{\star}\left(E\setminus\bigcup_{n=1}^{N}I_{n}\right)>\varepsilon.
\]

\end{lemma}

\begin{proof} Omitted. Selection process for the subcollection: Choose
$\{I_{n}\}$ inductively as follows. Let $I_{1}\in\mathcal{I}$ be
arbitrary. Assume $I_{1},\dots,I_{n}$ have been chosen. Let 
\[
k_{n}:=\sup\left\{ \ell(I)\mid I\in\mathcal{I},\,I\cap\left(\bigcup_{k=1}^{n}I_{k}\right)=\emptyset\right\} .
\]

If $E\subseteq\bigcup_{k=1}^{n}I_{k}$, we are done. Otherwise choose
$I_{n+1}\in\mathcal{I}$ such that $\ell(I_{n+1})>k_{n}/2$ and $I_{n+1}\cap\left(\bigcup_{k=1}^{n}I_{k}\right)=\emptyset$.
\end{proof}

\begin{defn}\label{defn2} Let $f:E\to\mathbb{R}$, $E\in\mathcal{M}$,
and $x\in E$. Define the \textbf{derivatives }$D^{+}$ and $D_{+}$
of $f$ at $x\in E$ by 
\[
D^{+}f(x)=\mbox{limsup}_{h\to0^{+}}\frac{f(x+h)-f(x)}{h}\quad\mbox{and}\quad D_{+}f(x)=\mbox{liminf}_{h\to0}\frac{f(x+h)-f(x)}{h}.
\]

Define the \textbf{derivatives }$D^{-}$ and $D_{-}$ of $f$ by
\[
D^{-}f(x)=\mbox{limsup}_{h\to0^{+}}\frac{f(x)-f(x-h)}{h}\quad\mbox{and}\quad D_{-}f(x)=\mbox{liminf}_{h\to0}\frac{f(x)-f(x-h)}{h}.
\]

We say $f$ is \textbf{differentiable }at $x$ if
\[
D^{+}f(x)=D_{+}f(x)=D^{-}f(x)=D_{-}f(x)\neq\pm\infty.
\]

In this case, the common value is called the \textbf{derivative }of
$f$ at $x$, and is denoted $f'(x)$. We say $f$ has a \textbf{right-hand
derivative} at $x$, denoted $f'(x_{+})$ if
\[
D^{+}f(x)=D_{+}f(x)\neq\pm\infty
\]

Similarly, we say $f$ has a \textbf{left-hand derivative}, denoted
$f^{1}(x_{-})$, if
\[
D^{-}f(x)=D_{-}f(x)\neq\pm\infty
\]

\end{defn}

\begin{prop}\label{prop2.8} If $f$ is continuous on $[a,b]$ and
one of its derivatives is everywhere greater than or equal to $0$
on $[a,b]$, then $f$ is nondecreasing on $[a,b]$. \end{prop}

\begin{theorem}\label{theorem27} Let $f$ be an increasing (or decreasing)
real-valued function on $[a,b]$. Then $f$ is differentiable a.e.
on $[a,b]$ and the derivative $f'$ is measurable. Moreover, 
\begin{enumerate}
\item $\int_{a}^{b}f'(x)dx\leq f(b)-f(a)$ if $f$ is increasing,
\item $\int_{a}^{b}f'(x)dx\geq f(b)-f(a)$ if $f$ is decreasing. 
\end{enumerate}
\end{theorem}

\begin{rem}\label{rem3} The inequality does not hold in general.
Here's a counterexample: Let $f(x):[0,1]\to\{0,1\}$ be given by 
\[
f(x)=\begin{cases}
0 & \mbox{if }x\in[0,1/2]\\
1 & \mbox{if }x\in(1/2,1].
\end{cases}
\]

Then $\int_{0}^{1}f'(x)dx=0$, however $f(1)-f(0)=1$. Even if $f$
is continuous, the reverse inequality need not hold. A counterexample
in this case is given by the Cantor function. \end{rem}

\textbf{Question }$1$: What kind of functions are differentiable
a.e?

\textbf{Question $2$}: When do these equalities hold? 

\subsection{Functions of bounded variation}

\begin{defn}\label{defn3} Let $f:[a,b]\to\mathbb{R}$ and let $\mathcal{P}=\{x_{i}\}_{i=0}^{k}$
be a partition of $[a,b]$ with $a=x_{0}<x_{1}<\cdots<x_{k}=b$. Define 
\begin{enumerate}
\item $p=p(\mathcal{P})=\sum_{i=1}^{k}(f(x_{i})-f(x_{i-1}))^{+}$, $n=n(\mathcal{P})=\sum_{i=1}^{k}(f(x_{i})-f(x_{i-1}))^{-}$,
and $t=t(\mathcal{P})=n+p=\sum_{i=1}^{k}|f(x_{i})-f(x_{i-1})|$.
\item $P=P_{a}^{b}(f)=\sup_{\mathcal{P}}(p),$ $N=N_{a}^{b}(f)=\sup_{\mathcal{P}}(n),$
and $T=T_{a}^{b}(f)=\sup_{\mathcal{P}}(t).$ $P,N$ and $T$ are the
\textbf{positive}, \textbf{negative}, and \textbf{total variations
}of $f$ over $[a,b]$.
\item If $T<\infty$, $f$ is said to be of \textbf{bounded variation }over
$[a,b]$ and is denoted $f\in BV([a,b])$. 
\end{enumerate}
\end{defn}

\begin{example}\label{example3} Let $f:[0,1]\to\mathbb{R}$ be defined
as
\[
f(x)=\begin{cases}
x\cos\left(\frac{\pi}{2x}\right) & \mbox{if }0<x\leq1\\
0 & \mbox{if }x=0.
\end{cases}\begin{aligned}x\cos\left(\frac{\pi}{2x}\right) & \mbox{if }0<x\leq1\\
0 & \mbox{if }x=0.
\end{aligned}
\]

Then $f$ is continuous but is \emph{not }of bounded variation. Proof:
Let $\mathcal{P}_{n}=\left\{ 0,\frac{1}{2n},\frac{1}{2n-1},\dots,\frac{1}{3},\frac{1}{2},1\right\} $.
Then $t(\mathcal{P}_{n})=V(f,\mathcal{P}_{n})=1+\frac{1}{2}+\cdots+\frac{1}{n}$
and so $T_{0}^{1}(f)=\infty$. \end{example}

\begin{rem}\label{rem4} \hfill
\begin{enumerate}
\item $f(b)-f(a)=p-n=\sum_{i=1}^{k}(f(x_{i})-f(x_{i-1}))$ is a telescoping
sum. In particular, if $f$ is increasing, then $TV(f)=f(b)-f(a)<\infty$. 
\item $\max\{P,N\}\leq T\leq P+N$ since
\begin{align*}
T & =\sup_{\mathcal{P}}(t)\\
 & =\sup_{\mathcal{P}}(p+n)\\
 & \le\sup_{\mathcal{P}}(p)+\sup_{\mathcal{P}}(n)\\
 & =P+N.
\end{align*}
\end{enumerate}
\end{rem}

\begin{lemma}\label{lemma4.5} Let $f\in BV([a,b])$. Then 
\begin{enumerate}
\item $f(b)-f(a)=P_{a}^{b}-N_{a}^{b}$
\item $T_{a}^{b}=P_{a}^{b}+N_{a}^{b}$.
\end{enumerate}
\end{lemma}

\begin{proof} By Remark~(\ref{rem4}) for all $\mathcal{P}$, $p=n+f(b)-f(a)$,
Taking the supremum over all $\mathcal{P}$, we have $P=N+f(b)-f(a)$
implies $f(b)-f(a)=P-N$. Also, $t=p+n=2p-(f(b)-f(a))$. Taking supremum
over all $\mathcal{P}$, we have $T=2P-(P-N)=P+N$. \end{proof}

\begin{theorem}\label{theorem5.5} (Lemma 5 and Jordan's Theorem)
$f\in BV([a,b])$ if and only if $f$ is the difference of two monotone
real-valued functions on $[a,b]$. \end{theorem}

\begin{proof} $(\implies)$ Assume $f\in BV([a,b])$ and define $g(x):=P_{a}^{x}(f)$
and $h(x):=N_{a}^{x}(f)$. Then $g$ and $h$ are increasing and real
valued. Since 
\[
0\le P_{a}^{x}\leq T_{a}^{x}\leq T_{a}^{b}<\infty,
\]
\[
0\le N_{a}^{x}\leq T_{a}^{x}\leq T_{a}^{b}<\infty,
\]

By Lemma~(\ref{lemma4.5}), $f(x)=g(x)-(h(x)-f(a))$. Therefore $f$
is a difference of two monotone functions. 

~~~$(\impliedby)$ Assume $f=g-h$ with $g$ and $h$ both monotone,
say decreasing (other cases are the same). Then for all $\mathcal{P}=\{x_{i}\}_{i=0}^{k}$,
\begin{align*}
t & =\sum_{i=1}^{n}|f(x_{i})-f(x_{i-1})|\\
 & \leq\sum_{i=1}^{n}|g(x_{i})-g(x_{i-1})|+\sum_{i=1}^{n}|h(x_{i})-h(x_{i-1})|\\
 & \leq g(a)-g(b)+h(a)-h(b)
\end{align*}

implies $T<\infty$ and $f\in BV([a,b])$. \end{proof}

\begin{cor}\label{cor6} If $f\in BV([a,b])$ then $f'(x)$ exists
a.e. \end{cor}

\begin{rem}\label{rem5}Note that $f(x)=\sin(1/x)\notin BV([0,1])$
but $f'(x)$ exists a.e. \end{rem}

\subsection{$6.4-6.5$ continued}

\begin{defn}\label{defn4} Let $f:[a,b]\to\mathbb{R}$. We say $f$
is \textbf{absolutely continuous }on $[a,b]$ if for all $\varepsilon>0$,
there exists $\delta>0$ such that 
\[
\sum_{i=1}^{n}|f(x_{i}')-f(x_{i})|<\varepsilon
\]

for every finite collection $\{(x_{i},x_{i}')\}$ of disjoint intervals
with 
\[
\sum_{i=1}^{n}|x_{i}'-x_{i}|<\delta.
\]

Notation: $f\in AC[a,b]$.\end{defn}

\begin{rem}\label{rem5} \hfill
\begin{enumerate}
\item Absolute continuity implies continuity.
\item Every indefinite integral is absolutely continuous. 
\item $f,g\in AC[a,b]$ implies $f\pm g\in AC[a,b]$. 
\item We say $f$ is \textbf{Lipschitz} on $[a,b]$ if there exists a consant
$C$ such that $|f(x)-f(y)|\leq C|x-y|$ for all $x,y\in[a,b]$. If
$f$ is Lipschitz, then $f$ is absolutely continuous. 
\item The Cantor function is continuous but belongs to $BV[0,1]\setminus AC[0,1]$. 
\end{enumerate}
\end{rem}

\begin{lemma}\label{lemma11} $f\in AC[a,b]$ implies $f\in BV[a,b]$.
\end{lemma}

\begin{cor}\label{cor12} $f\in AC[a,b]$ implies $f$ is differentiable
almost everywhere on $[a,b]$. \end{cor}

\begin{lemma}\label{lemma13} If $f\in AC[a,b]$ and $f'(x)=0$ almost
everywhere then $f$ is constant on $[a,b]$. \end{lemma}

\begin{proof} We need to show $f(c)=f(a)$ for all $c\in[a,b]$.
Let $E\subseteq(a,c)$ such that $m(E)=c-a$ and $f'(x)=0$ for all
$x\in E$. Let $\varepsilon>0$ and $\eta>0$ be arbitrary. For all
$x\in E$, there exists arbitrarily small intervals $[x,x+h]\subset(a,c)$
such that 
\[
|f(x+h)-f(x)|<\eta h
\]
since $f'(x)=0$. By the Vitali Covering Lemma, there exists finite
subcollection $\{[x_{k},y_{k}]\}$ of disjoint intervals such that
\[
m((a,c)\setminus\bigcup[x_{k},y_{k}])=m(E\setminus\bigcup[x_{k},y_{k}])<\delta,
\]
where $\delta>0$ is the positive number in the definition of $AC$.
Rearranging the $x_{k}$ if necessary so that $x_{k}\leq x_{k+1}$,
we have 
\[
a=y_{0}\leq x_{1}<y_{1}<x_{2}<\cdots<y_{n}<c=x_{n+1}.
\]

So 
\[
\sum_{k=0}^{n}|x_{k+1}-y_{k}|<\delta.
\]

By $AC$ we have 
\[
\sum_{k=0}^{n}|f(x_{k+1})-f(y_{k})|<\varepsilon.
\]

Also, 
\[
\sum_{k=1}^{n}|f(y_{k})-f(x_{k})|\leq\eta\sum(y_{k}-x_{k})\leq\eta(c-a).
\]

And by triangle's inequality, we have
\[
|f(c)-f(a)|\leq\sum_{k=0}^{n}|f(x_{k+1})-f(y_{k})|+\sum_{k=1}^{n}|f(y_{k})-f(x_{k})|.
\]

Since $\varepsilon$ and $\eta$ were arbitrary, $f(c)=f(a)$. 

\end{proof}

\begin{theorem}\label{theorem14} A function $F$ is an indefinite
integral if and only if $F$ is $AC$. \end{theorem}

\begin{proof} $(\implies)$ Let 
\[
F(x)=\int_{a}^{x}f(t)dt+F(a)
\]

$(\impliedby)$ Assume $F\in AC[a,b]\subseteq BV[a,b]$. Then $F(x)=F_{1}(x)-F_{2}(x)$
where $F_{1},F_{2}$ are monotone increasing. Thus, $F'$ exists almost
everywhere and $|F'(x)|\leq F_{1}'(x)-F_{2}'(x)$ implies
\begin{align*}
\int_{a}^{b}|F'(x)| & \leq\int_{a}^{b}F_{1}(x)+\int_{a}^{b}F_{2}'(x)\\
 & \leq F_{1}(b)-F_{1}(a)+F_{2}(b)-F_{2}(a),
\end{align*}

which implies $F'$ is integrable. Define 
\[
G(x):=\int_{a}^{x}F'(t)dt.
\]

Then $G\in AC[a,b]$, which implies $f=F-G\in AC[a,b]$. Therefore
$f'=F'-G'=0$, and by Lemma(\ref{lemma13}), this implies $f$ is
constant. So $F(x)=G(x)+f(a)$, or $F(x)=\int_{a}^{x}F'(x)dt+f(a)$.
\end{proof}

\begin{cor}\label{cor14.5} If $F\in AC[a,b]$, then $F(x)=\int_{a}^{x}F'(t)dt+F(a)$.
In particular, 
\[
\int_{a}^{b}F(t)dt=F(b)-F(a).
\]
\end{cor}

\section{Convex Functions}

\begin{defn}\label{defn5} We say $\varphi:(a,b)\to\mathbb{R}$ is
\textbf{convex }if 
\[
\varphi(\lambda x+(1-\lambda)y)\leq\lambda\varphi(x)+(1-\lambda)\varphi(y).
\]

for all $x,y\in(a,b)$ and $\lambda\in[0,1]$. \end{defn}

\begin{rem} The term $\lambda x+(1-\lambda)y$ is the line segment
from $x$ to $y$, but it can also be thought of as a weighted average
of $x$ and $y$. \end{rem}

\begin{prop}\label{prop15} If $\varphi$ is continuous on $(a,b)$
and if one derivative (say $D^{+}$) of $\varphi$ is nondecreasing,
then $\varphi$ is convex. In particular,
\begin{enumerate}
\item If $\varphi$ is differentiable on $(a,b)$ and $\varphi'$ is increasing,
then $\varphi$ is convex. 
\item Let $\varphi$ have a second derivative on $(a,b)$. Then $\varphi$
is convex on $(a,b)$ if and only if $\varphi''(x)\geq0$ for each
$x$ in $(a,b)$. 
\end{enumerate}
\end{prop}

\begin{example}\label{example4} The following functions are convex. 
\begin{enumerate}
\item $\varphi(x)=x^{p}$ on $(0,\infty)$ where $p>1$.
\item $\varphi(x)=e^{ax}$ on $(-\infty,\infty)$ 
\item $\varphi(x)=\log(1/x)$ on $(0,\infty)$. 
\end{enumerate}
\end{example}

\begin{theorem}\label{theoremjensen} (Jensen's Inequality) Let $\varphi$
be a convex function on $(-\infty,\infty)$ and let $f$ be integrable
over $[0,1]$. Then 
\[
\varphi\left(\int_{0}^{1}f(t)dt\right)\leq\int_{0}^{1}\varphi(f(t))dt.
\]
\end{theorem}

\begin{rem} The term $\int_{0}^{1}f(t)dt$ is the average value of
the function $f$ on $[0,1]$, and the term $\int_{0}^{1}\varphi(f(t))dt$
is the average value of the function $\varphi\circ f$ on $[0,1]$.
So this theorem is saying $\varphi$ of the average value of $f$
is less than or equal to the average value of $\varphi\circ f$. \end{rem}

\[
AC[a,b]
\]

\section{$L^{p}$-spaces}

~~~Throughout this section, we assume $0<p<\infty$. Why study
$L^{p}$-spaces. Here are three reasons:
\begin{enumerate}
\item Enlarging the space of integrable functions.
\item As base spaces for defining other important spaces.
\item Norm (measure distance between functions).
\end{enumerate}

\subsection{Normed linear spaces}

\begin{defn}\label{defn7.1} Let $E\in\mathcal{M}$. For $0<p<\infty$,
define 
\[
L^{p}=L^{p}(E):=\{f:E\to[-\infty,\infty]\mid f\mbox{ is measurable, }\int_{E}|f|^{p}<\infty\}.
\]

\end{defn}

\begin{rem}\label{rem7.1}\hfill
\begin{enumerate}
\item $L^{1}(E)$ is the space of integrable functions on $E$.
\item $1/x\notin L^{1}[0,1]$, but $1/x\in L^{p}[0,1]$ for all $0<p<1$. 
\end{enumerate}
\end{rem}

\begin{theorem}\label{theorem7.1} Let $E\in\mathcal{M}$. Then for
$0<p<\infty$, $L^{p}(E)$ is a vector space. \end{theorem}

\begin{proof} $0\in L^{p}(E)$. We'll only prove that $L^{p}$ is
closed under addition. We claim that
\begin{align*}
|f+g|^{p} & \leq2^{p}(|f|^{p}+|g|^{p})
\end{align*}
 for all $0<p<\infty$ and for all $x\in E$. By triangle inequality,
we have 
\[
|f+g|\leq|f|+|g|\leq\begin{cases}
2|f| & \mbox{if }|f|\geq|g|,\\
2|g| & \mbox{if }|g|\geq|f|.
\end{cases}
\]

In either case, this implies 
\[
\int_{E}|f+g|^{p}\leq2^{p}(|f|^{p}+|g|^{p})
\]

Therefore 
\[
\int_{E}|f+g|^{p}\leq2^{p}\left(\int_{E}|f|^{p}+\int_{E}|g|^{p}\right)<\infty.
\]

\end{proof}

\begin{defn}\label{defn7.2} A vector space $(X,\|\cdot\|)$ is a
\textbf{normed linear space }if $\|\cdot\|:X\to[0,\infty)$ satisfies 
\begin{enumerate}
\item $\|f\|=0$ if and only if $f=0$ for all $f\in X$. 
\item $\|cf\|=|c|\|f\|$ for all $f\in X$ and $c\in\mathbb{R}$.
\item $\|f+g\|\leq\|f\|+\|g\|$ for all $f$ and $g$ in $X$. 
\end{enumerate}
$\|\cdot\|$ is called a \textbf{norm}. \end{defn}

\begin{defn}\label{defn7.3} Let $E\in\mathcal{M}$ and $f\in L^{p}(E)$.
Define 
\[
\|f\|=\|f\|_{p}=\left(\int_{E}|f|^{p}\right)^{\frac{1}{p}}<\infty.
\]
 \end{defn}

\begin{rem}\label{rem7.3} It's easy to show that 
\begin{enumerate}
\item $\|f\|_{p}=0$ if and only if $f=0$ almost everywhere. 
\item $\|cf\|_{p}=|c|\|f\|_{p}$ for all $c$ in $\mathbb{R}$. 
\item $\|f+g\|_{p}\leq\|f\|_{p}+\|g\|_{p}$ for all $1\leq p<\infty$. 
\end{enumerate}
\end{rem}

~~~Let $E\in\mathcal{M}$. Define 
\[
\mathcal{F}=\mathcal{F}(E):=\{f:E\to[-\infty,\infty]\mid f\mbox{ is measurable and finite almost everywhere on }E\}.
\]

We will identify all functions in $\mathcal{F}$ that are equal to
$0$ almost everywhere by defining an equivalence relation $\sim$
on $\mathcal{F}$ as follows:
\[
f\sim g\mbox{ if and only if }f-g=0\mbox{ almost everywhere.}
\]

\begin{ex} Show $\sim$ is an equivalence relation on $\mathcal{F}$
and $(L^{p}/\sim,\|\cdot\|_{p})$ is a normed linear space. \end{ex}

~~~We'll denote $(L^{p}/\sim,\|\cdot\|_{p})$ simply by $(L^{p},\|\cdot\|_{p})$
and say that $L^{p}$ is a normed linear space. 

\begin{defn}\label{defn7.4} Let $E\in\mathcal{M}$. Define 
\begin{enumerate}
\item $L^{\infty}=L^{\infty}(E):=\{f:E\to[-\infty,\infty]\mid f\mbox{ is measurable and }f\mbox{ is bounded almost everywhere.}\}$
\item For $f\in L^{\infty}$, $\|f\|_{\infty}=\mbox{esssup}|f|=\mbox{inf}\left\{ M\geq0\mid\mu\left(\left\{ t\mid|f(t)|>M\right\} \right)=0\right\} $,
where $\mbox{esssup}(g):=\mbox{inf}\left\{ M\geq0\mid\mu\left(\left\{ t\mid|g(t)>M\right\} \right)=0\right\} $
\end{enumerate}
For $f$ to belong to $L^{\infty}(E)$, $\|f\|_{\infty}$ must be
less than $\infty$. 

\end{defn}

\begin{rem}\label{rem7.4} \hfill
\begin{enumerate}
\item $(L^{\infty},\|\cdot\|_{\infty})$ is a normed linear space (by identifying
functions that are equal almost everywhere). 
\item Let 
\[
f(x)=\begin{cases}
\frac{1}{x} & \mbox{if }x\in[0,1]\cap\mathbb{Q},\\
e^{x} & \mbox{if }x\in[0,1]\setminus\mathbb{Q}.
\end{cases}
\]
 then $\|f\|_{\infty}=e$, so $f\in L^{\infty}[0,1]$. 
\end{enumerate}
\end{rem}

\begin{theorem}\label{theorem7.2} Let $E\in\mathcal{M}$ with $\mu(E)<\infty$.
If $0<p_{1}\leq p_{2}\leq\infty$, then $L^{p_{1}}(E)=L^{p_{2}}(E)$.
\end{theorem}

\begin{rem}\label{rem7.5} This is \emph{not }true if $\mu(E)=\infty$:
Let $f(x)=1/x.$ Then $f$ does not belong to $L^{1}[1,\infty)$ but
$f$ does belong to $L^{p}[1,\infty)$ if $p>1$. \end{rem}

\begin{proof} Let $f\in L^{p_{2}}(E)$ and define 
\begin{align*}
E_{1} & :=\{x\in E\mid|f(x)|>1\},\\
E_{2} & :=\{x\in E\mid|f(x)|\leq1\}.
\end{align*}
Then $E=E_{1}\cup E_{2}$ is a disjoint union, so 
\begin{align*}
\int_{E}|f|^{p_{1}} & =\int_{E_{1}}|f|^{p_{1}}+\int_{E_{2}}|f|^{p_{2}}\\
 & \leq\int_{E_{1}}|f|^{p_{2}}+\int_{E_{2}}1\\
 & \leq\|f\|_{p_{2}}+\mu(E_{2})\\
 & <\infty.
\end{align*}
\end{proof}

\begin{defn}\label{defn4.5} \hfill
\begin{enumerate}
\item For $1\leq p<\infty$, define 
\[
\ell^{p}:=\left\{ \{a_{k}\}_{k=1}^{\infty}\mid a_{k}\in\mathbb{R},\mbox{ }\sum_{k=1}^{\infty}|a_{k}|^{p}<\infty\right\} .
\]
for $a=\{a_{k}\}_{k=1}^{\infty}\in\ell^{p}$, define 
\[
\|a\|_{p}=\left(\sum_{k=1}^{\infty}|a_{k}|^{p}\right)^{\frac{1}{p}}
\]
\item Define 
\[
\ell^{\infty}:=\left\{ \{a_{k}\}_{k=1}^{\infty}\mid a_{k}\in\mathbb{R},\mbox{ }\sup\left\{ |a_{k}|\mid k\geq1\right\} <\infty\right\} 
\]
for $a=\{a_{k}\}_{k=1}^{\infty}\in\ell^{\infty}$, define 
\[
\|a\|_{\infty}:=\sup\left\{ |a_{k}|\mid k\geq1\right\} .
\]
\end{enumerate}
\end{defn}

~~~$\ell^{p}$ and $\ell^{\infty}$ are normed linear spaces. Check
that $a=\{a_{k}\}_{k=1}^{\infty}=0$ if and only if $a_{k}=0$ for
all $k\geq1$. 

\begin{theorem}\label{theorem7.3} (Minkowski inequality) Let $1\leq p\leq\infty$
and $f,g\in L^{p}(E)$. Then $f+g\in L^{p}(E)$ and 
\[
\|f+g\|_{p}\leq\|f\|_{p}+\|g\|_{p}.
\]
\end{theorem}

\begin{theorem}\label{theorem7.4} (H�lder's inequality) Let $E\in\mathcal{M}$
and $p,q\in[1,\infty]$ such that 
\[
\frac{1}{p}+\frac{1}{q}=1,
\]
 if $f\in L^{p}(E)$ and $g\in L^{q}(E)$, then $fg\in L^{1}(E)$
and 
\[
\int_{E}|fg|\leq\|f\|_{p}\|g\|_{q}.
\]
 \end{theorem}

\begin{rem}\label{rem7.8} This is a generalization of Cauchy-Schwartz
inequality. \end{rem}
\end{document}
