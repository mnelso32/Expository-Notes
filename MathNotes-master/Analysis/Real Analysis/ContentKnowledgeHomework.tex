%% LyX 2.2.3 created this file.  For more info, see http://www.lyx.org/.
%% Do not edit unless you really know what you are doing.
\documentclass[12pt,english]{article}
\usepackage[osf]{mathpazo}
\renewcommand{\sfdefault}{lmss}
\renewcommand{\ttdefault}{lmtt}
\usepackage[T1]{fontenc}
\usepackage[latin9]{inputenc}
\usepackage[paperwidth=30cm,paperheight=35cm]{geometry}
\setlength{\parindent}{0bp}
\usepackage{amsmath}
\usepackage{amssymb}

\makeatletter
%%%%%%%%%%%%%%%%%%%%%%%%%%%%%% User specified LaTeX commands.
\usepackage{tikz}
\usetikzlibrary{matrix,arrows,decorations.pathmorphing}
\usetikzlibrary{shapes.geometric}
\usepackage{tikz-cd}
\usepackage{amsthm}
\theoremstyle{plain}
\newtheorem{theorem}{Theorem}[section]
\newtheorem{lemma}[theorem]{Lemma}
\newtheorem{prop}{Proposition}[section]
\newtheorem*{cor}{Corollary}
\theoremstyle{definition}
\newtheorem{defn}{Definition}[section]
\newtheorem{ex}{Exercise} 
\newtheorem{example}{Example}[section]
\theoremstyle{remark}
\newtheorem*{rem}{Remark}
\newtheorem*{note}{Note}
\newtheorem{case}{Case}
\usepackage{graphicx}
\usepackage{amssymb}
\usepackage{tikz-cd}
\usetikzlibrary{calc,arrows,decorations.pathreplacing}
\tikzset{mydot/.style={circle,fill,inner sep=1.5pt},
commutative diagrams/.cd,
  arrow style=tikz,
  diagrams={>=latex},
}

\usepackage{babel}
\usepackage{hyperref}
\hypersetup{
    colorlinks,
    citecolor=black,
    filecolor=black,
    linkcolor=black,
    urlcolor=black
}
\usepackage{pgfplots}
\usetikzlibrary{decorations.markings}
\pgfplotsset{compat=1.9}

\makeatother

\usepackage{babel}
\begin{document}

\title{Homework Assignment 5}

\author{Michael Nelson}

\maketitle
$(1):$ Let $E\subseteq\mathbb{R}$ be measurable and $g$ be a nonnegative
function that is integrable over $E$. Suppose $\{f_{n}\}$ is a sequence
of measurable functions on $E$ such that for each $n$, $|f_{n}|\leq g$
almost everywhere on $E$. Show that 
\begin{equation}
\int_{E}\mbox{liminf}(f_{n})\leq\mbox{liminf}\left(\int_{E}f_{n}\right)\leq\mbox{limsup}\left(\int_{E}f_{n}\right)\leq\int_{E}\mbox{limsup}(f_{n}).\label{eq:inequality}
\end{equation}

\begin{proof} By possibly excising from $E$ a set of measure zero,
we may assume $|f_{n}|\leq g$ everywhere on $E$. Let $h_{n}$ be
the function given by $h_{n}(x)=\mbox{inf}\{f_{k}(x)\mid k\geq n\}$.
Then $\{h_{n}\}$ is a sequence of measurable functions on $E$ which
is dominated by the integral function $g$. Also $\{h_{n}\}\to\mbox{liminf}(f_{n})$
pointwise on $E$, so by the Lebesgue Dominated Convergence Theorem,
we have 
\[
\lim_{n\to\infty}\int_{E}h_{n}=\int_{E}\mbox{liminf}(f_{n})
\]

Since $h_{n}\leq f_{n}$, we have $\int_{E}h_{n}\leq\int_{E}f_{n}$
by monotonicity. Therefore 
\begin{align*}
\mbox{liminf}\left(\int_{E}f_{n}\right) & \geq\mbox{liminf}\left(\int_{E}h_{n}\right)\\
 & =\lim_{n\to\infty}\left(\int_{E}h_{n}\right)\\
 & =\int_{E}\mbox{liminf}(f_{n}).
\end{align*}
 Where $\mbox{liminf}\left(\int_{E}h_{n}\right)=\lim_{n\to\infty}\left(\int_{E}h_{n}\right)$
since $\left\{ \int_{E}h_{n}\right\} $ is a monotone increasing sequence. 

~~~Similarly, let $p_{n}$ be the function given by $p_{n}(x)=\mbox{sup}\{f_{k}(x)\mid k\geq n\}$.
Then $\{p_{n}\}$ is a sequence of measurable functions on $E$ which
is dominated by the integral function $g$. Also $\{p_{n}\}\to\mbox{limsup}(f_{n})$
pointwise on $E$, so by the Lebesgue Dominated Convergence Theorem,
we have 
\[
\lim_{n\to\infty}\int_{E}p_{n}=\int_{E}\mbox{limsup}(f_{n})
\]

Since $p_{n}\geq f_{n}$, we have $\int_{E}p_{n}\geq\int_{E}f_{n}$
by monotonicity. Therefore 
\begin{align*}
\mbox{limsup}\left(\int_{E}f_{n}\right) & \leq\mbox{limsup}\left(\int_{E}p_{n}\right)\\
 & =\lim_{n\to\infty}\left(\int_{E}p_{n}\right)\\
 & =\int_{E}\mbox{limsup}(f_{n}).
\end{align*}
 Where $\mbox{liminf}\left(\int_{E}p_{n}\right)=\lim_{n\to\infty}\left(\int_{E}p_{n}\right)$
since $\left\{ \int_{E}p_{n}\right\} $ is a monotone decreasing sequence. 

~~~The middle inequality in (\ref{eq:inequality}) follows from
the properties of $\mbox{liminf}$ and $\mbox{limsup}$. 

\end{proof}

\hfill

$(2):$ Prove the following Proposition in the lecture notes: If $f$
is continuous on $[a,b)$ and one of its derivates (say $D^{+}$)
is everywhere nonnegative on $(a,b)$, then $f$ is nondecreasing
on $[a,b]$; i.e., $f(x)\le f(y)$ for all $x,y\in[a,b]$ with $x\le y$.
(You only need to consider the derivate $D^{+}$.) Hint: First show
this for a function $g$ for which $D^{+}g\ge\epsilon>0$. Apply this
to the function $g(x)=f(x)+\epsilon x$.

\begin{proof} Let $g:[a,b]\to\mathbb{R}$ denote the function $g(x)=f(x)+\varepsilon x$.
For all $x\in[a,b]$, 
\begin{align*}
(D^{+}g)(x) & =\mbox{limsup}\limits _{h\to0^{+}}\left(\frac{g(x+h)-b(x)}{h}\right)\\
 & =\mbox{limsup}\limits _{h\to0^{+}}\left(\frac{f(x+h)-f(x)}{h}+\frac{\varepsilon(x+h)-\varepsilon x}{h}\right)\\
 & =\mbox{limsup}\limits _{h\to0^{+}}\left(\frac{f(x+h)-f(x)}{h}+\varepsilon\right)\\
 & =\mbox{limsup}\limits _{h\to0^{+}}\left(\frac{f(x+h)-f(x)}{h}\right)+\varepsilon\\
 & =(D^{+}f)(x)+\varepsilon\\
 & \geq\varepsilon,
\end{align*}

implies $D^{+}g\geq\varepsilon$. 

~~~Now we show $g$ is nondecreasing. To do this, we show for all
$c\in(a,b)$ that $g(x)\geq g(c)$ for all $x\in(c,b)$. To obtain
a contradiction, suppose there exists a $d\in(c,b)$ such that $g(d)<g(c)$.
Since $g$ is continuous, there exists a neighborhood $(d-\delta,d+\delta)\subset(c,b)$
such that $g(x)<g(c)$ for all $x\in(d-\delta,d+\delta)$. Define
$e:=\inf\{y\mid y\in(c,d),\mbox{ }g(x)<g(c),\mbox{ for all }x\in(y,d]\}$.
Continuity of $g$ implies $g(e)>g(c)$. Also, if $g(e)<g(c)$, then
there exists a neighborhood $(e-\delta,e+\delta)\subset(c,b)$ such
that $g(x)<g(c)$ for all $x\in(e-\delta,e+\delta)$, but this contradicts
the choice of $e$ as the infinum. Therefore $g(e)=g(c)$. Since $g(x)<g(c)=g(e)$
for all $x\in(e,d)$, we have
\[
\frac{g(x)-g(e)}{x-e}<0\mbox{ for all }x\in[e,d].
\]

Letting $x\to e^{+}$ gives 
\[
(D^{+}g)(e)=\mbox{limsup}\limits _{x\to e^{+}}\left(\frac{g(x)-g(e)}{x-e}\right)\leq0.
\]

This contradicts the fact that $D^{+}g\geq\varepsilon>0.$ Therefore
$g(x)\geq g(c)$ for all $x\in(c,b)$, which implies $g$ is nondecreasing. 

~~~Let $c,d\in[a,b]$ such that $c<d$. Then
\begin{align*}
f(c)+\varepsilon c & =g(c)\\
 & \leq g(d)\\
 & =f(d)+\varepsilon d.
\end{align*}

Letting $\varepsilon\to0^{+}$ gives $f(c)\leq f(d)$. Therefore $f$
is nondecreasing on $[a,b]$. \end{proof}
\end{document}
