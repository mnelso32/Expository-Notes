%% LyX 2.2.3 created this file.  For more info, see http://www.lyx.org/.
%% Do not edit unless you really know what you are doing.
\documentclass[12pt,english]{article}
\usepackage[osf]{mathpazo}
\renewcommand{\sfdefault}{lmss}
\renewcommand{\ttdefault}{lmtt}
\usepackage[T1]{fontenc}
\usepackage[latin9]{inputenc}
\usepackage[paperwidth=30cm,paperheight=35cm]{geometry}
\geometry{verbose,tmargin=3cm,bmargin=3cm}
\setlength{\parindent}{0bp}
\usepackage{amsmath}
\usepackage{amssymb}

\makeatletter
%%%%%%%%%%%%%%%%%%%%%%%%%%%%%% User specified LaTeX commands.
\usepackage{tikz}
\usetikzlibrary{matrix,arrows,decorations.pathmorphing}
\usetikzlibrary{shapes.geometric}
\usepackage{tikz-cd}
\usepackage{amsthm}
\theoremstyle{plain}
\newtheorem{theorem}{Theorem}[section]
\newtheorem{lemma}[theorem]{Lemma}
\newtheorem{prop}{Proposition}[section]
\newtheorem*{cor}{Corollary}
\theoremstyle{definition}
\newtheorem{defn}{Definition}[section]
\newtheorem{ex}{Exercise} 
\newtheorem{example}{Example}[section]
\theoremstyle{remark}
\newtheorem*{rem}{Remark}
\newtheorem*{note}{Note}
\newtheorem{case}{Case}
\usepackage{graphicx}
\usepackage{amssymb}
\usepackage{tikz-cd}
\usetikzlibrary{calc,arrows,decorations.pathreplacing}
\tikzset{mydot/.style={circle,fill,inner sep=1.5pt},
commutative diagrams/.cd,
  arrow style=tikz,
  diagrams={>=latex},
}

\usepackage{babel}
\usepackage{hyperref}
\hypersetup{
    colorlinks,
    citecolor=blue,
    filecolor=blue,
    linkcolor=blue,
    urlcolor=blue
}
\usepackage{pgfplots}
\usetikzlibrary{decorations.markings}
\pgfplotsset{compat=1.9}

\makeatother

\usepackage{babel}
\begin{document}

\title{Complex Analysis}

\maketitle
\newpage{}

\tableofcontents{}

\newpage{}

\section{Preliminaries to Complex Analysis}

We denote $\mathbb{C}$ to be the field of complex numbers. Every
complex number $z\in\mathbb{C}$ can be written as $z=x+iy$ for unique
$x,y\in\mathbb{R}$. Given $z=x+iy\in\mathbb{C}$, we denote $\bar{z}$
to be $\bar{z}=x-iy$. 
\begin{enumerate}
\item $\bar{z}=z$
\item $|z|=\sqrt{z\bar{z}}$
\item $|z_{1}z_{2}|=|z_{1}||z_{2}|$
\item $|z_{1}+z_{2}|\leq|z_{1}|+|z_{2}|$
\item $||z_{1}|-|z_{2}||\leq|z_{1}-z_{2}|$
\item $|\text{Re}(z_{1})-\text{Re}(z_{2})|\leq|z_{1}-z_{2}|$
\item $|\text{Im}(z_{1})-\text{Im}(z_{2})|\leq|z_{1}-z_{2}|$
\end{enumerate}
Every complex number $z\neq0$ can be written as $z=re^{i\theta}$,
where $r=|z|$ and $\theta=\text{Arg}(z)\in[0,2\pi)$. 
\begin{enumerate}
\item $\text{Arg}(z_{1}z_{2})=\text{Arg}(z_{1})+\text{Arg}(z_{2})$
\item $\text{Arg}(z^{n})=n\text{Arg}(z)$
\item Each $z\neq0$ has precisely $n$ distinct $n$th roots. 
\end{enumerate}
\begin{defn}\label{defnopendisc} The \textbf{open disc} centered
at $z_{0}$ is $D_{r}(z_{0})=\{z\in\mathbb{C}\mid|z-z_{0}|<r\}$.
\end{defn}

\begin{defn}\label{defnlimit} Let $\{z_{n}\}:=\{z_{n}\}_{n\in\mathbb{N}}$
be a sequence in $\mathbb{C}$. We say $\{z_{n}\}$ \textbf{converges
}to limit $\ell$ if for all $\varepsilon>0$, there exists $N\in\mathbb{N}$
such that $|z_{n}-\ell|<\varepsilon$ for all $n\geq N$. \end{defn}

\begin{rem} $z_{n}\to\ell$ if and only if $\text{Re}(z_{n})\to\text{Re}(\ell)$
and $\text{Im}(z_{n})\to\text{Im}(\ell)$.\end{rem}

\begin{prop} Suppose $z_{n}\to\ell$ and $w_{n}\to\ell'$. Then 
\begin{enumerate}
\item $az_{n}+bw_{n}\to a\ell+b\ell'$
\item $z_{n}w_{n}\to\ell\ell'$ 
\item $|z_{n}|\to|\ell|$ 
\end{enumerate}
\end{prop}

\begin{defn}\label{defn1.2} A complex sequence $\{z_{n}\}$ is \textbf{Cauchy
}if for all $\varepsilon>0$, there exists $N\in\mathbb{N}$ such
that $|z_{n}-z_{m}|<\varepsilon$ for all $n,m\geq N$.\end{defn}

\begin{theorem}\label{theoremcauchyconvergence} The following hold\hfill
\begin{enumerate}
\item In a complete metric space, a sequence is convergent if and only if
it is cauchy.
\item If $\{z_{n}\}$ is a bounded sequence, then there exists a convergent
subsequence of $\{z_{n}\}$.
\end{enumerate}
\end{theorem}

\begin{defn} An \textbf{open cover }of a subset $\Omega\subset\mathbb{C}$
is a family of open sets $\{U_{\alpha}\}$ such that 
\[
\Omega\subset\bigcup_{\alpha}U_{\alpha}.
\]

A set $\Omega\subset\mathbb{C}$ is said to be \textbf{compact }if
every open cover of $\Omega$ has a finite subcover. \end{defn}

\begin{theorem}\label{theorem1.3}\hfill
\begin{enumerate}
\item A set $\Omega\subset\mathbb{C}$ is compact if and only it is closed
and bounded.
\item A set $\Omega\subset\mathbb{C}$ is compact if and only if every sequence
$\{z_{n}\}\subset\Omega$ has a convergent subsequence.
\end{enumerate}
\end{theorem}

\begin{defn}\label{defndiameter} The \textbf{diameter }of a set $\Omega$
is $\text{diam}(\Omega)=\text{sup}\{|x-y|\mid x,y\in\Omega\}$. \end{defn}

\begin{prop}\label{propsequenceofcompactsets} If $\Omega_{1}\supset\Omega_{2}\supset\cdots$
is a sequence of nonempty compact sets in $\mathbb{C}$ with the property
that $\text{diam}(\Omega_{i})\to0$ as $i\to\infty$, then there is
a unique point $\omega\in\mathbb{C}$ such that $\omega\in\Omega_{i}$
for all $i$. \end{prop}

\begin{proof} If $\bigcap_{i}\Omega_{i}$ is empty, then $\bigcap_{i}(\mathbb{C}\backslash\Omega_{i})\cap\Omega_{1}$
gives an open cover of $\Omega_{1}$ with no finite subcover. Therefore
$\bigcap_{i}\Omega_{i}$ is nonempty. Now suppose $\omega,\omega'\in\bigcap_{i}\Omega_{i}$.
Then $|\omega-\omega'|=0$ since $\text{diam}(\Omega_{i})\to0$. Therefore
$\omega=\omega'$. \end{proof}

\section{Holomorphic Functions and Power Series}

\begin{defn} Given a sequence $\{z_{1},z_{2},\dots\}$ in $\mathbb{C}$,
the \textbf{series} $\sum z_{n}$ is defined to be the limit of the
partial sums $s_{N}:=\sum_{n=1}^{N}z_{n}$ as $N\to\infty$. \end{defn}

~~~In any metric space, if a sequence is convergent, then it is
cauchy. In a complete metric space, a sequence is convergent if and
only if it is cauchy. Since $\mathbb{C}$ is complete under $|\cdot|$,
we will use the cauchy condition quite a lot to determine whether
a sequence is convergent or not. In terms of the partial sums $s_{n}$,
the cauchy condition says this: The sequence $\{s_{1},s_{2},\dots\}$
converges to a limit if and only if for every $\varepsilon>0$, there
is an $N$, such that 
\[
|s_{n}-s_{m}|=\left|\sum_{k=m+1}^{n}z_{k}\right|<\varepsilon
\]

for all $n\ge m\geq N$. To summarize this in a more compact form,
the series $\sum z_{n}$ converges if and only if the sum
\[
\left|\sum_{n=M}^{N}z_{n}\right|
\]

tends to $0$ as $M,N\to\infty$.

\begin{defn} We say that the series $\sum z_{n}$ is \textbf{absolutely
convergent }if the series $\sum|z_{n}|$ is converges. \end{defn}

Using the triangle inequality together with the cauchy condition,
it's easy to show that absolute convergence implies convergence:
\[
\left|\sum_{n=M}^{N}z_{n}\right|\le\sum_{n=M}^{N}|z_{n}|,
\]

and the right side tends to $0$ as $M,N\to\infty$. 

\begin{defn}A \textbf{power series }in $z$ is a series of the form
$\sum_{k=0}^{\infty}a_{k}z^{k}$. \end{defn}

\begin{lemma} A power series $\sum c_{n}z^{n}$ centered at the origin
that converges at $z_{0}\neq0$ is absolutely convergent at every
$z$ with $|z|<|z_{0}|$. \end{lemma}

\begin{proof} Let $r=|z/z_{0}|<1$ and $|c_{n}z_{0}^{n}|\leq B$
for some bound $B$. Then 
\begin{align*}
\sum|c_{n}z^{n}| & =\sum|c_{n}z_{0}^{n}||z/z_{0}|^{n}\\
 & \leq\sum Br^{n}\\
 & =B/(1-r)\\
 & <\infty.
\end{align*}

\end{proof}

\begin{lemma} Let $a_{n}\geq0$ and assume the series $\sum a_{n}$
converges, say to $S$. For every permutaiton $\pi$ of the index
set, the series $\sum a_{\pi(n)}$ also converges to $S$. \end{lemma}

\begin{proof} Choose $\varepsilon>0$. For all large $N$, say $N\geq M$
(where $M$ depends on $\varepsilon$), 
\[
S-\varepsilon\leq\sum_{n=1}^{N}a_{n}\leq S+\varepsilon.
\]

The permutation $\pi$ takes on all values $1,2,\dots,M$ among some
initial segment of the positive integers, say 
\[
\{1,2,\dots,M\}\subset\{\pi(1),\pi(2),\dots,\pi(K)\}
\]

for some $K$. For $N\geq K$, the set $\{a_{\pi(1)},\dots,a_{\pi(N)}\}$
contains $\{a_{1},\dots,a_{M}\}$. Let $J$ be the maximal value of
$\pi(n)$ for $n\le N$. So for $N\geq K$, 
\begin{equation}
S-\varepsilon\le\sum_{n=1}^{M}a_{n}\le\sum_{n=1}^{N}a_{\pi(n)}\le\sum_{n=1}^{J}a_{n}\le S+\varepsilon.\label{eq:sumsquash}
\end{equation}

So for every $\varepsilon$, $\sum_{n=1}^{N}a_{\pi(n)}$ is within
$\varepsilon$ of $S$ for all large $N$. Therefore $\sum a_{\pi(n)}=S$.

\end{proof}

~~~To get a feel for how the proof above works, let's take the
case where our permutation of the sum $\sum z_{n}$ starts out as
\[
z_{12}+z_{2}+z_{4}+z_{3}+z_{6}+z_{8}+z_{5}+z_{10}+z_{1}+z_{7}+z_{14}+z_{16}+\cdots
\]

and let's assume that we chose $\varepsilon>0$ in such a way that
we can take $M=5$. In this case, $K=9$ since the first nine terms
in our rearranged sum contain the first five terms in our original
sum. Then taking $N=12$, (\ref{eq:sumsquash}) tells us 
\[
S-\varepsilon\le\sum_{n=1}^{5}z_{n}\le z_{12}+z_{2}+z_{4}+z_{3}+z_{6}+z_{8}+z_{5}+z_{10}+z_{1}+z_{7}+z_{14}+z_{16}\le\sum_{n=1}^{16}z_{n}\le S+\varepsilon.
\]

Note that the reason we have 
\[
\sum_{n=1}^{5}z_{n}\le z_{12}+z_{2}+z_{4}+z_{3}+z_{6}+z_{8}+z_{5}+z_{10}+z_{1}+z_{7}+z_{14}+z_{16}\le\sum_{i=1}^{16}z_{i}
\]

is because all of the $z_{i}$ are positive. 

\begin{theorem} If $f(z)=\sum c_{n}z^{n}$ converges at the point
$z_{0}$, then $f(z_{0})$ is the limit of $f(z)$ as $z\to z_{0}$
along a radial path from the origin. In particular, if $\sum c_{n}$
converges, then 
\[
\lim_{x\to1^{-}}\sum c_{n}x^{n}=\sum c_{n}
\]

\end{theorem}

\begin{proof} The case of a series at $z_{0}$ is easily reduced
to the case $z_{0}=1$ by a scaling and a rotation. So we assume $z_{0}=1$.
Since $\sum c_{n}z^{n}$ converges at $z=1$, the series converges
on the open unit disc. Let $b_{n}=c_{0}+\cdots+c_{n}$, $b=\lim_{n\to\infty}b_{n}$,
$0<x<1$. Then 
\[
\sum_{n=0}^{N}c_{n}x^{n}=\sum_{n=0}^{N}b_{n}x^{n}-x\sum_{n=0}^{N-1}b_{n}x^{n}=(1-x)\sum_{n=0}^{N-1}b_{n}x^{n}+b_{N}x^{N}.
\]

Let $N\to\infty$. Since $b_{N}\to b$ and $x^{N}\to0$, we get 
\[
\sum_{n\geq0}c_{n}x^{n}=(1-x)\sum_{n\geq0}b_{n}x^{n}.
\]

Since $\sum c_{n}x^{n}-b=(1-x)\sum(b_{n}-b)x^{n}$, we choose $\varepsilon>0$
and then $M$ so that $|b_{n}-b|\leq\varepsilon$ for $n>M$. Then
\[
\left|\sum_{n\geq0}c_{n}x^{n}-b\right|\leq(1-x)\sum_{n=0}^{M}\left|b_{n}-b\right|x^{n}+\varepsilon\leqq(1-x)\sum_{n=0}^{M}|b_{n}-b|+\varepsilon.
\]

For $|x-1|$ small enough, the first term on the right side can be
made $\leq\varepsilon$. \end{proof}

\begin{defn}\label{defnuniformconvergence} Let $f_{n}:D\to\mathbb{C}$
where $D\subseteq\mathbb{C}$. Then
\begin{enumerate}
\item The sequence $\{f_{n}\}$ converges \textbf{pointwise }on\textbf{
}$D$ to a function $f$ if $f_{n}(z)\to f(z)$ for all $z\in D$.
\item The sequence $\{f_{n}\}$ converges \textbf{uniformly }on $D$ to
a function $f$ if for all $\varepsilon>0$, there exists $N\in\mathbb{N}$
such that for all $n\geq N$ and for all $z\in D$, we have $|f_{n}(z)-f(z)|<\varepsilon$. 
\item The sequence $\{f_{n}\}$ is \textbf{uniformly cauchy }on $D$ if
given any $\varepsilon>0$, there exists $N\in\mathbb{N}$ such that
for all $m,n\geq N$ and for all $z\in D$, we have $|f_{n}(z)-f_{m}(z)|<\varepsilon$.
\item The series $\sum_{n=1}^{\infty}f_{n}$ converges \textbf{uniformly
}on $D$ to a function $f$ if the sequence of partial sums $\left\{ \sum_{n=1}^{N}f_{n}\right\} $
converges uniformly on $D$ to $f$.
\item The series $\sum_{n=1}^{\infty}f_{n}$ converges \textbf{uniformly
cauchy }on $D$ to a function $f$ if the sequence of partial sums
$\left\{ \sum_{n=1}^{N}f_{n}\right\} $ converges uniformly cauchy
on $D$ to $f$.
\end{enumerate}
\end{defn}

\begin{theorem}\label{theorem1.5} Let $f_{n}:D\to\mathbb{C}$ where
$D\subseteq\mathbb{C}$. Then 
\begin{enumerate}
\item $\{f_{n}\}$ is uniformly convergence on $D$ if and only if $\{f_{n}\}$
is uniformly cauchy on $D$.
\item (Weierstrass $M$-test) If $|f_{n}(z)|\leq M_{n}$ for all $n\in\mathbb{N}$,
$z\in D$, and if $\sum_{n=1}^{\infty}M_{n}<\infty$, then $\sum_{n=1}^{\infty}f_{n}$
converges uniformly on $D$. 
\end{enumerate}
\end{theorem}

\begin{proof}\hfill
\begin{enumerate}
\item $(\implies)$ is an easy exercise. $(\impliedby)$ Assume $\{f_{n}\}$
is uniformly Cauchy on $D$. Then given $\varepsilon>0$, there exists
$N\in\mathbb{N}$ such that for all $m,n\geq N$ and for all $z\in D$,
we have $|f_{n}(z)-f_{m}(z)|<\varepsilon$. Fix $z\in D$. Then $\{f_{n}(z)\}$
is a Cauchy sequence in $\mathbb{C}$, and hence converges to some
limit $f(z)$. Now keep $n$ fixed and let $m\to\infty$. Then $|f_{n}(z)-f(z)|\leq\varepsilon$
for all $n\geq N$, $z\in D$. 
\item It suffices to show that $\left\{ \sum_{n=1}^{N}f_{n}\right\} $ converges
uniformly Cauchy on $D$. Note that for all $z\in D$, we have
\begin{align*}
\left|\sum_{k=1}^{n}f_{k}(z)-\sum_{k=1}^{m}f_{k}(z)\right| & =\left|\sum_{k=m+1}^{n}f_{k}(z)\right|\\
 & \leq\sum_{k=m+1}^{n}\left|f_{k}(z)\right|\\
 & \leq\sum_{k=m+1}^{n}M_{k}\\
 & =\sum_{k=1}^{n}M_{k}-\sum_{k=1}^{m}M_{k}
\end{align*}
 But $\left\{ \sum_{k=1}^{n}M_{k}\right\} $ is a Cauchy sequence.
Hence $\left\{ \sum_{k=1}^{n}f_{k}(z)\right\} $ is uniformly Cauchy. 
\end{enumerate}
\end{proof}

\subsection{Limit Supremum}

~~~To study the convergence of a power series, we recall the notion
of $\text{limsup}$ of a positive real-valued sequence. That is, if
$\{a_{k}\}_{k=1}^{\infty}$ is a sequence of positive real-valued
numbers, then 
\[
\text{limsup}\left(a_{k}\right)=\lim_{n\to\infty}\left(\sup_{k\ge n}\left(a_{k}\right)\right).
\]

Since $\sup_{k\geq n}\left(a_{k}\right)$ is a non-increasing function
of $n$, the limit always exists or equals $+\infty$. The properties
of the $\text{limsup}$ which will be of interest to us are the following:

\begin{prop}\label{proplimsup} Suppose $\{a_{k}\}_{k=1}^{\infty}$
is a sequence of positive real-valued numbers such that $\text{limsup}\left(a_{k}\right)=L$.
Then 
\begin{enumerate}
\item for each $\varepsilon>0$ and $N\in\mathbb{N}$, there exists some
$k\geq N$ such that $a_{k}\geq L-\varepsilon$.
\item for each $\varepsilon>0$, there exists $N\in N$ such that $a_{k}\leq L+\varepsilon$
for all $k>N$. 
\item $\text{limsup}\left(ca_{k}\right)=cL$ for any nonnegative constant
$c$. 
\end{enumerate}
\end{prop}

\begin{proof}
\begin{enumerate}
\item Choose $N$ and $\varepsilon>0$ and suppose that there does not exist
a $k>N$ such that $a_{k}\geq L-\varepsilon$. Then $L-\varepsilon>a_{k}$
for all $k>N$. This implies $\sup_{k\geq N}\left(a_{k}\right)<L$.
This is a contradiction since $\sup_{k\ge n}\left(a_{k}\right)$ is
a non-increasing function of $n$. 
\item Choose $\varepsilon>0$ and suppose that there does not exist an $N\in\mathbb{N}$
such that $a_{k}\leq L+\varepsilon$ for all $k>N$. Then $\sup_{k\geq N}\left(a_{k}\right)\geq L+\varepsilon$
for all $N\in\mathbb{N}$. This implies $\text{limsup}\left(a_{k}\right)\geq L+\varepsilon$,
which is a contradiction.
\item This follows since 
\[
\lim_{n\to\infty}\left(\sup_{k\ge n}\left(ca_{k}\right)\right)=\lim_{n\to\infty}\left(c\sup_{k\ge n}\left(a_{k}\right)\right)=c\lim_{n\to\infty}\left(\sup_{k\ge n}\left(a_{k}\right)\right).
\]
\end{enumerate}
\end{proof}

\begin{rem}Here are some additional properties of limsup, which are
easy to verify:
\begin{enumerate}
\item If $\{a_{n}\}$ and $\{b_{n}\}$ are two bounded real sequences, then
$\text{limsup}(a_{n}+b_{n})\leq\text{limsup}(a_{n})+\text{limsup}(b_{n})$
\item If $\{a_{n}\}$ and $\{b_{n}\}$ are two bounded real sequences such
that $\{b_{n}\}$ converges to $b$, then $\text{limsup}(a_{n}+b_{n})=\text{limsup}(a_{n})+b$.
\item If $\{a_{n}\}$ and $\{b_{n}\}$ are two bounded real sequences such
that $\{b_{n}\}$ converges to $b$, then $\text{limsup}(a_{n}b_{n})=b\text{limsup}(a_{n})$.
\end{enumerate}
\end{rem}

\begin{theorem} Suppose $\text{limsup}\left(\left|a_{k}\right|^{1/k}\right)=L$.
\begin{enumerate}
\item If $L=0$, $\sum a_{k}z^{k}$ converges for all $z$.
\item If $L=\infty$, $\sum a_{k}z^{k}$ converges for $z=0$ only.
\item If $0<L<\infty$, set $R=1/L$. Then $\sum a_{k}z^{k}$ converges
for $|z|<R$ and diverges for $|z|>R$. ($R$ is called the radius
of convergence of the power series.) 
\end{enumerate}
\end{theorem}

\begin{proof} We prove (3). Assume first that $|z|<R$ and set $|z|=R(1-2\delta)$.
Then since
\[
\text{limsup}\left(|z|\left|a_{k}\right|^{1/k}\right)=|z|\text{limsup}\left(\left|a_{k}\right|^{1/k}\right)=1-2\delta,
\]
$|z||a_{k}|^{1/k}<1-\delta$ for sufficiently large $k$. This implies
$|a_{k}z^{k}|<(1-\delta)^{k}$ for sufficiently large $k$. Hence
$\sum a_{k}z^{k}$ is absolutely convergent. On the other hand, if
$|z|>R$, then
\[
\text{limsup}\left(|z|\left|a_{k}\right|^{1/k}\right)>1,
\]
so that for infinitely many values of $k$, $a_{k}z^{k}$ has absolute
value greater than $1$ and $\sum a_{k}z^{k}$ diverges. \end{proof}

\begin{example} Since $n^{1/n}\to1$, $\sum_{n=1}^{\infty}nz^{n}$
converges for $|z|<1$ and diverges for $|z|>1$. The series also
diverges for $|z|=1$ for then $|nz^{n}|=n\to\infty$. \end{example}

\begin{example} $\sum_{n=0}^{\infty}z^{n^{2}}=1+z+z^{4}+z^{9}+z^{16}+\cdots$
has radius of convergence $1$. In this case $\overline{\lim}|a_{n}|^{1/n}=\lim1=1$.
\end{example}

\begin{example} Any series of the form $\sum a_{n}z^{n}$ with $|a_{n}|=1$
has radius of convergence equal to $1$. \end{example}

\begin{example} The generating function for the Catalan numbers $C_{n}$
is 
\[
f(z)=(z^{2}+z)^{2}+z)^{2}+z)^{2}+\cdots)=z+2z^{2}+5z^{3}+14z^{4}+\cdots.
\]

Since $|C_{n}|^{1/n}=4$, we see that this power series converges
for all $z$ such that $|z|<1/4$. \end{example}

\begin{example} Consider the series 
\begin{enumerate}
\item $\sum_{n=1}^{\infty}z^{n}$
\item $\sum_{n=1}^{\infty}\frac{z^{n}}{n}$
\item $\sum_{n=1}^{\infty}\frac{z^{n}}{n^{2}}$
\end{enumerate}
Then 
\begin{enumerate}
\item $\text{limsup}\left(1^{1/n}\right)=1$
\item $\text{limsup}\left(\frac{1}{n^{1/n}}\right)=1$
\item $\text{limsup}\left(\frac{1}{n^{2/n}}\right)=1$
\end{enumerate}
Therefore the radius of convergence for all three series are $R=1$.
What happens when $|z|=1?$ 
\begin{enumerate}
\item This series diverges on $|z|=1$ because $|z^{n}|=1\not\to0$ as $n\to\infty$. 
\item This series diverges at $z=1$ since this is the harmonic series.
On the other hand, this series converges everywhere else by the Dirichlet
test, which says if $a_{n}\in\mathbb{C}$ with $\left|\sum_{i=1}^{n}a_{i}\right|\leq M$,
where $M$ is a constant, and $b_{n}\in\mathbb{R}$ such that $b_{n}\to0$
and $\{b_{n}\}$ is decreasing, then $\sum_{n=1}^{\infty}a_{n}b_{n}$
converges. Take $b_{n}=\frac{1}{n}$ and $a_{n}=e^{in\theta}$. Then
\[
\left|\sum_{i=1}^{n}a_{i}\right|=\left|\frac{e^{i\theta}\left(e^{in\theta}-1\right)}{e^{i\theta}-1}\right|\leq\frac{2}{\left|e^{i\theta}-1\right|}\leq M
\]
 when $\theta\neq0$.
\item This converges at every point $|z|=1$ since $\left|\frac{z^{n}}{n^{2}}\right|\leq\frac{1}{n^{2}}$,
and $\sum_{n=1}^{\infty}\frac{1}{n^{2}}$ converges.
\end{enumerate}
\end{example}

\begin{rem} The ratio test can also be applied to series of complex
numbers. Given a power series 
\[
\sum_{n=0}^{\infty}a_{n}z^{n},
\]
we set $b_{n}:=a_{n}|z|^{n}$ and study the limit 
\[
\frac{b_{n+1}}{b_{n}}=\frac{a_{n+1}}{a_{n}}|z|
\]
as $n\to\infty$. If for some $\varepsilon>0$ and $N\in\mathbb{N}$,
we have 
\[
\left|\frac{b_{n+1}}{b_{n}}\right|\geq|z|-\varepsilon>1
\]
for all $n\geq N$, then 
\[
b_{n+1}\geq(|z|-\varepsilon)b_{n}\geq(|z|-\varepsilon)^{2}b_{n-1}\geq\cdots\geq(|z|-\varepsilon)^{n-N+1}|b_{N}|.
\]

\end{rem}

\begin{example} We find the radius of convergence of 
\[
\sum_{n=0}^{\infty}\left(\frac{n+2}{2n^{2}+1}\right)z^{n}
\]
using the ratio test. Let 
\[
b_{n}:=\left(\frac{n+2}{2n^{2}+1}\right)|z|^{n}.
\]
Then 
\[
\frac{b_{n+1}}{b_{n}}=\frac{(n+3)(2n^{2}+1)}{(2(n+1)^{2}+1)(n+2)}|z|=\frac{2n^{3}+6n^{2}+n+3}{2n^{3}+8n^{2}+11n+6}|z|\to|z|,
\]
so the series converges absolutely for $|z|<1$, and diverges for
$|z|>1$. \end{example}

\begin{rem} The power series $\sum_{n=0}^{\infty}a_{n}(z-z_{0})^{n}$
and $\sum_{n=1}^{\infty}na_{n}(z-z_{0})^{n-1}$ have the same radius
of convergence since 
\[
R=\frac{1}{\text{limsup}\left(\left|a_{n}\right|^{\frac{1}{n}}\right)}=\frac{1}{\text{limsup}\left(\left|na_{n}\right|^{\frac{1}{n}}\right)}
\]
\end{rem}

\subsection{Limits and Continuity}

\begin{defn} Let $D$ be a nonempty subset of $\mathbb{C}$. A point
$z_{0}\in\mathbb{C}$ is an \textbf{accumulation point }of $D$ if
there exists a sequence of points $z_{n}\in D\backslash\{z_{0}\}$
such that $z_{n}\to z_{0}$. Equivalently, this means that for each
$r>0$, we have $D\cap D_{r}(z_{0})\neq\emptyset$. A point $z_{0}\in\mathbb{C}$
is an \textbf{isolated point }of $D$ if there exists $r>0$ such
that $D\cap D_{r}(z_{0})=\emptyset$. \end{defn}

\begin{defn} Let $f:D\to\mathbb{C}$ be a function and let $z_{0}$
be an accumulation point of $D$. We say that $\text{lim}_{z\to z_{0}}f(z)=\ell$
(or $f(z)\to\ell$ as $z\to z_{0}$) if given any $\varepsilon>0$,
there exists $\delta>0$ such that $|f(z)-\ell|<\varepsilon$ for
all $z\in D\cap D_{\delta}(z_{0})$. \end{defn}

\begin{rem}\label{rem} Let $f,g:D\to\mathbb{C}$ be two functions
such that $f(z)\to\ell$ and $g(z)\to\ell'$ as $z\to z_{0}$. Then
$\alpha f(z)+\beta g(z)\to\alpha\ell+\beta\ell'$ and $f(z)g(z)\to\ell\ell'$
as $z\to z_{0}$.\end{rem}

\begin{defn}\label{defn} Let $f:D\to\mathbb{C}$ be a function. We
say that $f$ is \textbf{continuous }at $z_{0}\in D$ if given any
$\varepsilon>0$, there exists $\delta>0$ such that $f(z)\in D_{\varepsilon}(f(z_{0}))$
whenever $z\in D_{\delta}(z_{0})\cap D$. \end{defn}

\begin{prop}\label{prop} Let $f_{n}:D\to\mathbb{C}$ be a sequence
of continuous functions. If $f_{n}$ converges to $f$ uniformly on
$D$, then $f$ is continuous on $D$. \end{prop}

\begin{proof} Take $z_{0}\in D$. Given $\varepsilon>0$, choose
$N\in\mathbb{N}$ such that 
\[
|f_{n}(z)-f(z)|<\frac{\varepsilon}{3},
\]
for all $n\geq N$ and $z\in D$. Since $f_{N}$ is continuous at
$z_{0}$, there exists $\delta>0$ such that 
\[
|f_{N}(z)-f_{N}(z_{0})|<\frac{\varepsilon}{3}
\]

whenever $|z-z_{0}|<\delta$. Now if $|z-z_{0}|<\delta$, $z\in D$,
then 
\begin{align*}
|f(z)-f(z_{0})| & \leq|f(z)-f_{N}(z))|+|f_{N}(z)-f_{N}(z_{0})|+|f_{N}(z_{0})-f(z_{0})|\\
 & <\frac{\varepsilon}{3}+\frac{\varepsilon}{3}+\frac{\varepsilon}{3}\\
 & =\varepsilon.
\end{align*}
\end{proof}

\subsection{Holomorphic Functions}

\begin{defn}\label{defn} Let $\Omega$ be an open set in $\mathbb{C}$
and $f$ a complex-valued function on $\Omega$. The function $f$
is \textbf{holomorphic at the point }$z_{0}\in\Omega$ if the quotient
\[
\frac{f(z_{0}+h)-f(z_{0})}{h}
\]
converges to a limit when $h\to0$. Here $h\in\mathbb{C}$ and $h\neq0$
with $z_{0}+h\in\Omega$, so that the quotient is well defined. The
limit of the quotient, when it exists, is denoted by $f'(z_{0})$,
and is called the \textbf{derivative of $f$ at $z_{0}$}:
\[
f'(z_{0}):=\lim_{h\to0}\frac{f(z_{0}+h)-f(z_{0})}{h}
\]

The function $f$ is said to be \textbf{holomorphic on $\Omega$ }if
$f$ is holomorphic at every point of $\Omega$. If $C$ is a closed
subset of $\mathbb{C}$, we say that $f$ is \textbf{holomorphic on
$C$ }if $f$ is holomorphic in some open set containing $C$. We
also say $f$ is \textbf{holomorphic }at $z_{0}$ if $f$ is differentiable
on some open neighborhood $D_{r}(z_{0})$ of $z_{0}$. Finally, if
$f$ is holomorphic in all of $\mathbb{C}$ we say that $f$ is \textbf{entire}.
\end{defn}

\begin{example}\label{example} \hfill
\begin{enumerate}
\item Polynomials and rational functions are entire functions.
\item Let $f:\mathbb{C}\to\mathbb{C}$ be given by $f(z)=\overline{z}$.
Then $f$ is continuous everywhere in $\mathbb{C}$ but is not differentiable
anywhere in $\mathbb{C}$. To see this, let $z_{0}\in\mathbb{C}$.
Then
\[
\lim_{h\to0}\frac{f(z_{0}+h)-f(z_{0})}{h}=\lim_{h\to0}\frac{\overline{h}}{h}.
\]
Letting $h=\varepsilon$ where $\varepsilon\in\mathbb{R}$, we get
\[
\lim_{h\to0}\frac{\overline{h}}{h}=1.
\]
On the other hand, letting $h=i\varepsilon$ where $\varepsilon\in\mathbb{R}$,
we get
\[
\lim_{h\to0}\frac{\overline{h}}{h}=-1\neq1.
\]
Therefore the limit does not exist. 
\item Let $f:\mathbb{C}\to\mathbb{C}$ be given by $f(z)=|z|^{2}$. Then
$f$ is differentiable at $z_{0}=0$, but it not differentiable whenever
$z_{0}\neq0$. In particular, $f$ is not holomorphic at $0$. To
see this, let $z_{0}=0$. Then
\begin{align*}
\lim_{h\to0}\frac{f(z_{0}+h)-f(h)}{h} & =\lim_{h\to0}\frac{|h|^{2}}{h}\\
 & =\lim_{h\to0}\frac{h\overline{h}}{h}\\
 & =\lim_{h\to0}\overline{h}\\
 & =0.
\end{align*}
So $f$ is differentiable at $0$. Now assume $f$ were differentiable
at some $z_{0}\neq0$. Let $g:\mathbb{C}\to\mathbb{C}$ be given by
$g(z)=z$. Then since $g$ is differentiable at $z_{0}$ and $g(z_{0})=z_{0}\neq0$,
the quotient $f/g$ must be differentiable at $z_{0}$. But $f(z)/g(z)=\overline{z}$,
which we know is not differentiable anywhere. Therefore we have a
contradiction, and so $f$ is not differentiable at $z_{0}\neq0$. 
\end{enumerate}
\end{example}

\begin{prop}\label{prop} Let $\Omega$ be an open set in $\mathbb{C}$
and $f$ a complex-valued function on $\Omega$. Then $f$ is holomorphic
at $z_{0}\in\Omega$ if and only if there exists a complex number
$a$ such that 
\begin{equation}
f(z_{0}+h)=f(z_{0})+ah+\psi(h)h,\label{eq:holomorphic}
\end{equation}
where $\psi$ is a function defined for all small $h$ and $\lim_{h\to0}\psi(h)=0$.
\end{prop}

\begin{proof} Assume (\ref{eq:holomorphic}) holds. Then 
\[
\lim_{h\to0}\frac{f(z_{0}+h)-f(z_{0})}{h}=\lim_{h\to0}\frac{ah+\psi(h)h}{h}=a,
\]
so $f'(z_{0})$ exists and $f'(z_{0})=a$. Conversely, assume $f'(z_{0})$
exists. Define $\psi$ as 
\[
\psi(h)=\frac{f(z_{0}+h)-f(z_{0})-f'(z_{0})h}{h}.
\]

Then $\psi$ satisfies the desired properties. \end{proof}

\begin{prop}\label{prop} Let $f:\Omega\to U$ and $g:U\to\mathbb{C}$
be holomorphic functions on the open sets $\Omega$ and $U$ respectively.
Then the chain rule holds
\[
(g\circ f)'(z)=g'(f(z))f'(z)
\]
for all $z\in\Omega$. \end{prop}

\begin{proof} Suppose $z\in\Omega$. Since $f$ is holomorphic at
$z$, we have 
\[
f(z+h)=f(z)+f'(z)h+\psi_{1}(h)h
\]

where $\psi_{1}$ is a function defined for all small $h$ and $\lim_{h\to0}\psi_{1}(h)=0$.
Since $g$ is holomorphic at $f(z)\in U$, we have 
\[
g(f(z)+h)=g(f(z))+g'(f(z))h+\psi_{2}(h)h
\]
where $\psi_{2}$ is a function defined for all small $h$ and $\lim_{h\to0}\psi_{1}(h)=0$.
\begin{align*}
(g\circ f)(z+h) & =g(f(z+h))\\
 & =g(f(z)+f'(z)h+\psi_{1}(h)h)\\
 & =g(f(z))+g'(f(z))(f'(z)h+\psi_{1}(h)h)+\psi_{2}(h)(f'(z)h+\psi_{1}(h)h)\\
 & =g(f(z))+g'(f(z))f'(z)h+\psi_{3}(h)h
\end{align*}

where $\psi_{3}(h)=g'(f(z))\psi_{1}(h)h+\psi_{2}(h)f'(z)h+\psi_{1}(h)\psi_{2}(h)h$.
Since $\psi_{3}$ is a function defined for all small $h$ and $\lim_{h\to0}\psi_{3}(h)=0$,
it follows that 
\[
(g\circ f)'(z)=g'(f(z))f'(z)
\]
for all $z\in\Omega$. \end{proof}

\begin{prop}\label{prop} Let $\Omega$ be an open subset of $\mathbb{C}$
and let $f,g:\Omega\to\mathbb{C}$ be holomorphic functions. Then
the product rule holds
\[
(fg)'(z)=f'(z)g(z)+f(z)g'(z)
\]
for all $z\in\Omega$. \end{prop}

\begin{proof} Suppose $z\in\Omega$. Since $f$ is holomorphic at
$z$, we have 
\[
f(z+h)=f(z)+f'(z)h+\psi_{1}(h)h
\]

where $\psi_{1}$ is a function defined for all small $h$ and $\lim_{h\to0}\psi_{1}(h)=0$.
Since $g$ is holomorphic at $f(z)\in U$, we have 
\[
g(f(z)+h)=g(f(z))+g'(f(z))h+\psi_{2}(h)h
\]
where $\psi_{2}$ is a function defined for all small $h$ and $\lim_{h\to0}\psi_{1}(h)=0$.
Then 

\begin{align*}
f(z+h)g(z+h) & =(f(z)+f'(z)h+\psi_{1}(h)h)(g(z)+g'(z)h+\psi_{2}(h)h)\\
 & =f(z)g(z)+(f(z)g'(z)+f'(z)g(z))h+\psi_{3}(h)h,
\end{align*}
where $\psi_{3}(h)=f(z)\psi_{2}(h)+f'(z)g'(z)h+g(z)\psi_{1}(h)$.
Since $\psi_{3}$ is a function defined for all small $h$ and $\lim_{h\to0}\psi_{3}(h)=0$,
it follows that 
\[
(fg)'(z)=f'(z)g(z)+f(z)g'(z)
\]
 for all $z\in\Omega$. \end{proof}

~~~Let $f(z)=u(z)+iv(z)$ and $z=x+iy$. We may treat $u$ and
$v$ as real-valued functions in the variable $x$ and $y$. 

\begin{theorem}\label{theorem1.7} Let $f$ be holomorphic at $z_{0}=x_{0}+iy_{0}$.
Then the partial derivatives $\partial_{x}u$ , $\partial_{y}u$,
$\partial_{x}v$, and $\partial_{y}v$ exist at $z_{0}$ and 
\[
f'(z_{0})=(\partial_{x}u)(x_{0},y_{0})+i(\partial_{x}v)(x_{0},y_{0})=(\partial_{y}u)(x_{0},y_{0})-i(\partial_{y}v)(x_{0},y_{0}).
\]
Consequently, we have the \textbf{Cauchy-Riemann Equations}:
\begin{align*}
\partial_{x}u & =\partial_{y}v\\
-\partial_{y}u & =\partial_{x}v
\end{align*}
\end{theorem}

\begin{proof} We have
\[
f'(z_{0})=\lim_{h\to0}\frac{f(z_{0}+h)-f(z_{0})}{h}.
\]
First let $h\to0$ where $h\in\mathbb{R}$. Then 
\begin{align*}
f'(z_{0}) & =\lim_{h\to0}\frac{u(x_{0}+h,y_{0})+iv(x_{0}+h,y_{0})-u(x_{0},y_{0})-iv(x_{0},y_{0})}{h}\\
 & =\lim_{h\to0}\frac{u(x_{0}+h,y_{0})-u(x_{0},y_{0})}{h}+i\lim_{h\to0}\frac{v(x_{0}+h,y_{0})-v(x_{0},y_{0})}{h}\\
 & =(\partial_{x}u)(x_{0},y_{0})+i(\partial_{x}v)(x_{0},y_{0}).
\end{align*}
Similarly, let $h\to0$ where $h$ is purely imaginary. Then we obtain
\[
f'(z_{0})=-i(\partial_{y}u)(x_{0},y_{0})+(\partial_{y}v)(x_{0},y_{0}).
\]

Equating the two formulas for $f'(z_{0})$ above yields the Cauchy-Riemann
equations. \end{proof}

~~~Recall that a function $F(x,y)=(u(x,y),v(x,y))$ is said to
be differentiable at a point $P_{0}=(x_{0},y_{0})$ if there exists
a linear transformation $J:\mathbb{R}^{2}\to\mathbb{R}^{2}$ such
that 
\[
\frac{|F(P_{0}+H)-F(P_{0})-J(H)|}{|H|}\to0
\]
as $|H|\to0$ where $H\in\mathbb{R}^{2}$. Equivalently, we can write
\[
F(P_{0}+H)-F(P_{0})=J(H)+|H|\Psi(H),
\]
with $|\Psi(H)|\to0$ as $|H|\to0$. The linear transformation $J$
is unique and is called the derivative of $F$ at $P_{0}$. If $F$
is differentiable, the partial derivatives exist, and the linear transformation
$J$ is described in the standard basis of $\mathbb{R}^{2}$ by the
Jacobian matrix of $F$ 
\[
J=J_{F}(x,y)=\begin{pmatrix}\partial_{x}u & \partial_{y}u\\
\partial_{x}v & \partial_{y}v
\end{pmatrix}.
\]
In the case of complex differentiation the derivative is a complex
number $f'(z_{0})$, while in the case of real derivatives, it is
a matrix. 

~~~We can clarify the situation further by defining two differential
operators
\[
\partial_{z}:=\frac{1}{2}\left(\partial_{x}-i\partial_{y}\right)\quad\text{and}\quad\partial_{\overline{z}}:=\frac{1}{2}\left(\partial_{x}+i\partial_{y}\right)
\]

\begin{example}\label{example} Let $f(z)=\overline{z}=x-iy$. Then
\[
\partial_{x}u=1\neq-1=\partial_{y}v,
\]
so $f$ is not differentiable. \end{example}

\begin{cor}\label{cor} \hfill
\begin{enumerate}
\item If $f$ is holomorphic on $D_{r}(z_{0})$ with $f'(z)=0$ for all
$z\in D_{r}(z_{0})$, then $f$ is constant on $D_{r}(z_{0})$.
\item If $f$ is holomorphic on $D_{r}(z_{0})$ and real-valued on $D_{r}(z_{0})$,
then $f$ is constant.
\end{enumerate}
\end{cor}

\begin{proof}\hfill
\begin{enumerate}
\item Since $f'(z)=0$, we have $\partial_{x}u+i\partial_{x}v=0=\partial_{y}v-i\partial_{y}u$.
This implies $\partial_{x}u=\partial_{y}u=\partial_{x}v=\partial_{y}v=0$.
Therefore $u$ and $v$ are constant functions, and hence, $f$ is
constant. 
\item Let $f=u+iv$. Then $v=0$ on $D_{r}(z_{0})$. So $\partial_{x}v=\partial_{y}v=0$
implies $\partial_{x}u=\partial_{y}u=0$. Therefore $u$ and $v$
are constant, and so $f$ is constant. 
\end{enumerate}
\end{proof}

\begin{example}\label{example} The function $f(z)=|z|^{2}$ is not
analytic anywhere because it is real-valued and non-constant. \end{example}

\begin{theorem}\label{theorem1.9} Let $f(z)=\sum_{n=0}^{\infty}a_{n}(z-z_{0})^{n}$
have radius of convergence $R>0$. Then $f$ is analytic on $D_{R}(z_{0})$
with 
\[
f'(z)=\sum_{n=1}^{\infty}na_{n}(z-z_{0})^{n-1}
\]
for $|z-z_{0}|<R$. \end{theorem}

\section{Integration along curves}

~~~A \textbf{parametrized curve }is a function $\gamma:[a,b]\to\mathbb{C}$.
We say the parametrized curve is \textbf{smooth }if $\dot{\gamma}(t)$
exists and is continuous on $[a,b]$, and $\dot{\gamma}(t)\neq0$
for $t\in[a,b]$. At the points $t=a$ and $t=b$, the quantities
$\dot{\gamma}(a)$ and $\dot{\gamma}(b)$ are interpreted as one-sided
limits
\[
\dot{\gamma}(a)=\lim_{\substack{h\to0\\
h>0
}
}\frac{\gamma(a+h)-\gamma(a)}{h}\quad\text{and}\quad\dot{\gamma}(b)=\lim_{\substack{h\to0\\
h<0
}
}\frac{\gamma(b+h)-\gamma(b)}{h}.
\]

In general, these quantities are called the right-hand derivative
of $\gamma(t)$ at $a$, and the left-hand derivative of $\gamma(t)$
at $b$, respectively.

~~~Similarly, we say that the parametrized curve is \textbf{piecewise-smooth
}if $\gamma$ is continuous on $[a,b]$ and if there exist points
\[
a=a_{0}<a_{1}<\cdots<a_{n}=b,
\]
where $\gamma(t)$ is smooth in the intervals $[a_{k},a_{k+1}]$.
In particular, the right-hand derivative at $a_{k}$ may differ from
the left-hand derivative at $a_{k}$, for $k=1,\dots,n-1$. 

~~~Two parametrizations 
\[
\gamma_{1}:[a,b]\to\mathbb{C}\quad\text{and}\quad\gamma_{2}:[c,d]\to\mathbb{C}
\]
are \textbf{equivalent }if there exists a continuously differentiable
bijection $s\mapsto\varphi(s)$ from $[c,d]$ to $[a,b]$ so that
$\dot{\varphi}(s)>0$ and 
\[
\gamma_{2}(s)=\gamma_{1}(\varphi(s)).
\]
The condition $\dot{\varphi}(s)>0$ says precisely that the orientation
is preserved: as $s$ travels from $c$ to $d$, then $\varphi(s)$
travels from $a$ to $b$. The family of all parametrizations that
are equivalent to $\gamma(t)$ determines a \textbf{smooth curve $\Gamma\subset\mathbb{C}$},
namely the image of $[a,b]$ under $\gamma$ with the orientation
given by $\gamma$ as $t$ travels from $a$ to $b$. We can define
a curve $\Gamma^{-}$ obtained from the curve $\Gamma$ by reversing
the orientation. As a particular parametrization for $\Gamma^{-}$,
we can take $\gamma^{-}:[a,b]\to\mathbb{C}$ defined by 
\[
\gamma^{-}(t)=\gamma(b+a-t).
\]

~~~It is also clear how to define a \textbf{piecewise-smooth curve}.
The point $\gamma(a)$ and $\gamma(b)$ are called the \textbf{end-points
}of the curve and are independent on the parametrization. Since $\gamma$
carries an orientation, it is natural to say that $\gamma$ begins
at $\gamma(a)$ and ends at $\gamma(b)$. 

~~~A smooth or piecewise-smooth curve is \textbf{closed }if $\gamma(a)=\gamma(b)$
for any of its parametrizations. Finally, a smooth or piecewise-smooth
curve is \textbf{simple }if it is not self-intersecting, that is,
$\gamma(t)\neq\gamma(s)$ unless $s=t$. 

~~~For brevity, we shall call any piecewise-smooth curve a \textbf{curve},
since these will be the objects we shall be primarily concerned with. 

\begin{example}\label{example} Let $C_{r}(z_{0})=\{z\in\mathbb{C}\mid|z-z_{0}|=r\}$.
Then $C_{r}(z_{0})$ is the circle centered at $z_{0}$ and of radius
$r$. The \textbf{positive orientation }(counterclockwise) is the
one that is given by the standard parametrization
\[
\gamma(t)=z_{0}+re^{2\pi it},\quad\text{where }t\in[0,1],
\]

while the \textbf{negative orientation }(clockwise)
\[
\gamma(t)=z_{0}+re^{-2\pi it},\quad\text{where }t\in[0,1].
\]

We usually denote by $C$ a general \emph{positively }oriented circle.
\end{example}

~~~Given a smooth curve $\Gamma$ in $\mathbb{C}$ parametrized
by $\gamma:[a,b]\to\mathbb{C}$, and $f$ a continuous function on
$\gamma$, we define the \textbf{integral of $f$ along $\gamma$
}by 
\[
\int_{\Gamma}f(z)dz=\int_{a}^{b}f(\gamma(t))\dot{\gamma}(t)dt.
\]
In order for this definition to be meaningful, we must show that the
right-hand integral is independent of the parametrization chosen for
$\Gamma$. Say that $\gamma_{1}$ and $\gamma_{2}$ are two equivalent
parametrizations for $\Gamma$. Then the change of variables formula
and the chain rule imply that 
\[
\int_{a}^{b}f(\gamma_{1}(t))\dot{\gamma}_{1}(t)dt=\int_{c}^{d}f(\gamma_{1}(\varphi(s)))\dot{\gamma}_{1}(\varphi(s))\dot{\varphi}(s)ds=\int_{a}^{b}f(\gamma_{2}(s))\dot{\gamma}_{2}(s)ds.
\]
This proves that the integral of $f$ over $\gamma$ is well defined. 

~~~If $\Gamma$ is piecewise smooth, then the integral of $f$
over $\Gamma$ is simply the sum of the integrals of $f$ over the
smooth parts of $\Gamma$, so if $\gamma(t)$ is a piecewise-smooth
parametrization as before, then 
\[
\int_{\Gamma}f(z)dz=\sum_{k=0}^{n-1}\int_{a_{k}}^{a_{k+1}}f(\gamma(t))\dot{\gamma}(t)dt.
\]
By definition, the \textbf{length }of the smooth curve $\Gamma$ is
\[
\text{length}(\Gamma)=\int_{a}^{b}|\dot{\gamma}(t)|dt.
\]
Arguing as we just did, it is clear that this definition is also independent
of the parametrization. Also, if $\gamma$ is only piecewise-smooth,
then its length is the sum of the lengths of its smooth parts. 

\begin{prop}\label{prop} Integration of continuous functions over
curves satisfies the following properties: 
\begin{enumerate}
\item It is linear, that is, if $\alpha,\beta\in\mathbb{C}$, then 
\[
\int_{\Gamma}(\alpha f(z)+\beta g(z))dz=\alpha\int_{\Gamma}f(z)dz+\beta\int_{\Gamma}g(z)dz.
\]
\item If $\Gamma^{-}$ is $\Gamma$ with reverse orientation, then 
\[
\int_{\Gamma}f(z)dz=-\int_{\Gamma^{-}}f(z)dz.
\]
\item One has the inequality 
\[
\left|\int_{\Gamma}f(z)dz\right|\leq\sup_{z\in\Gamma}\left|f(z)\right|\cdot\text{length}\left(\Gamma\right).
\]
\end{enumerate}
\end{prop}

\begin{proof}\hfill
\begin{enumerate}
\item Follows from the definition and the linearity of the Riemann integral.
\item Left as an exercise.
\item Let $\gamma$ be a parametrization of $\Gamma$. Then
\begin{align*}
\left|\int_{\Gamma}f(z)dz\right| & =\left|\int_{a}^{b}f(\gamma(t))\dot{\gamma}(t)dt\right|\\
 & \leq\int_{a}^{b}\left|f(\gamma(t))\dot{\gamma}(t)\right|dt\\
 & =\int_{a}^{b}\left|f(\gamma(t))\right|\left|\dot{\gamma}(t)\right|dt\\
 & \leq\sup_{t\in[a,b]}\left|f(\gamma(t))\right|\int_{a}^{b}\left|\dot{\gamma}(t)\right|dt\\
 & =\sup_{z\in\Gamma}\left|f(z)\right|\cdot\text{length}\left(\Gamma\right)
\end{align*}
\end{enumerate}
\end{proof}

\begin{defn}\label{defn} Suppose $f$ is a function on the open set
$\Omega$. A \textbf{primitive }for $f$ on $\Omega$ is a function
$F$ that is holomorphic on $\Omega$ and such that $F'(z)=f(z)$
for all $z\in\Omega$. \end{defn}

\begin{theorem}\label{theorem} If a continuous function $f$ has
a primitive $F$ in $\Omega$, and $\Gamma$ is a curve in $\Omega$
that begins at $w_{1}$ and ends at $w_{2}$, then
\[
\int_{\Gamma}f(z)dz=F(w_{2})-F(w_{1}).
\]

\end{theorem}

\begin{proof} If $\Gamma$ is smooth, the proof is a simple application
of the chain rule and the fundamental theorem of calculus. Indeed,
if $\gamma(t):[a,b]\to\mathbb{C}$ is a parametrization for $\Gamma$,
then $\gamma(a)=w_{1}$ and $\gamma(b)=w_{2}$, and we have 
\begin{align*}
\int_{\Gamma}f(z)dz & =\int_{a}^{b}f(\gamma(t))\dot{\gamma}(t)dt\\
 & =\int_{a}^{b}F'(\gamma(t))\dot{\gamma}(t)dt\\
 & =\int_{a}^{b}\frac{d}{dt}\left(F(\gamma(t))\right)dt\\
 & =F(\gamma(b))-F(\gamma(a)).
\end{align*}

If $\Gamma$ is only piecewise-smooth, then arguing as we just did,
we obtain a telescopic sum, and we have 
\begin{align*}
\int_{\Gamma}f(z)dz & =\sum_{k=0}^{n-1}F(\gamma(a_{k+1}))-F(\gamma(a_{k}))\\
 & =F(\gamma(a_{n}))-F(\gamma(a_{0}))\\
 & =F(\gamma(b))-F(\gamma(a)).
\end{align*}

\end{proof}

\begin{cor}\label{corintegralclosedcurve} If $\Gamma$ is a closed
curve in an open set $\Omega$, and $f$ is continuous and has a primitive
in $\Omega$, then 
\[
\int_{\Gamma}f(z)dz=0.
\]
\end{cor}

\begin{proof} This is immediate since the end-points of a closed
curve coincide. \end{proof}

\begin{defn}\label{defn} A connected open subset in $\mathbb{C}$
is called a \textbf{region}. \end{defn}

\begin{rem}\label{rem} In $\mathbb{C}$, a connected open subset
is path-connected. \end{rem}

\begin{cor}\label{cor} If $f$ is holomorphic in a region $\Omega$
and $f'=0$, then $f$ is constant. \end{cor}

\begin{proof} Fix a point $w_{0}\in\Omega$. It suffices to show
that $f(w)=f(w_{0})$ for all $w\in\Omega$. Since $\Omega$ is connected,
for any $w\in\Omega$, there exists a curve $\Gamma$ which joins
$w_{0}$ to $w$. Since $f$ is clearly a primitive for $f'$, we
have 
\[
\int_{\Gamma}f'(z)dz=f(w)-f(w_{0}).
\]
By assumption, $f'=0$, so the integral on the left is $0$, and we
conclude that $f(w)=f(w_{0})$ as desired. \end{proof}

\subsection{Cauchy's Theorem and its Applications}

~~~Roughly speaking, Cauchy's Theorem states that if $f$ is holomorphic
in an open set $\Omega$ and $\Gamma\subset\Omega$ is a closed curve
whose interior is also contained in $\Omega$, then 
\[
\int_{\Gamma}f(z)dz=0.
\]

Many results that follow, and in particular the calculus of residues,
are related in one way or another to this fact. We first prove this
in the special case that our curve $\Gamma$ is a triangle:

\begin{theorem}\label{theorem} (Goursat's Theorem) If $\Omega$ is
an open set in $\mathbb{C}$, and $T\subset\Omega$ is a triangle
whose interior is also contained in $\Omega$, then 
\[
\int_{T}f(z)dz=0,
\]
whenever $f$ is holomorphic in $\Omega$. \end{theorem}

\begin{proof} We call $T^{(0)}$ our original triangle (with a fixed
orientation which we choose to be positive), and let $d^{(0)}$ and
$p^{(0)}$ denote the diameter and perimeter of $T^{(0)}$, respectively.
The first step in our construction consists of bisecting each side
of the triangle and connected the midpoints. This creates four new
smaller triangles, denote $T_{1}^{(1)}$, $T_{2}^{(1)}$, $T_{3}^{(1)}$,
and $T_{4}^{(1)}$ that are similar to the original triangle. The
orientation is chosen to be consistent with that of the original triangle,
and so after cancellations arising from integration over the same
side in two opposite directions, we have 
\[
\int_{T^{(0)}}f(z)dz=\int_{T_{1}^{(1)}}f(z)dz+\int_{T_{2}^{(1)}}f(z)dz+\int_{T_{3}^{(1)}}f(z)dz+\int_{T_{4}^{(1)}}f(z)dz.
\]
By triangle inequality, we must have 
\[
\left|\int_{T^{(0)}}f(z)dz\right|\leq4\left|\int_{T_{j}^{(1)}}f(z)dz\right|
\]

for some $j\in\{1,2,3,4\}$. Without loss of generality, assume $j=1$.
Observe that if $d^{(1)}$ and $p^{(1)}$ denote the diameter and
perimeter of $T^{(1)}$, respectively, then $d^{(1)}=(1/2)d^{(0)}$
and $p^{(1)}=(1/2)p^{(0)}$. We now repeat this process for the triangle
$T^{(1)}$, bisecting into four smaller triangles. Continuing this
process, we obtain a sequence of triangles
\[
T^{(0)},T^{(1)},\dots,T^{(n)},\dots
\]
with the properties that 
\[
\left|\int_{T^{(0)}}f(z)dz\right|\leq4^{n}\left|\int_{T^{(n)}}f(z)dz\right|
\]
and 
\[
d^{(n)}=2^{-n}d^{(0)},\quad p^{(n)}=2^{-n}p^{(0)},
\]
where $d^{(n)}$ and $p^{(n)}$ denote the diameter and perimeter
of $T^{(n)}$, respectively. We also denote $\mathcal{T}^{(n)}$ the
\emph{solid }closed triangle with boundary $T^{(n)}$, and observe
that our construction yields a sequence of nested compact sets
\[
\mathcal{T}^{(0)}\supset\mathcal{T}^{(1)}\supset\cdots\supset\mathcal{T}^{(n)}\supset\cdots
\]
whose diameter goes to $0$. Thus, by Proposition~(\ref{propsequenceofcompactsets}),
there exists a unique point $z_{0}$ that belongs to all the solid
triangles $\mathcal{T}^{(n)}$. Since $f$ is holomorphic at $z_{0}$,
we can write 
\[
f(z)=f(z_{0})+f'(z_{0})(z-z_{0})+\psi(z)(z-z_{0}),
\]
where $\psi(z)\to0$ as $z\to z_{0}$. Since the constant $f(z_{0})$
and the linear function $f'(z_{0})(z-z_{0})$ have primitives, we
can integrate the above equality and, using Corollary~(\ref{corintegralclosedcurve}),
we obtain 
\[
\int_{T^{(n)}}f(z)dz=\int_{T^{(n)}}\psi(z)(z-z_{0})dz.
\]

Now $z_{0}$ belongs to the closure of the triangle $\mathcal{T}^{(n)}$
and $z$ to its boundary, so we must have $|z-z_{0}|\leq d^{(n)}$,
and so we estimate 
\[
\left|\int_{T^{(n)}}f(z)dz\right|\leq\varepsilon_{n}d^{(n)}p^{(n)},
\]
where $\varepsilon_{n}=\sup_{z\in T^{(n)}}|\psi(z)|\to0$ as $n\to\infty$.
Therefore
\[
\left|\int_{T^{(n)}}f(z)dz\right|\le\varepsilon_{n}4^{-n}d^{(0)}p^{(0)},
\]
which yields our final estimate 
\[
\left|\int_{T^{(0)}}f(z)dz\right|\leq4^{n}\left|\int_{T^{(n)}}f(z)dz\right|\le\varepsilon_{n}d^{(0)}p^{(0)}.
\]
Letting $n\to\infty$ concludes the proof since $\varepsilon_{n}\to0$.
\end{proof}

\begin{cor}\label{cor} If $f$ is holomorphic in an open set $\Omega$
that contains a rectangle $R$ and its interior, then 
\[
\int_{R}f(z)dz=0.
\]
\end{cor}

\begin{proof} We simply decompose the rectangle into two triangles,
and integrate over the two triangles. \end{proof}

\subsection{Local existence of primitives and Cauchy's theorem in a disc}

\begin{defn}\label{defn} A set $E\subset\mathbb{C}$ is said to be
\textbf{convex }if whenever $x,y\in E$ and $0<t<1$, the point $(1-t)x+ty$
also lies in $E$. \end{defn}

\subsubsection{Existence of primitives for holomorphic functions}

\begin{theorem}\label{theorem2.1} A holomorphic function in an open
disc has a primitive in that disc. \end{theorem}

\begin{proof} After a translation, we may assume without loss of
generality that the disc, say $D$, is centered at the origin. Given
a point $z\in D$, consider the piecewise-smooth curve that joins
$0$ to $z$ first by moving in the horizontal direction form $0$
to $\text{Re}(z)$, and then in the vertical direction from $\text{Re}(z)$
to $z$. We choose the orientation from $0$ to $z$, and denote this
polygonal line by $\Gamma_{z}$. Define 
\[
F(z)=\int_{\Gamma_{z}}f(w)dw.
\]
The choice of $\Gamma_{z}$ gives an unambigious definition of the
function $F(z0$. We contend that $F$ is holomorpic in $D$ and $F'(z)=f(z)$.
To prove this, fix $z\in D$ and let $h\in\mathbb{C}$ be so small
that $z+h$ also belongs to the disc. Now consider the difference
\[
F(z+h)-F(z)=\int_{\Gamma_{z+h}}f(w)dw-\int_{\Gamma_{z}}f(w)dw.
\]

The function $f$ is first integrated along $\Gamma_{z+h}$ with the
original orientation, and then along $\Gamma_{z}$ with the reverse
orientation. Since we integrate $f$ over the line segment starting
at the origin in two opposite directions, it cancels. Then we complete
the square and triangle, so that after an application of Goursat's
theorem for triangles and rectangles, we are left with the line segment
from $z$ to $z+h$. Hence, we have 
\[
F(z+h)-F(z)=\int_{\eta}f(w)dw,
\]
where $\eta$ is the straight line segment from $z$ to $z+h$. Since
$f$ is continuous at $z$ we can write 
\[
f(w)=f(z)+\psi(w)
\]
where $\psi(w)\to0$ as $w\to z$. Therefore
\[
F(z+h)-F(z)=\int_{\eta}f(z)dw+\int_{\eta}\psi(w)dw=f(z)\int_{\eta}dw+\int_{\eta}\psi(w)dw.
\]
On the one hand, the constant $1$ has $w$ as a primitive, so the
first integral is simply $h$. On the other hand, we have
\[
\left|\int_{\eta}\psi(w)dw\right|\leq\sup_{w\in\eta}|\psi(w)|\cdot|h|.
\]
Since the supremum above goes to $0$ as $h$ tends to $0$, we conclude
that 
\[
\lim_{h\to0}\frac{F(z+h)-F(z)}{h}=f(z),
\]
thereby proving that $F$ is a primitive for $f$ on the disc. 

\end{proof}

\begin{theorem}\label{theorem} If $f$ is holomorphic in a disc,
then 
\[
\int_{\Gamma}f(z)dz=0
\]
for any closed curve $\Gamma$ in that disc. \end{theorem}

\begin{proof} Since $f$ has a primitive, we can apply Corollary~(\ref{corintegralclosedcurve}).
\end{proof}

\begin{example}\label{example} We show that if $y\in\mathbb{R}$,
then 
\[
e^{-\pi y^{2}}=\int_{\mathbb{R}}e^{-\pi x^{2}}e^{-2\pi ixy}dx.
\]
This gives a new proof of the fact that $e^{-\pi x^{2}}$ is its own
Fourier transform. If $y=0$, then the equality becomes 
\[
1=\int_{\mathbb{R}}e^{-\pi x^{2}}dx.
\]
Now suppose that $y>0$ and consider the function $f(z)=e^{-\pi z^{2}}$,
which is entire, and in particular holomorphic in the interior of
the toy contour $\Gamma_{R}$, where $\Gamma_{r}$ consists of a rectangle
with vertices $R$, $R+iy$, $-R+iy$, $-R$ and the positive counterclockwise
orientation. By Cauchy's theorem, 
\[
\int_{\Gamma_{R}}f(z)dz=0.
\]
The integral over the real segment is simply 
\[
\int_{-R}^{R}e^{-\pi x^{2}}dx,
\]
which converges to $1$ as $R\to\infty$. The integral on the vertical
side on the right is 
\[
I(R)=\int_{0}^{y}f(R+it)idt=\int_{0}^{y}e^{-\pi(R^{2}+2iRt-t^{2})}idt.
\]
This integral goes to $0$ as $R\to\infty$ since $y$ is fixed and
we may estimate it by 
\[
\left|I(R)\right|\leq Ce^{-\pi R^{2}}.
\]
Similarly, the integral over the vertical segment on the left also
goes to $0$ as $R\to\infty$ for the same reasons. Finally, the integral
over the horizontal segment on top is 
\[
\int_{R}^{-R}e^{-\pi(x+iy)^{2}}dx=e^{-\pi x^{2}}\int_{-R}^{R}e^{-\pi x^{2}}e^{-2\pi ixy}dx.
\]
Therefore we find in the limit as $R\to\infty$ that 
\[
0=1-e^{-\pi x^{2}}\int_{-R}^{R}e^{-\pi x^{2}}e^{-2\pi ixy}dx,
\]
and our desired formula is established. In the case $y<0$, we then
consider the symmetric rectangle in the lower half-plane. \end{example}

\subsubsection{Cauchy's Integral Formula}

\begin{theorem}\label{theorem} (Cauchy's Integral Formula) Suppose
$f$ is holomorphic in an open set that contains the closure of a
disc $D$. If $C$ denotes the boundary circle of this disc with positive
orientation, then 
\[
f(z)=\frac{1}{2\pi i}\int_{C}\frac{f(\zeta)}{\zeta-z}d\zeta
\]
for any point $z\in D$. \end{theorem}

\begin{proof} Fix $z\in D$ and consider the ``keyhole'' $\Gamma_{\delta,\varepsilon}$
which omits the point $z$. Here $\delta$ is the width of the corridor,
and $\varepsilon$ is the radius of the small circle centered at $z$.
Since the function $F(\zeta)=f(\zeta)/(\zeta-z)$ is holomorphic away
from the point $\zeta=z$, we have 
\[
\int_{\Gamma_{\delta,\varepsilon}}F(\zeta)d\zeta=0
\]
by Cauchy's theorem for the chosen toy contour. Now we make the corridor
narrower by letting $\delta$ tend to $0$, and use the continuity
of $F$ to see that in the limit, the integrals over the two sides
of the corridor cancel out. The remaining part consists of two curves,
the large boundary circle $C$ with the positive orientation, and
a small circle $C_{\varepsilon}$ centered at $z$ of radius $\varepsilon$
and oriented negatively, that is, clockwise. To see what happens to
the integral over the small circle, we write
\[
F(\zeta)=\frac{f(\zeta)-f(z)}{\zeta-z}+\frac{f(z)}{\zeta-z}
\]
and note that since $f$ is holomorphic, the first term on the right-hand
is bounded so that its integral over $C_{\varepsilon}\to0$ as $\varepsilon\to0$.
To conclude the proof, it suffices to observe that 
\begin{align*}
\int_{C_{\varepsilon}}\frac{f(z)}{\zeta-z}d\zeta & =f(z)\int_{C_{\varepsilon}}\frac{d\zeta}{\zeta-z}\\
 & =-f(z)\int_{0}^{2\pi}\frac{\varepsilon ie^{-it}}{\varepsilon e^{-it}}dt\\
 & =-f(z)2\pi i
\end{align*}
so that in the limit we find 
\[
0=\int_{C}\frac{f(\zeta)}{\zeta-z}d\zeta-2\pi if(z),
\]
as was to be shown. \end{proof}

\begin{rem}\label{rem} The integral over the keyhole contour looks
like this:

\[
\int_{\Gamma_{\delta,\varepsilon}}F(z)dz=2\pi i\int_{\delta}^{1}F(e^{2\pi it})e^{2\pi it}dt+\int_{1}^{\varepsilon}F(t)dt-2\pi i\int_{\delta}^{1}F(\varepsilon e^{2\pi it})\varepsilon e^{2\pi it}dt-\int_{1}^{\varepsilon}F(t+i\delta)dt
\]

By continuity of $F$, the terms $\int_{1}^{\varepsilon}F(t)dt$ and
$\int_{1}^{\varepsilon}F(t+i\delta)dt$ cancel as $\delta\to0$. \end{rem}

\begin{cor}\label{cor} If $f$ is holomorphic in an open set $\Omega$,
then $f$ has infinitely many complex derivatives in $\Omega$. Moreover,
if $C\subset\Omega$ is a circle whose interior is also contained
in $\Omega$, then 
\[
f^{(n)}(z)=\frac{n!}{2\pi i}\int_{C}\frac{f(\zeta)}{(\zeta-z)^{n+1}}d\zeta
\]
for all $z$ in the interior of $C$. \end{cor}

\begin{proof} The proof is by induction on $n$, the case $n=0$
being simply the Cauchy integral formula. Suppose that $f$ has up
to $n-1$ complex derivatives and that 
\[
f^{(n-1)}(z)=\frac{(n-1)!}{2\pi i}\int_{C}\frac{f(\zeta)}{(\zeta-z)^{n}}d\zeta.
\]
Now for small $h$, the difference quotient for $f^{(n-1)}$ takes
the form 
\[
\frac{f^{(n-1)}(z+h)-f^{(n-1)}(z)}{h}=\frac{(n-1)!}{2\pi i}\int_{C}f(\zeta)\frac{1}{h}\left(\frac{1}{(\zeta-z-h)^{n}}-\frac{1}{(\zeta-z)^{n}}\right)d\zeta.
\]
We now recall that 
\[
A^{n}-B^{n}=(A-B)(A^{n-1}+A^{n-2}B+\cdots+AB^{n-2}+B^{n-1})
\]
with $A=1/(\zeta-z-h)$ and $B=1/(\zeta-z)$, we see that the term
in brackets is equal to 
\[
\frac{h}{(\zeta-z-h)(\zeta-z)}(A^{n-1}+A^{n-2}B+\cdots+AB^{n-2}+B^{n-1}).
\]
But observe that if $h$ is small, then $z+h$ and $z$ stay at a
finite distance from the boundary circle $C$, so in the limit as
$h$ tends to $0$, we find that the quotient converges to 
\[
\frac{(n-1)!}{2\pi i}\int_{C}f(\zeta)\left(\frac{1}{(\zeta-z)^{2}}\right)\left(\frac{n}{(\zeta-z)^{n-1}}\right)d\zeta=\frac{n!}{2\pi i}\int_{C}\frac{f(\zeta)}{(\zeta-z)^{n+1}}d\zeta,
\]
which completes the induction argument and proves the theorem.\end{proof}

\subsubsection{Taylor's Theorem}

\begin{theorem}\label{theorem} (Taylor's Theorem) Suppose $f$ is
holomorphic in an open set $\Omega$. If $D$ is a disc centered at
$z_{0}$ and whose closure is contained in $\Omega$, then $f$ has
a power series expansion at $z_{0}$ 
\[
f(z)=\sum_{n=0}^{\infty}a_{n}(z-z_{0})^{n}
\]
for all $z\in D$, and the coefficients are given by 
\[
a_{n}=\frac{f^{(n)}(z_{0})}{n!}
\]
for all $n\geq0$. \end{theorem}

\begin{proof} Fix $z\in D$. By the Cauchy integral formula, we have
\[
f(z)=\frac{1}{2\pi i}\int_{C}\frac{f(\zeta)}{\zeta-z}d\zeta,
\]
where $C$ denotes the boundary of the disc and $z\in D$. The idea
is to write 
\[
\frac{1}{\zeta-z}=\frac{1}{\zeta-z_{0}-(z-z_{0})}=\left(\frac{1}{\zeta-z_{0}}\right)\left(\frac{1}{1-\left(\frac{z-z_{0}}{\zeta-z_{0}}\right)}\right),
\]
and use the geometric series expansion. Since $\zeta\in C$ and $z\in D$
is fixed, there exists $0<r<1$ such that 
\[
\left|\frac{z-z_{0}}{\zeta-z_{0}}\right|<r,
\]
therefore
\[
\frac{1}{1-\left(\frac{z-z_{0}}{\zeta-z_{0}}\right)}=\sum_{n=0}^{\infty}\left(\frac{z-z_{0}}{\zeta-z_{0}}\right)^{n}
\]
where the series converges uniformly for $\zeta\in C$. This allows
us to interchange the infinite sum with the integral, thereby obtaining
\begin{align*}
f(z) & =\frac{1}{2\pi i}\int_{C}\frac{f(\zeta)}{\zeta-z}d\zeta\\
 & =\frac{1}{2\pi i}\int_{C}f(\zeta)\left(\frac{1}{\zeta-z_{0}}\right)\left(\frac{1}{1-\left(\frac{z-z_{0}}{\zeta-z_{0}}\right)}\right)d\zeta\\
 & =\frac{1}{2\pi i}\int_{C}f(\zeta)\left(\frac{1}{\zeta-z_{0}}\right)\sum_{n=0}^{\infty}\left(\frac{z-z_{0}}{\zeta-z_{0}}\right)^{n}d\zeta\\
 & =\sum_{n=0}^{\infty}\left(\frac{1}{2\pi i}\int_{C}\frac{f(\zeta)}{(\zeta-z_{0})^{n+1}}d\zeta\right)(z-z_{0})^{n}.\\
 & =\sum_{n=0}^{\infty}\left(\frac{f^{(n)}(z_{0})}{n!}\right)(z-z_{0})^{n}.\\
 & =\sum_{n=0}^{\infty}a_{n}(z-z_{0})^{n}.
\end{align*}

\end{proof}

\subsubsection{Cauchy's Inequalities}

\begin{cor}\label{cor} (Cauchy's inequality) If $f$ is holomorphic
in a given set that contains the closure of a disc $D$ centered at
$z_{0}$ and of radius $R$, then 
\[
\left|f^{(n)}(z_{0})\right|\leq\frac{n!}{R^{n}}\sup_{z\in C}\left|f(z)\right|.
\]
\end{cor}

\begin{proof} Applying Cauchy's Integral Formula for $f^{(n)}(z_{0})$,
we have 
\begin{align*}
\left|f^{(n)}(z_{0})\right| & =\left|\frac{n!}{2\pi i}\int_{C}\frac{f(\zeta)}{(\zeta-z)^{n+1}}d\zeta\right|\\
 & =\frac{n!}{2\pi}\left|\int_{0}^{2\pi}\frac{f(z_{0}+Re^{it})}{R^{n}e^{int}}d\zeta\right|\\
 & \leq\frac{n!}{R^{n}}\sup_{z\in C}\left|f(z)\right|.
\end{align*}

\end{proof}

\subsubsection{Louiville's Theorem}

\begin{theorem}\label{theoremlouiville} (Louiville's Theorem) Every
bounded entire function must be constant. \end{theorem}

\begin{proof} The theorem follows from the fact that holomorphic
functions are analytic. If $f$ is an entire function, it can be represented
by its Taylor series about $0:$ 
\[
f(z)=\sum_{k=0}^{\infty}a_{k}z^{k}
\]
where 
\begin{align*}
a_{k} & =\frac{f^{(k)}(0)}{k!}\\
 & =\frac{1}{2\pi i}\int_{\Gamma_{r}}\frac{f(z)}{z^{k+1}}dz\\
 & =\int_{0}^{1}\frac{f(re^{2\pi it})}{r^{k}e^{2\pi ikt}}dt.
\end{align*}
and $\Gamma_{r}$ is the circle about $0$ of radius $r>0$. Suppose
$f$ is bounded, i.e. there exists a constant $M$ such that $|f(z)|\leq M$
for all $z$. Then 
\begin{align*}
|a_{k}| & =\left|\int_{0}^{1}\frac{f(re^{2\pi it})}{r^{k}e^{2\pi ikt}}dt\right|\\
 & \leq\int_{0}^{1}\left|\frac{f(re^{2\pi it})}{r^{k}e^{2\pi ikt}}\right|dt\\
 & \leq\frac{M}{r^{k}}.
\end{align*}

Letting $r$ tend to $\infty$ gives us $a_{k}=0$ for all $k\geq1$.
Thus $f(z)=a_{0}$, which proves the theorem. \end{proof}

\subsubsection{Fundamental Theorem of Algebra}

\begin{cor}\label{cor} Every non-constant polynomial $P(z)=a_{n}z^{n}+\cdots+a_{1}z+a_{0}$
with complex coefficients has a root in $\mathbb{C}$. \end{cor}

\begin{proof} If $P(z)$ has no roots, then $Q(z):=1/P(z)$ is a
bounded holomorphic function. To see this, we can of course assume
that $a_{n}\neq0$ and write 
\[
Q(z)=\frac{1}{a_{n}z^{n}+\cdots+a_{1}z+a_{0}}=\left(\frac{1}{z^{n}}\right)\left(\frac{1}{\frac{a_{0}}{z^{n}}+\frac{a_{1}}{z^{n-1}}+\cdots+a_{n}}\right).
\]

As $z\to\infty$, the denominator of the second term in the round
brackets converges to $a_{n}\neq0$, hence the second term itself
goes to $1/a_{n}$. But the first term tends to zero, hence 
\[
\lim_{z\to\infty}Q(z)=0.
\]
In particular, $|Q(z)|$ is bounded by $1$ outside of some circle
$|z|=r$. Inside this circle, $|Q(z)|$ is continuous, hence bounded.
Thus $|Q(z)|$, and therefore $Q(z)$ itself is bounded on the whole
complex plane. By Liousville's theorem, we then conclude that $Q(z)$
is constant. This contradicts our assumption that $P(z)$ is nonconstant
and proves the corollary. \end{proof}

\begin{theorem}\label{theoremidentitytheorem} (Identity Theorem)
Let $f,g$ be holomorphic functions on a connected open set $D$ of
$\mathbb{C}$. If $f=g$ on a nonempty open subset of $D$, then $f=g$
on $D$. \end{theorem}

\begin{rem}\label{remidentitytheorem} This says that a holomorphic
function is completely determined by its values on a (possibly quite
small) neighborhood in $D$. This is not true for real-differentiable
functions. In comparison, holomorphy is a much more rigid notion.
\end{rem}

\begin{proof} Let $S$ be the set of all $z\in D$ such that $f(z)=g(z)$.
We show that $S$ is open and closed, and hence must be $D$. Since
$f-g$ is continuous, and $S=(f-g)^{-1}\{0\}$, we see that $S$ is
closed. To show, that $S$ is open, suppose $w$ lies in $S$. Then,
because the Taylor series of $f$ and $g$ at $w$ have non-zero radius
of convergence, the open disk $B_{r}(w)$ also lies in $S$ for some
$r$. \end{proof}

\subsection{Further Applications}

\subsubsection{Morera's Theorem}

\begin{theorem}\label{morerastheorem} Suppose $f$ is a continuous
function in the open disc $D$ such that for any triangle $T$ cotained
in $D$, 
\[
\int_{T}f(z)dz=0,
\]
then $f$ is holomorphic. \end{theorem}

\begin{proof} By the proof of Theorem~(\ref{theorem2.1}), the function
$f$ has a primitive $F$ in $D$ that satisfies $F'=f.$ By the regularity
theorem, we know that $F$ is indefinitely (and hence twice) complex
differentiable, and therefore $f$ is holomorphic. \end{proof}

\begin{theorem}\label{theorem5.2} If $\{f_{n}\}_{n=1}^{\infty}$
is a sequence of holomorphic functions that converges uniformly to
a function $f$ in every compact subset of $\Omega$, then $f$ is
holomorphic in $\Omega$. \end{theorem}

\begin{proof} Let $D$ be any disc whose closure is contained in
$\Omega$ and $T$ any triangle in that disc. Then, since each $f_{n}$
is holomorphic, Goursat's theorem implies 
\[
\int_{T}f_{n}(z)dz=0
\]
for all $n$. By assumption, $f_{n}\to f$ uniformly in the closure
of $D$, so $f$ is continuous and 
\[
\int_{T}f_{n}(z)dz\to\int_{T}f(z)dz.
\]
As a result, we find $\int_{T}f(z)dz=0$ and by Morera's theorem,
we conclude that $f$ is holomorphic in $D$. Since this conclusion
is true for every $D$ whose closure is contained in $\Omega$, we
find that $f$ is holomorphic in all of $\Omega$. \end{proof}

\subsubsection{Sequence of Holomorphic Functions}

\begin{theorem}\label{theorem5.4} Let $F(z,s)$ be defined for $(z,s)\in\Omega\times[0,1]$
where $\Omega$ is an open set in $\mathbb{C}$. Suppose $F$ satisfies
the following properties:
\begin{enumerate}
\item $F(z,s)$ is holomorphic in $z$ for each $s$.
\item $F$ is continuous on $\Omega\times[0,1]$.
\end{enumerate}
Then the function $f$ defined on $\Omega$ by 
\[
f(z)=\int_{0}^{1}F(z,s)ds
\]
is holomorphic. \end{theorem}

\begin{rem}\label{rem} The second condition says that $F$ is jointly
continuous in both arguments. To prove this result, it suffices to
prove that $f$ is holomorphic in any disc $D$ contained in $\Omega$,
and by Morera's theorem this could be achieved by showing that for
any triangle $T$ contained in $D$ we have 
\[
\int_{T}\int_{0}^{1}F(z,s)dsdz=0.
\]
Interchanging the order of integration, and using property (1) would
then yield the desired result. We can, however, get around the issue
of justifying the change in the order of integration by arguing differently.
The idea is to interpret the integral as a ``uniform'' limit of
Riemann sums, and then apply the results of the previous section.
\end{rem}

\begin{proof} For each $n\geq1,$ we consider the Riemann sum
\[
f_{n}(z)=(1/n)\sum_{k=1}^{n}F(z,k/n).
\]
Then $f_{n}$ is holomorphic in all of $\Omega$ by property (1),
and we claim that on any disc $D$ whose closure is contained in $\Omega,$
the sequence $\{f_{n}\}_{n=1}^{\infty}$ converges uniformly to $f$.
To see this, we recall that a continuous function on a compact set
is uniformly continuous, so if $\varepsilon>0$ there exists $\delta>0$
such that 
\[
\sup_{z\in D}\left|F(z,s_{1})-F(z,s_{2})\right|<\varepsilon\quad\text{whenever }\left|s_{1}-s_{2}\right|<\delta.
\]
Then if $n>1/\delta,$ and $z\in D$ we have 
\begin{align*}
\left|f_{n}(z)-f(z)\right| & =\left|\sum_{k=1}^{n}\int_{(k-1)/n}^{k/n}F(z,k/n)-F(z,s)ds\right|\\
 & \leq\sum_{k=1}^{n}\int_{(k-1)/n}^{k/n}\left|F(z,k/n)-F(z,s)\right|ds\\
 & <\sum_{k=1}^{n}\frac{\varepsilon}{n}\\
 & <\varepsilon.
\end{align*}
This proves the claim, and by Theorem~(\ref{theorem5.2}), we conclude
that $f$ is holomorphic in $D$. As a consequence, $f$ is holomorphic
in $\Omega$, as was to be shown. \end{proof}

\subsubsection{Scwarz reflection principle}

~~~Let $\Omega$ be an open subset of $\mathbb{C}$ that is symmetric
with respect to the real line, that is $z\in\Omega$ if and only if
$\overline{z}\in\Omega$. Let $\Omega^{+}$ denote the part of $\Omega$
that lies in the upper half-plane and $\Omega^{-}$ that part that
lies in the lower half-plane. Also, let $I=\Omega\cap\mathbb{R}$
so that $I$ denotes the interior of that part of the boundary of
$\Omega^{+}$ and $\Omega^{-}$ that lies on the real axis. Then we
have 
\[
\Omega^{+}\cup I\cup\Omega^{-}=\Omega
\]
and the only interesting case of the next theorem occurs, of course,
when $I$ is nonempty. 

\begin{theorem}\label{theoremsymmetryprinciple} If $f^{+}$ and $f^{-}$
are holomorphic functions in $\Omega^{+}$ and $\Omega^{-}$ respectively,
that extend continuously to $I$ and $f^{+}(x)=f^{-}(x)$ for all
$x\in I$, then the function $f$ defined on $\Omega$ by 
\[
f(z)=\begin{cases}
f^{+}(z) & \text{if }z\in\Omega^{+},\\
f^{+}(z)=f^{-}(z) & \text{if }z\in I,\\
f^{-}(z) & \text{if }z\in\Omega^{-}
\end{cases}
\]
is holomorphic on all of $\Omega$. \end{theorem}

\begin{proof} One first notes that $f$ is continuous on $\Omega$.
The only difficulty is to prove that $f$ is holomorphic at points
of $I$. Suppose $D$ is a disc centered at a point on $I$ and entirely
contained in $\Omega$. We prove that $f$ is holomorphic in $D$
by Morera's theorem. Suppose $T$ is a triangle in $D$. If $T$ does
not intersect $I$, then 
\[
\int_{T}f(z)dz=0
\]
since $f$ is holomorphic in the upper and lower half-discs. Suppose
now that one side or vertex of $T$ is contained in $I$, and the
rest of$T$ is in, say the upper half-disc. If $T_{\varepsilon}$
is the triangle obained from $T$ by slightly raising the edge or
vertex which lies on $I$, we have 
\[
\int_{T_{\varepsilon}}f(z)dz=0
\]
since $T_{\varepsilon}$ is entirely contained in the upper half-disc.
We then let $\varepsilon\to0$, and by continuity we conclude that
\[
\int_{T}f(z)dz=0.
\]
If the interior of $T$ intersects $I$, we can reduce the siutation
to the previous one by writing $T$ as the union of triangles each
of which has an edge or vertex on $I$. By Morera's theorem, we conclude
that $f$ is holomorphic in $D$, as was to be shown. \end{proof}

\begin{theorem}\label{theorem} (Schwarz reflection principle) Suppose
that $f$ is a holomorphic function in $\Omega^{+}$ that extends
continuously to $I$ and such that $f$ is real-valued on $I$. Then
there exists a function $F$ holomorphic in all of $\Omega$ such
that $F=f$ on $\Omega^{+}$. \end{theorem}

\begin{proof} The idea is simply to define $F(z)$ for $z\in\Omega^{-}$
by $F(z)=\overline{f(\overline{z})}.$ To prove that $F$ is holomorphic
in $\Omega^{-}$ we note that if $z,z_{0}\in\Omega^{-}$, then $\overline{z},\overline{z}_{0}\in\Omega^{+}$
and hence, the power series expansion of $f$ near $\overline{z}_{0}$
gives 
\[
f(\overline{z})=\sum_{n=0}^{\infty}a_{n}\left(\overline{z}-\overline{z}_{0}\right)^{n}.
\]
As a consequence we see that 
\[
F(z)=\sum_{n=0}^{\infty}\overline{a}_{n}\left(z-z_{0}\right)^{n}
\]
and $F$ is holomorphic in $\Omega^{-}$. Since $f$ is real valued
on $I$, we have $\overline{f(x)}=f(x)$ whenever $x\in I$ and hence
$F$ extends continuously up to $I$. The proof is complete once we
invoke the symmetry principle. \end{proof}

\begin{theorem}\label{theorem4.8} Suppose $f$ is a holomorphic function
in a region $\Omega$ that vanishes on a sequence of distinct points
with a limit point in $\Omega$. Then $f$ is identically $0$. \end{theorem}

\begin{proof} Suppose that $z_{0}\in\Omega$ is a limit point for
the sequence $\{w_{k}\}_{k=1}^{\infty}$ and that $f(\omega_{k})=0$.
First, we show that $f$ is idenitcally zero in a small disc containing
$z_{0}$. For that, we choose a disc $D$ centered at $z_{0}$ and
contained in $\Omega$, and consider the power series expansion of
$f$ in that disc
\[
\sum_{n=0}^{\infty}a_{n}(z-z_{0})^{n}.
\]
If $f$ is not identically zero, there exists a smallest integer $m$
such that $a_{m}\neq0$. But then we can write 
\[
f(z)=a_{m}(z-z_{0})^{m}(1+g(z-z_{0})),
\]
where $g(z-z_{0})$ converges to $0$ as $z\to z_{0}$. Taking $z=w_{k}\neq z_{0}$
for a sequence of points converging to $z_{0}$, we get a contradiction
since $a_{m}(w_{k}-z_{0})^{m}\neq0$ and $1+g(w_{k}-z_{0})\neq0$,
but $f(w_{k})=0$.

~~~We conclude the proof using the fact that $\Omega$ is connected.
Let $U$ denote the interior of the set of points where $f(z)=0$.
Then $U$ is open by definition and nonempty by the argument just
given. The set $U$ is also closed since if $z_{n}\in U$ and $z_{n}\to z$,
then $f(z)=0$ by continuity, and $f$ vanishes in a neighborhood
of $z$ by the argument above. Hence, $z\in U$. Now if we let $V$
denote the complement of $U$ in $\Omega$, we conclude that $U$
and $V$ are both open, disjoint, and 
\[
\Omega=U\cup V.
\]
Since $\Omega$ is connected we conclude that either $U$ or $V$
is empty. Since $z_{0}\in U$, we find that $U=\Omega$ and the proof
is complete. 

\end{proof}

\begin{cor}\label{cor} Suppose $f$ and $g$ are holomorphic in a
region $\Omega$ and $f(z)=g(z)$ for all $z$ in some nonempty open
subset of $\Omega$ (or more generally for $z$ in some sequence of
distinct points with limit point in $\Omega$). Then $f(z)=g(z)$
througout $\Omega$. \end{cor}

~~~Suppose we are given a pair of functions $f$ and $F$ analytic
in regions $\Omega$ and $\Omega'$, respectively, with $\Omega\subset\Omega'$.
If the two functions agree on the smaller set $\Omega$, we say that
$F$ is an \textbf{analytic continuiation }of $f$ into the region
$\Omega'$ The corollary then guarantees that there can be only one
such analytic continuation, since $F$ is uniquely determined by $f$. 

\section{Meromorphic Functions and the Logarithm}

\subsection{Zeros and poles}

~~~A \textbf{point singularity }of a function $f$ is a complex
number $z_{0}$ such that $f$ is defined in a neighborhood of $z_{0}$
but not a the point $z_{0}$ itself. We shall also call such points
\textbf{isolated singularities}. 

\begin{theorem}\label{theorem} (Weierstrass Identity Theorem) Let
$f$ be an analytic function in a region $\Omega$ such that $f$
is not identically zero in $\Omega$. Then each zero of $f$ is isolated.
In particular if $g,h$ are analytic on $\Omega$ and if the set 
\[
E=\{z\in\Omega\mid g(z)=h(z)\}
\]
has an accumulation point in $\Omega$. Then $g(z)=h(z)$ for all
$z\in\Omega$. \end{theorem}

\begin{proof} Define 
\[
\Omega_{1}:=\left\{ z\in\Omega\mid f^{(n)}(z)=0\text{ for all }n\in\mathbb{Z}_{\geq0}\right\} \quad\text{and}\quad\Omega_{2}:=\left\{ z\in\Omega\mid f^{(n)}(z)\neq0\text{ for some }n\in\mathbb{Z}_{\geq0}\right\} .
\]
If $w\in\Omega_{1}$, then $f=0$ on $D_{r}(w)$ for some $r>0$.
Hence, $D_{r}(w)\subset\Omega_{1}$, so $\Omega_{1}$ is open. Similarly,
$\Omega_{2}$ is open since it is the union of the open sets $\Omega_{2,n}:=\left\{ z\in\Omega\mid f^{(n)}(z)\neq0\right\} $
for all $n\in\mathbb{Z}_{n\geq0}$. Since $\Omega=\Omega_{1}\cup\Omega_{2}$
and $\Omega$ is connected, either $\Omega_{1}$ or $\Omega_{2}$
is empty. But $\Omega_{2}\neq\emptyset$ since $f$ is not identically
zero in $\Omega$. Hence $\Omega_{1}=\emptyset$, i.e. $\Omega=\Omega_{2}$.
Therefore all zeros of $f$ in $\Omega$ are isolated. 

~~~To prove the second part of the theorem, simply apply the first
part of the proof to the difference $g-h$. 

\end{proof}

\begin{theorem}\label{theorem} Suppose that $f$ is holomorphic in
a connected open set $\Omega$, has a zero at $z_{0}\in\Omega$, and
does not vanish identically in $\Omega$. Then there exists a neighborhood
$U\subset\Omega$ of $z_{0}$, a nonvanishing holomorphic function
$g$ on $U$, and a unique positive integer $n$ such that $f(z)=(z-z_{0})^{n}g(z)$
for all $z\in U$. \end{theorem}

\begin{proof} Since $\Omega$ is connected and $f$ is not identically
zero, we conclude that $f$ is not identically zero in a neighborhood
of $z_{0}$. In a small disc centered at $z_{0}$ the function $f$
has a power series expansion
\[
f(z)=\sum_{k=0}^{\infty}a_{k}(z-z_{0})^{k}.
\]
Since $f$ is not identically zero near $z_{0}$, there exists a smallest
integer $n$ such that $a_{n}\neq0$. Then, we can write 
\[
f(z)=(z-z_{0})^{n}\left[a_{n}+a_{n+1}(z-z_{0})+\cdots\right]=(z-z_{0})^{n}g(z),
\]
where $g$ is defined by the series in the brackets, and hence is
holomorphic, and is nowhere vanishing for all $z$ close to $z_{0}$.
To prove uniqueness of the integer $n$, suppose that we can also
write 
\[
f(z)=(z-z_{0})^{n}g(z)=(z-z_{0})^{m}h(z)
\]
where $h(z_{0})\ne0$. If $m>n$, then we may divide by $(z-z_{0})^{n}$
to see that 
\[
g(z)=(z-z_{0})^{m-n}h(z)
\]
and letting $z\to z_{0}$ yields $g(z_{0})=0$, a contradiction. We
obtain a similar contradiction if $m<n$. We conclude that $m=n$,
thus $h=g$, and the theorem is proved. 

\end{proof}

\begin{example}\label{example} We want to prove 
\[
\sin(z+w)=\sin(z)\cos(w)+\cos(z)\sin(w),
\]
for all complex numbers $z,x\in\mathbb{C}$. First, fix $w\in\mathbb{R}$
and define 
\[
g(z):=\sin(z+w)\quad\text{and}\quad h(z):=\sin(z)\cos(w)+\cos(z)\sin(w).
\]
These are two holomorphic functions in $z$ and they agree on the
real line, which has an accumulation point. Therefore they must agree
everywhere by the identity theorem. Now, fix $z\in\mathbb{C}$ and
use $w\in\mathbb{C}$ as a variable and use the same argument to get
the formula for all $z,w\in\mathbb{C}$. \end{example}

\begin{rem}\label{rem} \hfill
\begin{enumerate}
\item If $f\neq0$ is analytic on $\Omega$ and $E\subseteq\Omega$ is compact,
then $f$ can have at most finitely many zeros in $E$. This follows
from Bolzano-Weirestrass theorem and Weirestrass identity theorem.
\item Let $f(z)=\sin\left(\frac{1}{z}\right)$ for all $z\in\mathbb{C}\backslash\{0\}$.
Then $f(z)=0$ if $z=\frac{1}{n\pi}$ where $n\in\mathbb{Z}$, however
$f\neq0$ since the accumulation point of the set $\left\{ \frac{1}{n\pi}\mid n\in\mathbb{Z}\right\} $
is $0$, which is not in the region $\mathbb{C}\backslash\{0\}$. 
\end{enumerate}
\end{rem}

\begin{defn}\label{defn} We say $z_{0}$ is an \textbf{isolated singularity
}for $f$ if $f$ is not analytic at $z_{0}$ but there exists $r>0$
such that $f$ is analytic on $D_{r}(z_{0})\backslash\{z_{0}\}$.
\end{defn}

\subsection{Review of Laurent Series Expansion}

~~~Consider the series of the form 
\[
\sum_{n=1}^{\infty}b_{n}(z-z_{0})^{-n}=\sum_{n=1}^{\infty}b_{n}w^{n},
\]
where $w=\frac{1}{z-z_{0}}$. The series converges for $|w|<R$ and
diverges for $|w|>R$ where $R\in[0,\infty]$ is the radius of convergence.
Note that $|w|<R$ if and only if $|z-z_{0}|>\frac{1}{R}$. Hence,
the series converges whenever $|z-z_{0}|>\frac{1}{R}$ and diverges
whenever $|z-z_{0}|<\frac{1}{R}$. 

~~~The series $\sum_{n=-\infty}^{\infty}a_{n}(z-z_{0})^{n}$ is
formally
\[
\sum_{n=-\infty}^{\infty}a_{n}(z-z_{0})^{n}=\sum_{n=1}^{\infty}a_{-n}(z-z_{0})^{-n}+\sum_{n=0}^{\infty}a_{n}(z-z_{0})^{n}.
\]
We call $\sum_{n=0}^{\infty}a_{n}(z-z_{0})^{n}$ the \textbf{regular
part }of $\sum_{n=-\infty}^{\infty}a_{n}(z-z_{0})^{n}$, and we call
$\sum_{n=1}^{\infty}a_{-n}(z-z_{0})^{-n}$ the \textbf{principal part
}of $\sum_{n=-\infty}^{\infty}a_{n}(z-z_{0})^{n}$. There exists $R_{1},R_{2}\in[0,\infty]$
such that the regular part converges for $|z-z_{0}|<R_{1}$, diverges
for $|z-z_{0}|>R_{1}$, and the principal part converges for $|z-z_{0}|>R_{2}$,
and diverges for $|z-z_{0}|<R_{2}$. If $R_{1}>R_{2}$, we get an
\textbf{annulus of convergence }which is
\[
\left\{ z\in\mathbb{C}\mid R_{2}<|z-z_{0}|<R_{1}\right\} .
\]
If $R_{2}>R_{1}$, then the series $\sum_{n=-\infty}^{\infty}a_{n}(z-z_{0})^{n}$
diverges everywhere. If $R_{1}=R_{2}$, then the only possible place
for convergence of the series $\sum_{n=-\infty}^{\infty}a_{n}(z-z_{0})^{n}$
is the circle $|z-z_{0}|=R_{1}=R_{2}$. In this case, we obtain no
information.

\begin{example}\label{example} Let $f(z)=(2z-1)^{-1}(1-z)^{-1}$.
We want to express this as a doubly infinite power series for $\left\{ z\in\mathbb{C}\mid|z|<\frac{1}{2}\right\} $,
$\left\{ z\in\mathbb{C}\mid\frac{1}{2}<|z|<1\right\} $, and $\left\{ z\in\mathbb{C}\mid|z|>1\right\} $. 

\hfill

\textbf{Case 1: }Write
\begin{align*}
f(z) & =\frac{-2}{1-2z}+\frac{1}{1-z}\\
 & =-2\sum_{n=0}^{\infty}\left(2z\right)^{n}+\sum_{n=0}^{\infty}z^{n}.
\end{align*}
The first series converges if $|z|<\frac{1}{2}$ and the second converges
if $|z|<1$. Therefore the sum of the two series converges in the
set $\left\{ z\in\mathbb{C}\mid|z|<\frac{1}{2}\right\} $.

\hfill

\textbf{Case 2: }Write 
\begin{align*}
f(z) & =\frac{z^{-1}}{1-(2z)^{-1}}+\frac{1}{1-z}\\
 & =z^{-1}\sum_{n=0}^{\infty}\left(2z\right)^{-n}+\sum_{n=0}^{\infty}z^{n}.
\end{align*}

The first series converges if $|z|>\frac{1}{2}$ and the second series
converges if $|z|<1$. Therefore the sum of the two series converges
in the set $\left\{ z\in\mathbb{C}\mid\frac{1}{2}<|z|<1\right\} $.

\hfill

\textbf{Case 3: }Write 
\begin{align*}
f(z) & =\frac{z^{-1}}{1-(2z)^{-1}}+\frac{-z^{-1}}{1-z^{-1}}\\
 & =z^{-1}\sum_{n=0}^{\infty}\left(2z\right)^{-n}-z^{-1}\sum_{n=0}^{\infty}z^{-n}.
\end{align*}

The first series converges if $|z|>\frac{1}{2}$ and the second series
converges if $|z|>1$.Therefore the sum of the two series converges
in the set $\left\{ z\in\mathbb{C}\mid|z|>1\right\} $.

\end{example}

\begin{rem}\label{rem} Observe that the function $f$ in the previous
example has a pole at $z=\frac{1}{2}$ and $z=1$. \end{rem}

~~~Let $a,b,r,r_{1},r_{2}\in[0,\infty]$ such that $a<r_{2}<r<r_{1}<b$.
Suppose $\sum_{n=-\infty}^{\infty}a_{n}(z-z_{0})^{n}$ converges for
$a<|z-z_{0}|<b$. Let $\Gamma=\left\{ z_{0}+re^{it}\mid0\leq t\leq2\pi\right\} $,
where $a<r<b$. The regular part of the series converges uniformly
in the disc $\overline{D}_{r_{1}}(z_{0})$. The principal part of
the series converges uniformly for $|z-z_{0}|\geq r_{2}$. Hence the
series converges uniformly on the annulus $\left\{ z\in\mathbb{C}\mid r_{2}\leq|z-z_{0}|\leq r_{1}\right\} $,
and hence also on $\Gamma$. Therefore 
\begin{align*}
\int_{\Gamma}\left(\sum_{n=-\infty}^{\infty}a_{n}(z-z_{0})^{n}\right)dz & =\sum_{n=-\infty}^{\infty}a_{n}\int_{\Gamma}(z-z_{0})^{n}dz\\
 & =a_{-1}\int_{\Gamma}(z-z_{0})^{-1}dz\\
 & =2\pi ia_{-1}.
\end{align*}

~~~Next, let $m\in\mathbb{Z}$ and let $g$ be given by
\begin{align*}
g(z) & =(z-z_{0})^{-m-1}f(z)\\
 & =\sum_{n=-\infty}^{\infty}a_{n}(z-z_{0})^{n-m-1}\\
 & =\sum_{n=-\infty}^{\infty}a_{n+m+1}(z-z_{0})^{n}.
\end{align*}
Then by the previous argument, we have $\int_{\Gamma}g(z)dz=2\pi ia_{m}$.
Therefore
\[
a_{m}=\frac{1}{2\pi i}\int_{\Gamma}\frac{f(z)}{(z-z_{0})^{m+1}}dz,.
\]
where the $a_{m}$ are uniformly determined. 
\[
|a_{m}|\leq\frac{1}{r^{m}}\left\Vert f\right\Vert _{\Gamma}.
\]

\begin{theorem}\label{theorem} (Laurent's Theorem) Let $\Omega=\{z\in\mathbb{C}\mid a<|z-z_{0}|<b\}$,
where $0\leq a<b\leq\infty$. If $f$ is analytic on $\Omega$, then
$f$ can be expanded as a power series of the form 
\[
f(z)=\sum_{n=-\infty}^{\infty}a_{n}(z-z_{0})^{n},
\]
where $z\in\Omega$. The series converges uniformly on any closed
annulus
\[
\left\{ z\in\mathbb{C}\mid a_{1}\leq|z-z_{0}|\leq b_{1}\right\} ,
\]
where $a<a_{1}\leq b_{1}<b$. If $a<r<b$ and $\Gamma=\{z_{0}+re^{it}\mid0\leq t\leq2\pi\},$
then for all $m\in\mathbb{Z}$, we have 
\[
a_{m}=\frac{1}{2\pi i}\int_{\Gamma}\frac{f(z)}{(z-z_{0})^{m+1}}dz.
\]

\end{theorem}

\begin{example}\label{example} By considering the Laurent series
of $\left(z+\frac{1}{z}\right)^{m}$ where $m\in\mathbb{N}$, we evaluate
\[
\int_{-\pi}^{\pi}\cos(n\theta)\cos^{m}(\theta)d\theta,
\]
where $n\in\mathbb{N}$. The function $f(z)=\left(z+\frac{1}{z}\right)^{m}$
is analytic except at $z=0$. Using binomial theorem, write
\begin{align*}
f(z) & =\sum_{k=0}^{m}{m \choose k}z^{m-k}z^{-k}\\
 & =\sum_{k=0}^{m}{m \choose k}z^{m-2k}.
\end{align*}
By Laurent's Theorem, $f(z)=\sum_{n=-\infty}^{\infty}a_{n}z^{n}$,
where 
\[
a_{n}=\frac{1}{2\pi i}\int_{\Gamma}\frac{\left(z+\frac{1}{z}\right)^{m}}{z^{n+1}}dz\quad\text{and}\quad\Gamma=\left\{ e^{i\theta}\mid-\pi\leq\theta\leq\pi\right\} .
\]
 So 
\begin{align*}
a_{n} & =\frac{1}{2\pi i}\int_{-\pi}^{\pi}\frac{(e^{i\theta}+e^{-i\theta})^{m}}{e^{i(n+1)\theta}}ie^{i\theta}d\theta\\
 & =\frac{1}{2\pi}\int_{-\pi}^{\pi}\frac{2^{m}\cos^{m}(\theta)}{e^{in\theta}}d\theta\\
 & =\frac{2^{m-1}}{\pi}\left(\int_{-\pi}^{\pi}\cos^{m}(\theta)\cos(n\theta)d\theta-\int_{-\pi}^{\pi}\cos^{m}(\theta)\sin(n\theta)d\theta\right)\\
 & =\frac{2^{m-1}}{\pi}\int_{-\pi}^{\pi}\cos^{m}(\theta)\cos(n\theta)d\theta\\
 & =0
\end{align*}
where $\int_{-\pi}^{\pi}\cos^{m}(\theta)\sin(n\theta)d\theta=0$ since
$\sin(n\theta)\cos^{m}(\theta)$ is odd. 

\end{example}

~~~Suppose $f$ is analytic on $\Omega=\{0<|z-z_{0}|<r\}$. Recall
that the \textbf{Laurent series }for $f$ at $z_{0}$ is given by
\[
f(z)=\sum_{n=-\infty}^{\infty}a_{n}(z-z_{0})^{n}=\sum_{n=-\infty}^{-1}a_{n}(z-z_{0})^{n}+\sum_{n=0}^{\infty}a_{n}(z-z_{0})^{n},
\]
where $\sum_{n=-\infty}^{-1}a_{n}(z-z_{0})^{n}$ is the \textbf{principal
part} of $f$ and $\sum_{n=0}^{\infty}a_{n}(z-z_{0})^{n}$ is the
\textbf{regular part }of $f$. 

\begin{defn}\label{defn} Let $f$ be analytic on $\Omega=\{0<|z-z_{0}|<r\}$
and assume that $f$ has an isolated singularity at $z_{0}$. We have
the following trichotomy of isolated singularities: 
\begin{enumerate}
\item \textbf{Removable Singularity}: The principal part of the Laurent
series of $f$ is identically zero. 
\item \textbf{Pole of Order $n\geq1$}: The principal part of the Laurent
series of $f$ is a polynomial in $(z-z_{0})^{-1}$ of order $n$,
i.e. $a_{k}=0$ for all $k<-n$ but $a_{-n}\neq0$. 
\item \textbf{Essential Singularity}: The principal part of the Laurent
series of $f$ is is an infinite series, i.e. $a_{n}\neq0$ for infinitely
many negatives values of $n$. 
\end{enumerate}
\end{defn}

\begin{rem}\label{rem} \hfill
\begin{enumerate}
\item Suppose $f$ has a removable singularity at $z_{0}$. Then 
\[
f(z)=\sum_{n=0}^{\infty}a_{n}(z-z_{0})^{n}\quad\text{for }0<|z-z_{0}|<r.
\]
If we redefine $f(z_{0})$ to be $a_{0}$, then we remove the singularity,
i.e. $f$ is then analytic at $z_{0}$.
\item Suppose $f$ has a pole of order $n$ at $z_{0}$. Then 
\begin{align*}
f(z) & =a_{-n}(z-z_{0})^{-n}+a_{-n+1}(z-z_{0})^{-n+1}+\cdots\\
 & =(z-z_{0})^{-n}(a_{-n}+a_{-n+1}(z-z_{0})+\cdots)\\
 & =(z-z_{0})^{-n}g(z),
\end{align*}
where $g(z)=a_{-n}+a_{-n+1}(z-z_{0})+\cdots$ is analytic at $z_{0}$
and $g(z_{0})=a_{-n}\neq0$. So we can choose $\delta>0$ such that
$g(z)\neq0$ for all $D_{\delta}(z_{0})$. Taking reciprocals, we
have 
\begin{equation}
\frac{1}{f(z)}=(z-z_{0})^{n}\frac{1}{g(z)}\quad\text{for all }z\in D_{\delta}(z_{0})\backslash\{z_{0}\}.\label{eq:recip}
\end{equation}
If we define the left-hand side of (\ref{eq:recip}) to be zero at
$z_{0}$, then we see that $\frac{1}{f}$ has a zero of order $n$
at $z_{0}$. 
\item Suppose $f$ has a zero of order $n$ at $z_{0}$. Then 
\[
f(z)=(z-z_{0})^{n}h(z),
\]
where $h(z_{0})\neq0$ and $h$ is analytic at $z_{0}$. Then
\[
\frac{1}{f(z)}=(z-z_{0})^{-n}\frac{1}{h(z)}\quad\text{for all }0<|z-z_{0}|<r
\]
for some $r>0$. Then 
\begin{align*}
\frac{1}{f(z)} & =(z-z_{0})^{-n}\sum_{n=0}^{\infty}b_{n}(z-z_{0})^{n}\\
 & =(z-z_{0})^{-n}b_{0}+\cdots,
\end{align*}
has a pole of order $n$ at $z_{0}$.
\end{enumerate}
\end{rem}

\begin{defn}\label{defn} \hfill
\begin{enumerate}
\item A zero of order $1$ is called a \textbf{simple zero}. 
\item A pole of order $1$ is called a \textbf{simple pole}. 
\end{enumerate}
\end{defn}

\begin{theorem}\label{theorem} Let $z_{0}$ be an isolated singularity
for $f$. Then
\begin{enumerate}
\item $z_{0}$ is removable if and only if $f$ is bounded on $D_{r}(z_{0})\backslash\{z_{0}\}$
for some $r>0$.
\item $z_{0}$ is a pole if and only if $|f(z)|\to\infty$ as $z\to z_{0}$. 
\item (Casorati-Weierstrass) $z_{0}$ is an essential singularity if and
only if for each $w\in\mathbb{C}$ and any $\rho,\varepsilon>0$,
there exists $z\in D_{\rho}(z_{0})\backslash\{z_{0}\}$ such that
$f(z)\in D_{\varepsilon}(w)$. 
\end{enumerate}
\end{theorem}

\begin{rem}\label{rem} The result in (3) means that $f$ takes values
arbitrarily close to every complex number in every deleted disc with
center $z_{0}$. In other words, the image of $D_{\rho}(z_{0})\backslash\{z_{0}\}$
under $f$ is dense in $\mathbb{C}$. \end{rem}

\begin{proof} \hfill
\begin{enumerate}
\item If $z_{0}$ is a removable singularity, then $f$ becomes analytic
at $z_{0}$ if we define $f$ at $z_{0}$ properly. Hence, for $r>0$
sufficiently small, $f$ is bounded on $D_{r}(z_{0})$. Conversely,
let $f$ be bounded, say, $|f(z)|\leq M$ for $z\in D_{r}(z_{0})\backslash\{z_{0}\}$.
Let $\Gamma$ be the circle of radius $r/2$ centered at $z_{0}$.
Then by Cauchy's inequality for Laurent coefficients
\begin{align*}
|a_{n}| & \leq\frac{2^{n}}{r^{n}}\sup_{z\in\Gamma}f(z) & |a_{-n}| & \leq\frac{r^{n}}{2^{n}}\sup_{z\in\Gamma}f(z)\\
 & \leq\left(\frac{2}{r}\right)^{n}M &  & \leq\left(\frac{r}{2}\right)^{n}M
\end{align*}
 for all $n\in\mathbb{Z}$. Letting $r\to0^{+}$, we get $a_{-n}=0$
for all $n\geq1$. 
\item Assume $z_{0}$ is a pole of $f$. Then $z_{0}$ is a zero for $g:=1/f$.
So $|g(z)|\to0$ as $z\to z_{0}$. Hence $|f(z)|\to\infty$ as $z\to z_{0}$.
Conversely, if $|f(z)|\to\infty$ as $z\to z_{0}$, then there exists
$r>0$ such that $|f(z)|\geq1$ for all $z\in D_{r}(z_{0})$. Hence,
$g$ is analytic in $D_{r}(z_{0})\backslash\{z_{0}\}$ and $|g(z)|\leq1$
for all $z\in D_{r}(z_{0})$. By (1), $g$ has a removable singularity
at $z_{0}$ and $g(z)\to0$ as $z\to z_{0}$. Define $g(z_{0}):=0$.
Then $g$ is analytic at $z_{0}$ with a zero of some order $n$.
Hence, $f$ has a pole of order $n$ at $z_{0}$. 
\item If $f$ has the given property, then $f$ is not bounded near $z_{0}$
and $|f(z)|\not\to\infty$ as $z\to z_{0}$. Hence, $z_{0}$ must
be an essential singuarility by (1) and (2). Converseley, let $z_{0}$
be an essential singularity for $f$. Suppose that for some $w,\rho,\varepsilon$,
we cannot find a suitiable $z$. Hence, for all $z\in D_{\rho}(z_{0})\backslash\{z_{0}\}$,
we have $|f(z)-w|\geq\varepsilon$. Define
\begin{align*}
F(z) & :=\frac{1}{f(z)-w}.
\end{align*}
Then $f(z)=w+\frac{1}{F(z)}$ for all $z\in D_{\rho}(z_{0})\backslash\{z_{0}\}$
and $F$ is analytic in $D_{\rho}(z_{0})\backslash\{z_{0}\}$ and
$|F(z)|\leq\frac{1}{\varepsilon}$. Hence, $F$ has a removable singularity
at $z_{0}$, by (1). Now redefine $F$ at $z_{0}$ so that it becomes
analytic in $D_{\rho}(z_{0})$. 
\begin{enumerate}
\item \textbf{Case 1}: Suppose $F(z_{0})\neq0$. Then $f(z)\to w+\frac{1}{F(z_{0})}$
as $z\to z_{0}$. Hence $f$ has a removable singularity at $z_{0}$,
which is impossible.
\item \textbf{Case 2}: Suppose $F(z_{0})=0$. Then $|f(z)|\to\infty$ as
$z\to z_{0}$. Therefore $f$ as a pole at $z_{0}$, which is impossible. 
\end{enumerate}
\end{enumerate}
\end{proof}

\begin{theorem}\label{theorem} (Picard's Great Theorem) If $z_{0}$
is an essential (isolated) singularity of $f$ with at most one exception,
$f$ takes every complex value in every deleted disc $D_{r}(z_{0})\backslash\{z_{0}\}$.
\end{theorem}

\begin{proof} Omitted. \end{proof}

\begin{example}\label{example} \hfill
\begin{enumerate}
\item Let $f(z)=e^{1/z}$. Then $0$ is an essential singularity for $f$.
Moreover, in any $D_{r}(0)\backslash\{0\}$, $f$ takes every complex
value except $0$.
\item Let $g(z)=\sin(1/z)$. Then $0$ is an essential singularity for $g$.
Moreover, in any $D_{r}(0)\backslash\{0\}$, $g$ takes every complex
value. 
\end{enumerate}
\end{example}

\subsection{The Riemann Sphere}

~~~The extended complex plane, which consists of $\mathbb{C}$
and the point at infinity, has a convenient geometric interpetation,
which we briefly discuss here. 

~~~Consider the Euclidean space $\mathbb{R}^{3}$ with coordinates
$(X,Y,Z)$ where the $XY$-plane is identified with $\mathbb{C}$.
We denote by $\mathbb{S}$ the sphere centered at $(0,0,1/2)$ and
of radius $1/2$; this sphere is of unit diameter and lies on top
of the origin of the complex plane. Also, we let $\mathcal{N}=(0,0,1)$
denote the north pole of the sphere. Note that $(X,Y,Z)\in\mathbb{S}$
if and only if $X^{2}+Y^{2}+\left(Z-\frac{1}{2}\right)^{2}=\frac{1}{4}$. 

~~~Given any point $W=(X,Y,Z)$ on $\mathbb{S}$ different from
the north pole, the line joining $\mathcal{N}$ and $W$ intersects
the $XY$-plane in a single point which we denote by $w=x+iy$; $w$
is called the \textbf{stereographic projection }of $W$. Conversely,
given any point $w$ in $\mathbb{C}$, the line joining $\mathcal{N}$
and $w=(x,y,0)$ intersects the sphere at $\mathcal{N}$ and another
point, which we call $W$. This geometric construction gives a bijective
correspondence between points on the punctured sphere $\mathbb{S}\backslash\{\mathcal{N}\}$
and the complex plane; it is described analytically by the formulas
\[
x=\frac{X}{1-Z}\quad\text{and}\quad y=\frac{Y}{1-Z},
\]
giving $w$ in terms of $W$, and 
\[
X=\frac{x}{x^{2}+y^{2}+1},\quad Y=\frac{y}{x^{2}+y^{2}+1},\quad\text{and}\quad Z=\frac{x^{2}+y^{2}}{x^{2}+y^{2}+1},
\]
giving $W$ in terms of $w$. Intuitively, we have wrapped the complex
plane onto the punctured sphere $\mathbb{S}\backslash\{\mathcal{N}\}$.
For instance, the complex number $it\in\mathbb{C}$ maps to the point
$\left(0,\frac{t}{t^{2}+1},\frac{t^{2}}{t^{2}+1}\right)\in\mathbb{S}$.

~~~As the point $w$ goes to infinity in $\mathbb{C}$ (in the
sense that $|w|\to\infty$), the corresponding point $W$ on $\mathbb{S}$
comes arbitrarily close to $\mathcal{N}$. This simple observation
makes $\mathcal{N}$ a natural candidate for the so-called ``point
at infinity''. Identifying infinity with the point $\mathcal{N}$
on $\mathbb{S}$, we see that the extended complex plane can be visualized
as the full two-dimensional sphere $\mathbb{S}$; this is the \textbf{Riemann
sphere}. Since this construction takes the unbounded set $\mathbb{C}$
onto the compact set $\mathbb{S}$ by adding one point, the Riemann
sphere is sometimes called the \textbf{one-point compactification}
of $\mathbb{C}$. 

\subsubsection{Singularities}

\begin{defn}\label{defn} $\infty$ is called an \textbf{isolated
singularity }for $f$ if $f$ is analytic on some neighborhood of
$\infty$. \end{defn}

\begin{example}\label{example} Let $f(z)=\frac{1}{z^{2}-1}$ and
$g(z)=\frac{1}{\sin z}$. Then $f$ is not analytic at $z=\pm1$,
but outside $\pm1$, it is analytic. On the other hand, $g$ is not
analytic in any neighborhood of $\infty$. \end{example}

\begin{defn}\label{defn} Assume $f$ has an isolated singularity
at $\infty$. Define $g(z):=f(1/z)$ for $0<|z|<1/r$, where $f$
is analytic on $D_{r}(\infty)$. $f$ is said to have a
\begin{enumerate}
\item \textbf{removable singularity at} $\infty$ if $g$ has a removable
singularity at $0$.
\item \textbf{pole of order $n$ at} $\infty$ if $g$ has a pole of order
$n$ at $0$.
\item \textbf{essential singularity at} $\infty$ if $g$ has an essential
singularity at $0$.
\end{enumerate}
If the Laurent Series for $g$ is 
\[
g(z)=\sum_{n=-\infty}^{\infty}a_{n}z^{n},
\]
where $0<|z|<1/r$. Then principal part of $g$ is 
\[
\sum_{n=-\infty}^{-1}a_{n}z^{n}=\sum_{n=1}^{\infty}a_{-n}z^{-n}.
\]
We define the \textbf{Laurent series }for $f$ at $\infty$ to be
\[
f(z)=g\left(\frac{1}{z}\right)=\sum_{n=-\infty}^{\infty}a_{n}z^{-n},
\]
where $|z|>r$. And we define the \textbf{principal part }of the Laurent
series for $f$ at $\infty$ to be 
\[
\sum_{n=1}^{\infty}a_{-n}z^{n}.
\]

\end{defn}

\begin{rem}\label{rem} \hfill
\begin{enumerate}
\item $f$ has a removable singularity at $\infty$ if and only if the principal
part of the Laurent series for $f$ at $\infty$ vanishes, i.e. $a_{-n}=0$
for all $n\geq1$.
\item $f$ has a pole at $\infty$ of order $n$ if and only if the principal
part of the Laurent series for $f$ at $\infty$ is a polynomial in
$z$ of order $n$. 
\item $f$ has an essential singularity at $\infty$ if and only if the
principal part of the Laurent series for $f$ at $\infty$ is an infinite
power series in $z$, i.e. $a_{-n}\neq0$ for infinitely many $n$. 
\end{enumerate}
\end{rem}

\begin{example}\label{example} Let $f$ be given by $f(z)=e^{z}$.
Then 
\[
f\left(\frac{1}{z}\right)=\sum_{n=0}^{\infty}\frac{z^{-n}}{n!}
\]
is the Laurent Series for $f$ at $\infty$. Since the principal part
of the Laurent Series for $f$ at $\infty$ has infinitely many terms,
we see that $f$ has an essential singularity at $\infty$. \end{example}

\begin{example}\label{example} Let $f$ be given by $f(z)=\frac{1}{z}+1+2z+z^{3}$.
Then $f$ has a pole at $\infty$ of order $3$. \end{example}

\begin{example}\label{example} Let $f$ be given by $f(z)=\frac{1}{\sin z}$.
Then $f$ does not have an isolated singularity at $\infty$. \end{example}

\begin{defn}\label{defn} Let $f:D\to\mathbb{C}$, where $D$ is an
open set in $\mathbb{C}_{\infty}$. We say that $f$ is \textbf{meromorphic
}on $D$ if it is analytic on $D$, except possibly for a discrete
set of poles. \end{defn}

\begin{theorem}\label{theorem}\hfill
\begin{enumerate}
\item $f$ is analytic on $C_{\infty}$ if and only $f$ is constant. 
\item $f$ is entire with a pole at $\infty$ if and only if $f$ is a polynomial.
\item $f$ is meromorphic on $C_{\infty}$ if and only if $f$ is a rational
function. 
\end{enumerate}
\end{theorem}

\begin{proof}\hfill
\begin{enumerate}
\item First assume $f$ is analytic on $C_{\infty}$. Since $f$ is analytic
at $\infty$, it must be bounded on some $\{z\mid|z|>r\}$. Since
$f$ is analytic on $\mathbb{C}$, it must be bounded everywhere.
Now apply Liouville's theorem. The converse direction is trivial.
\item First assume $f$ is entire. Write
\[
f(z)=a_{0}+\sum_{m=1}^{\infty}a_{m}z^{m}.
\]
Since $f$ has a pole at $\infty$ of order $n$, we must have $a_{m}=0$
for all $m>n$. Therefore $f$ is a polynomial. The converse direction
is trivial.
\item First assume $f(z)=\frac{p(z)}{q(z)}$, where $p,q$ are polynomials
with no common factors. Then $f$ is analytic in $\mathbb{C}$ except
possibly at finitely many points $z$ where $g(z)=0$. Suppose $p(z)=a_{n}z^{n}+a_{n-1}z^{n-1}\cdots+a_{0}$
and $g(z)=b_{m}z^{m}+b_{m-1}z^{m-1}\cdots+b_{0}.$ Then
\[
f(z)=z^{n-m}\left(\frac{a_{n}+\frac{a_{n-1}}{z}\cdots+\frac{a_{0}}{z^{n}}}{b_{m}+\frac{b_{m-1}}{z}\cdots+\frac{b_{0}}{z^{m}}}\right)=z^{n-m}h(z),
\]
where $h$ is analytic at $\infty$. If $n>m$, then $f$ has a pole
at $\infty$. If $n\leq m$, then $f$ is analytic at $\infty$. Hence,
$f$ is meromorphic on $\mathbb{C}_{\infty}$. Conversely, assume
$f$ is meromorphic on $\mathbb{C}_{\infty}$. First we claim that
$f$ can have at most finitely many poles $z_{1},\dots,z_{n}\in\mathbb{C}_{\infty}$.
To see this, note that $f$ is analytic on $\{|z|>r\}$ for some $r>0$.
In $\overline{B_{r}(0)}$, there exists a limit point $\xi$. If $\xi$
is not a pole, then $f$ cannot be analytic at $\xi$, which is a
contradiction. (If $\xi$ is a pole, it is not an isolated singularity).
Now, take $P_{r}(z)$ to be the principal part of the Laurent series
of $f$ at $z_{r}$. If $z_{r}=\infty$, then $P_{r}(z)$ is a polynomial.
If $z_{r}\neq\infty$, then 
\[
P_{r}(z)=\sum_{j=1}^{m_{r}}b_{j}(z-z_{r})^{-j}.
\]
Define $g:=f-\sum_{r=1}^{n}P_{r}$. Then $g$ is a function which
has removable singularities at the points $z_{r}$, and $g$ is analytic
elsewhere. Now remove the singularities so that $g$ is analytic on
$\mathbb{C}_{\infty}$. Therefore $g$ must be constant by part 1.
Therefore 
\[
f=\text{constant}+\sum_{r=1}^{n}P_{r},
\]
which is a rational function. 
\end{enumerate}
\end{proof}

\begin{example}\label{example} Let $f$ be entire and assume $f(z)\leq K+L|z|^{m}$
for some constants $K,L$ and $m\in\mathbb{N}$. We will show that
$f$ is a polynomial of degree $\leq m$. Let $g$ be given by 
\[
g(z):=\frac{f(z)}{|z|^{m}}\leq\frac{K}{|z|^{m}}+L.
\]

Then $g$ is meromorphic on $\mathbb{C}$ and $g$ is bounded near
$\infty$. Hence, $g$ is meromorphic in $\mathbb{C}_{\infty}$. Therefore
$g$ is a rational function and hence, so is $f$. Since $f$ remains
bounded as $z$ tends to any finite limit, $f$ has no poles in $\mathbb{C}$,
so $f$ must be a polynomial. Finally, since $\frac{f(z)}{|z|^{m}}$
is bounded as $z\to\infty$, $\text{deg}(f)\leq m$. \end{example}

\section{Residues}

\begin{theorem}\label{theorem} (General form of Cauchy's Theorem)
Let $f$ be analytic on a region $\Omega$ and let $\Gamma$ be a
contour with $\Gamma\cup\text{Int}(\Gamma)\subseteq\Omega$. Then
\[
\int_{\Gamma}f(z)dz=0.
\]
\end{theorem}

\begin{proof} Omitted.\end{proof}

\begin{defn}\label{defn} Let $\sum_{n=-\infty}^{\infty}a_{n}(z-z_{0})^{n}$
be the Laurent series for $f$ at $z_{0}$. The coefficient $a_{-1}$
is called the \textbf{residue }of $f$ at $z_{0}$. We denote this
as $\text{res}_{z_{0}}(f)$ or $\text{res}(f,z_{0})$. \end{defn}

\begin{rem}\label{rem} Recall that 
\[
a_{-1}=\frac{1}{2\pi i}\int_{\Gamma}f(z)dz.
\]
\end{rem}

\subsubsection{Cauchy's Residue Theorem}

\begin{theorem}\label{theoremcauchyresiduetheorem} (Cauchy's residue
theorem) Let $\Gamma$ be a positively oriented simple contour in
$\mathbb{C}$ and let $f$ be analytic on $\Gamma$ and meromorphic
on $\text{Int}(\Gamma)$ with poles at $z_{1},\dots,z_{n}\in\text{Int}(\Gamma)$.
Then 
\[
\frac{1}{2\pi i}\int_{\Gamma}f(z)dz=\sum_{i=1}^{n}\text{res}_{z_{i}}(f).
\]

\end{theorem}

\begin{proof} Let $P_{r}(z)$ to be the principal part of the Laurent
series for $f$ at $z_{r}$. Then define $g$ by 
\[
g(z):=f(z)-\sum_{r=1}^{n}P_{r}(z).
\]
 Then $g$ is analytic on $\Gamma\cup\text{Int}(\Gamma)$ except for
removable singularity. Now we apply Cauchy's Theorem to get 
\[
\int_{\Gamma}g(z)dz=0,
\]
so 
\begin{align*}
\frac{1}{2\pi i}\int_{\Gamma}f(z)dz & =\frac{1}{2\pi i}\sum_{r=1}^{n}\int_{\Gamma}P_{r}(z)dz\\
 & =\sum_{r=1}^{n}\text{res}(f,z_{r}).
\end{align*}
\end{proof}

\subsubsection{Argument Principle}

\begin{theorem}\label{theoremargumentprinciple} (Argument Principal)
Let $\Omega\subseteq\mathbb{C}$ be a region, $\Gamma$ be a positive
contour with $\Gamma\cup\text{Int}(\Gamma)\subseteq\Omega$, and let
$f$ be meromorphic on $\Omega$ with no zeros or poles on $\Gamma$.
Suppose $f$ has zeros at $a_{j}$ of order $\omega_{j}$ in $\text{Int}(\Gamma)$
and poles at $b_{k}$ of order $\rho_{k}$ in $\text{Int}(\Gamma)$.
Then
\[
\frac{1}{2\pi i}\int_{\Gamma}\frac{f'(z)}{f(z)}dz=\sum_{j=1}^{m}\omega_{j}-\sum_{k=1}^{n}\rho_{k}.=N-P,
\]
where $N$ is the sum of orders of zeros in $\text{Int}(\Gamma)$
and $P$ is the sum of orders of poles of $f$ in $\text{Int}(\Gamma)$.
\end{theorem}

\begin{rem}\label{rem} If $\gamma$ paramatrizes $\Gamma$, then
$\frac{1}{2\pi i}\int_{\Gamma}f'(z)$ counts the number of winds $\gamma$
makes about $0$. \end{rem}

\begin{proof} The function $\frac{f'}{f}$ is analytic except at
points the points $a_{j}$ and $b_{k}$. Near each $a_{j}$, we can
factor $f$ as 
\begin{equation}
f(z)=(z-a_{j})^{\omega_{j}}g(z),\label{eq:f23}
\end{equation}
where $g$ is analytic at $a_{j}$ and $g(a_{j})\neq0$. Taking derivatives
on both sides of (\ref{eq:f23}), we obtain 
\begin{equation}
f'(z)=\ell_{j}(z-a_{j})^{\omega_{j}-1}g(z)+(z-a_{j})^{\omega_{j}}g'(z).\label{eq:f24}
\end{equation}
Combining (\ref{eq:f23}) and (\ref{eq:f24}), we obtain
\[
\frac{f'(z)}{f(z)}=\frac{\omega_{j}}{z-a_{j}}+\frac{g'(z)}{g(z)}.
\]

So $\frac{f'}{f}$ has a simple pole at $a_{j}$ with residue $\omega_{j}$. 

~~~Near each $b_{k}$ we can factor $f$ as 
\begin{equation}
f(z)=(z-b_{k})^{-\rho_{k}}h(z),\label{eq:f23-1}
\end{equation}
where $h$ is analytic at $b_{k}$ and $h(b_{k})\neq0$. Taking derivatives
on both sides of (\ref{eq:f23-1}), we obtain 
\begin{equation}
f'(z)=\rho_{k}(z-b_{k})^{\rho_{k}-1}h(z)+(z-b_{k})^{\rho_{k}}h'(z).\label{eq:f24-1}
\end{equation}
Combining (\ref{eq:f23-1}) and (\ref{eq:f24-1}), we obtain
\[
\frac{f'(z)}{f(z)}=\frac{-\rho_{k}}{z-b_{k}}+\frac{h'(z)}{h(z)}.
\]
So $\frac{f'}{f}$ has a simple pole at $b_{k}$ with residue $-\rho_{k}$. 

~~~Now we apply Theorem~(\ref{theoremcauchyresiduetheorem}) and
obtain 
\[
\frac{1}{2\pi i}\int_{\Gamma}\frac{f'(z)}{f(z)}dz=\sum_{j=1}^{m}\omega_{j}-\sum_{k=1}^{n}\rho_{k}.
\]

\end{proof}

\subsubsection{Rouche's Theorem}

\begin{theorem}\label{theoremrouchestheorem} (Rouche's Theorem) Suppose
that $f$ and $g$ are holomorphic in an open set containing a circle
$C$ and its interior. If 
\[
|f(z)|>|g(z)|\quad\text{for all }z\in C,
\]
then $f$ and $f+g$ have the same number of zeros inside the circle
$C$. \end{theorem}

\begin{rem}\label{rem} The Theorem says that an analytic function
can be perturbed slightly without changing the number of zeros. \end{rem}

\begin{proof} For each $t\in[0,1]$, define a function $f_{t}$ by
$f_{t}(z)=f(z)+tg(z)$. Let $n_{t}$ denote the number of zeros of
$f_{t}$ inside the circle counted with multiplicities, so that in
particular, $n_{t}$ is an integer. The condition $|f(z)|>|g(z)|$
for all $z\in C$ clearly implies that $f_{t}$ has no zeros on the
circle, and the argument principle implies
\[
n_{t}:=\frac{1}{2\pi i}\int_{C}\frac{f'_{t}(z)}{f_{t}(z)}dz.
\]

To prove that $n_{t}$ is constant, it suffices to show that it is
a continuous function of $t$. Then we could argue that if $n_{t}$
were not constant, the intermediate value theorem would guarantee
existence of some $t_{0}\in[0,1]$ with $n_{t_{0}}$ not integral,
contradicting the fact that $n_{t}\in\mathbb{Z}$ for all $t$. 

~~~To prove continuity of $n_{t}$, observe that $f'_{t}(z)/f_{t}(z)$
is jointly continuous for $t\in[0,1]$ and $z\in C$. This joint continuity
follows from the fact that it holds for both the numerator and denominator,
and our assumptions guarantee that $f_{t}(z)$ does not vanish on
$C$. Hence, $n_{t}$ is integer-valued and continuous, and it must
be constant. We conclude that $n_{0}=n_{1}$, which is Rouche's theorem.
\end{proof}

\subsubsection{Local Mapping Theorem}

\begin{theorem}\label{theoremlocalmappingtheorem} (Local Mapping
Theorem) Suppose $f$ is analytic at $z_{0}$ and that $f(z)-f(z_{0})$
has a zero of order $n$ at $z_{0}$. Then if $\varepsilon>0$ is
sufficiently small, there exists $\delta>0$ such that whenever $0<|w-f(z_{0})|<\delta$,
then there exists exactly $n$ distinct points $z\in D_{\varepsilon}(z_{0})\backslash\{z_{0}\}$
such that $f(z)=w$. \end{theorem}

\begin{proof} Choose $\varepsilon>0$ small enough so that $f(z)-f(z_{0})\neq0$
in $D_{2\varepsilon}(z_{0})\backslash\{z_{0}\}$ and $f'(z)\neq0$
for $z\in D_{2\varepsilon}(z_{0})\backslash\{z_{0}\}$. Let $\Gamma:=\{z_{0}+\varepsilon e^{it}\mid0\leq t\leq2\pi\}$
and $\delta:=\text{inf}\{|f(z)-f(z_{0})|\mid z\in\Gamma\}$. Then
if $|f(z_{0})-w|<\delta$, then $|f(z_{0})-w|<|f(z)-f(z_{0})|$ for
all $z\in\Gamma$. Apply Theorem~(\ref{theoremrouchestheorem}) to
get
\[
f(z)-w=f(z)-f(z_{0})+f(z_{0})-w
\]
has the same number of zeros as $f(z)-f(z_{0})$, i.e. $n$ (counted
with multiplicity). Finally, $(f(z)-w)'=f'(z)\ne0$ for $0<|z-z_{0}|<\varepsilon$.
Hence, since $f(z_{0})\neq w$ and the zeros of $f(z)-w$ inside $\Gamma$
have order $1$, there exists $n$ distinct zeros of $f(z)-w$ inside
$\Gamma$. \end{proof}

\begin{cor}\label{cor} Let $f$ be analytic and nonconstant on a
region $\Omega$. 
\begin{enumerate}
\item (Open Mapping Theorem) If $G\subseteq\Omega$ is open, then $f(G)$
is open. 
\item Let $z_{0}\in\Omega$. Then $f'(z_{0})\neq0$ if and only if there
exists $r>0$ such that $f$ is 1-1 on $D_{r}(z_{0})$.
\item (Inverse Function Theorem) If $f$ is 1-1 on $\Omega$, then $f^{-1}$
is analytic on $f(\Omega)$ and
\[
(f^{-1})'(f(z))=\frac{1}{f'(z)}
\]
for all $z\in\Omega$. 
\end{enumerate}
\end{cor}

\begin{proof} \hfill
\begin{enumerate}
\item Let $z_{0}\in G$ so that $f(z_{0})\in f(G)$. By Theorem~(\ref{theoremlocalmappingtheorem}),
there exists $\varepsilon>0$ such that $D_{\varepsilon}(z_{0})\subset G$
and there exists $\delta>0$ such that $D(f(z_{0}),\delta)\subseteq f(D_{\varepsilon}(z_{0}))\subseteq f(G)$.
Therefore $f(G)$ is open. 
\item Use notation in the Local Mapping Theorem. Note that $f(z)-f(z_{0})=(z-z_{0})^{n}h(z)$
and $f'(z)=n(z-z_{0})^{n-1}h(z)+(z-z_{0})^{n}h'(z)$ where $h(z_{0})\neq0$.
Therefore $f'(z_{0})\neq0$ if and only if $n=1$. Converseley, $f'(z_{0})\neq0$
implies $n=1$. Then $f^{-1}(D(f(z_{0})),\delta)$ is an open set
containing $z_{0}$ and hence containing $D_{r}(z_{0})$ for some
$r>0$. By the Theorem~(\ref{theoremlocalmappingtheorem}), $f$ is
1-1 on $D_{r}(z_{0})$. 
\item Write $w=f(z)$. Since $f$ is 1-1, $f'(z)\neq0$ by part (2). Then
\begin{align*}
\lim_{w\to f(z_{0})}\frac{f^{-1}(w)-f^{-1}(f(z_{0}))}{w-f(z_{0})} & =\lim_{f(z)\to f(z_{0})}\frac{z-z_{0}}{f(z)-f(z_{0})}\\
 & =\lim_{z\to z_{0}}\frac{1}{\frac{f(z)-f(z_{0})}{z-z_{0}}}\\
 & =\frac{1}{f'(z_{0})}.
\end{align*}
 
\end{enumerate}
\end{proof}

\textbf{Warning:} $f'(z)\neq0$ for all $z\in\Omega$ does not imply
$f$ is 1-1 on $\Omega$. For example, $e^{z}$ has nonvanishing derivative
everywhere on $\mathbb{C}$, but $e^{z}$ is not 1-1 on $\mathbb{C}$.

\subsubsection{Maximum Modulus Theorem}

\begin{theorem}\label{theoremmaximummodulustheorem} (Maximum Modulus
Theorem) Let $f$ be analytic and nonconstant on a region $\Omega$.
Then $|f(z)|$ does not obtain its maximum value on $\Omega$. In
particular, if $\Omega=\text{Int}(\Gamma)$, where $\Gamma$ is a
simple contour, and $f$ is continuous on $\Gamma\cup\text{Int}(\Gamma)$,
then if $|f(z)|\leq M$ for all $z\in\Gamma$, then $|f(z)|<M$ for
all $z\in\text{Int}(\Gamma)$. 

\end{theorem}

\begin{proof} Suppose that $f$ did attain a maximum at $z_{0}$.
Since $f$ is holomorphic it is an open mapping, and therefore, if
$D\subset\Omega$ is a small disc centered at $z_{0}$, its image
$f(D)$ is open and contains $f(z_{0})$. This proves that there are
points in $z\in D$ such that $|f(z)|>|f(z_{0})|$, a contradiction.
\end{proof}

\begin{cor}\label{cor} Suppose that $\Omega$ is a region with compact
closure $\overline{\Omega}$. If $f$ is holomorphic on $\Omega$
and continuous on $\overline{\Omega}$, then 
\[
\sup_{z\in\Omega}|f(z)|\leq\sup_{z\in\overline{\Omega}\backslash\Omega}|f(z)|.
\]
\end{cor}

\begin{proof} In fact, since $f(z)$ is continuous on the compact
set $\overline{\Omega}$, then $|f(z)|$ attains a maximum in $\overline{\Omega}$;
but this cannot be in $\Omega$ if $f$ is nonconstant. If $f$ is
constant, the conclusion is trivial. \end{proof}

\begin{example}\label{example} \hfill
\begin{enumerate}
\item If $f$ is analytic on $\Omega$ and $f(z)\neq0$ for all $z\in\Omega$,
then $|f(z)|$ does attain its minimum value in $\Omega$. The idea
is simply to apply Theorem~(\ref{theoremmaximummodulustheorem}) to
$1/f$. 
\item If $f$ is analytic and nonconstant on $\Omega$. Then $\text{Re}(f)$
and $\text{Im}(f)$ do not attain maximum values on $\Omega$. The
idea is to consider 
\begin{align*}
|e^{f}| & =e^{\text{Re}(f)}\\
|e^{-f}| & =e^{-\text{Re}(f)}\\
|e^{if}| & =e^{-\text{Im}(f)}\\
|e^{-if}| & =e^{-\text{Im}(f)}.
\end{align*}
\end{enumerate}
\end{example}

\subsubsection{Schwarz's Lemma}

\begin{theorem}\label{theoremschwarzslemma} (Schwarz Lemma) Assume
$f$ is analytic on $D_{1}(0)$ with $|f(z)|\leq1$ for all $|z|<1$
and $f(0)=0$. Then $|f(z)|\leq|z|$ for all $|z|<1$. Moreover, if
$|f(z_{0})|=|z_{0}|$ for some $z_{0}\in D_{1}(0)$, then $f(z)=cz$
for some $c$ with $|c|=1$. \end{theorem}

\begin{proof} Define $g$ by 
\[
g(z)=\begin{cases}
\frac{f(z)}{z} & \text{if }0<|z|<1\\
f'(0) & \text{if }z=0.
\end{cases}
\]

Then $g$ is analytic on $D_{1}(0)$. Let 
\[
f(z)=\sum_{n=1}^{\infty}\frac{f^{(n)}(0)}{n!}z^{n}
\]
Then
\[
g(z)=\sum_{n=0}^{\infty}\frac{f^{(n+1)}(0)}{(n+1)!}z^{n}.
\]
which is valid for $|z|<1$. Take any $r\in(0,1)$. Apply the Maximum
Modulus Theorem to $g$ on $\Gamma_{r}:=\{re^{i\theta}\mid0\leq\theta\leq2\pi\}.$
Then on $\Gamma$, $|g(z)|\leq\frac{1}{r}$, hence $|f(z)|\leq\frac{1}{r}|z|$
for all $z\in\Gamma\cup\text{Int}(\Gamma)$. Letting $r\to1^{-}$,
we get |
\[
|g(z)|\leq1\quad\text{and}\quad|f(z)|\leq|z|
\]
for all $z\in D_{1}(0)$. Finally, if $|f(z_{0})|=|z_{0}|$, then
$|g(z_{0})|=1$. By the Maximum Modulus Theorem, $g$ is a constant,
i.e. $g(t)=c$ with $|c|=1$. Hence, $f(z)=cz$. 

\end{proof}

\subsection{Residue Calculus (Theory)}

~~~If $f$ is analytic on $\Gamma\cup\text{Int}(\Gamma)$ except
for some poles possibly in $\text{Int}(\Gamma)$. Then by Cauchy's
Residue Theorem, 
\[
\frac{1}{2\pi i}\int_{\Gamma}f(z)dz=\sum\text{residues of }f\text{ at each pole in }\text{Int}(\Gamma).
\]

Let $f$ have a pole of order $k$ at $z_{0}$. Then 
\[
f(z)=\sum_{n=-k}^{\infty}a_{n}(z-z_{0})^{n}
\]
for all $0<|z-z_{0}|<r$, with $a_{-k}\neq0$ and $a_{-1}=\text{res}(f,z_{0})$.
Let $g$ be defined as 
\[
g(z):=(z-z_{0})^{k}f(z)=\sum_{n=0}^{\infty}a_{n-k}(z-z_{0})^{n},
\]
for all $0<|z-z_{0}|<r$. Then $g(z_{0})=a_{-k}$ and $g$ is analytic
at $z_{0}$. Now $a_{-1}$ is the coefficient of $(z-z_{0})^{k-1}$
in the Taylor series expansion of $g$ at $z_{0}$. Thus, 
\[
a_{-1}=\frac{1}{(k-1)!}g^{(k-1)}(z_{0})=\frac{1}{(k-1)!}\lim_{z\to z_{0}}g^{(k-1)}(z).
\]
Therefore 
\[
\text{res}(f,z_{0})=a_{-1}=\frac{1}{(k-1)!}\lim_{z\to z_{0}}\frac{d^{k-1}}{dz^{k-1}}\left((z-z_{0})^{k}f(z)\right).
\]
In the special case that $k=1$, we have 
\[
\text{res}(f,z_{0})=\lim_{z\to z_{0}}(z-z_{0})f(z).
\]
We will call this result, combined with Cauchy's Residue Theorem,
the \textbf{Residue Theorem}. 

\subsubsection{Integrating using residues method 1}

~~~Suppose we want to integrate something of the form
\[
I=\int_{0}^{2\pi}R(\cos\theta,\sin\theta)d\theta,
\]
where $R$ is a rational function in $\cos\theta$ and $\sin\theta$.
To evaluate $I$, we put $z=e^{i\theta}$, so that 
\[
\cos\theta=\frac{1}{2}\left(z+\frac{1}{z}\right)\quad\text{and}\quad\sin\theta=\frac{1}{2i}\left(z-\frac{1}{z}\right).
\]
Let $\Gamma=\{e^{i\theta}\mid0\leq\theta\leq2\pi\}.$ Then 
\[
I=\int_{\Gamma}R\left(\frac{1}{2}\left(z+\frac{1}{z}\right),\frac{1}{2i}\left(z-\frac{1}{z}\right)\right)\left(\frac{-i}{z}\right)dz.
\]
\begin{example}\label{example} Let 
\[
I=\int_{0}^{\pi}\frac{d\theta}{a+\cos\theta}.
\]
Since $\cos\theta$ is symmetric about the line $\theta=\pi$, it's
easy to see that $\frac{1}{a+\cos\theta}$ is also symmetric about
this line. So 
\begin{align*}
I & =\frac{1}{2}\int_{0}^{2\pi}\frac{d\theta}{a+\cos\theta}\\
 & =\frac{1}{2}\int_{\Gamma}\frac{1}{a+\frac{1}{2}\left(z+\frac{1}{z}\right)}\cdot\frac{-i}{z}dz\\
 & =-i\int_{\Gamma}\frac{1}{z^{2}+2az+1}dz.
\end{align*}
Now we solve $z^{2}+2az+1=0$ to get $z=-a\pm\sqrt{a^{2}-1}$. The
only pole inside $\Gamma$ is $-a+\sqrt{a^{2}-1}$. It is also a simple
pole, thus its residue is 
\[
\lim_{z\to-a+\sqrt{a^{2}-1}}\left(\frac{z+a-\sqrt{a^{2}-1}}{z^{2}+2az+1}\right)=\lim_{z\to-a+\sqrt{a^{2}-1}}\left(\frac{1}{z+a+\sqrt{a^{2}-1}}\right)=\frac{1}{2\sqrt{a^{2}-1}}.
\]
So, by the Residue Theorem, 
\[
I=-i\cdot2\pi i\cdot\text{res}(f,-a+\sqrt{a^{2}-1})=\frac{\pi}{\sqrt{a^{2}-1}}.
\]
\end{example}

\subsubsection{Integrating using residues method 2}

\begin{theorem}\label{theoremresidueintegral2} Let $p,q\in\mathbb{R}[x]$
such that $p$ and $q$ share no common factor, $\text{deg}(q)\geq\text{deg}(p)+2$,
and $q(x)\ne0$ for all $x\in\mathbb{R}$. Then the integral
\[
\int_{-\infty}^{\infty}\frac{p(x)}{q(x)}dx=2\pi i\sum_{r=1}^{k}\text{res}\left(\frac{p}{q},z_{r}\right)
\]
where $z_{r}$ denotes the zeros of $q$ in the upper half-plane.
\end{theorem}

\begin{proof} Omitted. \end{proof}

\begin{ex} Evaluate $I=\int_{-\infty}^{\infty}\frac{1}{\left(1+x^{2}\right)^{2}}dx$.
\end{ex}

\textbf{Solution: }For this problem, $p(z)=1$ and $q(z)=\left(1+z^{2}\right)^{2}=(1+iz)^{2}(1-iz)^{2}$.
The conitions in Theorem~(\ref{theoremresidueintegral2}) are satisfied.
The only zero of $q$ in the upper-half plane is $z=i$ with order
$2$. Then
\begin{align*}
\text{res}\left(\frac{p}{q},i\right) & =\lim_{z\to i}\frac{d}{dz}\left(\frac{\left(z-i\right)^{2}}{\left(1+z^{2}\right)^{2}}\right)\\
 & =\lim_{z\to i}\frac{d}{dz}\left(\frac{1}{(z+i)^{2}}\right)\\
 & =\lim_{z\to i}\left(\frac{-2}{(z+i)^{3}}\right)\\
 & =\frac{1}{4i},
\end{align*}
and therefore 
\begin{align*}
\int_{-\infty}^{\infty}\frac{1}{\left(1+x^{2}\right)^{2}}dx & =2\pi i\cdot\text{res}\left(\frac{p}{q},i\right)\\
 & =2\pi i\cdot\frac{1}{4i}\\
 & =\frac{\pi}{2}.
\end{align*}

\subsubsection{Integrating using residues method 3}

~~~~~~Suppose we want to integrate 
\[
I=\int_{-\infty}^{\infty}\frac{p(x)}{q(x)}\cos(x)dx\quad\text{or}\quad J=\int_{-\infty}^{\infty}\frac{p(x)}{q(x)}\cos(x)dx
\]

where $p,q$ are real polynomials and relatively prime real polynomials,
$\text{deg}(q)\geq\text{deg}(p)+1$, and $q(x)\neq0$ for all $x\in\mathbb{R}$.

\begin{theorem}\label{theoremtheoremresidueintegral3} Under the above
conditions, the integral $I=\int_{-\infty}^{\infty}\frac{p(x)}{q(x)}e^{ix}dx$
exists and equals $2\pi i\sum_{r=1}^{k}\text{res}\left(\frac{p(z)}{q(z)}e^{iz},z_{r}\right)$,
where $z_{r}$ denotes the zeros of $q$ in the upper half-plane. 

\end{theorem}

\begin{proof} Omitted. \end{proof}

\begin{ex} Evaluate $I=\int_{0}^{\infty}\frac{x\sin(x)}{1+x^{2}}dx$.
\end{ex}

\textbf{Solution: }The function $\frac{x\sin(x)}{1+x^{2}}$ is even.
Therefore
\begin{align*}
I & =\frac{1}{2}\int_{-\infty}^{\infty}\frac{x\sin(x)}{1+x^{2}}dx\\
 & =\frac{1}{2}\text{Im}\left(\int_{-\infty}^{\infty}\frac{xe^{ix}}{1+x^{2}}dx\right).
\end{align*}
The conitions in Theorem~(\ref{theoremresidueintegral3}) are satisfied,
with $p(z)=z$ and $q(z)=1+z^{2}$. The only zero of $q$ in the upper-half
plane is $z=i$ with order $1$. Then
\begin{align*}
\text{res}\left(\frac{ze^{iz}}{1+z^{2}},i\right) & =\lim_{z\to i}\left(\frac{(z-i)ze^{iz}}{1+z^{2}}\right)\\
 & =\lim_{z\to i}\left(\frac{ze^{iz}}{z+i}\right)\\
 & =\frac{ie^{-1}}{2i}\\
 & =\frac{1}{2e},
\end{align*}
and therefore 
\begin{align*}
\int_{-\infty}^{\infty}\frac{xe^{ix}}{1+x^{2}}dx & =2\pi i\cdot\text{res}\left(\frac{ze^{iz}}{1+z^{2}},i\right)\\
 & =2\pi i\cdot\frac{1}{2e}\\
 & =\frac{\pi i}{e}.
\end{align*}

\section{The Complex Logarithm}

~~~Recall that each complex number $z\neq0$ can be written as
$z=re^{i\theta}$ where $r=|z|$ and $\theta=\text{arg}(z)$. The
value $r$ is unique, but the value $\theta$ is only unique modulo
$2\pi\mathbb{Z}$. In the real case, $e^{x}=y$ with $y>0$, then
$x=\log(y)$. In the complex case, $e^{z}=w$ with $w\in\mathbb{C}\backslash\{0\}$.
Setting $z=x+iy$, we have $e^{x}e^{iy}=e^{x+iy}=|w|e^{i\text{arg}(w)}$.
The solutions are $x=\log|w|$ and $y=\text{arg}(w)+2n\pi$. There
are infinitely choices of what $\log(w)$ could be, but we only choose
one of them. We define $\log(w)=\log|w|+i\text{arg}(w)$, where $\text{arg}(w)\in(-\pi,\pi]$.
Notice that $\log(w)$ is continuous except on the half line $(-\infty,0]$. 

\begin{defn}\label{defn} Let $w\in\mathbb{C}\backslash\{0\}$. Then
we define $w^{z}:=e^{z\log(w)}$, which makes sense as a function
as long as we restrict to the principal branch. \end{defn}

\begin{ex} \hfill
\begin{enumerate}
\item $\log(z_{1})+\log(z_{2})=\log(z_{1}z_{2})$ modulo $2\pi i\mathbb{Z}$. 
\item $w^{z_{1}+z_{2}}=w^{z_{1}}w^{z_{2}}$ if we take the principal value
of $\log$. 
\end{enumerate}
\end{ex}

\begin{theorem}\label{theorem3.16} Let $\Omega$ be a simply connected
region with $1\in\Omega$ and $0\notin\Omega$. Then in $\Omega$
there is a branch of the logarithm $F(z)=\log_{\Omega}(z)$ so that 
\begin{enumerate}
\item $F$ is holomorphic in $\Omega$,
\item $e^{F(z)}=z$ for all $z\in\Omega$,
\item $F(r)=\log(r)$ whenever $r$ is a real number and near $1$. 
\end{enumerate}
\end{theorem}

\begin{rem}\label{rem} In other words, each branch $\log_{\Omega}(z)$
is an extension of the standard logarithm defined for positive numbers.
\end{rem}

\begin{proof} We shall construct $F$ as a primitive of the function
$1/z$. Since $0\notin\Omega$, the function $f(z)=1/z$ is holomorphic
in $\Omega$. We define
\[
\log_{\Omega}(z)=F(z)=\int_{\gamma}f(w)dw,
\]
where $\gamma$ is any curve in $\Omega$ connecting $1$ to $z$.
Since $\Omega$ is simply connected, this definition does not depend
on the path chosen. Arguing as in the proof of Theorem 5.2, we find
that $F$ is holomorphic and $F'(z)=1/z$ for all $z\in\Omega$. This
proves (1). To prove (2), it suffices to show that $ze^{-F(z)}=1$.
For that, we differentiate the left-hand side, obtaining
\[
\frac{d}{dz}\left(ze^{-F(z)}\right)=e^{-F(z)}-zF'(z)e^{-F(z)}=\left(1-zF'(z)\right)e^{-F(z)}=0.
\]
Since $\Omega$ is connected, we conclude that $ze^{-F(z)}$ is constant.
Evaluating this expression at $z=1$, and noting that $F(1)=0$, we
find that this constant must be $1$. 

~~~Finally, if $r$ is real and close to $1$ we can choose as
a path from $1$ to $r$ a line segment on the real axis so that
\[
F(r)=\int_{1}^{r}\frac{dx}{x}=\log r,
\]
by the usual formula for the standard logarithm. This completes the
proof of the theorem.\end{proof}

\section{Conformal Mappings}

~~~We fix some terminology that we shall use in the rest of this
chapter. A bijective holomorphic function $f:U\to V$ is called a
\textbf{conformal map }or \textbf{biholomorphism}. Given such a mapping
$f$, we say that $U$ and $V$ are \textbf{conformally equivalent
}or simply \textbf{biholomorphic}. An important fact is that the inverse
of $f$ is then automatically holomorphic.

~~~Let $\gamma_{1}$ and $\gamma_{2}$ be paths in $\mathbb{C}$
with $z_{0}=\gamma_{1}(0)=\gamma_{2}(0)$. The \textbf{angle }between
$\gamma_{1}$ and $\gamma_{2}$ at $z_{0}$ is written
\[
\langle\gamma_{1},\gamma_{2}\rangle_{z_{0}}=\lim_{t\to0^{+}}\left(\arg(\gamma_{2}(t)-z_{0})\right)-\lim_{s\to0^{+}}\left(\arg(\gamma_{1}(s)-z_{0})\right),
\]
assuming that these limits exist. Let $f$ be analytic on $\gamma_{1}\cup\gamma_{2}$
and write $\widetilde{\gamma}_{j}=f\circ\gamma_{j}$. 

\begin{rem}\label{rem} \hfill 
\begin{enumerate}
\item If $\gamma_{1}'(0)\ne0$ and $\gamma_{2}'(0)\neq0$, then $\langle\gamma_{1},\gamma_{2}\rangle_{z_{0}}=\text{arg}\left(\gamma_{1}'(0)\right)-\text{arg}\left(\gamma_{2}'(0)\right)$.
To see this, note that 
\begin{align*}
\lim_{t\to0^{+}}\left(\arg\left(\gamma_{1}(t)-\gamma_{1}(0)\right)\right) & =\lim_{t\to0^{+}}\left(\arg\left(\frac{\gamma_{1}(t)-\gamma_{1}(0)}{t}\right)\right)\\
 & =\arg\left(\lim_{t\to0^{+}}\left(\frac{\gamma_{1}(t)-\gamma_{1}(0)}{t}\right)\right)\\
 & =\arg\left(\gamma_{1}'(0)\right).
\end{align*}
\end{enumerate}
\end{rem}

\begin{defn}\label{defn} We say $f$ is \textbf{conformal }on a region
$\Omega\subseteq\mathbb{C}$ if $f$ is analytic on $\Omega$ and
$f'(z)\neq0$ for all $z\in\Omega$. \end{defn}

\begin{rem}\label{rem} The book's definition is equivalent to our
definition + 1-1. Thus is it stronger. \end{rem}

\begin{theorem}\label{theorem4.1} Let $f$ be analytic at $z_{0}$
with $f(z)-f(z_{0})$ having a zero of order $n$ at $z_{0}$. Then
if $\langle\gamma_{1},\gamma_{2}\rangle_{z_{0}}$ and $\langle\widetilde{\gamma}_{1},\widetilde{\gamma}_{2}\rangle_{f(z_{0})}$
exist, then 
\[
\langle\gamma_{1},\gamma_{2}\rangle_{z_{0}}=\langle\widetilde{\gamma}_{1},\widetilde{\gamma}_{2}\rangle_{f(z_{0})}.
\]
\end{theorem}

\begin{proof} Write $f(z)-f(z_{0})=(z-z_{0})^{n}h(z)$, for $|z-z_{0}|<\delta$
where $\delta>0$ is sufficiently small, and $h$ is analytic and
nonzero in $\{|z|<\delta\}$. Then 
\begin{equation}
\text{arg}\left(f(\gamma_{2}(t))-f(z_{0})\right)=n\text{arg}\left(\gamma_{2}(t)-z_{0}\right)+\text{arg}\left(h(\gamma_{2}(t))\right).\label{eq:arg1}
\end{equation}
\begin{equation}
\text{arg}\left(f(\gamma_{1}(t))-f(z_{0})\right)=n\text{arg}\left(\gamma_{1}(t)-z_{0}\right)+\text{arg}\left(h(\gamma_{1}(t))\right).\label{eq:arg2}
\end{equation}

Subtracting (\ref{eq:arg1}) from (\ref{eq:arg2}), then letting $t\to0^{+}$,
we obtain
\[
\langle\widetilde{\gamma}_{1},\widetilde{\gamma}_{2}\rangle_{f(z_{0})}=n\langle\gamma_{1},\gamma_{2}\rangle_{z_{0}}+\lim_{t\to0^{+}}\text{arg}\left(h(\gamma_{2}(t)\right)-\lim_{t\to0^{+}}\text{arg}\left(h(\gamma_{1}(t)\right)
\]

\end{proof}

\begin{rem}\label{rem} If $f'(z_{0})\neq0$, then $n=1$ in Theorem~(\ref{theorem4.1}),
hence $f$ preserves angles. \end{rem}

\subsection{Mobius Transformations}

~~~Let $a,b,c,d\in\mathbb{C}$ such that $ad-bc\neq0$. A function
of the form 
\[
f(z)=\frac{az+b}{cz+d}
\]
is called a \textbf{M�bius transformation }(or \textbf{bilinear transformation
}or \textbf{linear fractional transformation}). It is easy to check
that 
\[
f'(z)=\frac{ad-bc}{\left(cz+d\right)^{2}}\neq0,
\]
hence $f$ is conformal on $\mathbb{C}\backslash\{-d/c\}$. We define
$f$ on $\mathbb{C}_{\infty}=\mathbb{C}\cup\{\infty\}$ by $f(-d/c)=\infty$
and $f(\infty)=a/c$. We can check that the equation 
\[
f(z)=w
\]
can be solved uniquely for each $w\in\mathbb{C}_{\infty}$ as 
\[
z=\frac{dw-b}{-cw+a}=f^{-1}(w).
\]
This implies $f^{-1}$ is also a m�bius transformation. 

\begin{rem}\label{rem} The set of M�bius transformations forms a
group under composition as multiplication. \end{rem}

~~~Here are some special types of Mobius transformations. 
\begin{enumerate}
\item $f(z)=z+b$ (translation) $\left(\begin{smallmatrix}1 & b\\
0 & 1
\end{smallmatrix}\right)$
\item $f(z)=e^{i\alpha}z$, $\alpha\in\mathbb{R}$ (rotation) $\left(\begin{smallmatrix}e^{i\alpha} & 0\\
0 & 1
\end{smallmatrix}\right)$
\item $f(z)=az$, $a>0$ (homothety) $\left(\begin{smallmatrix}a & 0\\
0 & 1
\end{smallmatrix}\right)$
\item $f(z)=\frac{1}{z}$ (inversion) $\left(\begin{smallmatrix}0 & 1\\
1 & 0
\end{smallmatrix}\right)$
\end{enumerate}
In fact, any Mobius transformation is the composition of these special
transformations. The proof of this is as follows: assume $c\neq0$,
otherwise it is easy. Then
\begin{align*}
z & \mapsto z+\frac{d}{c}\\
 & \mapsto\frac{1}{z+\frac{d}{c}}\\
 & \mapsto\frac{bc-ad}{c^{2}\left(z+\frac{d}{c}\right)}\\
 & \mapsto\frac{a}{c}+\frac{bc-ad}{c^{2}\left(z+\frac{d}{c}\right)}\\
 & =\frac{az+b}{cz+d}
\end{align*}

where the transformations are given by translation, inversion, rotation/homothety,
and translation respectively. 

~~~Mobius transformations maps $\{\text{circles}\cup\text{lines}\}$
to $\{\text{circles}\cup\text{lines}\}$. In particular, the special
types (1), (2), and (3) takes lines into lines and circles into circle.
Whereas the special type (4) takes circles to lines and lines to circles.
We summarize everything we've discussed in the form of a theorem.

\begin{theorem}\label{theorem4.2} Let $f$ be a M�bius transformation.
Then
\begin{enumerate}
\item $f$ is a composition of the $4$ special types (translation, rotation,
homothety, and inversion).
\item $f:\mathbb{C}_{\infty}\to\mathbb{C}_{\infty}$ is one-to-one and onto,
and $f^{-1}$ is also a M�bius transformation.
\item $f$ maps $\{\text{circles}\cup\text{lines}\}$ to $\{\text{circles}\cup\text{lines}\}.$
\item $f$ is conformal at every point except its pole. 
\end{enumerate}
\end{theorem}

\begin{prop}\label{prop} Let $z_{1},z_{2},z_{3}\in\mathbb{C}_{\infty}$
be distinct and let $w_{1},w_{2},w_{3}\in\mathbb{C}_{\infty}$ be
distinct. Then there exists a unique M�bius transformation $f$ such
that $f(z_{i})=w_{i}$ for $i=1,2,3$. \end{prop}

\begin{proof} Let 
\[
g(z)=\frac{(z-z_{1})(z_{2}-z_{3})}{(z-z_{3})(z_{2}-z_{1})}\quad\text{and}\quad h(z)=\frac{(w-w_{1})(w_{2}-w_{3})}{(w-w_{3})(w_{2}-w_{1})}.
\]
Then $g(z_{1})=0$, $g(z_{2})=1$, and $g(z_{3})=\infty$ and $g$
is the unique M�bius transformation with these values. We can represent
as matrix as
\[
\left(\begin{smallmatrix}z_{2}-z_{3} & -(z_{2}-z_{3})z_{1}\\
z_{2}-z_{1} & -(z_{2}-z_{1})z_{3}
\end{smallmatrix}\right)
\]
The determinant of this matrix is given by $(z_{2}-z_{1})(z_{3}-z_{2})(z_{3}-z_{1})\ne0$.
Now let $f:=h^{-1}\circ g$. Then $f(z_{i})=w_{i}$ for $i=1,2,3$.
This proves existence. ~~~For uniqueness, suppose $\widetilde{f}$
is another M�bius transformation such that $\widetilde{f}(z_{i})=w_{i}$
for $i=1,2,3$. Then $(h\circ\widetilde{f})(z_{1})=0$, $(h\circ\widetilde{f})(z_{2})=1$,
and $(h\circ\widetilde{f})(z_{3})=\infty$. By uniqueness of $g$,
we have $h\circ\widetilde{f}=g$. So $\widetilde{f}=h^{-1}\circ g=f$.
\end{proof}

\begin{ex} There exists a M�bius transformation $f$ mapping $D_{1}(0)$
onto $\{x+iy\mid ax+by+c>0\}$, where $a,b,c\in\mathbb{R}$ and $a^{2}+b^{2}>0$.
\end{ex}

\begin{ex} Find all M�bius transformations mapping $D_{1}(0)$ onto
$\{z\mid\text{Re}(z)>0\}$ and such that it maps $0$ to $1$. \end{ex}

\begin{prop} If $f$ is a one-one analytic function from $D_{1}(0)$
onto $D_{1}(0)$, then $f$ is a M�bius transformation. If $f(0)=0$,
then $f(z)=e^{i\theta}z$. \end{prop}

\begin{proof} Let $g$ be given by 
\[
g(z):=\frac{z-f(0)}{1-\overline{f(0)}z}.
\]
Then $g$ maps $D_{1}(0)$ onto $D_{1}(0)$, $g$ is injective, and
$g(f(0))=0$. Let $h:=g\circ f$. Then $h$ is one-one analytic from
$D_{1}(0)$ to $D_{1}(0)$ with $h(0)=0$. The same holds for $h^{-1}$.
Hence by Schwarz lemma, we get $|h(z)|\leq|z|$ and similarly $|h^{-1}(w)|\leq|w|$
for all $|z|,|w|<1$. Hence $|h^{-1}(h(z))|\leq|h(z)|$ if and only
if $|z|\leq|h(z)|$. Therefore $|z|=|h(z)|$. By Schwarz lemma again,
$h(z)=e^{i\alpha}z$. Now, $f=g^{-1}\circ f$ is a M�bius transformation.
\end{proof}

\begin{ex} The general M�bius transformation from $D_{1}(0)$ onto
$D_{1}(0)$ has the form 
\[
f(z)=e^{i\alpha}\frac{z-\beta}{1-\overline{\beta}z}
\]
\end{ex}

\begin{ex} Find all M�bius transformations mapping $D_{r_{1}}(z_{1})$
onto $D_{r_{2}}(z_{2})$. \end{ex}

\subsubsection{Riemann Mapping Theorem}

~~~Our goal in this section is to prove the Riemann Mapping Theorem.
We first need some defintions.

\begin{defn}\label{defn} Let $\Omega\subseteq\mathbb{C}$ be open
and $\mathcal{F}$ be a family of holomorphic functions on $\Omega$.
Then 
\begin{enumerate}
\item $\mathcal{F}$ is said to be \textbf{uniformly bounded on compact
subsets of $\Omega$} if for all compact sets $K\subset\Omega$, there
exists a constant $B=B(K)>0$ such that 
\[
|f(z)|\leq B\qquad\text{for all }z\in K\text{ and }f\in\mathcal{F}.
\]
\item $\mathcal{F}$ is said to be \textbf{equicontinuous on a compact set}
$K\subset\Omega$ if for all $\varepsilon>0$, there exists $\delta>0$
such that for all $z,w\in K$ and $f\in\mathcal{F}$, 
\[
|z-w|<\delta\text{ implies }|f(z)-f(w)|<\varepsilon.
\]
\item $\mathcal{F}$ is said to be \textbf{normal }if every sequence in
$\mathcal{F}$ has a subsequence that converges uniformly on every
compact subset of $\Omega$. The limit need not be in $\mathcal{F}$.
\end{enumerate}
\end{defn}

\begin{rem}\label{rem} Remark about (3). The limit is holomorphic
by Morera. \end{rem}

\begin{defn}\label{defn} Let $\Omega\subseteq\mathbb{C}$ be open.
A sequence $\{K_{\ell}\}_{\ell=1}^{\infty}$ of compact subsets is
called an \textbf{exhaustion }if 
\begin{enumerate}
\item $K_{\ell}\subseteq\text{Int}(K_{\ell+1})$ for all $\ell\geq1$. 
\item Any compact set $K\subseteq\Omega$ is contained in $K_{\ell}$ for
some $\ell$. In particular, $\Omega=\bigcup_{\ell=1}^{\infty}K_{\ell}$. 
\end{enumerate}
\end{defn}

\begin{example}\label{example} $\overline{D_{1-\frac{1}{\ell}}(0)}_{\ell=1}^{\infty}$
is an exhaustion of $D=D_{1}(0)$. \end{example}

\begin{lemma}\label{lemma} Any open set $\Omega$ in $\mathbb{C}$
has an exhaustion. \end{lemma}

\begin{proof}\hfill

\textbf{Case 1}: Assume $\Omega$ is bounded. Let $K_{\ell}:=\{z\in\Omega\mid|z-w|\geq1/\ell\text{ for all }w\in\partial\Omega\}$.
Then clearly $\{K_{\ell}\}_{\ell=1}^{\infty}$ is an exhaustion of
$\Omega$.

\hfill

\textbf{Case 2}: Assume $\Omega$ is not bounded. Let $K_{\ell}:=\{z\in\Omega\mid|z-w|\geq1/\ell\text{ for all }w\in\partial\Omega\}\cap\overline{D_{\ell}(0)}$.
It is straightforward to verify that $\{K_{\ell}\}_{\ell=1}^{\infty}$
is an exhaustion of $\Omega$. 

\end{proof}

\begin{theorem}\label{theorem} (Montel's Theorem) Assume that $\mathcal{F}$
is a family of holomorphic functions on $\Omega$ that is uniformly
bounded on compact subsets of $\Omega$. Then 
\begin{enumerate}
\item $\mathcal{F}$ is equicontinuous on every compact subset of $\Omega$.
\item $\mathcal{F}$ is a normal family. 
\end{enumerate}
\end{theorem}

\begin{proof} \hfill
\begin{enumerate}
\item Let $K\subseteq\Omega$ be compact and choose $r>0$ sufficiently
smallet such that $D_{3r}(z)\subseteq\Omega$ for all $z\in K$. Let
$z,w\in K$ such that $|z-w|<r$ and let $\Gamma$ be the boundary
of the disc $D_{2r}(w)$. By Cauchy's Integral Formula, we have 
\[
f(z)-f(w)=\frac{1}{2\pi i}\int_{\Gamma}f(\zeta)\left(\frac{1}{\zeta-z}-\frac{1}{\zeta-w}\right)d\zeta.
\]
Note that 
\[
\left|\frac{1}{\zeta-z}-\frac{1}{\zeta-w}\right|=\frac{|z-w|}{|\zeta-z||\zeta-w|}\leq\frac{|z-w|}{r^{2}}.
\]
Therefore 
\begin{align*}
|f(z)-f(w)| & \leq\frac{1}{2\pi}\int_{\Gamma}|f(\zeta)|\frac{|z-w|}{r^{2}}d\zeta\\
 & \leq\frac{1}{2\pi}\cdot\frac{2\pi r}{r^{2}}\cdot B\cdot|z-w|,
\end{align*}
where $B=\sup_{f\in\mathcal{F}}\left\{ |f(\zeta)|\mid\text{dist}(\zeta,K)\leq2r\right\} <\infty$
and$\text{dist}(\zeta,K)_{z\in K}=\inf\left\{ |\zeta-z|\right\} $.
Hence, $|f(z)-f(w)|<C|z-w|$ for all $z,w\in K$ with $|z-w|<r$ and
for all $f\in\mathcal{F}$. 
\item Let $\{f_{n}\}_{n=1}^{\infty}$ be a sequence in $\mathcal{F}$. \textbf{Step
1: }For any compact set $K\subseteq\Omega$, we extract a subsequence
that converges uniformly on $K$. Let $\{w_{i}\}_{i=1}^{\infty}$
be a sequence of points that is dense in $\Omega$. Since $\{f_{n}\}$
is uniformly bounded, there exists a subsequence $\{f_{n,1}\}$ of
$\{f_{n}\}$ such that $\{f_{n,1}(w_{1})\}$ converges as a sequence
of complex numbers. From $\{f_{n,1}\}$, we can extract a subsequence
$\{f_{n,2}\}$ such that $\{f_{n,2}(w_{2})\}$ converges. Continuing
this pro\{cess by extracting a subsequence $\{f_{n,j}\}_{n\in\mathbb{N}}$
from $\{f_{n,j-1}\}_{n\in\mathbb{N}}$ such that $\{f_{n,j}(w_{j})\}_{n\in\mathbb{N}}$
converges. Now let $g_{n}:=f_{n,n}$. By construction, $\{g_{n}(w_{j})\}_{n=1}^{\infty}$
converges for each $j$. In fact, for $n\geq j$, $g_{n}(w_{j})=f_{n,n}(w_{j})$
is a subsequence of $\{f_{n,j}(w_{j})\}$. We claim $\{g_{n}\}$ converges
uniformly on $K$ by showing $\{g_{n}\}$ is uniformly Cauchy on $K$.
Given $\varepsilon>0$, by equicontinuity, there exists $\delta>0$
such that 
\begin{equation}
|g_{n}(z)-g_{n}(w)|<\frac{\varepsilon}{3}\quad\text{whenever }|z-w|<\delta\quad\text{ for all }z,w\in\Omega\text{ and }n\in\mathbb{N}.\label{eq:gnz}
\end{equation}
Since $K$ is compact and 
\[
\bigcup_{i=1}^{\infty}D_{\delta}(w_{i})\supseteq K,
\]
there exists $L\in\mathbb{N}$ such that 
\[
L\subseteq\bigcup_{i=1}^{L}D_{\delta}(w_{i}).
\]
Let $N\in\mathbb{N}$ be sufficiently large so that for all $m,n\geq N$,
we have 
\begin{equation}
|g_{n}(w_{j})-g_{n}(w_{j})|<\frac{\varepsilon}{3}\quad\text{for all }j=1,\dots,L.\label{eq:gnz2}
\end{equation}
Now let $z\in K$. Then there exists $1\leq j\leq L$ such that $z\in D_{\delta}(w_{j})$.
By (ref{eq:gnz}) and (ref{eq:gnz2}), we have 
\begin{align*}
|g_{n}(z)-g_{m}(z)| & \leq|g_{n}(z)-g_{n}(w_{j})|+|g_{n}(w_{j})-g_{m}(w_{j})|+|g_{m}(w_{j})-g_{m}(z)|\\
 & <\frac{\varepsilon}{3}+\frac{\varepsilon}{3}+\frac{\varepsilon}{3}\\
 & =\varepsilon.
\end{align*}
Therefore $\{g_{n}\}$ is uniformly Cauchy on $K$ and proves the
claim. \textbf{Step 2: }We extract a subsequence that converges uniformly
in every compact subset of $\Omega$. Let 
\[
K_{1}\subseteq K_{2}\subseteq\cdots
\]
be an exhaustion of $\Omega$. Let $\{g_{n,1}\}$ be a sequence of
$\{f_{n}\}$ that converges uniformly on $K_{1}$ from Step 1. Extract
from $\{g_{n,1}\}$ a subsequence $\{g_{n,2}\}$ that converges uniformly
on $K_{2}$. Continuing this process, then $\{g_{n,n}\}$ is a subsequence
of $\{f_{n}\}$ that converges uniformly on every $K_{\ell}$. In
fact, for all $n\geq\ell$, $\{g_{n,n}\}_{n\geq\ell}$ is a subsequence
of $\{g_{n,\ell}\}$. Finally, since every compact set $K\subseteq\Omega$
is contained in some $K_{\ell}$, $\{g_{n,n}\}$ converges uniformly
on every compact subset of $\Omega$. 
\end{enumerate}
\end{proof}

\begin{prop}\label{prop} Let $\Omega$ be a region in $\mathbb{C}$
and $\{f_{n}\}$ be a sequence of injective holomorphic functions
in $\Omega$ that converges uniformly on every compact subset of $\Omega$.
Then $\lim_{n}(f_{n})=f$ is either injective or constant.\end{prop}

\begin{proof} Omitted, in text.\end{proof}

\begin{defn}\label{defn} \hfill
\begin{enumerate}
\item Two open sets $U,V\in\mathbb{C}$ are said to be \textbf{conformally
equivalent }if there exists a bijective holomorphic map $f$ from
$U$ to $V$. 
\item We call a subset $\Omega\subseteq\mathbb{C}$ \textbf{proper }if $\Omega\neq\emptyset$
and $\Omega\neq\mathbb{C}$. 
\end{enumerate}
\end{defn}

\begin{rem}\label{rem} Conformal equivalence is an equialence relation.
\end{rem}

\begin{theorem}\label{theorem} (Riemann Mapping Theorem). Assume
$\Omega\subseteq\mathbb{C}$ is proper simply connected region. Then
for any $z_{0}\in\mathbb{C}$, there exists a unique conformal bijection
$F:\Omega\to\mathbb{D}$ such that $F(z_{0})=0$ and $F'(z_{0})>0$.
\end{theorem}

\begin{proof}\hfill

\textbf{Step 1: }We show that $\Omega$ is conformally equivalent
to some open subset of $\mathbb{D}$ that contains the origin. Choose
$\alpha\in\mathbb{C}\backslash\Omega$ so that $z-\alpha\neq0$ for
all $z\in\Omega$. By Theorem~(\ref{theorem3.16}), we can define
a holomorphic branch of the logarithm $f:\Omega\to\mathbb{C}$ as
\[
f(z):=\log(z-\alpha),
\]
so that $e^{f(z)}=z-\alpha$ and thus in particular, $f$ is injective
($f(z_{1})=f(z_{2})$ implies $e^{f(z_{1})}=e^{f(z_{2})}$ implies
$z_{1}-\alpha=z_{2}-\alpha$ implies $z_{1}=z_{2}$). Take $w\in\Omega$
and observe that $f(z)\neq f(w)+2\pi i$ for all $z\in\Omega$, for
otherwise, we would get $e^{f(z)}=e^{f(w)+2\pi i}=e^{f(w)}$ implies
$z=w$, which is impossible. We claim there exists $D:=D_{r}(f(w)+2\pi i)$
such that $D\cap f(\Omega)=\emptyset$. Otherwise, there exists a
sequence $\{z_{n}\}\subseteq\Omega$ such that $f(z_{n})\to f(w)+2\pi i$.
But this implies $e^{f(z_{n})}\to e^{f(w)+2\pi i}=e^{f(w)}$ which
implies $f(z_{n})\to f(w)$, contradicting $f(z_{n})\to f(w)+2\pi i$. 

~~~Now define 
\[
F(z)=\frac{1}{f(z)-(f(w)+2\pi i)}.
\]
Since $f$ is injective, so is $F$. Hence, $F:\Omega\to F(\Omega)$
is a conformal bijection. Moreover, $F(\Omega)$ is bounded. Therefore,
upon translating and scaling $F$ if necessary, we obtain a conformal
bijection mapping $\Omega$ onto an open subset of $\mathbb{D}$ containing
the origin. 

\textbf{Step 2: }We show that the family of all injective holomorphic
functions from $\Omega$ to $\mathbb{D}$ that fix the origin contains
an element that maximizes $|f'(0)|$. Let 
\[
\mathcal{F}:=\{f:\Omega\to\mathbb{D}\mid f\text{ is holomorphic, injective, and }f(0)=0\}.
\]
Then $\mathcal{F}\neq\emptyset$ since it contains the identity. Obviously,
$\mathcal{F}$ is uniformly bounded on $\Omega$. By Cauchy's inequality,
if $\Gamma=\partial D_{r}(0)$ where $0<r<1$, then for all $f\in\mathcal{F}$,
\[
|f'(0)|=\left|\frac{1}{2\pi i}\int_{\Gamma}\frac{f'(\zeta)}{\zeta}d\zeta\right|\leq1.
\]
Let $s:=\sup_{f\in\mathcal{F}}|f'(0)|$ and let $\{f_{n}\}\in\mathcal{F}$
be a sequence such that $|f_{n}'(0)|\to s$ as $n\to\infty$. By Montel's
Theorem, there exists a subsequence that converges uniformly to a
holomorphic funciton $f$ on $\Omega$. By Theorem~(\ref{theorem2.14}),
$f_{n}'\to f'$ on every compact subset of $\Omega$. Thus, 
\[
|f'(0)|=\lim_{n\to\infty}|f_{n}'(0)|=s.
\]
Since the identity belongs to $\mathcal{F}$, $s\geq1$. Thus, $f$
is nonconstant. Hence, it is injective by Proposition~(\ref{prop4.5}).
Also, by continuity, 
\[
|f(z)|\leq1\qquad\text{for all }z\in\Omega.
\]
It follows from the Maximum Modulus Principle that $|f(z)|<1$ for
all $z\in\Omega$. Clearly, 
\[
f(0)=\lim_{n\to\infty}f_{n}(0)=0.
\]
Therefore $f\in\mathcal{F}$. 

\textbf{Step 3:}

\end{proof}
\end{document}
