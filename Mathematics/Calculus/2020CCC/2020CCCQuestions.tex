%% LyX 2.3.3 created this file.  For more info, see http://www.lyx.org/.
%% Do not edit unless you really know what you are doing.
\documentclass[12pt,english]{article}
\usepackage[osf]{mathpazo}
\renewcommand{\sfdefault}{lmss}
\renewcommand{\ttdefault}{lmtt}
\usepackage[T1]{fontenc}
\usepackage[latin9]{inputenc}
\usepackage[paperwidth=30cm,paperheight=35cm]{geometry}
\geometry{verbose,tmargin=3cm,bmargin=3cm}
\setlength{\parindent}{0bp}
\usepackage{amsmath}
\usepackage{amssymb}

\makeatletter

%%%%%%%%%%%%%%%%%%%%%%%%%%%%%% LyX specific LaTeX commands.
%% Because html converters don't know tabularnewline
\providecommand{\tabularnewline}{\\}

%%%%%%%%%%%%%%%%%%%%%%%%%%%%%% User specified LaTeX commands.
\usepackage{tikz}
\usetikzlibrary{matrix,arrows,decorations.pathmorphing}
\usetikzlibrary{shapes.geometric}
\usepackage{tikz-cd}
\usepackage{amsthm}
\theoremstyle{plain}
\newtheorem{theorem}{Theorem}[section]
\newtheorem{lemma}[theorem]{Lemma}
\newtheorem{prop}{Proposition}[section]
\newtheorem*{cor}{Corollary}
\theoremstyle{definition}
\newtheorem{defn}{Definition}[section]
\newtheorem{ex}{Exercise} 
\newtheorem{example}{Example}[section]
\theoremstyle{remark}
\newtheorem*{rem}{Remark}
\newtheorem*{note}{Note}
\newtheorem{case}{Case}
\usepackage{graphicx}
\usepackage{amssymb}
\usepackage{tikz-cd}
\usetikzlibrary{calc,arrows,decorations.pathreplacing}
\tikzset{mydot/.style={circle,fill,inner sep=1.5pt},
commutative diagrams/.cd,
  arrow style=tikz,
  diagrams={>=latex},
}

\usepackage{babel}
\usepackage{hyperref}
\hypersetup{
    colorlinks,
    citecolor=blue,
    filecolor=blue,
    linkcolor=blue,
    urlcolor=blue
}
\usepackage{pgfplots}
\usetikzlibrary{decorations.markings}
\pgfplotsset{compat=1.9}

\makeatother

\usepackage{babel}
\begin{document}
\begin{center}
\textbf{\large{}Morning Exam - 2019 CCC - Version A}\textbf{ }
\par\end{center}

({*}) 1. Consider the ellipse $E$ defined by the set of all points
$(x,y)$ in the plane such that $\frac{x^{2}}{4}+y^{2}=1$:

\begin{center}\begin{tikzpicture}\begin{axis}[axis lines = middle, ticks=none, 
xlabel = $x$, 
ylabel = {$y$}, 
xmin=-3,xmax=3,
ymin=-3,ymax=3,  ]

\addplot [domain=0:2*pi,samples=100,color=black,thick] ({2*cos(deg(x))}, {sin(deg(x))});
\addplot [domain=-3:3,samples=100,color=black,dashed] {-2}; 


\node[circle, fill=black, inner sep=1.5pt, label=above right:$ ( 0 \text{,} 1 )$] (x) at (axis cs:0,1) {$$};
\node[circle, fill=black, inner sep=1.5pt, label=below right :$ (2 \text{,} 0 )$] (y) at (axis cs:2,0) {$$};



\end{axis}\end{tikzpicture}\end{center}

Find the volume of the elliptic torus $\widetilde{E}$ obtained by
rotating $E$ around the $y=-2$ line. 
\begin{center}
\begin{tabular}{lclclclcl}
(A) $8\pi$ &  & (B) $4\pi^{2}$ &  & (C) $8\pi^{2}$ &  & (D) ${\displaystyle 4\pi}$ &  & (E) none of these\tabularnewline
\end{tabular}
\par\end{center}

2. Let $\Gamma$ be the curve in the plane parametrized by $\gamma\colon[0,1]\to\mathbb{R}^{2}$
be given by
\[
\gamma(t)=\left(t,\left(1-t^{2/3}\right)^{3/2}\right)
\]
for all $t\in[0,1]$:

\begin{center}\begin{tikzpicture}\begin{axis}[axis lines = middle, xmin=-1,xmax=1,
ymin=-1,ymax=1,     
xtick={-1,1},     
ytick={-1,1},
]

\addplot [domain=0:(0.5)*pi,samples=100,color=black,thick] ({cos(deg(x))^3}, {sin(deg(x))^3});

 

\end{axis}\end{tikzpicture}\end{center}

Compute the arclength of $\Gamma$.
\begin{center}
\begin{tabular}{lclclclcl}
(A) $\frac{3}{2}$ &  & (B) $\frac{\pi}{2}$ &  & (C) $\frac{5\pi}{2}$ &  & (D) $\frac{5}{2}$ &  & (E) none of these\tabularnewline
\end{tabular}
\par\end{center}

3. Compute $\int_{0}^{\sin x}\arcsin t\mathrm{d}t$
\begin{center}
\begin{tabular}{lclclclcl}
(A) $x\sin x+\cos x+C$ &  & (B) $\sin x+x\cos x+C$ &  & (C) $-x\sin x+\cos x+C$ &  & (D) $\sin x-x\cos x+C$ &  & (E) none of these\tabularnewline
\end{tabular}
\par\end{center}

4. Compute $\sum_{n=1}^{\infty}\frac{4}{n^{2}+2n}$
\begin{center}
\begin{tabular}{lclclclcl}
(A) $2$ &  & (B) $3$ &  & (C) $4$ &  & (D) does not converge &  & (E) none of these\tabularnewline
\end{tabular}
\par\end{center}

6. Compute $-\frac{13}{2}\int_{0}^{\pi}e^{2x}\cos(3x)\mathrm{d}x$ 
\begin{center}
\begin{tabular}{lclclclcl}
(A) $1+3\pi$ &  & (B) $1+2\pi$ &  & (C) $-1+2\pi$ &  & (D) $-1+\pi$ &  & (E) none of these\tabularnewline
\end{tabular}
\par\end{center}

7. Compute $\int\frac{\ln(\ln x)}{x\ln x}\mathrm{d}x$
\begin{center}
\begin{tabular}{lclclclcl}
(A) $\ln^{2}(\ln x)+C$ &  & (B) $\frac{1}{2}\ln^{2}(\ln x)+C$ &  & (C) $\ln(\ln^{2}x)+C$ &  & (D) $\frac{1}{2}\ln(\ln^{2}x)+C$ &  & (E) none of these\tabularnewline
\end{tabular}
\par\end{center}

({*}) 8. Define $\varphi\colon\mathbb{N}\to\mathbb{N}$ by 
\[
\varphi(m)=\min\{n\in\mathbb{N}\mid2^{m}<3^{n}\}.
\]
for all $m\in\mathbb{N}$. Thus $\varphi(1)=1$, $\varphi(2)=2$,
$\varphi(3)=2$, $\varphi(4)=3$, and so on. The function $\varphi$
can be described more explicitely by
\begin{center}
\begin{tabular}{lclclclcl}
(A) $\left\lceil m\ln(2)/\ln(3)\right\rceil $ &  & (B) $\left\lceil m\ln(3)/\ln(2)\right\rceil $ &  & (C) $\left\lfloor m\ln(3)/\ln(2)\right\rfloor $ &  & (D) $\left\lfloor m\ln(2)/\ln(3)\right\rfloor $ &  & (E) none of these\tabularnewline
\end{tabular}
\par\end{center}

({*}{*}) 9. Determine whether the following series converges. Justify
your answer.
\[
\sum_{n=1}^{\infty}(-1)^{n}\sum_{i=n^{2}}^{(n+1)^{2}-1}\frac{1}{i}.
\]

If the series converges, then provide an upper bound for it. 
\end{document}
